\section{Strain-Sensitive Transport Approach}
\label{sec:transport:ssta}

\subsection{Model Development}
\label{sec:transport:ssta:framework}

Using the nonpremixed flamelet framework provided by \cref{eq:transport:overview:lei:flamelety,eq:transport:overview:lei:flamelett}, the proposed model applies molecular transport to particular species rather than to all species within regions of a certain Reynolds number. Selection of these species is through the strain-sensitivity parameter
\begin{equation}\label{eq:transport:ssta:framework:ssp}
  \zeta_k \equiv \frac{\rho\chi}{\dot{m}_k^{+}},
\end{equation}
where $\dot{m}_k^{+}$ is the chemical production rate of species $k$. When $\zeta_k > 1$, the rate of local mixing is greater than the formation chemistry rate (\textit{i.e.}, the chemistry is slow). Species $k$ is then identified as strain-sensitive and is confined to scales on the order of Kolmogorov or smaller, where molecular transport is dominant. Conversely, when $\zeta_k < 1$, the formation chemistry is fast, and the effective diffusivity of species $k$ is governed by the turbulent diffusivity. In this case, the length scales are now comparable to those of fuel-oxidizer mixing (such as for major species).

The strain-sensitivity parameter is shown for a selection of gas-phase species relevant to soot in \cref{fig:transport:ssta:framework:sspc7h16}. It is clear that $\zeta_{\text{A2}} > 1$ for any value of mixture fraction, indicating that differential diffusion for gas-phase naphthalene and larger PAH is important. Conversely, the strain-sensitivity parameters for acetylene, hydroxyl, and hydrogen are less than unity at slightly rich conditions, indicating that turbulent transport is dominant. Such a finding agrees with \cref{fig:transport:overview:lei:a2vsz,fig:transport:overview:lei:tvsz}, where differential diffusion is shown to be relevant for naphthalene but not for the other species. At this stage in model development, the minimum value of $\zeta_k$ is used to determine whether or not the species is strain-sensitive. This value provides the most conservative estimate for the species for which differential diffusion is important.

\begin{figure}[htb]
  \centering
  \includegraphics[width=0.43\linewidth]{ch-transport/figures/ZETAvsZ-C7H16-chi_st20}
  \caption[Strain-Sensitivity Parameter $\zeta_k$ for Various Species Within a \ce{C7H16}/\ce{N2} Mixture]{Strain-sensitivity parameter calculated from nonpremixed flamelet solutions for various species at $\chi_{st} = 20$ s$^{-1}$. The nitrogen-diluted, \textit{n}-heptane fuel mixture is the same as in \cref{fig:subfilter:zussp:chisensitivity}. The red line is for A2, the black line is for \ce{C2H2}, the blue line is for \ce{OH}, and the cyan line is for \ce{H}. The vertical black dashed line indicates the stoichiometric mixture fraction.}
  \label{fig:transport:ssta:framework:sspc7h16}
\end{figure}

A new definition of the effective species Lewis number $\check{Le}_k$ is required that depends on $\zeta_k$:
\begin{equation}\label{eq:transport:ssta:framework:lezetai}
  \check{Le}_k(\zeta_k) = \frac{Le_k}{Le_k + H(\min(\zeta_k) - 1)\cdot (1 - Le_k)},
\end{equation}
where $H(\cdot)$ is the Heaviside function. The flamelet equations for species and temperature become
\begin{equation}\label{eq:transport:ssta:framework:flamelety}
  \rho\pder[Y_k]{\tau} = \frac{\rho\chi}{2 \check{Le}_k(\zeta_k)}\sder[Y_k]{Z} + \dot{m}_k + \frac{\rho\chi}{2 \check{Le}_k(\zeta_k)}\frac{Y_k}{W}\sder[W]{Z} + V_k^{DD} - \dot{\rho}Y_k + (\dot{\rho}Z - \dot{m}_Z)\pder[Y_k]{Z}
\end{equation}
and
\begin{equation}\label{eq:transport:ssta:framework:flamelett}
  \begin{split}
    \rho c_p \pder[T]{\tau} &= \frac{\rho c_p \chi}{2}\sder[T]{Z} + \frac{\rho\chi}{2}\pder[c_p]{Z}\pder[T]{Z} - \sum\limits_{k} h_k\dot{m}_k \\
    &+ \sum\limits_{k} \frac{\rho\chi}{2 \check{Le}_k(\zeta_k)}\left( \pder[Y_k]{Z} + \frac{Y_k}{W}\pder[W]{Z} \right)(c_p - c_{p,k})\pder[T]{Z} + \dot{q}_{\text{RAD}} \\
    &+ \dot{H} + c_p(\dot{\rho}Z - \dot{m}_Z)\pder[T]{Z},
  \end{split}
\end{equation}
respectively. The last two terms of \cref{eq:transport:ssta:framework:flamelety} and the terms on the third line of \cref{eq:transport:ssta:framework:flamelett} are the same as those in \cref{eq:lesmodels:combust:flamelet:flamelety,eq:lesmodels:combust:flamelet:flamelett} to account for the removal of PAH from the gas-phase during nucleation and condensation. This framework for the effective Lewis numbers could also accommodate a Reynolds number dependence~\cite{wang2016}, but such a possibility has not been pursued in this thesis.

% This form is similar to that of Wang~\cite{wang2016}, so a Reynolds number dependence could also be added to the model in future work.


\subsection{\textit{A Priori} Analysis}
\label{sec:transport:ssta:dns}

%These profiles are plotted as a function of the Bilger mixture fraction~\cite{bilger1989} in \cref{fig:transport:ssta:dns:chi}.

%% \begin{figure}[htb]
%%   \centering
%%   \includegraphics[width=0.43\linewidth]{ch-transport/figures/chivsZBilger-C7H16-chi_st20}
%%   \caption[Scalar Dissipation Rate for Various Transport Approaches Within a \ce{C7H16}/\ce{N2} Mixture]{Scalar dissipation rates as a function of the Bilger mixture fraction~\cite{bilger1989}, calculated from nonpremixed flamelet solutions at $\chi_{st} = 20$ s$^{-1}$. The nitrogen-diluted, \textit{n}-heptane fuel mixture is the same as in \cref{fig:subfilter:zussp:chisensitivity}. The solid line is for transport with unity Lewis number, the dash-dotted line is for detailed transport, and the double-dashed line is for strain-sensitive transport. The vertical black dashed line indicates the stoichiometric mixture.}
%%   \label{fig:transport:ssta:dns:chi}
%% \end{figure}

%% The flame temperature and the mass fractions of acetylene and naphthalene are also plotted against the Bilger mixture fraction and are provided in \cref{fig:transport:ssta:dns:tc2h2a2vszbilger} for these transport approaches. In the plot for flame temperature, the profile of the proposed model nearly matches that of the unity Lewis number approach, replicating the trend from \cref{fig:transport:overview:lei:tvsz}. This behavior is expected, for the species that participate in the main heat-releasing chemistry have been modeled with unity Lewis numbers. On the other hand, the peak of the detailed transport approach is lower than those of the latter models. This phenomena is due to the high diffusivity of molecular hydrogen, which ``flattens'' out the temperature profile in mixture fraction space. Note that the detailed transport model is not appropriate in the highly turbulent regions of a nonpremixed jet flame.

%% \begin{figure}[ht]
%%   \centering
%%   \begin{subfigure}[b]{0.33\linewidth}
%%     \includegraphics[width=\linewidth]{ch-transport/figures/TvsZBilger-C7H16-chi_st20}
%%   \end{subfigure}%%
%%   \begin{subfigure}[b]{0.33\linewidth}
%%     \includegraphics[width=\linewidth]{ch-transport/figures/YC2H2vsZBilger-C7H16-chi_st20}
%%   \end{subfigure}%%
%%   \begin{subfigure}[b]{0.33\linewidth}
%%     \includegraphics[width=\linewidth]{ch-transport/figures/YA2vsZBilger-C7H16-chi_st20}
%%   \end{subfigure}
%%   \caption[\texorpdfstring{$T$}{T}, \texorpdfstring{$Y_{\ce{C2H2}}$}{YC2H2}, and \texorpdfstring{$Y_{\text{A2}}$}{YA2} for Various Transport Approaches Within a \ce{C7H16}/\ce{N2} Mixture]{Mass fractions of acetylene and naphthalene and flame temperature as a function of the Bilger mixture fraction~\cite{bilger1989}, calculated from nonpremixed flamelet solutions at $\chi_{st} = 20$ s$^{-1}$. The nitrogen-diluted, \textit{n}-heptane fuel mixture is the same as in \cref{fig:subfilter:zussp:chisensitivity}. All lines are defined in \cref{fig:transport:ssta:dns:chi}.}
%%   \label{fig:transport:ssta:dns:tc2h2a2vszbilger}
%% \end{figure}

Flamelet solutions using the strain-sensitive transport, detailed transport, and unity Lewis number transport models are obtained by evaluating all approaches at the same stoichiometric scalar dissipation rate for the nitrogen-diluted, \textit{n}-heptane mixture from \cref{sec:subfilter:dns}. The flame temperature and the mass fractions of acetylene and naphthalene are provided in \cref{fig:transport:ssta:dns:tc2h2a2vsz} for these transport approaches. In the plot of the flame temperature, the profile of the proposed model nearly matches that of the unity Lewis number approach, replicating the trend from \cref{fig:transport:overview:lei:tvsz}. This behavior is expected, for the species that participate in the main heat-releasing chemistry have been modeled with unity Lewis numbers. On the other hand, the peak of the detailed transport approach is shifted towards larger mixture fractions. This phenomenon is a result of \textit{n}-heptane's large Lewis number, which contributes to a convective velocity towards richer mixture fractions that is encapsulated by $V_k^{DD}$ in \cref{eq:transport:ssta:framework:flamelety}. Note that the detailed transport model is not appropriate in the highly turbulent regions of a nonpremixed jet flame and that this shifting phenomenon contradicts the results from the DNS~\cite{attili2016}.

% On the other hand, the peak of the detailed transport approach is lower than those of the latter models. This phenomena is due to the high diffusivity of molecular hydrogen, which ``flattens'' out the temperature profile in mixture fraction space.
% The rich-shifting of the peak is a result of \textit{n}-heptane's large Lewis number, which contributes to a convective velocity towards richer mixture fractions that is encapsulated by $V_k^{DD}$ in \cref{eq:transport:ssta:framework:flamelety}. Note that the detailed transport model is not appropriate in the highly turbulent regions of a nonpremixed jet flame.

\begin{figure}[ht]
  \centering
  \begin{subfigure}[b]{0.33\linewidth}
    \includegraphics[width=\linewidth]{ch-transport/figures/TvsZ-C7H16-chi_st20}
  \end{subfigure}%%
  \begin{subfigure}[b]{0.33\linewidth}
    \includegraphics[width=\linewidth]{ch-transport/figures/YC2H2vsZ-C7H16-chi_st20}
  \end{subfigure}%%
  \begin{subfigure}[b]{0.33\linewidth}
    \includegraphics[width=\linewidth]{ch-transport/figures/YA2vsZ-C7H16-chi_st20}
  \end{subfigure}
  \caption[$T$, $Y_{\ce{C2H2}}$, and $Y_{\text{A2}}$} for Various Transport Approaches Within a \ce{C7H16}/\ce{N2} Mixture]{Flame temperature and mass fractions of acetylene and naphthalene as a function of mixture fraction, calculated from nonpremixed flamelet solutions at $\chi_{st} = 20$ s$^{-1}$. The nitrogen-diluted, \textit{n}-heptane fuel mixture is the same as in \cref{fig:subfilter:zussp:chisensitivity}. The solid line is for transport with unity Lewis number, the dash-dotted line is for detailed transport, and the double-dashed line is for strain-sensitive transport. The vertical black dashed line indicates the stoichiometric mixture fraction.}
  \label{fig:transport:ssta:dns:tc2h2a2vsz}
\end{figure}

% All lines are defined in \cref{fig:transport:ssta:dns:chi}.
The acetylene and naphthalene mass fractions are available in the middle and right-hand plots of \cref{fig:transport:ssta:dns:tc2h2a2vsz}, respectively. Acetylene was identified as having a relatively fast chemical production rate in \cref{fig:transport:ssta:framework:sspc7h16}, so the profile from the proposed model closely follows the flamelet solution with unity Lewis number. Conversely, naphthalene was classified as being strain-sensitive. Note, in particular, that while the naphthalene mass fraction is significantly increased with the proposed model compared to unity Lewis numbers, it is still smaller than with full detailed transport. Overall, the Strain-Sensitive Transport Approach captures the trends that are observed in \cref{fig:transport:overview:lei:a2vsz}. 


\subsection{Strain-Sensitivity Parameter Dependencies}
\label{sec:transport:ssta:dependencies}

The LES in \cref{ch:lesresults} use a fuel mixture that is different from the nitrogen-diluted, \textit{n}-heptane fuel mixture of the DNS. Therefore, it is worthwhile to investigate the extent to which the strain-sensitivity parameter is dependent on the fuel mixture as well as other variables, such as the choice of chemical mechanism and stoichiometric scalar dissipation rate.

%In \cref{fig:transport:ssta:dependencies:fuelchem}, the strain sensitivity parameter has been plotted for the fuel mixtures of \ce{C2H4}/\ce{H2}/\ce{N2} at 40/41/19\% composition by volume~\cite{mahmoud2017}, pure \ce{C2H4}~\cite{shaddix2010,zhang2011}, and \textit{n}-\ce{C7H16}/\ce{N2} at 15/85\% composition by volume~\cite{bisetti2012,attili2014,attili2015}.
In \cref{fig:transport:ssta:dependencies:fuelchem}, the strain sensitivity parameter has been plotted for the fuel mixtures of \ce{C2H4}/\ce{H2}/\ce{N2} at 40/41/19\% composition by volume~\cite{mahmoud2017} and pure \ce{C2H4}~\cite{shaddix2010,zhang2011}. It is obvious that the species identified as strain-sensitive are the same across different fuel mixtures. The minimum value of the parameter for naphthalene is greater than unity for both mixtures, indicating that its transport should be modeled with molecular diffusion. Conversely, the minimum values of acetylene, hydroxyl, and hydrogen are less than unity. These species are not constrained to scales that are on the order of the Kolmogorov scales or smaller, so their transport is governed by turbulent eddies.
%It is obvious that the choice of the fuel mixture does not affect the ability of the strain-sensitivity parameter to identify species with slow production rates.

\begin{figure}[ht]
  \centering
  \begin{subfigure}[b]{0.33\linewidth}
    \includegraphics[width=\linewidth]{ch-transport/figures/ZETAvsZ-EHN-chi_st20}
  \end{subfigure}%%
  \begin{subfigure}[b]{0.33\linewidth}
    \includegraphics[width=\linewidth]{ch-transport/figures/ZETAvsZ-C2H4-chi_st20}
  \end{subfigure}%%
  \begin{subfigure}[b]{0.33\linewidth}
    \includegraphics[width=\linewidth]{ch-transport/figures/ZETAvsZ-C2H4-RedHept-chi_st20}
  \end{subfigure}
  %% \begin{subfigure}[b]{0.33\linewidth}
  %%   \includegraphics[width=\linewidth]{ch-transport/figures/ZETAvsZ-C7H16-chi_st20}
  %% \end{subfigure}
  \caption[Dependencies of Strain-Sensitivity Parameter $\zeta_k$ on Fuel Mixture and Chemical Mechanism]{Strain-sensitivity parameter calculated from nonpremixed flamelet solutions for various species at $\chi_{st} = 20$ s$^{-1}$. \textit{Left} - Fuel mixture of \ce{C2H4}/\ce{H2}/\ce{N2} (40/41/19\% by volume)~\cite{mahmoud2017}. \textit{Middle} and \textit{Right} - Fuel is pure \ce{C2H4}~\cite{shaddix2010,zhang2011}. The solutions of the left and middle plots are evaluated with a chemical mechanism that accounts for the formation and oxidation of PAH up to cyclopenta[cd]pyrene (\ce{C18H10})~\cite{blanquart2009588,narayanaswamy2010}, while the profiles in the right plot are from a reduced mechanism that accounts for PAH up to naphthalene (\ce{C10H8})~\cite{bisetti2012}. The lines are the same as in \cref{fig:transport:ssta:framework:sspc7h16}.} %\textit{Right} - Fuel mixture of \textit{n}-\ce{C7H16}/\ce{N2} (15/85\% by volume)~\cite{bisetti2012,attili2014,attili2015}. The solutions of the left and middle plots are evaluated with a chemical mechanism that accounts for the formation and oxidation of PAH up to cyclopenta[cd]pyrene (\ce{C18H10})~\cite{blanquart2009588,narayanaswamy2010}, while the profiles in the right plot are from a reduced mechanism that accounts for PAH up to naphthalene (\ce{C10H8})~\cite{bisetti2012}. The lines are the same as in \cref{fig:transport:ssta:framework:sspc7h16}.}
  \label{fig:transport:ssta:dependencies:fuelchem}
\end{figure}

Additionally, the strain-sensitivity parameter is robust to the choice of chemical mechanism. The left and middle plots of \cref{fig:transport:ssta:dependencies:fuelchem} use a detailed mechanism that contains chemistry for soot precursors with up to eighteen carbon atoms~\cite{blanquart2009588,narayanaswamy2010}, while the plot on the right uses a reduced mechanism that accounts for PAH only up to naphthalene~\cite{bisetti2012}. The same species are identified as strain-sensitive with each chemical mechanism. 

Analysis of the parameter's dependency on the stoichiometric scalar dissipation rate is also fruitful for understanding model performance in various regions of the turbulent flame. The strain-sensitivity parameter is plotted for $\chi_{st} = 0.1$, 1, and 10 s$^{-1}$ in the left, middle, and right plots of \cref{fig:transport:ssta:dependencies:chist}, respectively. It is clear that the same species are identified as strain-sensitive, irrespective of the scalar dissipation rate.

% It is interesting to note that as the stoichiometric scalar dissipation rate decreases, all lines shift downwards. As the rate of turbulent mixing decreases, the chemical production rates of all species become relatively substantial. It can be anticipated that at a low enough stoichiometric scalar dissipation rate, the minimum value of the parameter for naphthalene may become less than unity. In this situation, its rate of formation overtakes the rate of turbulent mixing, so it is no longer constrained to regions on the order of the Kolmogorov scale or smaller. At the same time, the Kolmogorov eddies begin to penetrate the flame's mixing zone, which is broadened at low scalar dissipation rates. Consequently, naphthalene and larger PAH would be transported with unity Lewis numbers under these conditions.

\begin{figure}[ht]
  \centering
  \begin{subfigure}[b]{0.33\linewidth}
    \includegraphics[width=\linewidth]{ch-transport/figures/ZETAvsZ-EHN-chi_st01}
  \end{subfigure}%%
  \begin{subfigure}[b]{0.33\linewidth}
    \includegraphics[width=\linewidth]{ch-transport/figures/ZETAvsZ-EHN-chi_st1}
  \end{subfigure}%%
  \begin{subfigure}[b]{0.33\linewidth}
    \includegraphics[width=\linewidth]{ch-transport/figures/ZETAvsZ-EHN-chi_st10}
  \end{subfigure}
  \caption[Dependency of Strain-Sensitivity Parameter $\zeta_k$ on $\chi_{st}$]{Strain-sensitivity parameter calculated from nonpremixed flamelet solutions at various values of stoichiometric scalar dissipation rate. The fuel mixture consists of \ce{C2H4}/\ce{H2}/\ce{N2} (40/41/19\% by volume)~\cite{mahmoud2017}, and the detailed chemical mechanism previously mentioned is used~\cite{blanquart2009588,narayanaswamy2010}. \textit{Left}: $\chi_{st} = 0.1$ s$^{-1}$. \textit{Middle}: $\chi_{st} = 1$ s$^{-1}$. \textit{Right}: $\chi_{st} = 10$ s$^{-1}$. The lines are the same as in \cref{fig:transport:ssta:framework:sspc7h16}.}
  \label{fig:transport:ssta:dependencies:chist}
\end{figure}

% At the other extreme, a high enough rate of turbulent mixing may cause the minimums of all lines to shift above unity. Note, however, that this condition may be unreachable for species like hydrogen and hydroxyl. For instance, the scalar dissipation would have to increase by three orders of magnitude in order for hydroxyl to be considered strain-sensitive. Such a value would certainly be far beyond the stoichiometric scalar dissipation rate at extinction. Additionally, naphthalene would cease to exist at such elevated levels of turbulence.

The corresponding plots for the mass fractions of acetylene and naphthalene evaluated with detailed transport, strain-sensitive transport, and unity Lewis number transport are available in \cref{fig:transport:ssta:dependencies:c2h2a2vschist}. As the stoichiometric scalar dissipation rate is increased from 0.1 s$^{-1}$ to 10 s$^{-1}$, the differences between the various transport models are generally magnified for both species. This is explained by the thinning of the flame structure, which intensifies the effects of molecular diffusion at high scalar dissipation rates. The abatement of gradients in all naphthalene profiles over mixture fraction space further confirms this phenomenon. Additionally, the acetylene mass fraction does not change by more than 20\% for any transport approach as the scalar dissipation rate is increased. Conversely, an increase of two orders of magnitude in the scalar dissipation rate leads to a drop in the mass fraction of naphthalene by roughly an order of magnitude.

\begin{figure}[ht]
  \centering
  \begin{subfigure}[b]{0.33\linewidth}
    \centering
    \includegraphics[width=\linewidth]{ch-transport/figures/YC2H2vsZ-EHN-chi_st01}
  \end{subfigure}%%
  \begin{subfigure}[b]{0.33\linewidth}
    \centering
    \includegraphics[width=\linewidth]{ch-transport/figures/YC2H2vsZ-EHN-chi_st1}
  \end{subfigure}%%
  \begin{subfigure}[b]{0.33\linewidth}
    \centering
    \includegraphics[width=\linewidth]{ch-transport/figures/YC2H2vsZ-EHN-chi_st10}
  \end{subfigure}
  \begin{subfigure}[b]{0.33\linewidth}
    \centering
    \includegraphics[width=\linewidth]{ch-transport/figures/YA2vsZ-EHN-chi_st01}
  \end{subfigure}%%
  \begin{subfigure}[b]{0.33\linewidth}
    \centering
    \includegraphics[width=\linewidth]{ch-transport/figures/YA2vsZ-EHN-chi_st1}
  \end{subfigure}%%
  \begin{subfigure}[b]{0.33\linewidth}
    \centering
    \includegraphics[width=\linewidth]{ch-transport/figures/YA2vsZ-EHN-chi_st10}
  \end{subfigure}
  \begin{subfigure}[b]{0.33\linewidth}
    \centering
    \includegraphics[width=\linewidth]{ch-transport/figures/YOHvsZ-EHN-chi_st01}
  \end{subfigure}%%
  \begin{subfigure}[b]{0.33\linewidth}
    \centering
    \includegraphics[width=\linewidth]{ch-transport/figures/YOHvsZ-EHN-chi_st1}
  \end{subfigure}%%
  \begin{subfigure}[b]{0.33\linewidth}
    \centering
    \includegraphics[width=\linewidth]{ch-transport/figures/YOHvsZ-EHN-chi_st10}
  \end{subfigure}
  \caption[Trends of $Y_{\ce{C2H2}}$, $Y_{\text{A2}}$, and $Y_{\ce{OH}}$ with $\chi_{st}$]{Mass fractions of acetylene, naphthalene, and hydroxyl as a function of mixture fraction at various values of stoichiometric scalar dissipation rate. \textit{Left}: $\chi_{st} = 0.1$ s$^{-1}$. \textit{Middle}: $\chi_{st} = 1$ s$^{-1}$. \textit{Right}: $\chi_{st} = 10$ s$^{-1}$. The fuel mixture and chemical mechanism are the same as in \cref{fig:transport:ssta:dependencies:chist}. The lines are the same as in \cref{fig:transport:ssta:dns:tc2h2a2vsz}.}
  \label{fig:transport:ssta:dependencies:c2h2a2vschist}
\end{figure}

For acetylene, the flamelet solution with strain-sensitive transport closely follows the profile from the unity Lewis number approach for all values of stoichiometric scalar dissipation rate. This behavior is attributed to acetylene's relatively fast production rate even in the presence of elevated turbulent mixing, as was shown in \cref{fig:transport:ssta:dependencies:chist}. On the other hand, naphthalene is identified as being strain-sensitive, so the proposed model tends to follow the profile from detailed transport. It is interesting to note that at $\chi_{st} = 0.1$ s$^{-1}$ and 1 s$^{-1}$, the proposed model predicts higher levels of naphthalene than the flamelet solution with full detailed transport. This may be the result of the relatively larger amounts of \ce{OH} with detailed transport versus with strain-sensitive transport, as shown in the bottom left and middle plots of \cref{fig:transport:ssta:dependencies:c2h2a2vschist}.

Ultimately, the Strain-Sensitive Transport Approach is an attempt to more accurately model the variety of species that play a role in the evolution of soot. In \cref{fig:transport:ssta:dependencies:c2h2a2vschist}, this proposed model has consistently predicted larger amounts of naphthalene than the unity Lewis number transport approach. The elevated levels of these soot precursors, sometimes by more than a factor of two at certain stoichiometric scalar dissipation rates, could potentially translate into an increase of the soot volume fraction by a factor of four or more over the unity Lewis number approach~\cite{hmom2009}. Results from Large Eddy Simulations with the proposed model as well as with the other transport models are presented in the following chapter.
