\chapter{Effect of Transport on PAH Evolution\label{ch:transport}}

The lifetime of soot is governed by the balance between growth and oxidation. Nucleation from collisions between PAH dimers~\cite{blanquart2009,schuetz2002,frenklach1991,wang2011}, condensation of PAH dimers on existing soot particles~\cite{blanquart2009,hmom2009}, and surface growth through the HACA surface reaction mechanism~\cite{frenklach1985,frenklach1991} are processes that increase soot mass by extracting species from the gas-phase. Conversely, oxidation by \ce{OH} and \ce{O2} removes carbon from particles~\cite{stanmore2001,neoh1981,kazakov1995} and can lead to their fragmentation and destruction~\cite{neoh1985,mueller2011}. Clearly, the dynamics of soot are governed by species with a broad range of chemical timescales and molecular weights. Therefore, accurately capturing interactions with gas-phase species is crucial when developing predictive models for soot evolution.

In \cref{ch:lesmodels}, the modeling foundation for LES was established. In particular, the classical nonpremixed flamelet equations for species and temperature~\cite{peters1984} were presented in \cref{sec:lesmodels:combust:flamelet} with modifications accounting for the formation of soot and radiative thermal losses. These equations were derived under the assumption that the species' effective Lewis numbers are unity. This assumption is based on the reasoning that turbulence mixes indiscriminately at sufficiently large Reynolds number~\cite{pitsch19981057}.

%% These equations were derived by transforming the physical space coordinate system of \cref{eq:lesmodels:combust:flamelet:consy,eq:lesmodels:combust:flamelet:const} to a mixture-fraction-based coordinate system. The latter conservation equations used Fick's law to evaluate the diffusion velocity, so all species were assumed to have the same constant diffusion coefficient $D$. A further simplication was made by selecting $D$ such that the species' effective Lewis numbers are unity, for it is often assumed that turbulence mixes all species indiscriminately at sufficiently high Reynolds numbers~\cite{pitsch19981057}.

However, DNS studies~\cite{bisetti2012,attili2014} have suggested that this assumption may be inappropriate for PAH, which are very sensitive to the local scalar dissipation rate due to their slow formation chemistry. These species are confined to spatially intermittent regions of low scalar dissipation rate that are on the order of the Kolmogorov scale or smaller~\cite{vaishnavi2008}. At such scales, transport is solely dictated by molecular diffusion. Therefore, a method to identify such strain-sensitive species and properly account for their interactions with turbulence is desired. The development of a model that addresses these points is the focus of this chapter.

%% Summary: Main point of chapter is to show that existing transport models do not properly account for the presence of soot. As a result, a strain sensitive transport model is developed.

%% Include a brief introduction explaining why accurately capturing the transport of gas-phase species is important for developing predictive models for soot evolution in LES (growth and oxixidation processes).

% include other files for sections of this chapter. These use the 'input' command since each section within a chapter should not start a new page.
% If you want to swap the order of sections, it is as simple as reversing the order you include them. 
\section{Overview of Gas-Phase Species Transport}
\label{sec:transport:overview}

Provide an overview of different transport modeling approaches and discuss their deficiencies.


\subsection{Equal Effective Diffusivities}
\label{sec:transport:overview:le1}

Summary of equal effective diffusivities approach.


\subsection{Considerations for PAH}
\label{sec:transport:overview:pah}

Explain why PAH requires another mode of transport (sensitivity to $\chi$ due to slow chemistry suggests molecular transport is more appropriate).


\subsection{Molecular Transport}
\label{sec:transport:overview:lei}

Summary of detailed transport approach.


\subsection{Bimodal Transport}
\label{sec:transport:overview:bimodal}

Summary of approaches that account for equal effective diffusivities as well as detailed transport (Wang).

\section{Strain-Sensitive Transport Approach}
\label{sec:transport:ssta}

Details on proposed strain-sensitive transport approach.


\subsection{Model Development}
\label{sec:transport:ssta:framework}

Introduce definitions of strain-sensitivity parameter as well as effective species Lewis number.


\subsection{\textit{A Priori} Analysis}
\label{sec:transport:ssta:dns}

Compare flamelet calculations of strain-sensitivity parameter to DNS database.


\subsection{Strain-Sensitivity Parameter Dependencies}
\label{sec:transport:ssta:dependencies}

Discuss dependencies on stoichiometric scalar dissipation rate, fuel mixture, chemical mechanism, and note any discrepancies between flamelet calculations and our intuition (ethylene, propane, etc.)

