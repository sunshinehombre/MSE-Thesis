\section{Soot-Chemistry-Turbulence Interactions}
\label{sec:intro:scti}

The importance of small-scale interactions between soot particles and chemistry has been highlighted previously. However, the evolution of soot in turbulent nonpremixed combustion is also heavily influenced by the underlying turbulent flow field. Turbulence is a flow regime characterized by seemingly chaotic variations in flow properties. It is a highly efficient mechanism for mixing and operates on a wide range of scales, from those on the order of the system geometry to the Kolmogorov scales, the smallest scales of turbulence where kinetic energy is dissipated into heat~\cite{pope2000}. It impacts the evolution of soot in the following ways. First, it is responsible for variations in the local strain rate due to turbulent vortices stretching and deforming one another. PAH, the soot precursors responsible for nucleation and condensation, possess relatively slow kinetics and only form in regions of low strain rate where the residence time is sufficiently long~\cite{bisetti2012,attili2014,attili2015}. Additionally, soot is characterized by a very large Schmidt number and is confined to very thin structures that are elongated by turbulent eddies. The combined effect of interactions with turbulence is a highly spatially and temporally intermittent soot field. This property has been observed in both experiments~\cite{lee2009,qamar2009,narayanaswamy2013,mahmoud2015} and in Direct Numerical Simulation (DNS)~\cite{lignell2007,lignell2008,bisetti2012,attili2014,attili2015}, a type of high-fidelity simulation where all scales of the flow are resolved.

Turbulence also indirectly affects the maturation process of soot by altering the local gas composition. Specifically, turbulence controls the local mixing between fuel and oxidizer, the extent of which can be quantified through the mixture fraction. Soot has been observed to exist only at fuel rich values of mixture fraction~\cite{bisetti2012,attili2014,mahmoud2015,park2017}, suggesting the importance of its location relative to the flame surface. In DNS studies of a turbulent nonpremixed ethylene jet flame by Lignell \etal~\cite{lignell2007,lignell2008}, it was noted that soot concentrations peaked when particles were convected into the flame. However, these studies used a 19-species reduced ethylene/air chemical mechanism and a semi-empirical soot model, where the nucleation and surface growth rates depended only on temperature and the acetylene concentration~\cite{leung1991}. Surface growth by acetylene was determined to be the greatest contributor to soot growth, for growth due to PAH was neglected.

Recently, Attili \etal~\cite{attili2014} performed the first three-dimensional DNS of a turbulent nonpremixed \textit{n}-heptane jet flame using a detailed chemical mechanism containing the PAH naphthalene and a soot model based on elementary physical processes and rates, implemented through a higher-order method of moments approach~\cite{hmom2009}. Through analysis of Lagrangian statistics, it was noted that turbulent mixing tended to push soot away from the flame, where soot mass growth occurred slowly due to condensation and surface reactions. In contrast to the findings of Lignell \etal~\cite{lignell2007,lignell2008}, particles that traveled towards the flame were rapidly oxidized by hydroxyl radicals. Additionally, surface growth by acetylene was observed to play a lesser role in mass growth compared to the contributions from PAH-based nucleation and condensation.

Previously, a statistical model for LES was developed to simulate the small-scale interactions between soot, chemistry, and turbulence~\cite{subfilterpdf2011}. It accounted for the intermittent nature of soot, but some simplifying assumptions were made about the relationship between soot quantities and the local mixture fraction. These assumptions are revisited in \cref{ch:subfilter} and a new model is proposed that attempts to more accurately capture the aforementioned properties observed in DNS. The impact of such a model on the source terms for oxidation and surface growth is assessed \textit{a priori} through an existing DNS database~\cite{attili2014}.


%% A previous model for these small-scale interactions between soot, combustion, and turbulence had proposed that the variables for soot should be independent of the variables for thermochemical quantities such as the mixture fraction. The reasoning behind this assumption was based on the timescale separation between the slow chemistry of soot and its precursors and the much faster heat-releasing chemistry. However, as was previously noted, the evolution of soot is intimately linked to the local mixture fraction, which is a thermochemical quantity. In \cref{ch:subfilter}, a model is developed that incorporates the relationship between soot and combustion chemistry while still accounting for the intermittent nature of soot. The impact of such a model on the source terms for oxidation and surface growth is assessed \textit{a priori} through an existing DNS database.

%% Main points:
%% Previous research has shown that turbulence constrains soot and its precursors to very thin, spatially intermittent regions. Experimental and computational evidence that support the previous statement. Point of work is to incorporate dependence of soot subfilter PDF on thermochemical variables - i.e. soot does not exist in lean regions, only in rich regions.

%% Previously proposed by Mueller and Pitsch that PAH had much slower formation chemistry compared to main heat-releasing chemistry (thermochemical variables). They proposed the timescales of soot scalars and the latter are disparate enough to be independent. They noted this timescale separation argument could be violated during soot oxidation, when there are enhanced interactions between soot and the major gas-phase chemistry, and during surface growth near the flame. Also, this argument only accounts for the timescales of PAH and soot formation, but soot, turbulence, and chemistry interact even after the nucleation of soot particles from PAH dimers.

%% The previous theory assumes soot is uniformly distributed in mixture fraction space, but this is not correct for non-smoking flames (DNS). Soot is oxidized before reaching lean mixtures. The local mixture fraction also guides the mode of evolution of soot. The previous theory will overpredict the rate of oxidation with a lesser impact on the rate of surface growth. The subfilter PDF must exclude the presence of soot at lean mixtures by incorporating the local ratios of fuel and oxidizer.

%% From Michael:
%% Turbulence affects soot:
%% 1) Confines to very thin structures that are stretched into long filaments
%% 2) PAH formed only in regions of low scalar dissipation rate due to its relatively slow chemistry

%% RANS studies with semi-empirical soot models, PAH-based inception models
%% LES studies with semi-empirical, acetylene-based inception model
%% DNS studies
%% Goal of work
