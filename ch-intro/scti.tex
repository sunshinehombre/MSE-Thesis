\section{Soot-Chemistry-Turbulence Interactions}
\label{sec:intro:scti}

The evolution of soot in turbulent nonpremixed combustion is heavily influenced by small-scale interactions between particles, combustion chemistry, and turbulence. Turbulence has been 

Main points:
Previous research has shown that turbulence constrains soot and its precursors to very thin, spatially intermittent regions. Experimental and computational evidence that support the previous statement. Point of work is to incorporate dependence of soot subfilter PDF on thermochemical variables - i.e. soot does not exist in lean regions, only in rich regions.

Previously proposed by Mueller and Pitsch that PAH had much slower formation chemistry compared to main heat-releasing chemistry (thermochemical variables). They proposed the timescales of soot scalars and the latter are disparate enough to be independent. They noted this timescale separation argument could be violated during soot oxidation, when there are enhance interactions between soot and the major gas-phase chemistry, and during surface growth near the flame. Also, this argument only accounts for the timescales of PAH and soot formation, but soot, turbulence, and chemistry interact even after the nucleation of soot particles from PAH dimers.

The previous theory assumes soot is uniformly distributed in mixture fraction space, but this is not correct for non-smoking flames (DNS). Soot is oxidized before reaching lean mixtures. The local mixture fraction also guides the mode of evolution of soot. The previous theory will overpredict the rate of oxidation with a lesser impact on the rate of surface growth. The subfilter PDF must exclude the presence of soot at lean mixtures by incorporating the local ratios of fuel and oxidizer.

From Michael:
Turbulence affects soot:
1) Confines to very thin structures that are stretched into long filaments
2) PAH formed only in regions of low scalar dissipation rate due to its relatively slow chemistry

RANS studies with semi-empirical soot models, PAH-based inception models
LES studies with semi-empirical, acetylene-based inception model
DNS studies
Goal of work
