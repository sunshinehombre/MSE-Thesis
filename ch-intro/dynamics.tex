\section{Dynamics of Soot}
\label{sec:intro:dynamics}
%\section{Modeling Framework for Large Eddy Simulation}
%\label{sec:intro:framework}

%% Summarize previous works on soot and turbulent combustion models as well as other closure approaches other than the presumed PDF approach.

%% There are three classes of statistical models that are generally utilized. The most accurate is the Monte Carlo (MC) approach, where a large population of notional particles is used to represent the NDF. The evolution of these particles is determined by assuming that all aerosol processes are governed by stochastic processes~\cite{balthasar2003} or that only the coagulation of particles occurs stochastically~\cite{lucchesi2017}. MC is able to capture the NDF with high accuracy, as thousands of internal coordinates can be used to provide highly detailed descriptions of aggregate structure and chemical composition~\cite{celnik2008,mosbach2009}. However, the computational cost associated with such accuracy constrains the application of MC to simple configurations such as homogeneous reactors~\cite{celnik2007} and one-dimensional laminar premixed flames~\cite{patterson2007}. Additionally, explicitly coupling MC to the gas-phase chemistry is not straightforward~\cite{celnik2007}.

%% The second class of statistical models comprises sectional methods, where the NDF is discretized into bins and equations are solved for the number of particles in each bin. Like MC simulations, sectional methods provide accurate depictions of the complete NDF. However, as the number of internal coordinates used to describe the NDF increases, the associated computational cost can become intractable due to the required number of bins~\cite{gelbard1980}. Sectional methods do possess an advantage over MC, as they are deterministic and can be explicitly coupled to gas-phase chemistry.

The life of a soot particle begins with the formation of Polycyclic Aromatic Hydrocarbons (PAH). The exact mechanism for this process is not yet fully understood, but in general, large aliphatic fuel molecules are oxidized into smaller hydrocarbons by $\beta$-scission and \ce{H}-abstraction~\cite{law2006}. Further reactions eventually lead to the creation of species such as acetylene (\ce{C2H2}), propargyl (\ce{C3H3}), cyclopentadienyl (\ce{C5H5}), phenyl (\ce{C6H5}), and benzene (\ce{C6H6})~\cite{wang1997,richter2000,wang2011}. Benzene, the first aromatic ring, plays an important role in the production of multi-ring aromatic species. Larger PAH with molecular weights of 500-1000 amu are considered to be the immediate precursors of soot, and their formation can occur through the attachment of \ce{C2}, \ce{C3}, and other small units to benzene~\cite{wang1997,richter2000}. Growth is further fostered through the addition of PAH radicals and through reactions between PAH species, including PAH-PAH radical recombination and addition reactions~\cite{richter2000,wang2011}. Collisions between these gas-phase PAH lead to dimerization, and the collisions between PAH dimers form solid-phase nascent soot particles known as primary particles~\cite{frenklach1991,richter2000,schuetz2002,blanquart2009,wang2011}. This process can take several milliseconds~\cite{richter200,wang2011} and is summarized in the first two frames of \cref{fig:intro:dynamics:sootdynamics}.

\begin{figure}[htb]
  \centering
  \includegraphics[width=\linewidth]{ch-intro/figures/soot-dynamics}
  \caption[Dynamics of Soot]{Various processes that govern the dynamics of soot.}
  \label{fig:intro:dynamics:sootdynamics}
\end{figure}

Once the first primary particles appear, their evolution is dictated by various physical and chemical processes. In the top right frame of \cref{fig:intro:dynamics:sootdynamics}, coagulation is depicted. During coagulation, the number density of particles decreases as existing particles collide and no mass is transferred from gas-phase species to solid-phase particles~\cite{kazakov1995,hmom2009}. There are two limits as to how coagulation can occur. In the limit of pure coalescence, a primary particle is assumed to undergo maximum deformation as it collides with another particle to form a larger spherical particle. This is facilitated by the liquid-like nature of nascent particles as noted in experimental studies~\cite{dobbins1998,dobbins2002}. In the limit of pure aggregation, the colliding particles do not deform and the total surface area is assumed to be preserved in the resulting particle.

Soot can also evolve through two different growth pathways, as shown in the bottom left and middle frames of \cref{fig:intro:dynamics:sootdynamics}. During condensation, a gas-phase PAH dimer collides with a soot particle and attaches to its surface~\cite{hmom2009,blanquart2009}. Surface growth, on the other hand, involves reactions with gas-phase acetylene. Carbon atoms are added to the surface of the soot particle through the \ce{H}-Abstraction \ce{C2H2}-Addition (HACA) mechanism~\cite{frenklach1985,frenklach1991}. Both growth processes influence soot morphology by rendering particles and aggregates more spherical~\cite{mitchell1998,mitchell2003,park2003}.

These growth modes are balanced by destruction, as illustrated in the bottom right frame of \cref{fig:intro:dynamics:sootdynamics}. Oxidation occurs when hydroxyl radicals (and molecular oxygen to a lesser extent) strip carbon atoms from the surface of soot particles, forming products such as \ce{CO} and \ce{CO2}~\cite{kazakov1995,neoh1981,stanmore2001}. Oxidation by hydroxyl radicals proceeds rapidly, but oxidation by molecular oxygen occurs slowly enough such that the internal structures of large aggregates are weakened. This leaves these aggregates susceptible to breaking apart in a process known as fragmentation~\cite{mueller2011,neoh1985}.

Clearly, the lifetime of soot is governed by interactions with species of various weights, lengthscales, and timescales. Accurately predicting the evolution of soot in a turbulent reacting flow requires accounting for these interactions as well as for the influence of the surrounding flow field. Previous research addressing these topics is explored in the upcoming sections.
