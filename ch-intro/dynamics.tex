\section{Dynamics of Soot}
\label{sec:intro:dynamics}
%\section{Modeling Framework for Large Eddy Simulation}
%\label{sec:intro:framework}

%% Summarize previous works on soot and turbulent combustion models as well as other closure approaches other than the presumed PDF approach.

%% There are three classes of statistical models that are generally utilized. The most accurate is the Monte Carlo (MC) approach, where a large population of notional particles is used to represent the NDF. The evolution of these particles is determined by assuming that all aerosol processes are governed by stochastic processes~\cite{balthasar2003} or that only the coagulation of particles occurs stochastically~\cite{lucchesi2017}. MC is able to capture the NDF with high accuracy, as thousands of internal coordinates can be used to provide highly detailed descriptions of aggregate structure and chemical composition~\cite{celnik2008,mosbach2009}. However, the computational cost associated with such accuracy constrains the application of MC to simple configurations such as homogeneous reactors~\cite{celnik2007} and one-dimensional laminar premixed flames~\cite{patterson2007}. Additionally, explicitly coupling MC to the gas-phase chemistry is not straightforward~\cite{celnik2007}.

%% The second class of statistical models comprises sectional methods, where the NDF is discretized into bins and equations are solved for the number of particles in each bin. Like MC simulations, sectional methods provide accurate depictions of the complete NDF. However, as the number of internal coordinates used to describe the NDF increases, the associated computational cost can become intractable due to the required number of bins~\cite{gelbard1980}. Sectional methods do possess an advantage over MC, as they are deterministic and can be explicitly coupled to gas-phase chemistry.

\begin{figure}[htb]
  \centering
  \includegraphics[width=\linewidth]{ch-intro/figures/soot-dynamics}
  \caption[Dynamics of Soot]{Various processes that govern the dynamics of soot.}
  \label{fig:intro:dynamics:sootdynamics}
\end{figure}
