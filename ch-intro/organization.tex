\section{Organization of Thesis}
\label{sec:intro:org}

This thesis is organized as follows. In \cref{ch:lesmodels}, the foundation for modeling soot in LES of turbulent nonpremixed combustion is presented. In \cref{ch:subfilter}, the model for small-scale interactions between soot, combustion chemistry, and turbulence is developed and validated \textit{a priori} against a recent DNS database. In \cref{ch:transport}, the transport model for species with relatively slow formation chemistry is presented and validated \textit{a priori} with solutions to the flamelet equations. In \cref{ch:lesresults}, these models are implemented in LES and validated against experimental measurements from a series of three laboratory-scale flames. Finally, in \cref{ch:conclusion}, the results from this thesis are summarized and suggestions for future work are proposed.

The accomplishments and new contributions of this thesis are outlined below.
\begin{itemize}
\item \textbf{$Z$-Activated Soot Subfilter PDF (ZASSP):} Developed a presumed subfilter PDF model to close the filtered transport equations for the soot model. It contains a dependence on the mixture fraction to address the lack of soot in fuel-lean regions and has the form of a double-delta distribution to account for the high spatial intermittency of soot (Chapter 3).
\item \textbf{Strain-Sensitive Transport Approach (SSTA):} Developed a model that transports species governed by relatively slow formation chemistry with molecular diffusion and transports species governed by fast kinetics with equal effective diffusivities. A strain-sensitivity parameter is used to categorize each species (Chapter 4).
\item \textbf{Validation of the Integrated Large Eddy Simulation Model:} Conducted Large Eddy Simulations with ZASSP and SSTA for a series of three laboratory-scale ethylene-hydrogen-nitrogen simple jet flames (Chapter 5).
\end{itemize}
