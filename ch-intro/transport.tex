\section{Effect of Transport on Soot Evolution}
\label{sec:intro:transport}

In addition to the effects discussed previously, turbulent eddies contribute to various levels of strain in the velocity field. As previously noted, PAH are highly sensitive to strain. In experiments of counterflow diffusion flames~\cite{}, an increase in the strain rate lead to a decrease in the amount of soot observed. Similarly, in turbulent 

Experiments~\cite{} and simulations~\cite{} have found that PAH are highly sensitive to the amount of strain.

Can discuss various approaches to modeling transport for sooting flames. Can discuss diffusion between soot and gas-phase species (DNS paper by Jackie Chen).

Main points:
Discussion about soot PAH precursors and other strain-sensitive species.
Classical theory is that turbulence mixes indiscriminately at sufficiently high Re.
PAH are very sensitive to the local scalar dissipation rate due to their slow formation chemistry
DNS studies suggest PAH are confined to spatially intermittent regions of low scalar dissipation rate that are on the order of the Kolmogorov scale or smaller
At such scales, transport is governed by molecular diffusion
Pitsch and Peters flamelet equations with full differential diffusion
Wang's bimodal transport
