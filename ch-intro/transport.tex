\section{Effect of Transport on Soot Evolution}
\label{sec:intro:transport}

The evolution of soot is dependent on its interactions with species such as \ce{C2H2}, \ce{OH}, PAH, and others mentioned in \cref{sec:intro:dynamics}. Therefore, accurately modeling the transport of these species throughout a system is vital to enabling better predictions of soot dynamics. The presence of turbulence can profoundly influence the nature of this transport and must be accounted for properly. In a classical model for turbulent combustion known as the flamelet approach, it is often assumed that the species' effective Lewis numbers are unity~\cite{peters1984}. This theory is based on the belief that turbulence mixes indiscriminately at sufficiently large Reynolds number, causing species to diffuse at the same rate~\cite{pitsch19981057}.

Turbulence is also responsible for large variations in the scalar dissipation rate field. As noted in the previous section, PAH are highly sensitive to the local strain rate due to their slow kinetics. In experiments and simulations of laminar nonpremixed counterflow flames, the total amount of PAH and soot was found to decrease with an increase in the strain rate~\cite{decroix2000,cuoci2009,huijnen2010,wang2016433}. Similarly, an inverse relationship between the global mixing/strain rate and the amount of PAH and soot has been observed for turbulent nonpremixed simple jet flames in experiments~\cite{kent1984,qamar2005,narayanaswamy2013,mahmoud2017} and simulation~\cite{attili2015}. Such a trend has also been noted with respect to the local strain rate for the latter system~\cite{bisetti2012,mueller2013,attili2014}. The consequence of this sensitivity is that PAH are confined to intermittent regions of low scalar dissipation rate. These regions are on the order of the Kolmogorov scale or smaller, where transport is governed by molecular diffusion~\cite{vaishnavi2008}. This insight conflicts with the frequently used assumption of unity Lewis number transport in turbulent combustion.

In \cref{ch:transport}, this discrepancy is addressed through the development of the strain-sensitive transport approach. A method to identify PAH and other species sensitive to strain is proposed and an \textit{a priori} analysis is conducted to assess its robustness.


%% talk about how pah is strain sensitive
%% mention they are governed by molecular diffusion
%% bring up evidence that soot/pah should be governed by differential diffusion
%% end with material in ch 4

%% When modeling turbulent combustion, a classical theory known as the flamelet approach often assumes

%% The presence of turbulence can largely influence the transportation of species throughout a system. 
%% Turbulence is responsible for large variations in the scalar dissipation rate field. As noted in the previous section, PAH are highly sensitive to the local strain rate due to 

%% In addition to the effects discussed previously, turbulent eddies contribute to various levels of strain in the velocity field. As previously noted, PAH are highly sensitive to strain. In experiments of counterflow diffusion flames~\cite{}, an increase in the strain rate lead to a decrease in the amount of soot observed. Similarly, in turbulent 

%% Experiments~\cite{} and simulations~\cite{} have found that PAH are highly sensitive to the amount of strain.

%% Can discuss various approaches to modeling transport for sooting flames. Can discuss diffusion between soot and gas-phase species (DNS paper by Jackie Chen).

%% want to expose a gap in existing works that this thesis will fill
%% various theories of how turbulence affects transport of species - unity Le, full DD
%% proof that PAH are highly sensitive to the strain rate


%% Main points:
%% Discussion about soot PAH precursors and other strain-sensitive species.
%% Classical theory is that turbulence mixes indiscriminately at sufficiently high Re.
%% PAH are very sensitive to the local scalar dissipation rate due to their slow formation chemistry
%% DNS studies suggest PAH are confined to spatially intermittent regions of low scalar dissipation rate that are on the order of the Kolmogorov scale or smaller
%% At such scales, transport is governed by molecular diffusion
%% Pitsch and Peters flamelet equations with full differential diffusion
%% Wang's bimodal transport
