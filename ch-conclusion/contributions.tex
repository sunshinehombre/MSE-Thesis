\section{Major Contributions}
\label{sec:conclusion:contributions}

\subsection{Subfilter Modeling Advancements}
\label{sec:conclusion:contributions:subfilter}

The spatially filtered governing equations in LES contain unclosed terms that represent small-scale interactions between soot, turbulence, and combustion chemistry. These unresolved terms, such as the filtered source term, are closed by convoluting unfiltered quantities with a density-weighted joint subfilter PDF of the thermochemical and soot variables. However, this joint subfilter PDF is unknown, so the presumed PDF approach was adopted in this thesis to assume a form of the PDF based on physical insights about the subfilter interactions.

First, Bayes' theorem was used to decompose the joint subfilter PDF into a thermochemical PDF and a PDF of the soot scalars conditioned on the thermochemical variables. The former was obtained from Ihme and Pitsch~\cite{ihme2008}, where the mixture fraction is modeled as a beta distribution that is parameterized with the filtered mixture fraction and its subfilter variance. Following Mueller and Pitsch~\cite{mueller2012}, this subfilter variance was obtained by evaluating a transport equation for the filtered square of the mixture fraction. 

Previous DNS studies~\cite{bisetti2012,attili2014,attili2015} observed that PAH are confined to narrow, spatially intermittent regions of low scalar dissipation rate due to their relatively slow chemistry. Soot was found to be even more intermittent since its source terms are non-linearly dependent on the amount of PAH precursors and it is strongly influenced by differential diffusion with respect to mixture fraction. As a result, Mueller and Pitsch~\cite{subfilterpdf2011} modeled the PDF of the soot scalars as a double delta distribution with non-sooting and sooting modes to capture the spatial inhomogeneity. It was also proposed that the soot subfilter PDF be independent of the thermochemical variables due to the timescale separation between the slower formation chemistry of soot and its PAH precursors and the faster chemistry represented by the thermochemical quantities. Such a simplification assumed that soot is uniformly distributed in mixture fraction space. However, this is inaccurate for non-smoking flames, where fuel-lean regions should be devoid of soot. Thus, the $Z$-Uniform Soot Subfilter PDF was likely to significantly overestimate the rate of oxidation due to its prediction of non-physical soot in those regions.

In \cref{ch:subfilter}, the $Z$-Activated Soot Subfilter PDF was developed to address this point. The soot subfilter PDF's dependence on the thermochemical variables was maintained with conditioning on the mixture fraction. In this model, the sooting mode is activated only for rich values of mixture fraction to account for the rapid oxidation of soot before it reaches lean mixtures. The sooting mode still presumes a uniform subfilter distribution of soot at rich values of mixture fraction, which is appropriate at later times in a nonpremixed jet flame when turbulent mixing has distributed soot throughout mixture fraction space~\cite{attili2014}. The dependence on the mixture fraction was also incorporated into the weights of the non-sooting and sooting modes, which are determined by solving a transport equation for the filtered second moment of the total number density.

The $Z$-Activated Soot Subfilter PDF was validated through an \textit{a priori} analysis against the DNS database of Attili \etal~\cite{attili2014}. The analysis revealed that the magnitudes of the largest filtered moment source terms for oxidation were half of those from the $Z$-Uniform Soot Subfilter PDF. The $Z$-Activated Soot Subfilter PDF tended to produce smaller oxidation source terms than the $Z$-uniform soot subfilter PDF, although it still overpredicted the values from DNS. Meanwhile, predictions of the filtered moment source term for surface growth were observed to be unaffected by the $Z$-Activated Soot Subfilter PDF. Increasing the normalized filter width lead to further attenuation of the oxidation source terms while the surface growth source terms were minimally affected.

\subsection{Transport for Strain-Sensitive Species}
\label{sec:conclusion:contributions:transport}

In \cref{ch:transport}, an advancement was made in understanding the effect of transport on the evolution of PAH. In the flamelet framework, the nonpremixed flamelet equations have been previously derived while assuming that the species' effective Lewis numbers are unity~\cite{peters1984}, that all species are governed by molecular diffusion~\cite{pitsch1998}, and that all species' effective Lewis numbers are functions of the Reynolds number~\cite{wang2016}. However, the species that dictate the flame temperature are governed by unity Lewis numbers while PAH are governed by molecular transport due to their relatively slow formation chemistry. PAH are restricted to narrow regions of low scalar dissipation rate that are on the order of the Kolmogorov scale or smaller, where transport is governed by molecular diffusion irrespective of the Reynolds number. The Strain-Sensitive Transport Approach was developed to properly model the transport of species with relatively slow chemistry, such as PAH.

Categorization of each species was achieved through the strain-sensitivity parameter, which compared the local rate of mixing to the species' rate of formation. When this parameter is greater than unity, the rate of formation is slower than the rate of mixing and the species is identified as strain-sensitive. It is confined to scales on the order of Kolmogorov or smaller, where molecular transport is dominant. Conversely, when the strain-sensitivity parameter is less than unity, the kinetics are fast and the species' effective diffusivity is governed by the turbulent diffusivity. The length scales of these species are comparable to those of fuel-oxidizer mixing, such as for the major species. This parameter was incorporated into the existing nonpremixed flamelet equations accounting for differential diffusion by replacing all instances of the species Lewis number with an effective Lewis number that depends on the minimum value of the strain-sensitivity parameter.

The Strain-Sensitive Transport Approach was validated through an \textit{a priori} analysis using flamelet solutions from the nitrogen-diluted, \textit{n}-heptane mixture of \cref{sec:subfilter:dns} at $\chi_{st} = 20$ s$^{-1}$. The flame temperature profile from the flamelet solution with the proposed model nearly matched that from the flamelet solution with unity effective Lewis numbers. A similar pattern was observed for the acetylene mass fraction profile since acetylene was identified as a species with a relatively fast chemical production rate. The mass fraction profile for naphthalene, on the other hand, was significantly increased in the flamelet solution with the proposed model when compared to the solution with unity effective Lewis numbers. Naphthalene and other species with relatively slow chemistry were consistently identified as strain-sensitive even as the fuel mixture or chemical mechanism was varied. Additionally, the aforementioned trends held as the stoichiometric scalar dissipation rate was increased by two orders of magnitude from 0.1 to 10 s$^{-1}$. It was observed that the acetylene mass fraction did not change by more than 20\% for any transport approach while the naphthalene mass fraction experienced an order of magnitude decrease.

\subsection{Results from Large Eddy Simulations}
\label{sec:conclusion:contributions:les}

In \cref{ch:lesresults}, the $Z$-Activated Soot Subfilter PDF and the Strain-Sensitive Transport Approach were validated against experimental data from a series of three turbulent nonpremixed ethylene-hydrogen-nitrogen jet flames with varying amounts of global strain at a constant Reynolds number~\cite{mahmoud2017}. Time-averaged centerline flame temperature and soot volume fraction measurements were compared to profiles from LES. In the first batch of simulations, the effective Lewis numbers of all gas-phase species were held constant at unity to evaluate the differences between the $Z$-Activated Soot Subfilter PDF and the $Z$-Uniform Soot Subfilter PDF. The LES with both soot subfilter PDFs were found to predict the experimental profile for flame temperature fairly well, although the simulated flames tended to be longer. However, both sets of LES underestimated the experimental soot volume fraction by two orders of magnitude. The profiles from LES tended to peak slightly upstream and approached zero sooner at $x/D \approx 130$, indicating the presence of intense oxidation further upstream. The LES with the $Z$-Activated Soot Subfilter PDF were observed to experience less oxidation, although the differences were minimal due to the activation location of the sooting mode in mixture fraction space.

The Strain-Sensitive Transport Approach and the $Z$-Activated Soot Subfilter PDF were both implemented in the LES of \cref{sec:lesresults:ssta} to evaluate the ability of the integrated modeling approach to accurately predict the evolution of soot. First, the latter was compared to LES with unity effective Lewis numbers and detailed transport for flame $1/\tau|_M$. The LES with the proposed models and the LES with unity effective Lewis numbers predicted the experimental centerline flame temperature fairly accurately, as was observed previously. In contrast, the peak mean temperature from the LES with full detailed transport was predicted to be further downstream due to the small Lewis number of the hydrogen radical, which resulted in a convective velocity towards lean regions in mixture fraction space.

The time-averaged, centerline soot volume fractions were still underpredicted by all LES. The volume fractions from the LES with the proposed models and the LES with unity effective Lewis numbers both peaked and approached zero earlier than the experimental measurement. However, the peak volume fraction from the LES with the proposed models was observed to be 1.6 to 2.1 times larger than that from the LES with the equal effective diffusivities approach. A closer look at the normalized volume fraction source terms revealed that the LES with the proposed models possessed a larger proportion of nucleation and condensaton occurring earlier at $x/D = 49$. The LES with full detailed transport was observed to have a larger soot volume fraction at downstream locations and a maximum value that was twice that of the LES with the proposed models. However, the hydrogen radical was not identified as a strain-sensitive species. Modeling its behavior with molecular diffusion is inappropriate, as evidenced by the inaccurate flame temperature profile. % These phenomena were the result of the elongated flame structure, which enabled soot to accumulate for an extended period of time before oxidation.

The sensitivity of soot evolution to the chemical kinetics was also investigated. The previously discussed results utilized the BN1 mechanism~\cite{blanquart2009,narayanaswamy2010}, which placed emphasis on the formation of soot precursors up to cyclopenta[cd]pyrene (\ce{C18H10}). LES with equal effective diffusivities and the Strain-Sensitive Transport Approach were also conducted with the KM2 mechanism~\cite{wang2013}, where the peak volume fractions were eight to ten times larger than the corresponding LES with the BN1 mechanism. The LES with the proposed models and the KM2 mechanism still underestimated the soot volume fraction by a factor of ten, suggesting that the integrated modeling framework is still missing some aspect of the physics of soot evolution and that the uncertainties in chemical kinetics do not fully explain the discrepancies.

Despite this, the LES with the proposed models were able to capture experimental trends with respect to the global strain rate. An increase in the exit strain rate from jet flame $1/\tau|_L$ to $1/\tau|_H$ did not significantly impact the maximum flame temperatures recorded experimentally because the gas-phase species that participate in the main exothermic reactions, such as \ce{CO} or \ce{CO2}, have fast chemical production rates that easily adjust to the global strain rate. The LES with the proposed models accurately identified these species as not strain-sensitive, and were able to reproduce the flame temperature profiles for all three flames. Conversely, the centerline soot volume fraction was experimentally observed to be inversely related to the global strain rate. The LES with the $Z$-Activated Soot Subfilter PDF and the Strain-Sensitive Transport Approach were generally able to capture this trend, although the normalized maximum centerline volume fraction for jet flame $1/\tau|_M$ was underpredicted. Analysis of the volume fraction source terms for all three flames revealed that the source term due to combined nucleation and condensation grew in importance downstream while the source term due to surface growth became less dominant at upstream locations as the global strain rate was decreased. The LES with the proposed models was observed to emphasize the importance of the PAH-based soot growth modes at lower global strain rates much more than the LES with equal effective diffusivities. The quantitative differences between the two transport approaches highlighted the importance of properly modeling the physics of PAH.


