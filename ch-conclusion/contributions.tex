\section{Major Contributions}
\label{sec:conclusion:contributions}


\subsection{Subfilter Modeling Advancements}
\label{sec:conclusion:contributions:subfilter}

The spatially filtered governing equations in LES contain unclosed terms that represent small-scale interactions between soot, turbulence, and combustion chemistry. These unresolved terms, such as the filtered source term, are closed by convoluting unfiltered quantities with a density-weighted joint subfilter PDF of the thermochemical and soot variables. However, this joint subfilter PDF is unknown, so the presumed PDF approach was adopted in this thesis to assume a form of the PDF based on physical insights about the subfilter interactions.

First, Bayes' theorem was used to decompose the joint subfilter PDF into a thermochemical PDF and a PDF of the soot scalars conditioned on the thermochemical variables. The former was obtained from Ihme and Pitsch~\cite{ihme2008}, where the mixture fraction is modeled as a beta distribution that is parameterized with the subfilter mixture fraction variance. Following Mueller and Pitsch~\cite{mueller2012}, this subfilter variance was obtained by evaluating a transport equation for the filtered square of the mixture fraction. 

Previous DNS studies~\cite{bisetti2012,attili2014,attili2015} observed that PAH are confined to narrow, spatially intermittent regions of low scalar dissipation rate due to their relatively slow chemistry. Soot was found to be even more intermittent since its source terms are non-linearly dependent on the amount of PAH precursors and it is strongly influenced by differential diffusion with respect to mixture fraction. As a result, Mueller and Pitsch~\cite{subfilterpdf2011} modeled the PDF of the soot scalars as a double delta distribution with non-sooting and sooting modes to capture the spatial inhomogeneity. It was also proposed that the soot subfilter PDF be independent of the thermochemical variables due to the timescale separation between the slower formation chemistry of soot and its PAH precursors and the faster chemistry represented by the thermochemical quantities. Such a simplification assumed that soot is uniformly distributed in mixture fraction space. However, this is inaccurate for non-smoking flames, where fuel-lean regions should be devoid of soot. Thus, the $Z$-uniform soot subfilter PDF was likely to significantly overestimate the rate of oxidation due to its prediction of non-physical soot in those regions.

In \cref{ch:subfilter}, the $Z$-activated soot subfilter PDF was developed to address this point. The soot subfilter PDF's dependence on the thermochemical variables was maintained, with conditioning on the mixture fraction. In this model, the sooting mode is activated only for rich values of mixture fraction to account for the rapid oxidation of soot before it reaches lean mixtures. The sooting mode still presumes a uniform subfilter distribution of soot at rich values of mixture fraction, which is appropriate at later times in a nonpremixed jet flame when turbulent mixing has distributed soot throughout mixture fraction space~\cite{attili2014}. The dependence on the mixture fraction was also incorporated into the weights of the non-sooting and sooting modes, which were determined by solving a transport equation for the filtered second moment of the total number density.

The $Z$-activated soot subfilter PDF was validated through an \textit{a priori} analysis against the DNS database of Attili \etal~\cite{attili2014}. The analysis revealed that the magnitudes of the largest filtered moment source terms for oxidation were half of those from the $Z$-uniform soot subfilter PDF. The $Z$-activated soot subfilter PDF tended to produce smaller oxidation source terms than the $Z$-uniform soot subfilter PDF, although it still overpredicted the values from DNS. Meanwhile, predictions of the filtered moment source term for surface growth were observed to be unaffected by the $Z$-activated soot subfilter PDF. Increasing the normalized filter width lead to further attenuation of the oxidation source terms while the surface growth source terms were minimally affected.

\subsection{Transport for Strain-Sensitive Species}
\label{sec:conclusion:contributions:transport}

Summarize advancements for transport modeling.
