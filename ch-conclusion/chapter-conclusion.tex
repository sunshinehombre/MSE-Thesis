\chapter{Conclusion\label{ch:conclusion}}

This thesis has presented several advancements in modeling the evolution of soot and its precursors in turbulent nonpremixed combustion. These models were developed for LES, where the geometry-dependent, large-scale phenomena are resolved and the small-scale features are modeled. LES provides improved predictions of turbulent mixing over simulations with Reynolds-Averaged Navier-Stokes (RANS), especially for large-scale phenomena such as swirl, separation, and recirculation that can impact soot dynamics. The evolution of soot is also heavily influenced by its physics and chemistry occurring at the unresolved scales, where the combustion community's understanding is far from complete. Therefore, the principal focus of this thesis was to provide novel insight and develop LES models for the interactions between soot, chemistry, and turbulence as well as for the transport of species with relatively slow chemistry at these small scales. The main results from model development, validation, and application are summarized below.

The framework for modeling the evolution of soot in LES of turbulent nonpremixed combustion was outlined in \cref{ch:lesmodels}. The Method of Moments was chosen to model the Number Density Function (NDF) of soot particles due to its lower computational expense. This statistical approach uses a bivariate volume $V$ and surface area $S$ description of soot to account for their geometrical complexities and involves solving equations that track the evolution of moments of the NDF. Closure of these equations was achieved through the Hybrid Method of Moments (HMOM)~\cite{hmom2009}, which accounts for the bimodal nature of the NDF resulting from the presence of small, incipient particles and larger, more mature aggregates. HMOM provides models for the physical and chemical processes that govern soot evolution including nucleation, coagulation, condensation, surface growth, oxidation, and fragmentation.

This soot model was integrated into a turbulent combustion model that describes the thermal and chemical structure of the nonpremixed flame and accounts for the formation of soot. In the flamelet approach~\cite{peters1984}, a three-dimensional turbulent flame is conceptualized as an ensemble of locally one-dimensional, in a transformed coordinate system, flame structures embedded in a turbulent flow field. Since thermal radiation can occur on similar temporal scales as soot evolution, the Radiative Flamelet/Progress Variable (RFPV) approach~\cite{ihme2008} with adaptions from Carbonell \etal~\cite{carbonell2009} was selected as the foundation for the turbulent combustion model. In this approach, the local thermochemical state is described by solutions to the steady flamelet equations that are augmented with radiative losses. These solutions are computed \textit{a priori} and are stored in a database that is accessed during LES through a reduced set of parameters that includes the filtered mixture fraction and its subfilter variance, the filtered progress variable, and the filtered heat loss parameter. Following Mueller and Pitsch~\cite{mueller2012}, the transport equation definitions for these parameters were modified to account for the extraction of PAH from the gas-phase and provide a unique parameterization of the thermochemical state.

However, previous works~\cite{attili2014,bisetti2012} have found that the use of the steady flamelet approximation leads to inaccurate predictions for the mass fractions of gas-phase PAH due to their relatively slow chemistry. To address this point, the transport equation model for PAH from Mueller and Pitsch~\cite{mueller2012} was incorporated into the LES modeling framework, where the chemical source term of the transport equation accounts for the production, consumption, and dimerization of PAH.

\section{Major Contributions}
\label{sec:conclusion:contributions}


\subsection{Subfilter Modeling Advancements}
\label{sec:conclusion:contributions:subfilter}

Summarize subfilter modeling advancements.


\subsection{Transport for Strain-Sensitive Species}
\label{sec:conclusion:contributions:transport}

Summarize advancements for transport modeling.
 % Contributions
\section{Recommendations for Future Work}
\label{sec:conclusion:future}

\subsection{Z-Activated Soot Subfilter PDF}
\label{sec:conclusion:future:zassp}

The $Z$-activated soot subfilter PDF was developed in \cref{sec:subfilter:zassp} to address the deficiencies of the $Z$-uniform soot subfilter PDF in capturing the connection between subfilter soot-turbulence interactions and combustion chemistry. Its form remains a double delta distribution to account for the non-sooting and sooting modes within an LES filter width. However, rather than decorrelate the soot scalars from the thermochemical variables, a dependence of the soot scalars on the mixture fraction was introduced in order to eliminate the presence of soot in below-stoichiometric regions from the sooting mode. Additionally, the sooting mode of the $Z$-activated soot subfilter PDF presumed a uniform distribution for rich values of mixture fraction. This model was shown to be more appropriate at later times in a nonpremixed jet when turbulent mixing had distributed soot throughout mixture fraction space~\cite{attili2014}. However, as demonstrated in \cref{fig:subfilter:leszussp:ysvsz}, the bulk of soot tends to be reside in narrower intervals of mixture fraction at the earlier stages of soot evolution.

An improvement to the $Z$-activated soot subfilter PDF can be made by concentrating the sooting mode near the location of peak soot inception in mixture fraction space at these earlier stages. This soot subfilter PDF could have a form similar to:
\begin{equation}\label{eq:conclusion:future:zassp:dd}
  \begin{split}
    P(\mc{M}_j|Z) &= \delta(\mc{M}_j)\left[ 1 - H(Z - Z_{st}) \right] \\
    &+ \omega\delta(\mc{M}_j)H(Z - Z_{st}) + (1 - \omega)f(\mc{M}_j, Z)H(Z - Z_{st}),
  \end{split}
\end{equation}
where the first term on the right-hand side represents the non-sooting mode at lean values of mixture fraction. The second and third terms on the second line represent the non-sooting and sooting modes at rich values of mixture fraction, respectively. Note that \cref{eq:subfilter:zassp:dd} had used delta distributions to model both the non-sooting and sooting modes. However, in order to capture the inhomogeneity of the sooting mode with respect to mixture fraction, the distribution $f(\mc{M}_j, Z)$ from \cref{eq:conclusion:future:zassp:dd} can no longer be a delta distribution. Additional analysis is required to determine the specific form of $f(\mc{M}_j, Z)$.

The $Z$-activated soot subfilter PDF can be further improved by adjusting the location of activation for the sooting mode in mixture fraction space. In \cref{sec:subfilter:zassp}, the sooting mode was activated at the stoichiometric mixture fraction to prevent non-physical oxidation from occurring in lean regions. Using this model, the LES from \cref{fig:lesresults:zassp:ctrlineleseval} revealed minimal increases in the predictions of the centerline soot volume fraction over those from the LES with the $Z$-uniform soot subfilter PDF. These results can be explained by examining \cref{fig:subfilter:leszussp:kvsz}, where it is clear that activating the sooting mode at the stoichiometric mixture fraction results in a rate of oxidation that is only slightly less than the global maximum.

The plots of soot mass fraction from DNS in \cref{fig:subfilter:leszussp:ysvsz} indicate that there is zero soot in fuel-lean regions of the flame. However, the plots also reveal that there is still no soot at slightly rich values of mixture fraction. To capture this physics, the activation of the sooting mode needs to occur at the right boundary of this narrow fuel-rich region containing no soot. This location is determined by finding the smallest mixture fraction where soot growth begins to dominate over oxidation, which corresponds to the intersection between the profiles for surface growth and oxidation in \cref{fig:conclusion:future:zassp:shiftedz}. It is evident that rich-shifting the location of activation leads to a decrease in the rate of oxidation by more than an order of magnitude. With this improved model for soot evolution, preliminary LES have demonstrated large increases in the time-averaged, centerline soot volume fractions. Future work is required to determine if the LES volume fraction profiles will eventually match those from experiments.
\begin{figure}[htb]
  \centering
  \includegraphics[width=0.43\linewidth]{ch-conclusion/figures/flamelet_sootcoeffs_EHN_chi01_le1}
  \caption[Shifted Activation of ZASSP]{Timescales of oxidation, surface growth, and combined nucleation and condensation from nonpremixed flamelet solutions for a fuel mixture of \ce{C2H4}/\ce{H2}/\ce{N2} [40/41/19 by volume]. The vertical gray dashed line indicates the location of the stoichiometric mixture fraction ($Z_{st} = 0.0834$). The vertical black dashed line marks the location of the intersection between the profiles for oxidation and surface growth ($Z = 0.133$). All other lines are the same as in \cref{fig:subfilter:leszussp:kvsz}.}
  \label{fig:conclusion:future:zassp:shiftedz}
\end{figure}


\subsection{Strain-Sensitive Transport Approach}
\label{sec:conclusion:future:ssta}

The strain-sensitive transport approach was developed in \cref{sec:transport:ssta} to properly account for the transport of species with chemical production rates that are slow relative to the rate of local turbulent mixing. A key aspect of this model is the strain-sensitivity parameter, which compares the rate of local mixing to the formation chemistry rate for a certain species. In this thesis, the minimum value of this parameter was used to identify whether the species should be governed by molecular transport. However, in \cref{fig:transport:ssta:framework:sspc7h16}, it is clear that the parameter varies with respect to the mixture fraction. Thus, this transport approach can be improved by incorporating the dependence on the local mixture fraction into the strain-sensitivity parameter.

The modification described above implies that a species can be governed by both molecular diffusion and turbulent transport with the exact mode contingent on its location in mixture fraction space. The effective Lewis number defined in \cref{eq:transport:ssta:framework:lezetai} acquires a value of unity or the species' Lewis number depending on whether the minimum value of the strain-sensitivity parameter is less than or greater than unity, respectively. Its current formulation allows for transitioning between the two modes of transport, albeit in a very abrupt manner through the Heaviside function. For a species like the hydrogen radical, whose Lewis number is approximately 0.18, the sharp transition in the effective Lewis number to a value of unity is unphysical and can cause numerical issues. The formulation of the effective Lewis number can be improved by replacing the Heaviside function with one that possesses smooth gradients. A simple definition similar to that from Wang~\cite{wang2016} is given by
\begin{equation}\label{eq:conclusion:future:ssta:lezetai}
  \check{Le}_k(\zeta_k) = \frac{Le_k}{Le_k + \theta(\zeta_k(Z)) \cdot (1 - Le_k)},
\end{equation}
where
\begin{equation}\label{eq:conclusion:future:ssta:theta}
  \theta(\zeta_k(Z)) = \frac{\zeta_k(Z)}{\zeta_k(Z) + C}
\end{equation}
and $C$ is a constant. Determining the exact form of \cref{eq:conclusion:future:ssta:theta} and understanding the effects of these modifications on predictions of the soot volume fraction from LES are suggestions for future work.

Lastly, the strain-sensitive transport approach was developed on the foundation of the one-dimensional nonpremixed flamelet equations accounting for differential diffusion effects and the presence of soot~\cite{pitsch1998,mueller2012}. These equations were derived under the assumption that the nonpremixed flame is one-dimensional in physical space \textit{a priori}, which is applicable in the limit of a thin, one-dimensional flat flame. However, this simplification is not appropriate for curved or thick flames, whose curvature effects can influence transport in mixture fraction space for species that are modeled with differential diffusion. These flamelet equations accounting for differential diffusion have been re-derived by Xuan \etal~\cite{xuan2014} while assuming that the nonpremixed flame is one-dimensional in mixture fraction space \textit{a posteriori} in order to capture the aforementioned effects. Integration of the strain-sensitive transport approach into this new framework and analysis of the impact of curvature on PAH evolution are tasks for future studies.
 % Future work
