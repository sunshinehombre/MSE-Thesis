\chapter{Conclusion\label{ch:conclusion}}

This thesis has presented several advancements in modeling the evolution of soot and its precursors in turbulent nonpremixed combustion. These models were developed for Large Eddy Simulation (LES), where the geometry-dependent, large-scale phenomena are resolved and the small-scale features are modeled. LES provides improved predictions of turbulent mixing over simulations with Reynolds-Averaged Navier-Stokes (RANS), especially for large-scale phenomena such as swirl, separation, and recirculation that can impact soot dynamics. The evolution of soot is also heavily influenced by its physics and chemistry occurring in the unresolved scales, where the combustion community's understanding is far from complete. Thus, the principal focus of this thesis was to provide novel insight and develop LES models for the interactions between soot, chemistry, and turbulence as well as for the transport of species with relatively slow chemistry at these small scales. The main results from model development, validation, and application are summarized below.

The framework for modeling the evolution of soot in LES of turbulent nonpremixed combustion was outlined in \cref{ch:lesmodels}. The Method of Moments was chosen to model the Number Density Function (NDF) of soot particles due to its lower computational expense. This statistical approach uses a bivariate volume $V$ and surface area $S$ description of soot to account for their geometrical complexities and involves solving equations that track the evolution of moments of the NDF. Closure of these equations was achieved through the Hybrid Method of Moments (HMOM)~\cite{hmom2009}, which accounts for the bimodal nature of the NDF resulting from the presence of small, incipient particles and larger, more mature aggregates. HMOM provides models for the physical and chemical processes that govern soot evolution including nucleation, coagulation, condensation, surface growth, oxidation, and fragmentation.

This soot model was integrated into a turbulent combustion model that describes the thermal and chemical structure of the nonpremixed flame and accounts for the formation of soot. In the flamelet approach~\cite{peters1984}, a three-dimensional turbulent flame is conceptualized as an ensemble of locally one-dimensional flame structures embedded in a turbulent flow field. Since thermal radiation can occur on similar temporal scales as soot evolution, the Radiation Flamelet/Progress Variable (RFPV) approach~\cite{ihme2008} with adaptions from Carbonell \etal~\cite{carbonell2009} was selected as the foundation for the turbulent combustion model. In this approach, the local thermochemical state is described by solutions to the steady flamelet equations that are augmented with radiative losses. These solutions are computed \textit{a priori} and are stored in a database that is accessed during LES through a reduced set of parameters that includes the filtered mixture fraction and its subfilter variance, the filtered progress variable, and the filtered heat loss parameter. Following Mueller and Pitsch~\cite{mueller2012}, the transport equation definitions for these parameters were modified to account for the extraction of PAH from the gas-phase and provide a unique parameterization of the thermochemical state.

However, previous works~\cite{attili2014,bisetti2012} have found that the use of the steady flamelet approximation leads to inaccurate predictions for the mass fractions of gas-phase PAH due to their relatively slow chemistry. To address this point, the transport equation model for PAH from Mueller and Pitsch~\cite{mueller2012} was incorporated into the LES modeling framework, where the chemical source term of the transport equation accounts for the production, consumption, and dimerization of PAH.

\section{Major Contributions}
\label{sec:conclusion:contributions}

Outline major contributions


\subsection{Subfilter Modeling Advancements}
\label{sec:conclusion:contributions:subfilter}

Summarize subfilter modeling advancements.


\subsection{Transport for Strain-Sensitive Species}
\label{sec:conclusion:contributions:transport}

Summarize advancements for transport modeling.
 % Contributions
\section{Recommendations for Future Work}
\label{sec:conclusion:future}

Discuss future work.


\subsection{Z-Activated Soot Subfilter PDF}
\label{sec:conclusion:future:zassp}

Future work for ZASSP.


\subsection{Strain-Sensitive Transport Approach}
\label{sec:conclusion:future:ssta}

Future work for SSTA.

%% Future work should include options in the template for a masters thesis or an undergraduate senior thesis. It should also support running headings in the headers using the `headings' pagestyle.  The print mode and proquest mode included in the template might also be candidates to include in the class itself.
 % Future work
