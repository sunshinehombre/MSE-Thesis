\section{$Z$-Uniform Soot Subfilter PDF}
\label{sec:subfilter:zussp}

%% Subfilter-scale interactions between soot, turbulence, and combustion chemistry are captured with a density-weighted joint subfilter PDF of the thermochemical and soot variables. To simplify the form of the joint PDF, a timescale argument is adapted from Mueller and Pitsch~\cite{subfilterpdf2011}. It is proposed that since the formation chemistry of soot is governed by long temporal scales relative to that of the main heat-releasing chemistry, then the soot scalars should not have any dependence on the thermochemical variables. Thus, the functional relation $\mc{J}$ can be replaced with \cref{eq:lesmodels:combust:producteos}, and the joint subfilter PDF can be expressed as the product of marginal PDFs for the thermochemical and soot components. The spatially Favre filtered functions of the thermochemical and soot variables are then defined by \cref{eq:lesmodels:presumedpdf:indep}, where the presumed PDF approach has been utilized to provide the form of the thermochemical subfilter PDF (\cref{eq:lesmodels:presumedpdf:trimarg}). However, the soot subfilter PDF $P(\mc{M}_j)$ is still unknown. Closure with the presumed PDF approach is the focus of this section.

In DNS studies of a nitrogen-diluted, \textit{n}-heptane/air turbulent nonpremixed planar jet flame~\cite{bisetti2012,attili2014,attili2015}, it was noted that PAH are extremely sensitive to the local scalar dissipation rate due to their relatively slow chemistry. As a result, they are confined to thin, spatially intermittent regions of low scalar dissipation rate. Soot, whose source terms are non-linearly dependent on the amount of gas-phase PAH precursors, is also characterized by a high Schmidt number and is strongly affected by differential diffusion with respect to the mixture fraction. Therefore, soot exhibits an even higher degree of spatial intermittency than PAH. These features are evident in \cref{fig:subfilter:zussp:chisensitivity} from the DNS of Bisetti \etal~\cite{bisetti2012}.

\begin{figure}[htb]
  \centering
  \includegraphics[width=\linewidth]{ch-subfiltermodeling/figures/chisensitivity}
  \caption[Spatial Variability of \ce{C10H8} and Soot Fields]{\textit{Left to right} - Scalar fields of scalar dissipation rate $\chi$, naphthalene mass fraction $Y_{\ce{C10H8}}$, and soot volume fraction $f_V$ sampled at 10 ms in a 2D DNS of an \textit{n}-heptane/nitrogen [15.6/84.4 by volume] and air turbulent nonpremixed planar jet flame at 1 atm and 300 K~\cite{bisetti2012}. The Taylor scale Reynolds number is $Re_{\lambda} = 170$ initially and the stoichiometric mixture fraction is $Z_{st} = 0.143$. The solid black line in each image denotes the $Z = 0.3$ iso-contour.} %It is clear that high concentrations of naphthalene and soot persist only in regions where the scalar dissipation rate is largely diminished. As a result, both are spatially intermittent, although soot exhibits this to a higher degree. Since soot does not diffuse, its transport is mainly controlled by convection. Thus, soot exists in more confined patches than naphthalene, which are stretched into elongated filaments by turbulent eddies over time.}
  \label{fig:subfilter:zussp:chisensitivity}
\end{figure}

The soot subfilter PDF must be able to capture this spatial inhomogeneity caused by interactions between soot and turbulence. Mueller and Pitsch~\cite{subfilterpdf2011} proposed the form of a double delta distribution to account for sooting and non-sooting modes within an LES filter width:
\begin{equation}\label{eq:subfilter:zussp:dd}
  P(\mc{M}_j) = \omega\delta(\mc{M}_j) + (1 - \omega)\delta(\mc{M}_j - \mc{M}_j^*),
\end{equation}
where $\omega$ is the subfilter intermittency and $\mc{M}_j^*$ is formulated to recover the filtered values of the soot scalars $\mean{\mc{M}}_j$ upon integration against the PDF. For the remainder of this thesis, \cref{eq:subfilter:zussp:dd} shall be referred to as the $Z$-Uniform Soot Subfilter PDF (ZUSSP). The subfilter intermittency is interpreted as the probability of not finding soot within an LES filter width, so it is expected to be closer to unity than zero due to the high spatial intermittency of soot. It is acquired from one second-order moment of a soot scalar:
\begin{equation}\label{eq:subfilter:zussp:omega}
  \omega = 1 - \frac{{\mean{M}_{x,y}}^2}{\mean{{M_{x,y}}^2}},
\end{equation}
where the total number density $M_{0,0}$ is selected to provide the best model performance~\cite{subfilterpdf2011}. Note that when $\omega$ is zero, \cref{eq:subfilter:zussp:dd} is reduced to a single delta distribution
\begin{equation}\label{eq:subfilter:zussp:sd}
  P(\mc{M}_j) = \delta(\mc{M}_j - \mean{\mc{M}}_j),
\end{equation}
where the soot scalars adopt their average values $\mean{\mc{M}}_j$ within an LES filter width.

Evaluation of the subfilter intermittency requires values for the filtered moments and a filtered second-order moment. The former are obtained through a transport equation, which is derived by filtering \cref{eq:lesmodels:soot:ndf:momtransport} and incorporating the total velocity of \cref{eq:lesmodels:soot:ndf:totvel}:
\begin{equation}\label{eq:subfilter:zussp:momtransport}
  \pder[\mean{M}_{x,y}]{t} + \pder[\tf{u_j^*}\mean{M}_{x,y}]{x_j} = -\pder[]{x_j}\left[ \mean{\rho}\tf{u_j^* \left( \frac{M_{x,y}}{\rho}\right)} - \mean{\rho}\tf{u_j^*}\tf{ \left( \frac{M_{x,y}}{\rho} \right)} \right] + \fst[M]{x,y}.
\end{equation}
The filtered second moment of the total number density in \cref{eq:subfilter:zussp:omega} is also determined by a transport equation
\begin{equation}\label{eq:subfilter:zussp:secondmomtransport}
  \pder[\mean{{M_{0,0}}^2}]{t} + \pder[\tf{u_j^*}\mean{{M_{0,0}}^2}]{x_j} = -\pder[]{x_j}\left[ \mean{\rho}\tf{u_j^* \left( \frac{{M_{0,0}}^2}{\rho} \right)} - \mean{\rho}\tf{u_j^*}\tf{\left( \frac{{M_{0,0}}^2}{\rho} \right)} \right] - \mean{{M_{0,0}}^2 \pder[u_j^*]{x_j}} + 2\mean{M_{0,0}\dot{M}_{0,0}}.
\end{equation}
The scalar fluxes in \cref{eq:subfilter:zussp:momtransport} and \cref{eq:subfilter:zussp:secondmomtransport} can be modeled with a dynamic procedure as previously discussed, and the correlation between the number density and the number density source term in \cref{eq:subfilter:zussp:secondmomtransport} is closed with the presumed PDF approach. The source term of \cref{eq:subfilter:zussp:momtransport} is also closed with the presumed PDF approach. Its form, shown for oxidation, can be expressed as
\begin{equation}\label{eq:subfilter:zussp:ox}
  \begin{split}
    \fst[M]{x,y}^{ox} &= \iint k_{ox}(Z)\mc{M}_j P(\mc{M}_j)\pz dZ d\mc{M}_j \\
    &= \tf{k}_{ox}\mean{M}_j,
  \end{split}
\end{equation}
where $k_{ox}(Z)$ is the oxidation coefficient, $\pz$ is modeled with a beta distribution as in \cref{eq:lesmodels:presumedpdf:trimarg}, and the soot subfilter PDF $P(\mc{M}_j)$ is provided in \cref{eq:subfilter:zussp:dd}. Contributions from the thermochemical and soot variables are made distinct in the second line, where $\tf{k}_{ox} = \int k_{ox}(Z)\pz dZ$ and $\mean{M}_j = \int \mc{M}_j P(\mc{M}_j)d\mc{M}_j$. The remaining unclosed term in \cref{eq:subfilter:zussp:secondmomtransport} is the correlation between the second moment and the divergence of the total velocity. Following Mueller and Pitsch~\cite{subfilterpdf2011}, this term will be closed by neglecting the correlation between the two quantities:
\begin{equation}\label{eq:subfilter:zussp:nocorrelation}
  \mean{{M_{0,0}}^2 \pder[u_j^*]{x_j}} \approx \mean{{M_{0,0}}^2}\pder[\tf{u_j^*}]{x_j}.
\end{equation}
