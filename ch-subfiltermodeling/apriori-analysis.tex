\section{\textit{A Priori} Analysis}
\label{sec:subfilter:dns}

\subsection{DNS Database}
\label{sec:subfilter:dns:database}

The effectiveness of the proposed $Z$-activated soot subfilter PDF at reducing the oxidation rate is validated through an \textit{a priori} analysis using the database from a three-dimensional DNS of a temporally evolving, turbulent nonpremixed planar jet flame at atmospheric pressure~\cite{attili2014}. In this simulation, a central fuel slab consisting of an \textit{n}-heptane/nitrogen [15/85 by volume] mixture at 400 K is surrounded on either side by an air coflow at 800 K. The initial velocity field of the fuel jet is obtained from an instantaneous realization of a turbulent channel flow at $Re_{\tau} = 390$ with a centerline value of $U_c = 8.74$ m/s. The surrounding air flows in the opposite direction of the fuel jet with a streamwise velocity of the same magnitude to give a jet Reynolds number $Re = 2U_c H/\nu \approx 15,000$.

Oxidation of \textit{n}-heptane is modeled with a reduced mechanism comprising 47 species and 290 reactions that accounts for the formation of PAH up to naphthalene. The soot model, as described in \cref{ch:lesmodels,ch:subfilter}, is used in both the DNS and the \textit{a priori} analysis. In the DNS, the soot population is described with seven statistical moments whereas only three moments and the weight of the delta function are used in the \textit{a priori} investigation. The choice to use a reduced set of moments allows for validation of the proposed soot model in a configuration that will most likely be implemented in LES.

The DNS domain is discretized with $N_x \times N_y \times N_z = 1024 \times 1024 \times 512$ grid points, where the homogeneous region of the domain ($|y/H| \le 2.8$) has a grid spacing of $h = 91\ \mu$m. For a filter width $\Delta$, the \textit{a priori} study utilizes a subset of the domain of size $N_x \times N_y \times (\Delta/h + 1)$, where low-pass filtering is done only within the homogeneous mesh region. Additionally, analysis is performed on a snapshot of the DNS at $t = 5$ ms and at a stoichiometric scalar dissipation rate of $\chi_{st} = 20$ s$^{-1}$. Other key properties of the DNS are summarized in \cref{tab:subfilter:dns:params}, and complete details can be found in Attili et al.~\cite{attili2014}.

\begin{table}[htbp]
\centering
\caption[Parameters for DNS of Turbulent Nonpremixed \ce{C7H16}/\ce{N2} Jet Flame]{Parameters for DNS of a turbulent nonpremixed \ce{C7H16}/\ce{N2} jet flame}
\label{tab:subfilter:dns:params}
\begin{tabular}{p{0.46\textwidth} p{0.15\textwidth} p{0.2\textwidth}}
\toprule
Initial jet width, $H$
& [mm] & 15 \\[0.2em]

Domain size, $L_x \times L_y \times L_z$
& [mm] & $94 \times 105 \times 47$ \\[0.2em]

Time step, $\Delta t$
& [$\mu$s] & 4 \\[0.2em]

Minimum Kolmogorov scale, $\eta$
& [$\mu$m] & 110 \\[0.2em]

Kinematic viscosity of fuel mixture, $\nu$
& [m$^2$s$^{-1}$] & $1.7 \times 10^{-5}$ \\[0.2em]

Stoichiometric mixture fraction, $Z_{st}$
& [--] & 0.147 \\

\bottomrule
\end{tabular}
\end{table}

In order to validate the proposed LES model against DNS, a low-pass filter must be applied to the chosen snapshot of the DNS database. A three-dimensional, clipped and renormalized Gaussian filter kernel is employed and is given by
\begin{equation}\label{eq:subfilter:dns:kernel}
  F(x,y,z) = \kappa^3\exp\left[ \frac{-6(x^2 + y^2 + z^2)}{\Delta^2} \right],
\end{equation}
where $\Delta$ is the filter width and $\kappa$ is a renormalization constant that ensures the following relation holds for a grid spacing $h$:
\begin{equation}\label{eq:subfilter:dns:unity}
  \sum\limits_{x = -\Delta/2h}^{\Delta/2h} \sum\limits_{y = -\Delta/2h}^{\Delta/2h} \sum\limits_{z = -\Delta/2h}^{\Delta/2h} F(x,y,z) = 1.
\end{equation}
Note that the above relations require a homogeneous mesh such that $\kappa = \kappa_x = \kappa_y = \kappa_z$, $\Delta = \Delta_x = \Delta_y = \Delta_z$, and $h = h_x = h_y = h_z$. As evident in \cref{eq:subfilter:dns:unity}, the filter kernel is active over a cube of $(\Delta/h + 1)^3$ grid points.


\subsection{Oxidation Source Term}
\label{sec:subfilter:dns:ox}

Analysis of oxidation source term.


\subsection{Surface Growth Source Term}
\label{sec:subfilter:dns:sg}

Analysis of surface growth source term.


\subsection{Effects of Filter Width}
\label{sec:subfilter:dns:fw}

Investigation of filter width variation effects.


