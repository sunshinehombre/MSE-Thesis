\section{\texorpdfstring{$Z$}{Z}-Activated Soot Subfilter PDF}
\label{sec:subfilter:zassp}

The timescale separation assertion in \cref{sec:lesmodels:presumedpdf} is used to eliminate the conditional PDF's dependence on the thermochemical variables. Only the timescales of PAH and soot formation are considered in this argument, yet soot, turbulence, and chemistry interact even after the nucleation of soot particles from PAH dimers. As mentioned previously, interactions between soot and gas-phase chemistry may be intensified during the oxidation of soot. Also, surface growth by the \ce{H}-Abstraction, \ce{C2H2}-Addition (HACA) mechanism~\cite{frenklach1985,frenklach1991} can be dominant in regions near the flame. Therefore, the subfilter soot-turbulence interactions and combustion chemistry interactions cannot be mutually exclusive; \cref{eq:subfilter:zussp:dd} must contain a connection to the thermochemical variables. More specifically, the soot subfilter PDF needs to account for the local ratios of fuel and oxidizer as described by the mixture fraction $Z$ since oxidation and surface growth should be restricted to fuel-rich locations where soot exists. A novel soot subfilter PDF is proposed
\begin{equation}\label{eq:subfilter:zassp:dd}
  P(\mc{M}_j|Z) = \omega\delta(\mc{M}_j) + (1 - \omega)\delta(\mc{M}_j - \mc{M}_j^*(Z)),
\end{equation}
where the sooting mode is activated only for rich values of mixture fraction through
\begin{equation}\label{eq:subfilter:zassp:mstar}
  \mc{M}_j^*(Z) = \mc{M}_j^{**}H(Z - Z_{st}).
\end{equation}
$H(\cdot)$ is the Heaviside function and $\mc{M}_j^{**}$ is selected so that $\mc{M}_j^*(Z)$ recovers the filtered values of the soot scalars upon integration of $\tf{P}(\xi,\mc{M}_j)$. It is evident that \cref{eq:subfilter:zassp:dd,eq:subfilter:zassp:mstar} still presume a uniform subfilter distribution of soot at rich values of mixture fraction. In the right-hand plot of \cref{fig:subfilter:leszussp:ysvsz}, the DNS of Attili et al.~\cite{attili2014} shows that such a model is more appropriate at later times as steady-state behavior is approached. For the earlier transitional stages, the delta distribution representing the sooting mode of \cref{eq:subfilter:zassp:dd} may be replaced with one that captures the nonuniformity of soot at rich mixture fractions~\cite{berger2017}. Note that the increased accuracy associated with such a modification is expected to be accompanied with additional computational expense, as more sophisticated distributions require a larger set of parameters.

The subfilter intermittency must also incorporate the dependence on mixture fraction so that
\begin{equation}\label{eq:subfilter:zassp:omega}
  \omega = 1 - \frac{1}{\int H(Z - Z_{st})\pz dZ} \cdot \frac{{\mean{M}_{x,y}}^2}{\mean{{M_{x,y}}^2}},
\end{equation}
where $\pz$ is modeled with a beta distribution as in \cref{eq:lesmodels:presumedpdf:trimarg} and the total number density $M_{0,0}$ is selected to provide the best model performance~\cite{subfilterpdf2011}. Note that while \cref{eq:subfilter:zussp:omega} is bounded between zero and unity, \cref{eq:subfilter:zassp:omega} is unbounded because $\int H(Z - Z_{st})\pz dZ$ could approach a value of zero. The resulting numerical challenges can be mitigated by introducing an arbitrarily small, nonzero threshold $\Upsilon$. When the value of the integral is less than this threshold, the subfilter distribution of mixture fraction is right-skewed, as nearly the entire mass of the distibution is concentrated at mixture fractions below stoichiometric. Soot cannot exist at such lean conditions, so $\omega$ is assigned a value of unity to indicate that no soot can be found within the LES filter width. When $\int H(Z - Z_{st})\pz dZ$ approaches unity, the subfilter distribution of mixture fraction is heavily left-skewed. These fuel-rich conditions can support the presence of soot, and \cref{eq:subfilter:zassp:omega} reduces to \cref{eq:subfilter:zussp:omega} in the limit that the integral evaluates to unity. For $\Upsilon < \int H(Z - Z_{st})\pz dZ < 1$, the subfilter intermittency $\omega$ can adopt negative values. This is nonsensical, as the subfilter intermittency should be interpreted as the probability of not finding soot within an LES filter width. As a consequence, $\omega$ is clipped to a minimum value of zero.

Lastly, \cref{eq:subfilter:zassp:dd} is utilized to derive new expressions for the filtered source terms of the LES transport equation for the moments (\cref{eq:subfilter:zussp:momtransport}). For example, the oxidation source term becomes
\begin{equation}\label{eq:subfilter:zassp:ox}
  \begin{split}
    \fst[M]{x,y}^{ox} &= \iint k_{ox}(Z)\mc{M}_j P(\mc{M}_j|Z)\pz dZ d\mc{M}_j \\
    &= \check{k}_{ox}\mean{M}_j,
  \end{split}
\end{equation}
where the integrated form of the oxidation coefficient now depends on the stoichiometric mixture fraction
\begin{equation}\label{eq:subfilter:zassp:kox}
  \check{k}_{ox} = \frac{\int k_{ox}(Z)H(Z - Z_{st})\pz dZ}{\int H(Z - Z_{st})\pz dZ}.
\end{equation}
It is discernable that this expression faces the same numerical issues as the subfilter intermittency of \cref{eq:subfilter:zassp:omega}. Therefore, the filtered moment source terms are practically implemented by expressing them in terms of $\omega$ and $\mc{M}_j^{**}$, which have been formulated to avoid those complications:
\begin{equation}\label{eq:subfilter:zassp:practicalox}
  \fst[M]{x,y}^{ox} = \breve{k}_{ox}f(\omega,\mc{M}_j^{**}),
\end{equation}
where $\breve{k}_{ox} = \int k_{ox}(Z)H(Z - Z_{st})\pz dZ$.
