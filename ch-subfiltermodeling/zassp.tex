\section{\texorpdfstring{$Z$}{Z}-Activated Soot Subfilter PDF}
\label{sec:subfilter:zassp}

%% I want to say that the ZUSSP model assumes a uniform distribution of soot in mixture fraction space. However, this is inconsistent with DNS (show Ys vs Z plot), where no soot exists at lean mixture fractions. Possible ramifications include the overprediction of the oxidation source term (write out the filtered moment source term for oxidation) - show inverse timescales of nucleation + condensation, surface growth, and oxidation to indicate that oxidation time scale is extremely fast. Therefore, the current model likely significantly overpredicts the rate of oxidation with a lesser impact on the surface growth rate. Consequently, I need a subfilter PDF that accounts for the dearth of soot at lean. Introduce ZASSP model.

As a result of the decorrelation of the soot scalars from the thermochemical variables, the soot subfilter PDF of \cref{sec:subfilter:zussp} implictly assumes that soot is uniformly distributed in mixture fraction space. However, for non-smoking flames, this is qualitatively incorrect as there should be zero soot in fuel-lean regions of the flame. This is evident in \cref{fig:subfilter:leszussp:ysvsz}, reproduced from the 3D DNS of a nitrogen-diluted, \textit{n}-heptane/air turbulent nonpremixed planar jet flame~\cite{attili2014}. Although soot is formed in the region $0.3 < Z < 0.5$, it is rapidly oxidized before reaching lean mixtures.

\begin{figure}[htb]
  \centering
  \includegraphics[width=0.7\linewidth]{ch-subfiltermodeling/figures/dns_Ysoot_vs_Z}
  \caption[DNS of Turbulent Nonpremixed \ce{C7H16}/\ce{N2} Jet Flame, \texorpdfstring{$\langle Y_{\text{s}}|Z \rangle$}{<Ys|Z>} vs. \texorpdfstring{$Z$}{Z}]{Mean soot mass fraction conditioned on mixture fraction at various times in a 3D DNS of an \textit{n}-heptane/nitrogen [15/85 by volume] and air turbulent nonpremixed planar jet flame, reproduced from Attili \etal~\cite{attili2014}. The stoichiometric mixture fraction ($Z_{st} = 0.147$) is demarcated with the vertical dashed line. \textit{Left} - 1 ms (filled squares), 2 ms (crosses), 3 ms (open squares), 4 ms (circles), and 5 ms (triangles). \textit{Right} - 6 ms (stars), 8 ms (circles), 10 ms (open squares), and 20 ms (filled squares). The small gray dots represent the soot mass fraction field at 20 ms.}
  \label{fig:subfilter:leszussp:ysvsz}
\end{figure}

The timescales of oxidation, surface growth, and combined nucleation and condensation, obtained from a nonpremixed flamelet solution and shown in \cref{fig:subfilter:leszussp:kvsz}, support the aforementioned point and further emphasize the effect of the local mixture fraction on the soot evolution pathways. It is clear that different soot evolution modes are dominant over distinct regions of mixture fraction even as the fuel mixture and stoichiometric scalar dissipation rate are varied. Soot growth through PAH-based nucleation and condensation prevails at very rich values of mixture fraction, whereas acetylene-based surface growth is more dominant at moderately rich mixture fractions. Soot oxidation is the dominant mode at mixture fractions below and slightly above the stoichiometric value. Since the rates associated with the high-temperature oxidation chemistry are comparable with those of the main heat-releasing chemistry~\cite{guo2016}, it is anticipated that the magnitude of the peak oxidation rate will not vary much as the fuel mixture or stoichiometric scalar dissipation rate is modified. Indeed, this phenomenon is evident in \cref{fig:subfilter:leszussp:kvsz}.

\begin{figure}[htb]
  \centering
  \includegraphics[width=\linewidth]{ch-subfiltermodeling/figures/flamelet_sootcoeffs_le1}
  \caption[Soot Growth and Oxidation Timescales, 1/\texorpdfstring{$\tau$}{t} vs. \texorpdfstring{$Z$}{Z}]{Timescales of oxidation, surface growth, and combined nucleation and condensation from nonpremixed flamelet solutions. The blue lines are the oxidation coefficients, the red lines represent the surface growth coefficients, and the green lines are the coefficients for nucleation and condensation. \textit{Left} - Fuel mixture \textit{n}-heptane/nitrogen [15/85 by volume] used in the DNS of \cref{fig:subfilter:zussp:chisensitivity,fig:subfilter:leszussp:ysvsz} at $\chi_{st} = 10\ s^{-1}$. \textit{Middle} \& \textit{Right} - Fuel mixture \ce{C2H4}/\ce{H2}/\ce{N2} [40/41/19 by volume] used in the LES of \cref{fig:subfilter:leszussp:zusspleseval} at $\chi_{st} = 10\ s^{-1}$ and $\chi_{st} = 0.1\ s^{-1}$, respectively.}
  \label{fig:subfilter:leszussp:kvsz}
\end{figure}

Additionally, the oxidation timescale at slightly lean conditions is very short, at least an order of magnitude smaller than the timescale for surface growth. This is concerning, for \cref{fig:subfilter:leszussp:ysvsz} shows that there is no soot to oxidize in less than stoichiometric conditions. However, the soot subfilter PDF given by \cref{eq:subfilter:zussp:dd} assumes that soot is uniformly distributed in mixture fraction space. Therefore, the current model is likely to significantly overpredict the rate of oxidation, with a lesser impact on the rate of surface growth. To prevent this non-physical oxidation from occurring, the subfilter PDF must exclude the presence of soot at lean mixtures $Z < Z_{st}$. The development of a soot subfilter PDF that addresses this point is the focus of this section.

%% Thus, the complete oxidation of soot by \ce{OH} (and \ce{O2} to a lesser degree) is expected during transport towards stoichiometric regions. However, the soot subfilter PDF given by \cref{eq:subfilter:zussp:dd} permits the existence of soot in fuel-lean areas, for it assumes that soot is uniformly distributed in mixture fraction space. Therefore, the presence of large, non-zero oxidation coefficients at lean values of mixture fraction is concerning, for it potentially leads to substantial filtered moment source terms (\cref{eq:subfilter:leszussp:ox}). This contribution to the oxidation rate is artificial and could be an explanation for the underpredicted soot volume fraction in \cref{fig:subfilter:leszussp:zusspleseval}. To prevent this non-physical oxidation from occurring, the subfilter PDF must depend on the thermochemical variables to exclude the presence of soot at lean mixtures $Z < Z_{st}$. A soot subfilter PDF that addresses this point is introduced in the following section.

The timescale separation assertion in \cref{sec:subfilter:joint} is used to eliminate the conditional PDF's dependence on the thermochemical variables. Only the timescales of PAH and soot formation are considered in this argument, yet soot, turbulence, and chemistry interact even after the nucleation of soot particles from PAH dimers. As mentioned previously, interactions between soot and gas-phase chemistry may be intensified during the oxidation of soot. Also, surface growth through the \ce{H}-Abstraction, \ce{C2H2}-Addition (HACA) mechanism~\cite{frenklach1985,frenklach1991} can be dominant in regions near the flame. Therefore, the subfilter soot-turbulence interactions and combustion chemistry interactions cannot be independent; \cref{eq:subfilter:zussp:dd} must contain a connection to the thermochemical variables. More specifically, the soot subfilter PDF needs to account for the local ratios of fuel and oxidizer as described by the mixture fraction $Z$, since oxidation and surface growth should be restricted to fuel-rich locations where soot exists. A soot subfilter PDF is proposed
\begin{equation}\label{eq:subfilter:zassp:dd}
  P(\mc{M}_j|Z) = \omega\delta(\mc{M}_j) + (1 - \omega)\delta(\mc{M}_j - \mc{M}_j^*(Z)),
\end{equation}
where the sooting mode is activated only for rich values of mixture fraction through
\begin{equation}\label{eq:subfilter:zassp:mstar}
  \mc{M}_j^*(Z) = \mc{M}_j^{**}H(Z - Z_{st}).
\end{equation}
$H(\cdot)$ is the Heaviside function and $\mc{M}_j^{**}$ is selected so that $\mc{M}_j^*(Z)$ recovers the filtered values of the soot scalars upon integration of $\tf{P}(\xi,\mc{M}_j)$. It is evident that the sooting mode of \cref{eq:subfilter:zassp:dd} still presumes a uniform subfilter distribution of soot with respect to mixture fraction on the rich sie of the flame. In the right-hand plot of \cref{fig:subfilter:leszussp:ysvsz}, the DNS of Attili \etal~\cite{attili2014} shows that such a model is more appropriate at later times when turbulent mixing has distributed soot throughout mixture fraction space. For the earlier stages of soot evolution, the sooting mode should be concentrated near the location of peak soot inception in mixture fraction space, and such a model should be the focus of future work. Note that the increased accuracy associated with such a modification is expected to be accompanied with additional computational expense, as more sophisticated distributions require a larger set of parameters.

%% the delta distribution representing the sooting mode of \cref{eq:subfilter:zassp:dd} may be replaced with one that captures the nonuniformity of soot at rich mixture fractions~\cite{berger2017}. Note that the increased accuracy associated with such a modification is expected to be accompanied with additional computational expense, as more sophisticated distributions require a larger set of parameters.

The subfilter intermittency must also incorporate the dependence on mixture fraction:
\begin{equation}\label{eq:subfilter:zassp:omega}
  \omega = 1 - \frac{1}{\int H(Z - Z_{st})\pz dZ} \cdot \frac{{\mean{M}_{x,y}}^2}{\mean{{M_{x,y}}^2}},
\end{equation}
where $\pz$ is modeled with a beta distribution as in \cref{eq:lesmodels:presumedpdf:trimarg} and, following previous work~\cite{subfilterpdf2011}, the total number density $M_{0,0}$ is selected for the filtered second moment. Note that while \cref{eq:subfilter:zussp:omega} is bounded between zero and unity, \cref{eq:subfilter:zassp:omega} is unbounded because $\int H(Z - Z_{st})\pz dZ$ could approach a value of zero. The resulting numerical challenges are mitigated by introducing an arbitrarily small, nonzero threshold $\Upsilon$. When the value of the integral is less than this threshold, the entire mass of the subfilter distribution of mixture fraction is concentrated at values less than stoichiometric. Soot cannot exist at such lean conditions, so $\omega$ is assigned a value of unity to indicate that no soot can be found within the LES filter width. When $\int H(Z - Z_{st})\pz dZ$ approaches unity, the mass of the subfilter distribution of mixture fraction is heavily concentrated at rich mixture fractions. These fuel-rich conditions can support the presence of soot, and \cref{eq:subfilter:zassp:omega} reduces to \cref{eq:subfilter:zussp:omega} in the limit that the integral evaluates to unity.

%For $\Upsilon < \int H(Z - Z_{st})\pz dZ < 1$, the subfilter intermittency $\omega$ can adopt negative values. This is nonsensical, as the subfilter intermittency should be interpreted as the probability of not finding soot within an LES filter width. As a consequence, $\omega$ is clipped to a minimum value of zero.

Lastly, \cref{eq:subfilter:zassp:dd} is utilized to derive new expressions for the filtered source terms of the LES transport equation for the moments (\cref{eq:subfilter:zussp:momtransport}). For example, the oxidation source term becomes
\begin{equation}\label{eq:subfilter:zassp:ox}
  \begin{split}
    \fst[M]{x,y}^{ox} &= \iint k_{ox}(Z)\mc{M}_j P(\mc{M}_j|Z)\pz dZ d\mc{M}_j \\
    &= \check{k}_{ox}\mean{M}_j,
  \end{split}
\end{equation}
where the integrated form of the oxidation coefficient now depends on the stoichiometric mixture fraction
\begin{equation}\label{eq:subfilter:zassp:kox}
  \check{k}_{ox} = \frac{\int k_{ox}(Z)H(Z - Z_{st})\pz dZ}{\int H(Z - Z_{st})\pz dZ}.
\end{equation}
Implementation of the $Z$-activated soot subfilter PDF in LES is very similar to the process for the $Z$-uniform soot subfilter PDF. During the generation of the database for the thermochemical states, each flamelet solution is convoluted with the beta distibution for the mixture fraction. However, the proposed model simply requires incorporation of the Heaviside function during the convolution process for quantities such as the oxidation coefficient (\cref{eq:subfilter:zassp:kox}). 

%% It is discernable that this expression faces the same numerical issues as the subfilter intermittency of \cref{eq:subfilter:zassp:omega}. Therefore, the filtered moment source terms are practically implemented by expressing them in terms of $\omega$ and $\mc{M}_j^{**}$, which have been formulated to avoid those complications:
%% \begin{equation}\label{eq:subfilter:zassp:practicalox}
%%   \fst[M]{x,y}^{ox} = \breve{k}_{ox}f(\omega,\mc{M}_j^{**}),
%% \end{equation}
%% where $\breve{k}_{ox} = \int k_{ox}(Z)H(Z - Z_{st})\pz dZ$.
