\section{Modeling the Particle Size Distribution}
\label{sec:subfilter:ndf}

The distribution of soot particle sizes in a system can be described by the Number Density Function (NDF). Previous studies have suggested that the soot NDF is bimodal, accounting for newly formed 'primary' particles from nucleation and larger aggregrates from various growth modes~\cite{zhao2003,zhao2005,netzell2007}. Incipient particles are roughly spherical with diameters on the order of tens of nanometers, while aggregates comprised of monodisperse primaries have fractal geometries with length scales on the order of hundreds of nanometers~\cite{vanderwal1999}. In order to account for the geometrical complexities, the NDF requires a multivariate parameterization such as volume and surface area~\cite{FILL IN}, volume and the number of primary particles~\cite{FILL IN}, or even volume, surface area, and the number of surface hydrogenated carbon sites~\cite{FILL IN}. In this work, the volume and surface area are used to provide a bivariate description of the NDF. 

The NDF evolves according to the Population Balance Equation (PBE)~\cite{friedlander2000}
\begin{equation}\label{eq:ndf:pbe}
  \tder{N_i} - \pder[]{x_j}\left( 0.55\frac{\nu}{T}\pder[T]{x_j}N_i \right) = \dot{N_i}
\end{equation}
where the third term is the thermophoresis of particles~\cite{waldmann1966} and the source term on the right hand side incorporates the various physical and chemical processes that govern soot evolution. Note that molecular diffusion has been neglected, as a soot particle with a diameter of 10 nm has a Schmidt number of 287 at standard atmospheric conditions~\cite{friedlander2000}. The corresponding molecular diffusion coefficient is $D = 5.24\times 10^{-4}$ cm$^2$/sec, so the diffusion velocity of soot is minimal as a result.
