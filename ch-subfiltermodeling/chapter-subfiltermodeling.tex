\chapter{Soot-Chemistry-Turbulence Interactions\label{ch:subfilter}}

LES is a technique where large-scale, unsteady turbulent motions are computed explicitly while small-scale motions are modeled. The separation of scales is achieved through filtering the velocity and scalar fields (generically represented by $Q(x_j,t)$) to decompose them into the sum of a resolved component $\mean{Q}(x_j,t)$ and a subfilter-scale component $Q(x_j,t)-\mean{Q}(x_j,t)$. A general filtering operation can be expressed as
\begin{equation}\label{eq:subfilter:filter}
  \mean{Q}(x_j,t) = \int F(r_j,x_j)Q(x_j - r_j, t)dr_j,
\end{equation}
where integration is over the entire domain. The convolution kernel $F(r_j,x_j)$ satisfies the normalization condition
\begin{equation}\label{eq:subfilter:kernel}
  \int F(r_j,x_j)dr_j = 1,
\end{equation}
and uses cutoff length and time scales to separate smaller quantities from $\mean{Q}(x_j,t)$. In combustion, large temperature fluctuations motivate the tracking of quantities correlated with density. Therefore, it is convenient to use density-weighted filtering (Favre filtering) in LES of turbulent combustion, where the resolved components are expressed as $\tf{Q}(x_j,t) = \mean{\rho(x_j,t)Q(x_j,t)}/\mean{\rho(x_j,t)}$ and the subfilter-scale components $q''(x_j,t)$ satisfy $\mean{\rho(x_j,t) q''(x_j,t)} = 0$.

When these filtering operations are applied to the conservation equations for mass, momentum, and energy or scalar transport equations, filtered source terms and correlations between ``fluctuating'' quantities arise. These unclosed terms represent interactions between subfilter-scale quantities and must be modeled. In this chapter, two models for interactions between soot, turbulence, and combustion chemistry at subfilter scales are presented. These models are validated in an \textit{a priori} analysis of filtered moment source terms for oxidation and surface growth using the DNS database of Attili \etal~\cite{attili2014}.

%% Summary: Whole point of the chapter is to show that including a mixture fraction dependence in the soot subfilter model may lead to better predictions of $f_v$.

% include other files for sections of this chapter. These use the 'input' command since each section within a chapter should not start a new page.
% If you want to swap the order of sections, it is as simple as reversing the order you include them. 
\section{Z-Uniform Soot Subfilter PDF}
\label{sec:subfilter:zussp}

Description of ZUSSP.

\section{Evaluation in LES}
\label{sec:subfilter:leszussp}

LES of a series of turbulent, nonpremixed, ethylene-hydrogen-nitrogen [40/41/19 by volume] jet flames~\cite{mahmoud2017} is used to validate the $Z$-uniform soot subfilter PDF (\cref{eq:subfilter:zussp:dd}) in the framework presented in \cref{ch:lesmodels}. This is achieved by comparing thermocouple readings of flame temperature and laser-induced incandescence (LII) measurements of soot volume fraction with the profiles from LES. In this series, three flames are maintained at a jet exit Reynolds number of 15,000 while the exit strain rate is varied by altering the jet diameter and fuel exit velocity. A uniform coflow of air at a velocity of 1.1 m/s surrounds these flames. \cref{tab:subfilter:leszussp:ehn} summarizes other key aspects of the experimental setup, and complete details can be obtained from Mahmoud et al.~\cite{mahmoud2017}.

\begin{table}[htbp]
\centering
\caption[Flow Conditions of Turbulent Nonpremixed \ce{C2H4}/\ce{H2}/\ce{N2} Jet Flames]{Flow conditions of turbulent nonpremixed \ce{C2H4}/\ce{H2}/\ce{N2} jet flames.}
\label{tab:subfilter:leszussp:ehn}
\begin{tabular}{p{0.35\textwidth} p{0.1\textwidth} p{0.12\textwidth} p{0.12\textwidth} p{0.12\textwidth}}
\toprule
\textbf{Flame} & & \bm{$1/\tau|_{H}$} & \bm{$1/\tau|_{M}$} & \bm{$1/\tau|_{L}$} \\
\midrule

Central jet diameter, $D$
& [mm] & 4.4 & 5.8 & 8.0 \\[0.2em]

Mean jet exit velocity, $U$
& [m/s] & 56.8 & 42.4 & 31.5 \\[0.2em]

Exit strain rate, $U/D$
& [s$^{-1}$] & 12,900 & 7300 & 4300 \\[0.2em]

Exit Reynolds number, $Re_D$
& [--] & 15,000 & 15,000 & 15,000 \\[0.2em]

Mean flame length, $L_f$
& [mm] & 710 & 930 & 1060 \\

\bottomrule
\end{tabular}
\end{table}

A brief description of the numerical approach is provided here and will be addressed fully in \cref{ch:lesresults}. Grid filtered LES is achieved through NGA, a low-Mach number flow solver~\cite{desjardins2008}. The computational domain of each flame consists of a structured grid with $192 \times 96 \times 32$ points in the streamwise, radial, and circumferential directions, respectively. This corresponds to a domain with lengths $200D$ in the streamwise direction and $60D$ in the radial direction, where the grid is stretched for both. The flamelet solutions in the database are computed with FlameMaster, a solver for 0D and 1D flame calculations~\cite{flamemaster}. These solutions utilize the chemical mechanism of Blanquart et al.~\cite{blanquart2009588}, which encompasses the high temperature combustion of fuels from methane to iso-octane and places emphasis on the formation of soot precursors up to cyclopenta[cd]pyrene (\ce{C18H10}). The mechanism of Narayanaswamy et al.~\cite{narayanaswamy2010} is incorporated into the base mechanism to model the high-temperature oxidation of aromatic species.

In \cref{fig:subfilter:leszussp:zusspleseval}, it is clear that time-averaged, centerline flame temperature is generally well-predicted by LES in all flames. In the recorded experiments, the flame temperature rises to a peak value around $x/D = 100$, at which point it declines due to the presence of soot. The flame with the highest exit strain rate ($1/\tau|_H$, left plot) possesses a small region of uniform temperature around $x/D = 0$ due to the lifting of the flame in LES. This feature is not present in the experiment, and therefore explains the slight deviation from the empirical values until about $x/D = 50$. In all three flames, the peak mean temperature from LES closely matches the magnitude of the corresponding experimental value (1800 to 1900 K), but is shifted slightly downstream.

\begin{figure}[htb]
  \begin{center}
    \includegraphics[width=\linewidth]{ch-subfiltermodeling/figures/combined-Tfv-S-ZUSSP-Le_1}
    \caption[LES Validation of \texorpdfstring{$Z$}{Z}-Uniform Soot Subfilter PDF]{\textit{Left to right} - LES of jet flames $1/\tau|_H$, $1/\tau|_M$, and $1/\tau|_L$, respectively. Circles indicate experimental data while solid lines are time-averaged values from LES. Profiles in black correspond to centerline flame temperatures while profiles in blue represent centerline soot volume fractions. Note that the soot volume fraction profiles from LES have been premultiplied by a factor of 10 for visibility.}
    \label{fig:subfilter:leszussp:zusspleseval}
  \end{center}
\end{figure}

Most obvious in \cref{fig:subfilter:leszussp:zusspleseval} is the underprediction of the time-averaged, centerline soot volume fraction by LES. In all three flames, the volume fraction is underestimated by more than two orders of magnitude relative to the empirical values. Furthermore, the mean soot volume fractions from LES peak slightly earlier and approach zero sooner at approximately $x/D = 130$. These phenomena signal the presence of intense oxidation further upstream compared to the experiments. Such a disparity between the predicted and measured profiles raises the question of what could be modeled inaccurately. To determine the source of this inconsistency, the assumptions and deficiencies of the models need to be revisited.

The soot volume fraction is a measure of the total volume of soot relative to the total volume of gas. The amount of soot that is present depends on the rates of nucleation, condensation, surface growth, and oxidation. An underprediction of the volume fraction suggests that soot growth through nucleation, condensation, and surface growth is insufficient, or that soot oxidation is overwhelmingly excessive. Expressions for the filtered moment source terms of these modes are similar in form to \cref{eq:lesmodels:presumedpdf:separated}. For example, the filtered moment source term for oxidation is given by
\begin{equation}\label{eq:subfilter:leszussp:ox}
  \begin{split}
    \mean{\dot{M}}_{x,y}^{ox} &= \iint k_{ox}(Z)\mc{M}_j P(\mc{M}_j)\pz dZ d\mc{M}_j \\
    &= \hat{k}_{ox}\mean{M}_j,
  \end{split}
\end{equation}
where $k_{ox}(Z)$ is the oxidation coefficient, $\pz$ is modeled with a beta distribution as in \cref{eq:lesmodels:presumedpdf:trimarg}, and the soot subfilter PDF $P(\mc{M}_j)$ is provided in \cref{eq:subfilter:zussp:dd}. Contributions from the thermochemical and soot variables are made distinct in the second line, where $\hat{k}_{ox} = \int k_{ox}(Z)\pz dZ$ and $\mean{M}_j = \int \mc{M}_j P(\mc{M}_j)d\mc{M}_j$. \cref{eq:subfilter:leszussp:ox} incorporates the joint PDF simplification of Mueller and Pitsch~\cite{subfilterpdf2011}, which postulates that the soot scalars should be independent of the thermochemical variables due to the long timescales of PAH and soot formation relative to the timescales of the highly exothermic combustion chemistry. However, they also noted that such an assumption could be violated when surface growth near the flame is the dominant growth mechanism or during the oxidation of soot, when interactions with gas-phase chemistry are enhanced. The likelihood of the former is smaller, as the DNS studies of Bisetti et al.~\cite{bisetti2012} demonstrated that soot growth through PAH-based nucleation and condensation dominates acetylene-based surface growth in turbulent nonpremixed combustion. Thus, this work hypothesizes that the excessive oxidation of soot, due to the inappropriate simplification of the joint subfilter PDF \cref{eq:lesmodels:presumedpdf:bayes}, is the reason for the dramatic underprediction of the soot volume fraction as shown in \cref{fig:subfilter:leszussp:zusspleseval}.

As a result of the decorrelation of the soot scalars from the thermochemical variables, the current soot subfilter PDF implictly assumes that soot is uniformly distributed in mixture fraction space. However, for non-smoking flames, this is qualitatively incorrect as there should be zero soot in fuel-lean regions of the flame. This is evident in \cref{fig:subfilter:leszussp:ysvsz}, reproduced from the 3D DNS of a nitrogen-diluted, \textit{n}-heptane/air turbulent nonpremixed planar jet flame~\cite{attili2014}. The lack of soot at mixture fractions below stoichiometric can be explained by the preferential transport of soot generated in the region $0.3 < Z < 0.5$ to richer regions by turbulent fluctuations and by the oxidation of all soot during transport towards the stoichiometric surface. 

\begin{figure}[htb]
  \begin{center}
    \includegraphics[width=0.7\linewidth]{ch-subfiltermodeling/figures/dns_Ysoot_vs_Z}
    \caption[DNS of Turbulent Nonpremixed \ce{C7H16}/\ce{N2} Jet Flame, \texorpdfstring{$\langle Y_{\text{s}}|Z \rangle$}{<Ys|Z>} vs. \texorpdfstring{$Z$}{Z}]{Mean soot mass fraction conditioned on mixture fraction at various times in a 3D DNS of an \textit{n}-heptane/nitrogen [15/85 by volume] and air turbulent nonpremixed planar jet flame, reproduced from Attili et al.~\cite{attili2014}. The stoichiometric mixture fraction ($Z_{st} = 0.147$) is demarcated with the vertical dashed line. \textit{Left} - 1 ms (filled squares), 2 ms (crosses), 3 ms (open squares), 4 ms (circles), and 5 ms (triangles). \textit{Right} - 6 ms (stars), 8 ms (circles), 10 ms (open squares), and 20 ms (filled squares). The small gray dots represent the soot mass fraction field at 20 ms.}
    \label{fig:subfilter:leszussp:ysvsz}
  \end{center}
\end{figure}

The non-uniform nature of soot in mixture fraction space is captured by the flamelet solutions that are accessed during LES as well. Filtered moment source term coefficients for oxidation, surface growth, and the combination of nucleation and condensation are plotted in \cref{fig:subfilter:leszussp:kvsz}. It is clear that different soot evolution modes are dominant over distinct regions of mixture fraction even as the fuel mixture and stoichiometric scalar dissipation rate are varied. Soot growth through PAH-based nucleation and condensation prevails at very rich values of mixture fraction, whereas acetylene-based surface growth is more dominant at moderately rich mixture fractions. It is noteable that for a fixed fuel mixture (middle and right plots), a reduction in the value of stoichiometric scalar dissipation rate induces a large increase in the coefficient for nucleation and condensation while the coefficient for surface growth is lessened. This trend is due to the increasing effectiveness of PAH chemistry at smaller values of scalar dissipation rate. Thus, PAH-based soot growth is supported at leaner conditions, which ultimately supplants acetylene-based surface growth.

\begin{figure}[htb]
  \begin{center}
    \includegraphics[width=\linewidth]{ch-subfiltermodeling/figures/flamelet_sootcoeffs_le1}
    \caption[Soot Growth and Oxidation Coefficients, 1/\texorpdfstring{$\tau$}{t} vs. \texorpdfstring{$Z$}{Z}]{Flamelet calculations of soot growth and oxidation coefficients for filtered moment source terms as a function of mixture fraction. The blue lines are the oxidation coefficients, the red lines represent the surface growth coefficients, and the green lines are the coefficients for nucleation and condensation. \textit{Left} - Fuel mixture \textit{n}-heptane/nitrogen [15/85 by volume] used in the DNS of \cref{fig:subfilter:zussp:chisensitivity,fig:subfilter:leszussp:ysvsz} at $\chi_{st} = 10\ s^{-1}$. \textit{Middle} \& \textit{Right} - Fuel mixture \ce{C2H4}/\ce{H2}/\ce{N2} [40/41/19 by volume] associated with the LES of \cref{fig:subfilter:leszussp:zusspleseval} at $\chi_{st} = 10\ s^{-1}$ and $\chi_{st} = 0.1\ s^{-1}$, respectively.}
    \label{fig:subfilter:leszussp:kvsz}
  \end{center}
\end{figure}

Soot oxidation is the dominant mode at mixture fractions below and slightly above the stoichiometric value. Since the rates associated with the high-temperature oxidation chemistry are comparable with those of the main heat-releasing chemistry~\cite{guo2016}, it is anticipated that the magnitude of the peak oxidation coefficient will not vary much as the fuel mixture or stoichiometric scalar dissipation rate is modified. Indeed, this phenomenon is evident in \cref{fig:subfilter:leszussp:kvsz}. Additionally, the oxidation coefficient at stoichiometric mixture fraction is at least an order of magnitude larger than the maximum value of the surface growth coefficient. Thus, the complete oxidation of soot by \ce{OH} (and \ce{O2} to a lesser degree) is expected during transport towards stoichiometric regions. However, the soot subfilter PDF given by \cref{eq:subfilter:zussp:dd} allows for the existence of soot in fuel-lean areas, as it assumes that soot is uniformly distributed in mixture fraction space. The presence of large, non-zero oxidation coefficients at lean values of mixture fraction is therefore concerning, as it potentially leads to substantial filtered moment source terms (\cref{eq:subfilter:leszussp:ox}). This contribution to the oxidation rate is artificial and could be an explanation for the underpredicted soot volume fraction in \cref{fig:subfilter:leszussp:zusspleseval}. To prevent this non-physical oxidation from occurring, the subfilter PDF must depend on the thermochemical variables to exclude the presence of soot at lean mixtures $Z < Z_{st}$. A soot subfilter PDF that addresses this point is introduced in the following section.

\section{Z-Activated Soot Subfilter PDF}
\label{sec:subfilter:zassp}

Develop formulation for ZASSP.

\section{\textit{A Priori} Analysis}
\label{sec:subfilter:dns}

\subsection{DNS Database}
\label{sec:subfilter:dns:database}

The effectiveness of the proposed $Z$-activated soot subfilter PDF at reducing the oxidation rate is validated through an \textit{a priori} analysis using the database from a three-dimensional DNS of a temporally evolving, turbulent nonpremixed planar jet flame at atmospheric pressure~\cite{attili2014}. In this simulation, a central fuel slab consisting of an \textit{n}-heptane/nitrogen [15/85 by volume] mixture at 400 K is surrounded on either side by an air coflow at 800 K. The initial velocity field of the fuel jet is obtained from an instantaneous realization of a turbulent channel flow at $Re_{\tau} = 390$ with a centerline value of $U_c = 8.74$ m/s. The surrounding air flows in the opposite direction of the fuel jet with a streamwise velocity of the same magnitude to give a jet Reynolds number $Re = 2U_c H/\nu \approx 15,000$.

Oxidation of \textit{n}-heptane is modeled with a reduced mechanism comprising 47 species and 290 reactions that accounts for the formation of PAH up to naphthalene. The soot model, as described in \cref{ch:lesmodels,ch:subfilter}, is used in both the DNS and the \textit{a priori} analysis. In the DNS, the soot population is described with seven statistical moments whereas only three moments and the weight of the delta function are used in the \textit{a priori} investigation. The choice to use a reduced set of moments allows for validation of the proposed soot model in a configuration that will most likely be implemented in LES.

The DNS domain is discretized with $N_x \times N_y \times N_z = 1024 \times 1024 \times 512$ grid points, where the homogeneous region of the domain ($|y/H| \le 2.8$) has a grid spacing of $h = 91\ \mu$m. For a filter width $\Delta$, the \textit{a priori} study utilizes a subset of the domain of size $N_x \times N_y \times (\Delta/h + 1)$, where low-pass filtering is done only within the homogeneous mesh region. Additionally, analysis is performed on a snapshot of the DNS at $t = 5$ ms and at a stoichiometric scalar dissipation rate of $\chi_{st} = 20$ s$^{-1}$. Other key properties of the DNS are summarized in \cref{tab:subfilter:dns:params}, and complete details can be found in Attili et al.~\cite{attili2014}.

\begin{table}[htbp]
\centering
\caption[Parameters for DNS of Turbulent Nonpremixed \ce{C7H16}/\ce{N2} Jet Flame]{Parameters for DNS of a turbulent nonpremixed \ce{C7H16}/\ce{N2} jet flame}
\label{tab:subfilter:dns:params}
\begin{tabular}{p{0.46\textwidth} p{0.15\textwidth} p{0.2\textwidth}}
\toprule
Initial jet width, $H$
& [mm] & 15 \\[0.2em]

Domain size, $L_x \times L_y \times L_z$
& [mm] & $94 \times 105 \times 47$ \\[0.2em]

Time step, $\Delta t$
& [$\mu$s] & 4 \\[0.2em]

Minimum Kolmogorov scale, $\eta$
& [$\mu$m] & 110 \\[0.2em]

Kinematic viscosity of fuel mixture, $\nu$
& [m$^2$s$^{-1}$] & $1.7 \times 10^{-5}$ \\[0.2em]

Stoichiometric mixture fraction, $Z_{st}$
& [--] & 0.147 \\

\bottomrule
\end{tabular}
\end{table}

In order to validate the proposed LES model against DNS, a low-pass filter must be applied to the chosen snapshot of the DNS database. A three-dimensional, clipped and renormalized Gaussian filter kernel is employed and is given by
\begin{equation}\label{eq:subfilter:dns:kernel}
  F(x,y,z) = \kappa^3\exp\left[ \frac{-6(x^2 + y^2 + z^2)}{\Delta^2} \right],
\end{equation}
where $\Delta$ is the filter width and $\kappa$ is a renormalization constant that ensures the following relation holds for a grid spacing $h$:
\begin{equation}\label{eq:subfilter:dns:unity}
  \sum\limits_{x = -\Delta/2h}^{\Delta/2h} \sum\limits_{y = -\Delta/2h}^{\Delta/2h} \sum\limits_{z = -\Delta/2h}^{\Delta/2h} F(x,y,z) = 1.
\end{equation}
Note that the above relations require a homogeneous mesh such that $\kappa = \kappa_j$, $\Delta = \Delta_j$, and $h = h_j$, where the index $j \in \{ x,y,z \}$. As evident in \cref{eq:subfilter:dns:unity}, the filter kernel is active over a cube of $(\Delta/h + 1)^3$ grid points.


\subsection{Oxidation Source Term}
\label{sec:subfilter:dns:ox}

The filtered moment source term for oxidation is evaluated with the soot subfilter PDFs of \cref{eq:subfilter:leszussp:ox,eq:subfilter:zassp:ox}. The goal of the proposed model is to increase the overall amount of soot, so this analysis will focus on the oxidation source term contribution to the total volume fraction, $\fst[M]{1,0}^{ox}$. Oxidation of only the outermost surface is considered to reduce modeling complexity and computational cost, so the expression for the source term depends on the filtered moment for the total soot surface area $\mean{M}_{0,1}$.

Several approximations need to be made with regards to \cref{eq:subfilter:zassp:ox} before the \textit{a priori} analysis can be performed. First, the oxidation coefficient $k_{ox}(Z)$ is not present in the DNS database, as it was originally obtained from a flamelet model. However, spatial fields of density and mixture fraction are available, and the oxidation coefficient can be calculated as a function of space. Thus, $k_{ox}(Z)$ will be estimated with a density-weighted conditional average
\begin{equation}\label{eq:subfilter:dns:condkox}
  \begin{split}
    k_{ox}(Z) &\approx \{ k_{ox}(x_j)|Z(x_j) \} \\
    &= \frac{<\rho(x_j)k_{ox}(x_j)|Z(x_j)>}{<\rho(x_j)|Z(x_j)>},
  \end{split}
\end{equation}
where the angle brackets $< \cdot >$ denote the conditional averaging operator and the curly brackets $\{ \cdot \}$ denote the density-weighted conditional averaging operator. This quantity is plotted in \cref{fig:subfilter:dns:kox}, where it is evident that the oxidation coefficient at mixture fractions $Z < 0.1$ and $Z > 0.4$ is well-approximated by \cref{eq:subfilter:dns:condkox}. However, the large spread of the DNS field in $0.1 < Z < 0.4$ indicates that additional conditional averaging against the scalar dissipation rate or another variable could be performed for a more accurate approximation.

\begin{figure}[htb]
  \centering
  \includegraphics[width=0.4\linewidth]{ch-subfiltermodeling/figures/koxvsz}
  \caption[Approximation for Oxidation Coefficient, \texorpdfstring{$k_{ox}(Z)$}{kox(Z)}]{Density-weighted, conditionally averaged approximation for the oxidation coefficient $k_{ox}(Z)$. The gray dots represent $k_{ox}(x_j)|Z(x_j)$ from DNS and the red circles are the approximation $\{ k_{ox}(x_j)|Z(x_j) \}$, evaluated with 200 bins. The vertical black dashed line marks the location of stoichiometric mixture fraction $Z_{st} = 0.147$.}
  \label{fig:subfilter:dns:kox}
\end{figure}

The second quantity that needs to be estimated is the thermochemical subfilter PDF $\pz$, since the DNS database does not have the infinite number of points required to construct the true subfilter distribution. As before, this work will presume the form of a beta PDF for $\pz$. The error associated with this assumption can be discerned by juxtaposing two versions of \cref{eq:subfilter:leszussp:ox} that use the same $Z$-uniform soot subfilter PDF but approximate $k_{ox}(Z)$ with and without $\pz$. The form of the filtered moment source term for oxidation that excludes $\pz$ shall be referred to as the ``ZUSSP without $\pz$'' case and is given by
\begin{equation}\label{eq:subfilter:dns:mzusspwithoutpz}
  \fst[M]{1,0}^{ox} = \{ \tf{k_{ox}(x_j)|Z(x_j)} \} \cdot \mean{M_{0,1}(x_j)},
\end{equation}
where the $\tf{(\cdot)}$ operator denotes the density-weighted version of the low-pass filtering operator $\mean{(\cdot)}$. Use of the $Z$-uniform soot subfilter PDF implies that the soot scalars are assumed to be independent of the thermochemical variables. This property is evident in \cref{eq:subfilter:dns:mzusspwithoutpz}, where the two components are distinctly separated.

The source term using $\pz$ will be referred to as the ``ZUSSP with $\pz$'' case, and is given by
\begin{equation}\label{eq:subfilter:dns:mzusspwithpz}
  \fst[M]{1,0}^{ox} = \int\limits_0^1 \{ k_{ox}|Z \}\pz dZ \cdot \mean{M_{0,1}(x_j)},
\end{equation}
where $\{ k_{ox}|Z \}$ is the density-weighted, conditionally averaged oxidation coefficient that is not a function of space. \cref{eq:subfilter:dns:mzusspwithoutpz,eq:subfilter:dns:mzusspwithpz} are first individually compared to the filtered moment source term from DNS, which will be referred to as the ``DNS'' case and is given by
\begin{equation}\label{eq:subfilter:dns:mdns}
  \fst[M]{1,0}^{ox} = \mean{\{ k_{ox}(x_j)|Z(x_j)\} \cdot M_{0,1}(x_j)}.
\end{equation}
These comparisons are visible for a filter width of $\Delta/h = 32$ in \cref{fig:subfilter:dns:erroronbetaox}. The top two plots show that \cref{eq:subfilter:dns:mzusspwithoutpz,eq:subfilter:dns:mzusspwithpz} generally overpredict the oxidation rate compared to the value from DNS. However, when they are compared with each other in the bottom left plot, the sample standard deviation is roughly an order of magnitude smaller. This suggests that the error associated with presuming the form of a beta distribution for $\pz$ is less than the error associated with using the $Z$-uniform soot subfilter PDF to evaluate the filtered moment source term for oxidation.

The validity of presuming the beta distribution is more easily elucidated in the bottom right plot of \cref{fig:subfilter:dns:erroronbetaox}, where the cumulative distribution function (CDF) is plotted for $\fst[M]{1,0}^{ox}$. It can be observed that the distance between the lines representing \cref{eq:subfilter:dns:mzusspwithoutpz,eq:subfilter:dns:mzusspwithpz} is generally much less than the deviation between the latter and the source term from DNS. However, this trend is not valid when the magnitude of the source term is on the order of $10^2$ to $10^3$, where the error of presuming a beta distribution is commensurate with the error from the form of the soot subfilter PDF. Nevertheless, the most important values of the source term are the largest, where the error associated with presuming the form of a beta distribution for $\pz$ is relatively small.  This trend holds true for larger filter widths as well.

\begin{figure}[ht]
  \centering
  \begin{subfigure}[b]{0.375\linewidth}
    \centering
    \includegraphics[width=\linewidth]{ch-subfiltermodeling/figures/lin-Mox3vsMox6-r3D-32}
    %\vspace{1ex}
  \end{subfigure}%%
  \begin{subfigure}[b]{0.375\linewidth}
    \centering
    \includegraphics[width=\linewidth]{ch-subfiltermodeling/figures/lin-Mox4vsMox6-r3D-32}
    %\vspace{1ex}
  \end{subfigure}
  \begin{subfigure}[b]{0.375\linewidth}
    \centering
    \includegraphics[width=\linewidth]{ch-subfiltermodeling/figures/lin-Mox4vsMox3-r3D-32}
  \end{subfigure}%%
  \begin{subfigure}[b]{0.375\linewidth}
    \centering
    \includegraphics[width=\linewidth]{ch-subfiltermodeling/figures/cdf-ox-ZUSSP-r3D-32}
  \end{subfigure}
  \caption[Error Associated with \texorpdfstring{$\pz = \beta(Z;\tf{Z},\tf{Z_V})$}{P(Z) = B(Z;Z,ZV)} for \texorpdfstring{$\fst[M]{1,0}^{ox}$}{M1,0ox}]{Filtered moment source term for oxidation at $t = 5$ ms evaluated with \cref{eq:subfilter:dns:mzusspwithoutpz,eq:subfilter:dns:mzusspwithpz,eq:subfilter:dns:mdns} for a filter width of $\Delta/h = 32$. In the first three plots, the red lines represent a one-to-one correspondence and the sample standard deviation is indicated at the top left corner. In the fourth, bottom right plot, the solid red line is the ``DNS'' case, the blue dashed line is the ``ZUSSP without $\pz$'' case, and the blue dash-dotted line is the ``ZUSSP with $\pz$'' case.}
  \label{fig:subfilter:dns:erroronbetaox}
\end{figure}

Now that the form of the thermochemical subfilter PDF $\pz$ has been confirmed, the filtered moment source term for oxidation using the $Z$-activated soot subfilter PDF can be evaluated. The latter shall be referred to as the ``ZASSP with $\pz$'' case and its expression is given by
\begin{equation}\label{eq:subfilter:dns:mzasspwithpz}
  \fst[M]{1,0}^{ox} = \frac{\int\limits_0^1 \{ k_{ox}|Z \}H(Z - Z_{st})\pz dZ}{\int\limits_0^1 H(Z - Z_{st})\pz dZ} \cdot \mean{M_{0,1}(x_j)}.
\end{equation}

This case is plotted against the source term from DNS in \cref{fig:subfilter:dns:zasspcomparison}. In the left-hand plot, it is clear that the filtered moment source term for oxidation evaluated with the proposed model still overpredicts the oxidation rate when compared to the values from DNS. However, when contrasted with the top right plot of \cref{fig:subfilter:dns:erroronbetaox}, the magnitudes of the largest source terms are reduced by nearly half and the standard deviation is decreased. A direct comparison between the source terms using the $Z$-uniform and $Z$-activated soot subfilter PDFs, available in the middle plot of \cref{fig:subfilter:dns:zasspcomparison}, demonstrates that the latter tends to produce smaller oxidation rates than the former. This is to be expected, as the $Z$-activated soot subfilter PDF was formulated to eliminate the unphysical contributions to the oxidation rate that the $Z$-uniform soot subfilter PDF possessed.

The CDF in the right-hand plot of \cref{fig:subfilter:dns:zasspcomparison} more clearly reveals the extent of the oxidation source term reduction. It is evident that the proposed model has decreased the largest values of the source term relative to the ``ZUSSP with $\pz$'' case, albeit the effect is not enough to reach the values of the source terms from DNS. Nevertheless, the effectiveness of the proposed model is expected to increase with the filter width due to the expanded presence of lean subfilter regions, which is a consequence of the enlargened variance in mixture fraction. This point will be explored in \cref{sec:subfilter:dns:fw}.

\begin{figure}[ht]
  \centering
  \begin{subfigure}[b]{0.33\linewidth}
    \centering
    \includegraphics[width=\linewidth]{ch-subfiltermodeling/figures/lin-Mox5vsMox6-r3D-32}
  \end{subfigure}%%
  \begin{subfigure}[b]{0.33\linewidth}
    \centering
    \includegraphics[width=\linewidth]{ch-subfiltermodeling/figures/lin-Mox5vsMox4-r3D-32}
  \end{subfigure}%%
  \begin{subfigure}[b]{0.33\linewidth}
    \centering
    \includegraphics[width=\linewidth]{ch-subfiltermodeling/figures/cdf-ox-ZASSP-r3D-32}
  \end{subfigure}
  \caption[Comparison of ZASSP with $\pz$ to DNS \& ZUSSP with $\pz$ for \texorpdfstring{$\fst[M]{1,0}^{ox}$}{M1,0ox}]{Comparison of the ``ZASSP with $\pz$'' case to the ``DNS'' and ``ZUSSP with $\pz$'' cases for the same conditions as in \cref{fig:subfilter:dns:erroronbetaox}. In the right-hand plot, the solid red line is the ``DNS'' case, the blue dash-dotted line is the ``ZUSSP with $\pz$'' case, and the magenta dash-dotted line is the ``ZASSP with $\pz$'' case.}
  \label{fig:subfilter:dns:zasspcomparison}
\end{figure}


\subsection{Surface Growth Source Term}
\label{sec:subfilter:dns:sg}

Analysis of surface growth source term.


\subsection{Effects of Filter Width}
\label{sec:subfilter:dns:fw}

Investigation of filter width variation effects.



