\section{Joint Subfilter PDF}
\label{sec:subfilter:joint}

In \cref{eq:lesmodels:presumedpdf:bayes}, the joint subfilter PDF was expressed as the product of the thermochemical PDF and the PDF of the soot scalars conditioned on the thermochemical variables. The form of the latter has been approximated previously through consideration of the dynamics of soot. In DNS studies of a nitrogen-diluted, \textit{n}-heptane/air turbulent nonpremixed planar jet flame, PAH were found to have much slower formation chemistry compared to the main heat-releasing chemistry, which is represented by the thermochemical variables $\xi$~\cite{attili2014,bisetti2012}. PAH are soot precursors, so soot itself is characterized by even longer timescales. Consequently, Mueller and Pitsch~\cite{subfilterpdf2011} proposed that the timescales of the soot scalars and thermochemical variables are disparate enough such that the former should not depend on the latter. This argument was used to simplify the subfilter PDF of the soot scalars conditioned on $\xi$ to a marginal PDF of only the soot scalars. The expression for the spatially Favre filtered functions of \cref{eq:lesmodels:presumedpdf:filteredfuncs} then becomes
\begin{equation}\label{eq:lesmodels:presumedpdf:separated}
  \tf{\phi}(\xi,\mc{M}_j) = \iint \mc{J}(\xi,\mc{M}_j)\tf{P}(\xi)P(\mc{M}_j) d\xi d\mc{M}_j.
\end{equation}
By replacing the functional relation $\mc{J}$ with \cref{eq:lesmodels:combust:producteos}, \cref{eq:lesmodels:presumedpdf:separated} can be further simplified to
\begin{equation}\label{eq:lesmodels:presumedpdf:indep}
  \tf{\phi}(Z,C,H,\mc{M}_j) = \iiint \mc{G}(Z,C,H)\tf{P}(Z,C,H) dHdCdZ \times \int \mc{K}(\mc{M}_j)P(\mc{M}_j) d\mc{M}_j,
\end{equation}
where the thermochemical and soot components are now completely independent. Mueller and Pitsch noted that the timescale separation argument could be violated during the oxidation of soot, when there are enhanced interactions between soot and the major gas-phase chemistry, and during surface growth near the flame. These situations will be examined more closely in \cref{sec:subfilter:zassp}.

The challenging task of modeling the joint subfilter PDF was then reduced to developing models for the thermochemical subfilter PDF and the soot subfilter PDF. Discussion of the latter will be deferred to \cref{ch:subfilter}. The thermochemical subfilter PDF for the RFPV approach is obtained from Ihme and Pitsch~\cite{ihme2008}. First, two quantities are introduced to uniquely identify each flamelet solution in the database of thermochemical states: $\Lambda = C(Z_{st})$ and $\Phi = H(Z_{st})$. The thermochemical equation of state (\cref{eq:lesmodels:combust:rfpv}) then becomes
\begin{equation}\label{eq:lesmodels:presumedpdf:eos}
  \xi = \mc{G}(Z, C, H) = \mg(Z, \Lambda, \Phi).
\end{equation}
The spatially Favre filtered thermochemical functions are given by
\begin{equation}\label{eq:lesmodels:presumedpdf:filteredeos}
  \begin{split}
    \tf{\xi} &= \iiint \mc{G}(Z,C,H)\tf{P}(Z,C,H) dHdCdZ \\
    &= \iiint \mg(Z,\Lambda,\Phi)\tf{P}(Z,\Lambda,\Phi) d\Phi d\Lambda dZ.
  \end{split}
\end{equation}
Since $Z$, $\Lambda$, and $\Phi$ are defined to be independent, the thermochemical subfilter PDF can be expressed as the product of three marginal distributions:
\begin{equation}\label{eq:lesmodels:presumedpdf:trimarg}
  \tf{P}(Z,\Lambda,\Phi) = \beta(Z;\tf{Z},\tf{Z_{\text{V}}})\delta(\Lambda - \tf{\Lambda})\delta(\Phi - \tf{\Phi}),
\end{equation}
where the mixture fraction has been modeled with a beta distribution~\cite{cook1994,jimenez1997,wall2000}, the subfilter mixture fraction variance is defined as $\tf{Z_{\text{V}}} = \tf{Z^2} - \tf{Z}^2$, and $\Lambda$ and $\Phi$ are modeled with delta distributions~\cite{ihme2008}. Assuming \cref{eq:lesmodels:combust:rfpv} is unique, a bijective inversion may be used to conveniently reexpress a dependence on $\tf{\Lambda}$ and $\tf{\Phi}$ as a dependence on $\tf{C}$ and $\tf{H}$.
