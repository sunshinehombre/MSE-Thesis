Soot is a carbonaceous product that is undesired in many practical combustion-based engineering systems. Turbulence is present in most of these systems, yet its impact on the evolution of soot is not well characterized. This thesis advances the understanding of small-scale interactions between soot, turbulence, and chemistry as well as the role of transport in the evolution of soot. The insights gained are manifested through the $Z$-activated soot subfilter PDF and the strain-sensitive transport approach, which are models developed for an integrated Large Eddy Simulation (LES) framework. The final objective of this thesis is to evaluate and validate these models through LES of a series of laboratory-scale turbulent nonpremixed jet flames.

The $Z$-activated soot subfilter PDF is a model for the unresolved small-scale interactions between soot, chemistry, and turbulence. It has the form of a double delta distribution to account for the intermittent nature of soot and contains a dependence on the mixture fraction to address the absence of soot in fuel-lean regions and prevent non-physical oxidation from occurring. This model is validated \textit{a priori} against a recent DNS database of a turbulent nonpremixed jet flame. At a fixed filter width, the filtered moment source term for oxidation is observed to be reduced compared to the source term from a soot subfilter PDF that is independent of the mixture fraction. Meanwhile, the filtered moment source term for surface growth is minimally influenced by the proposed model, for surface growth is present only in fuel-rich regions. Enlarging the filter width results in an intensified decrease in the oxidation source term, while the surface growth source term remains hardly affected.

The strain-sensitive transport approach is formulated to appropriately model the transport of gas-phase species such as Polycyclic Aromatic Hydrocarbons (PAH) that are governed by relatively slow chemistry. The slow formation kinetics of PAH confine these soot precursors to spatially intermittent regions of low scalar dissipation rate where the residence time is sufficiently long. These regions are on the order of the Kolmogorov scale or smaller, where transport is solely dictated by molecular diffusion. Identification of species such as PAH that should be modeled with detailed transport is achieved through the strain-sensitivity parameter, which compares the rate of local mixing to the formation chemistry rate of a particular species. A new effective species Lewis number with a dependence on this parameter is defined for straightforward integration into an existing nonpremixed flamelet framework accounting for differential diffusion effects. An \textit{a priori} analysis with flamelet solutions reveals that the flame temperature and acetylene mass fraction profiles from the proposed model closely match those from the equal effective diffusivities approach, while the naphthalene mass fraction profile from the proposed model approaches that from the full detailed transport approach. Sensitivity studies on the parameter reveal that the same species are identified as strain-sensitive as the fuel mixture, chemical mechanism, and stoichiometric scalar dissipation rate are varied.

The proposed models are validated against experimental data from a series of three turbulent nonpremixed ethylene-hydrogen-nitrogen simple jet flames with varying amounts of global strain at a constant Reynolds number of 15,000. LES with the $Z$-activated soot subfilter PDF and the strain-sensitive transport approach exhibit good agreement with the experimental profiles for the time-averaged, centerline flame temperature and show improved predictions of the time-averaged, centerline soot volume fraction over LES with the $Z$-activated soot subfilter PDF and unity effective Lewis numbers. LES results are found to be sensitive to the chemical mechanism, although the maximum soot volume fraction is underpredicted by a factor of ten at best, suggesting that the overall integrated modeling approach is still missing some aspect of the physics of soot evolution. Nevertheless, the LES with the proposed models are still able to capture the general trends with the global strain rate. 

%% Models for both topics are developed for an integrated Large Eddy Simulation (LES) framework. The final objective of this thesis is to evaluate and validate these models through LES of a series of laboratory-scale, turbulent nonpremixed simple jet flames.

%% Two major components developed are a model for the small-scale interactions between soot, turbulence, and chemistry as well as a model for the transport of strain-sensitive species in turbulent nonpremixed combustion. The former is based on the presumed subfilter PDF method, where the distribution of blah is assumed. It contains a dependence on theh mixture fraction to address the lack of soot in fuel-lean regions and has the form of a double-delta distribution to account for the high spatial intermittency of soot. The strain-sensitive transport approach is a model that transports species governed by relatively slow formation chemistry with molecular diffusion and transports species governed by fast kinetics with equal effective diffusivities. A strain-sensitivity parameter is formulated to categorize each species. The final objective of this thesis is to validate these models in Large Eddy Simulations (LES). A series of three laboratory-scale ethylene-hydrogen-nitrogen simple jet flames are used as the testbench.

%% In chapter 3, the Z-activated soot subfilter PDF is a model for the small-scale interactions between soot, chemistry, and turbulence that accounts for the spatial intermittency of soot as well as the distribution of soot in mixture fraction space. It has the form of a double delta distribution to account for sooting and non-sooting modes within an LES filter width. It excludes the presence of soot at lean mixtures in order to prevent non-physical oxidation from occurring. The filtered moment source terms for oxidation and surface growth are evaluated with this model. The Z-activated soot subfilter PDF was found to reduce the oxidation rate at a fixed filter width compared to the subfilter PDF that did not account for soot's distribution in mixture fraction space. Meanwhile, the source term for surface growth was minimally affected by this model, for surface growth is confined to fuel rich regions. A increase in the LES filter width results in a intensified decrease in the oxidation rate, while the surface growth source term is hardly affected.

% In chapter 4, the strain-sensitive transport approach was developed to appropriately model the transport of species that are governed by relatively slow formation chemistry like PAH. PAH are confined to spatially intermittent regions of low scalar dissipation rate that are on the order of the Kolmogorov scale or smaller, where transport is solely dictated by molecular diffusion. This approach is based on the nonpremixed flamelet equations accounting for differential diffusion effects. Identification of species that should be modeled with molecular diffusion is achieved through the strain-sensitivity parameter, which compares the rate of local mixing to the formation chemistry rate of a particular species. This parameter is incorporated in the the flamelet framework through a new definition of the effective species Lewis number. An \textit{a priori} analysis using flamelet solutions reveal that the proposed model matches the flame temperature and acetylene mass fraction profiles from solutions using a unity Lewis number assumption, while the naphthalene mass fraction profile with the proposed model approaches the profile using full detailed transport. Sensitivity studies with the strain-sensitivity parameter reveal that the same species are identified as strain-sensitive as the fuel mixture, chemical mechanism, and stoichiometric scalar dissipation rate are varied.

% In chapter 5, the proposed models are validated against experimental data from a series of turbulent nonpremixed ethylene-hydrogen-nitrogen simple jet flames with varying amounts of global strain at a constant Reynolds number of 15,000. LES with the Z-activated soot subfilter PDF and unity effective Lewis numbers demonstrated good agreement with experimental profiles for the centerline flame temperature but underpredicted the centerline soot volume fraction by roughly two orders of magnitude. Minimal differences improvements of LES with the Z-uniform soot subfilter PDF and unity effective Lewis numbers were observed due to only a slight reduction in the maximum rate of oxidation. LES with the Z-activated soot subfilter PDF and the strain-sensitive transport approach showed good agreement with experimental profiles for the flame temperature, and exhibited an improved prediction of the soot volume fraction over LES with the Z-activated soot subfilter PDF and unity effective Lewis numbers. LES results were found to be sensitive to the chemical mechanism, although the maximum volume fraction was underpredicted by a factor of 10 at best, suggesting that the model is still missing some aspect of the physics of soot evolution. Nevertheless, the LES with the proposed models were able to capture trends with the global strain rate.

%% This is a \LaTeX{} template and document class for Ph.D. dissertations at Princeton University. It was created in 2010 by Jeffrey Dwoskin, and adapted from a template provided by the math department. Their original version is available at: \url{http://www.math.princeton.edu/graduate/tex/puthesis.html}

%% This is \textbf{NOT} an official document. Please verify the current Mudd Library dissertation requirements~\cite{mudd2009} and any department-specific requirements before using this template or document class.


%% Your abstract can be any length, but should be a maximum of 350 words for a Dissertation for ProQuest's print indicies (or 150 words for a Master's Thesis); otherwise it will be truncated for those uses~\cite{proquest2006}.


%% Dwoskin Ph.D. Dissertation Template --- version 1.0, 5/19/2010
