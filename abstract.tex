

Two major components developed are a model for the small-scale interactions between soot, turbulence, and chemistry as well as a model for the transport of strain-sensitive species in turbulent nonpremixed combustion. The former is based on the presumed subfilter PDF method, where the distribution of blah is assumed. It contains a dependence on theh mixture fraction to address the lack of soot in fuel-lean regions and has the form of a double-delta distribution to account for the high spatial intermittency of soot. The strain-sensitive transport approach is a model that transports species governed by relatively slow formation chemistry with molecular diffusion and transports species governed by fast kinetics with equal effective diffusivities. A strain-sensitivity parameter is formulated to categorize each species. The final objective of this thesis is to validate these models in Large Eddy Simulations (LES). A series of three laboratory-scale ethylene-hydrogen-nitrogen simple jet flames are used as the testbench.



%% This is a \LaTeX{} template and document class for Ph.D. dissertations at Princeton University. It was created in 2010 by Jeffrey Dwoskin, and adapted from a template provided by the math department. Their original version is available at: \url{http://www.math.princeton.edu/graduate/tex/puthesis.html}

%% This is \textbf{NOT} an official document. Please verify the current Mudd Library dissertation requirements~\cite{mudd2009} and any department-specific requirements before using this template or document class.


%% Your abstract can be any length, but should be a maximum of 350 words for a Dissertation for ProQuest's print indicies (or 150 words for a Master's Thesis); otherwise it will be truncated for those uses~\cite{proquest2006}.


%% Dwoskin Ph.D. Dissertation Template --- version 1.0, 5/19/2010
