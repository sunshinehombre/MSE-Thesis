\section{Strain-Sensitive Transport Approach}
\label{sec:lesresults:ssta}

%% what i want to say in this section: start with a comparison of unity to zeta to lei for flame 2. discuss centerline T and fv.
%% Then bring in radial statistics for unity vs zeta vs lek.

%% next, discuss zeta vs unity across all three flames for centerline T and fv structure.
%% bring in radial statistics for zeta vs unity and discuss T and fv.
%% Then look at radial conditional statistics at xD = 49, 71, 94, and 117 for dfv/dt for nucleation + condensation, surface growth, and oxidation to compare zeta and lek?
%% lastly, bring up point of different chemical mechanism - km2.
%% Show centerline T and fv for KM2 with le1 and le zeta, compare with existing models for flame 2.
%% finally, compare radial statistics for KM2 flame 2 against stanford mechanisms.

Results from the LES using the $Z$-activated soot subfilter PDF were presented in \cref{sec:lesresults:zassp} under the assumption of unity effective Lewis numbers for all gas-phase species. This assumption is appropriate when turbulent transport dominates molecular diffusion within the fuel-oxidizer mixing zone, and when the species' length scales are comparable to the length scale of the latter. In \cref{ch:transport}, this was found to be inappropriate for PAH, whose formation chemistry is slow relative to turbulent mixing. PAH are confined to regions of low scalar dissipation rate that are on the order of the Kolmogorov scale or smaller~\cite{vaishnavi2008}, so they are governed by molecular transport. The strain-sensitive transport approach was developed in \cref{sec:transport:ssta} to more accurately model the behavior of PAH and similar species with relatively slow chemistry. In this section, LES results with the aforementioned approach will be compared to experimental data and LES with other transport models. The goal is to evaluate the ability of the proposed transport model to more accurately predict the evolution of soot, so the $Z$-activated soot subfilter PDF will be implemented in all cases of interest.

In \cref{fig:lesresults:ssta:f2lecomparison}, the time-averaged, centerline flame temperature is plotted for flame $1/\tau|_M$ of \cref{tab:subfilter:leszussp:ehn}. It is clear that the LES with unity effective Lewis numbers and the strain-sensitive transport approach predict the flame temperature fairly accurately, while the LES with detailed transport for gas-phase species predicts a much longer flame with a higher maximum temperature. The peak mean temperatures of the former two LES closely match the magnitude of the experimental value, but are shifted slightly downstream as in \cref{fig:lesresults:zassp:ctrlineleseval}. The LES with full detailed transport accounts for the high diffusivity of the hydrogen radical, which promotes its presence at locations far downstream and allows for more frequent participation in various reactions that contribute to an elevated maximum flame temperature.

\begin{figure}[H]
  \centering
  \includegraphics[width=0.6\linewidth]{ch-lesresults/figures/2f-Tfv-S-ZASSP-Le_comparison}
  \caption[Centerline \texorpdfstring{$\langle T \rangle$}{<T>} \& \texorpdfstring{$\langle f_V \rangle$}{<fV>} from LES of Flame \texorpdfstring{$1/\tau|_M$}{1/t|M} with Various Transport Approaches]{Time-averaged flame temperatures and soot volume fractions from LES of jet flame $1/\tau|_M$. Circles indicate experimental data, solid lines are LES with unity effective Lewis numbers for gas-phase species, dashed lines are LES with the strain-sensitive transport approach, and dash-dotted lines are LES with detailed transport for all gas-phase species. Profiles in black correspond to flame temperatures while profiles in blue represent soot volume fractions. Note that the volume fraction profiles from LES have been premultiplied by a factor of 10 for visibility.}
  \label{fig:lesresults:ssta:f2lecomparison}
\end{figure}

As before, the time-averaged, centerline soot volume fractions are underpredicted by all simulations. For the LES with unity effective Lewis numbers and the strain-sensitive transport approach, the volume fractions peak at $x/D \approx 80$, versus the experimental maximum at $x/D \approx 100$, and approach zero earlier at $x/D \approx 130$ to 140. On the other hand, the LES that models all gas-phase species with molecular transport predicts a larger soot volume fraction at locations further downstream and has a volume fraction maximum at nearly the same location as the maximum value from experiment. Its peak volume fraction is more than four times larger than that of the LES with unity effective Lewis numbers, and slightly more than twofold the maximum value from the LES with the strain-sensitive transport approach. These two phenomena result from modeling the hydrogen radical with molecular transport, which allows it to diffuse further downstream and contribute to soot growth through the \ce{H}-abstraction, \ce{C2H2}-addition surface reaction mechanism~\cite{frenklach1985,frenklach1991}.

\begin{figure}[H]
  \centering
  \includegraphics[width=\linewidth]{ch-lesresults/figures/dfvdt-S-ZASSP-Le_comparison-ncsplit-cropped}
  \caption[Radial \texorpdfstring{$\langle df_V/dt \rangle/|\langle df_V/dt|_{\text{max}} \rangle|$}{<dfV/dt>/|<dfV/dt>|max|} from LES with Various Transport Approaches]{\textit{Left to right} - Time-averaged soot volume fraction source terms from LES of jet flame $1/\tau|_M$ with unity effective Lewis numbers, strain-sensitive transport, and detailed transport, respectively. These are normalized by the absolute value of the maximum source term contribution within each flame. \textit{Bottom to top} - Increasing streamwise locations of $x/D = 49, 71, 94,$ and 117, respectively. Profiles in dark green are source terms from nucleation, profiles in green are source terms from condensation, profiles in red are source terms from surface growth, and profiles in blue are source terms from oxidation.}
  \label{fig:lesresults:ssta:radialdfvdtf2lecomparison}
\end{figure}

This behavior can be seen more clearly in \cref{fig:lesresults:ssta:radialdfvdtf2lecomparison}, where the LES with detailed transport possesses a normalized time-averaged soot volume fraction source term for surface growth that has a proportionally larger contribution when compared to LES with the other transport models. In fact, at $x/D = 94$, it becomes the most dominant component for the LES with detailed transport. Conversely, the LES with unity effective Lewis numbers and the strain-sensitive transport approach have oxidation by \ce{OH} and \ce{O2} as the most prominent source term at the same axial location.

However, the hydrogen radical is not identified as a species with relatively slow chemistry in \cref{fig:transport:ssta:dependencies:chist}. It is not confined to regions of low scalar dissipation rate that are on the order of the Kolmogorov scale or smaller, so modeling its behavior with molecular diffusion is inappropriate. The strain-sensitive transport approach more accurately models its physics as well as the physics of species with relatively slow chemistry. In \cref{fig:lesresults:ssta:f2lecomparison}, the LES with the latter model shows an improved prediction of the soot volume fraction over the LES with unity effective Lewis numbers. A closer look at the relative contributions of the various volume fraction source terms in \cref{fig:lesresults:ssta:radialdfvdtf2lecomparison} reveals that at $x/D = 49$, the LES with the proposed model has nucleation, condensation, surface growth, and oxidation occurring at roughly the same proportions as the LES with unity effective Lewis numbers. However, by $x/D = 71$, the LES with the strain-sensitive transport approach predicts a higher amount of condensation relative to nucleation for $r/D \le 2$ when compared to the LES with unity effective Lewis numbers. Such a trend is responsible for the increased volume fraction in the region around $x/D = 71$, as observed in \cref{fig:lesresults:ssta:f2lecomparison}.

The previously discussed results use the chemical mechanism from Blanquart \etal~\cite{blanquart2009} and Narayanaswamy \etal~\cite{narayanaswamy2010}, which places emphasis on the formation of soot precursors up to cyclopenta[cd]pyrene (\ce{C18H10}). The sensitivity of soot evolution to the kinetics was also assessed by performing LES with the chemical mechanism from Wang \etal~\cite{wang2013} (known as the KM2 mechanism) while modeling gas-phase species with unity effective Lewis numbers and with the strain-sensitive transport approach. In \cref{fig:lesresults:ssta:f2lemechcomparison}, it is clear that the time-averaged, centerline flame temperatures from these LES closely match the experimental data. However, the soot volume fractions are obviously increased. The peak time-averaged volume fraction from the LES using the KM2 mechanism with unity effective Lewis numbers is 29 ppb, which is roughly tenfold larger than the maximum value from the corresponding LES with the former chemical mechanism (2.9 ppb). The LES using the KM2 mechanism with the strain-sensitive transport approach has a peak volume fraction of 46.7 ppb, which is within a factor of ten of the peak experimental value and is about eight times larger than the maximum value from the corresponding LES with the other chemical mechanism (6.0 ppb). The KM2 mechanism accounts for PAH species up to coronene (\ce{C24H12}), which has seven aromatic rings compared to the five fused rings of cyclopenta[cd]pyrene. The LES results suggest that the kinetics of these larger PAH may play an important role in the evolution of soot, as was found in previous studies~\cite{wang2013,selvaraj2016}. 

\begin{figure}[htb]
  \centering
  \includegraphics[width=0.6\linewidth]{ch-lesresults/figures/2f-Tfv-KS-ZASSP-Le_comparison}
  \caption[Centerline \texorpdfstring{$\langle T \rangle$}{<T>} \& \texorpdfstring{$\langle f_V \rangle$}{<fV>} from LES of Flame \texorpdfstring{$1/\tau|_M$}{1/t|M} with Various Transport Approaches and Chemical Mechanisms]{Time-averaged flame temperatures and soot volume fractions from LES of jet flame $1/\tau|_M$. All symbols and lines are the same as in \cref{fig:lesresults:ssta:f2lecomparison}, where the chemical mechanism from Blanquart \etal~\cite{blanquart2009} and Narayanaswamy \etal~\cite{narayanaswamy2010} was used. However, results from LES with the chemical mechanism of Wang \etal~\cite{wang2013} have been added - double dotted lines are LES with unity effective Lewis numbers, and dash-double dotted lines are LES with the strain-sensitive transport approach. Note that the volume fraction profiles from LES have been premultiplied by a factor of 10 for visibility.}
  \label{fig:lesresults:ssta:f2lemechcomparison}
\end{figure}

The evolution of soot is clearly sensitive to the kinetics of PAH precursors, as evidenced by the increase in the maximum volume fraction by a factor of 7.7 to 9.8 with the KM2 chemical mechanism compared to that from Blanquart \etal~\cite{blanquart2009} and Narayanaswamy \etal~\cite{narayanaswamy2010}. However, note that accurately modeling the interactions of PAH and other species that influence the development of soot particles with turbulence is just as important. The strain-sensitive transport approach improves predictions of the peak volume fraction by a preliminary factor of 1.6 to 2.1 over the equal effective diffusivities approach. Improvements to the proposed models, which will be discussed in \cref{sec:conclusion:future}, could potentially make the sensitivity to the turbulent interactions on par with the sensitivity to chemical mechanism.

%% Include comparison of EHN, Stanford, ZASSP, and $Le = 1$ vs. $\check{Le}_k(\zeta_k)$ vs. $Le_k$.

%% Also include the effect of the chemical mechanism with a comparison of EHN, Stanford, ZASSP, and $\check{Le}_k(\zeta_k)$ with EHN, KM2, ZASSP, and $Le = 1$ vs. $\check{Le}_k(\zeta_k)$.

%% Talk about temperature, $Y_{PAH}$, $f_V$, conditional radial statistics at various distances downstream, etc.
