\section{Z-Activated Soot Subfilter PDF}
\label{sec:lesresults:zassp}

In this section, results from LES with the $Z$-activated soot subfilter PDF of \cref{sec:subfilter:zassp} are compared against experimental data and the results from LES with the $Z$-uniform soot subfilter PDF of \cref{sec:subfilter:zussp}. The focus of this section is to evaluate the differences between the two soot subfilter PDFs, so the effective Lewis numbers of all gas-phase species will be held constant at unity in both LES cases.
% ability of the $Z$-activated soot subfilter PDF to more accurately predict the evolution of soot, so the effective Lewis numbers of all gas-phase species will be held constant at unity in both LES cases.  

In \cref{fig:lesresults:zassp:ctrlineleseval}, it is clear that time-averaged, centerline flame temperature is generally well-predicted by both LES cases in all flames. In the recorded experiments, the flame temperature rises to a peak value around $x/D = 100$. The flame with the highest exit strain rate ($1/\tau|_H$, left plot) possesses a small region of uniform temperature around $x/D = 0$ due to the lifting of the flame in both LES cases. This feature is not present in the experiments and explains the slight downstream shift of the temperature compared to the experiment until about $x/D = 50$. In all three flames, the peak mean temperature from LES closely matches the magnitude of the corresponding experimental value (1800 to 1900 K), but is shifted slightly downstream. Overall, the simulated flames tend to be longer than their experimental counterparts.

\begin{figure}[htb]
  \centering
  \includegraphics[width=\linewidth]{ch-lesresults/figures/combined-Tfv-S-ZUSSP-vs-ZASSP-Le_1}
  \caption[Centerline \texorpdfstring{$\langle T \rangle$}{<T>} \& \texorpdfstring{$\langle f_V \rangle$}{<fV>} from LES with \texorpdfstring{$Z$}{Z}-Activated Soot Subfilter PDF]{\textit{Left to right} - Time-averaged flame temperatures and soot volume fractions from LES of jet flames $1/\tau|_H$, $1/\tau|_M$, and $1/\tau|_L$, respectively. Note that the soot volume fraction profiles from LES have been premultiplied by a factor of 100 for visibility. Circles indicate experimental data, solid lines are time-averaged values from LES with the $Z$-uniform soot subfilter PDF, and dashed lines are time-averaged values from LES with the $Z$-activated soot subfilter PDF. Profiles in black correspond to centerline flame temperatures while profiles in blue represent centerline soot volume fractions.}
  \label{fig:lesresults:zassp:ctrlineleseval}
\end{figure}

Most obvious in \cref{fig:lesresults:zassp:ctrlineleseval} is the underprediction of the time-averaged, centerline soot volume fraction by both LES cases. In all three flames, the volume fraction is underestimated by roughly two orders of magnitude relative to the experimental values. Furthermore, the mean soot volume fractions from each LES peak slightly earlier and approach zero sooner at approximately $x/D = 130$. These phenomena signal the presence of intense oxidation further upstream compared to the experiments. However, the LES with the $Z$-activated soot subfilter PDF do not experience as much oxidation as the LES with the $Z$-uniform soot subfilter PDF, which is evident by the elevated amount of soot volume fraction in the downstream regions of flames $1/\tau|_H$ and $1/\tau|_L$. In jet flame $1/\tau|_M$, the maximum volume fraction of the LES with the $Z$-uniform soot subfilter PDF is slightly larger than that from the LES with the proposed model, which is a result of the less-converged statistics of the former. Overall, the decreased rate of oxidation is expected, as turbulent mixing promotes the presence of regions with mixture fractions less than stoichiometric. The $Z$-activated soot subfilter PDF was formulated to eliminate the unphysical oxidation contributions from these fuel-lean areas, so soot is able to persist slightly further downstream than in the simulations with the $Z$-uniform soot subfilter PDF.

Ultimately, the differences in the predicted centerline volume fractions from the two soot subfilter PDFs are not dramatic. In \cref{fig:subfilter:leszussp:kvsz}, it is evident that implementing a cutoff at the stoichiometric mixture fraction does not appreciably decrease the maximum rate of oxidation. In fact, the rate of oxidation at the latter location is only slightly less than the global maximum of the oxidation rate. To address this point, modifications to the activation location of the $Z$-activated soot subfilter PDF in mixture fraction space are proposed in \cref{sec:conclusion:future:zassp}.

Time-averaged radial statistics of the soot volume fraction source terms are provided in \cref{fig:lesresults:zassp:radialdfvdt}. In jet flames $1/\tau|_M$ and $1/\tau|_L$, the maximum rates of oxidation from LES with the $Z$-activated soot subfilter PDF become smaller than those from LES with the $Z$-uniform soot subfilter PDF as the distance in the streamwise direction is increased. However, as previously mentioned, these differences are minimal. %This trend is especially observable in jet flames $1/\tau|_M$ and $1/\tau|_L$, where the maximum rates of oxidation from LES with the proposed model are initially larger than those from LES with the $Z$-uniform soot subfilter PDF at $x/D = 49$ and 71.

\begin{figure}[H]
  \centering
  \includegraphics[width=\linewidth]{ch-lesresults/figures/dfvdt-S-ZUSSP-vs-ZASSP-Le_1-edited2}
  \caption[Radial \texorpdfstring{$\langle df_V/dt \rangle$}{<dfV/dt>} from LES with \texorpdfstring{$Z$}{Z}-Activated Soot Subfilter PDF]{\textit{Left to right} - Time-averaged soot volume fraction source terms from LES of jet flames $1/\tau|_H$, $1/\tau|_M$, and $1/\tau|_L$, respectively. \textit{Bottom to top} - Increasing streamwise locations of $x/D = 49, 71, 94,$ and 117, respectively. Solid lines are LES with the $Z$-uniform soot subfilter PDF and dashed lines are LES with the $Z$-activated soot subfilter PDF. Profiles in green are source terms from nucleation and condensation, profiles in red are source terms from surface growth, and profiles in blue are source terms from oxidation. The soot volume fraction is provided as a reference in the background from $0 \le r/D \le 2$ of each flame, where the positions $x/D = 49, 71, 94,$ and 117 in the field are aligned at the corresponding middle of each plot.}
  \label{fig:lesresults:zassp:radialdfvdt}
\end{figure}

Both subfilter PDFs accurately capture the overall nature of soot evolution with respect to the streamwise distance. The position $x/D = 49$ is approximately located at the beginning of the region of concentrated volume fraction, as revealed by the background field in \cref{fig:lesresults:zassp:radialdfvdt}. Here, nucleation from PAH (and condensation to a lesser degree) are the main soot growth pathways. Condensation becomes increasingly important at $x/D = 71$ (investigation not shown), where nucleated particles are numerous enough for PAH to condense upon. At $x/D = 94$, surface growth overtakes nucleation and condensation as the main growth pathway for jet flames $1/\tau|_H$ and $1/\tau|_M$. Such a transition between the growth modes is reasonable, for turbulent mixing tends to decrease the mixture fraction in the downstream region and promote surface growth, as shown in \cref{fig:subfilter:leszussp:kvsz}. Additionally, the elevated global strain rates in the aforementioned flames tend to decrease the influence of the strain-sensitive PAH. %These observations suggest that the conclusions of previous DNS studies of nonpremixed turbulent flames~\cite{bisetti2012,attili2014} are premature, where these analyses found that the soot mass growth rate from surface reactions is dominated by the contributions from nucleation and condensation processes.

The differences in the surface growth source terms between the LES with the $Z$-activated soot subfilter PDF and the LES with the $Z$-uniform soot subfilter PDF are observed to be minimal at $x/D = 94$. This finding agrees with the \textit{a priori} analysis of \cref{sec:subfilter:dns:sg}, where a comparison of the two soot subfilter PDFs revealed small differences in the evaluation of the surface growth moment source term. Lastly, oxidation dominates all growth modes at $x/D = 117$. %, where the LES with the proposed subfilter PDF predict relatively smaller maximum volume fraction source terms for oxidation.

%% The soot volume fraction is a measure of the total volume of soot relative to the total volume of gas. The amount of soot that is present depends on the rates of nucleation, condensation, surface growth, and oxidation. An underprediction of the volume fraction suggests that soot growth through nucleation, condensation, and surface growth is insufficient, or that soot oxidation is overwhelmingly excessive. Expressions for the filtered moment source terms of these modes are similar in form to \cref{eq:lesmodels:presumedpdf:separated}. For example, the filtered moment source term for oxidation is given by
%% \begin{equation}\label{eq:subfilter:leszussp:ox}
%%   \begin{split}
%%     \mean{\dot{M}}_{x,y}^{ox} &= \iint k_{ox}(Z)\mc{M}_j P(\mc{M}_j)\pz dZ d\mc{M}_j \\
%%     &= \hat{k}_{ox}\mean{M}_j,
%%   \end{split}
%% \end{equation}
%% where $k_{ox}(Z)$ is the oxidation coefficient, $\pz$ is modeled with a beta distribution as in \cref{eq:lesmodels:presumedpdf:trimarg}, and the soot subfilter PDF $P(\mc{M}_j)$ is provided in \cref{eq:subfilter:zussp:dd}. Contributions from the thermochemical and soot variables are made distinct in the second line, where $\hat{k}_{ox} = \int k_{ox}(Z)\pz dZ$ and $\mean{M}_j = \int \mc{M}_j P(\mc{M}_j)d\mc{M}_j$. \cref{eq:subfilter:leszussp:ox} incorporates the joint PDF simplification of Mueller and Pitsch~\cite{subfilterpdf2011}, which postulates that the soot scalars should be independent of the thermochemical variables due to the long timescales of PAH and soot formation relative to the timescales of the highly exothermic combustion chemistry. However, they also noted that such an assumption could be violated when surface growth near the flame is the dominant growth mechanism or during the oxidation of soot, when interactions with gas-phase chemistry are enhanced. The likelihood of the former is smaller, as the DNS studies of Bisetti \etal~\cite{bisetti2012} demonstrated that soot growth through PAH-based nucleation and condensation dominates acetylene-based surface growth in turbulent nonpremixed combustion. Therefore, this work hypothesizes that the excessive oxidation of soot, due to the simplification of the joint subfilter PDF (\cref{eq:lesmodels:presumedpdf:bayes}), is the reason for the dramatic underprediction of the soot volume fraction as shown in \cref{fig:subfilter:leszussp:zusspleseval}.

%% As a result of the decorrelation of the soot scalars from the thermochemical variables, the current soot subfilter PDF implictly assumes that soot is uniformly distributed in mixture fraction space. However, for non-smoking flames, this is qualitatively incorrect as there should be zero soot in fuel-lean regions of the flame. This is evident in \cref{fig:subfilter:leszussp:ysvsz}, reproduced from the 3D DNS of a nitrogen-diluted, \textit{n}-heptane/air turbulent nonpremixed planar jet flame~\cite{attili2014}. The lack of soot at mixture fractions below stoichiometric can be explained by the preferential transport of soot generated in the region $0.3 < Z < 0.5$ to richer mixture fractions by turbulent fluctuations and by the oxidation of all soot during transport towards the stoichiometric surface. 

%% %% \begin{figure}[htb]
%% %%   \centering
%% %%   \includegraphics[width=0.7\linewidth]{ch-subfiltermodeling/figures/dns_Ysoot_vs_Z}
%% %%   \caption[DNS of Turbulent Nonpremixed \ce{C7H16}/\ce{N2} Jet Flame, \texorpdfstring{$\langle Y_{\text{s}}|Z \rangle$}{<Ys|Z>} vs. \texorpdfstring{$Z$}{Z}]{Mean soot mass fraction conditioned on mixture fraction at various times in a 3D DNS of an \textit{n}-heptane/nitrogen [15/85 by volume] and air turbulent nonpremixed planar jet flame, reproduced from Attili \etal~\cite{attili2014}. The stoichiometric mixture fraction ($Z_{st} = 0.147$) is demarcated with the vertical dashed line. \textit{Left} - 1 ms (filled squares), 2 ms (crosses), 3 ms (open squares), 4 ms (circles), and 5 ms (triangles). \textit{Right} - 6 ms (stars), 8 ms (circles), 10 ms (open squares), and 20 ms (filled squares). The small gray dots represent the soot mass fraction field at 20 ms.}
%% %%   \label{fig:subfilter:leszussp:ysvsz}
%% %% \end{figure}

%% The non-uniform nature of soot in mixture fraction space is captured by the flamelet solutions that are accessed during LES as well. Filtered moment source term coefficients for oxidation, surface growth, and the combination of nucleation and condensation are plotted in \cref{fig:subfilter:leszussp:kvsz}. It is clear that different soot evolution modes are dominant over distinct regions of mixture fraction even as the fuel mixture and stoichiometric scalar dissipation rate are varied. Soot growth through PAH-based nucleation and condensation prevails at very rich values of mixture fraction, whereas acetylene-based surface growth is more dominant at moderately rich mixture fractions. It is notable that for a fixed fuel mixture (middle and right plots), a reduction in the value of stoichiometric scalar dissipation rate induces a large increase in the coefficient for nucleation and condensation while the coefficient for surface growth is lessened. This trend is due to the increasing effectiveness of PAH chemistry at smaller values of scalar dissipation rate. Therefore, PAH-based soot growth is supported at leaner conditions, which ultimately supplants acetylene-based surface growth.

%% %% \begin{figure}[htb]
%% %%   \centering
%% %%   \includegraphics[width=\linewidth]{ch-subfiltermodeling/figures/flamelet_sootcoeffs_le1}
%% %%   \caption[Soot Growth and Oxidation Coefficients, 1/\texorpdfstring{$\tau$}{t} vs. \texorpdfstring{$Z$}{Z}]{Flamelet calculations of soot growth and oxidation coefficients for filtered moment source terms as a function of mixture fraction. The blue lines are the oxidation coefficients, the red lines represent the surface growth coefficients, and the green lines are the coefficients for nucleation and condensation. \textit{Left} - Fuel mixture \textit{n}-heptane/nitrogen [15/85 by volume] used in the DNS of \cref{fig:subfilter:zussp:chisensitivity,fig:subfilter:leszussp:ysvsz} at $\chi_{st} = 10\ s^{-1}$. \textit{Middle} \& \textit{Right} - Fuel mixture \ce{C2H4}/\ce{H2}/\ce{N2} [40/41/19 by volume] used in the LES of \cref{fig:subfilter:leszussp:zusspleseval} at $\chi_{st} = 10\ s^{-1}$ and $\chi_{st} = 0.1\ s^{-1}$, respectively.}
%% %%   \label{fig:subfilter:leszussp:kvsz}
%% %% \end{figure}

%% Soot oxidation is the dominant mode at mixture fractions below and slightly above the stoichiometric value. Since the rates associated with the high-temperature oxidation chemistry are comparable with those of the main heat-releasing chemistry~\cite{guo2016}, it is anticipated that the magnitude of the peak oxidation coefficient will not vary much as the fuel mixture or stoichiometric scalar dissipation rate is modified. Indeed, this phenomenon is evident in \cref{fig:subfilter:leszussp:kvsz}. Additionally, the oxidation coefficient at stoichiometric mixture fraction is at least an order of magnitude larger than the maximum value of the surface growth coefficient. Thus, the complete oxidation of soot by \ce{OH} (and \ce{O2} to a lesser degree) is expected during transport towards stoichiometric regions. However, the soot subfilter PDF given by \cref{eq:subfilter:zussp:dd} permits the existence of soot in fuel-lean areas, for it assumes that soot is uniformly distributed in mixture fraction space. Therefore, the presence of large, non-zero oxidation coefficients at lean values of mixture fraction is concerning, for it potentially leads to substantial filtered moment source terms (\cref{eq:subfilter:leszussp:ox}). This contribution to the oxidation rate is artificial and could be an explanation for the underpredicted soot volume fraction in \cref{fig:subfilter:leszussp:zusspleseval}. To prevent this non-physical oxidation from occurring, the subfilter PDF must depend on the thermochemical variables to exclude the presence of soot at lean mixtures $Z < Z_{st}$. A soot subfilter PDF that addresses this point is introduced in the following section.

%% Perhaps have a comparison of flamelets first to give hints at what to expect in terms of $Y_{PAH}$ and $f_V$. Include comparison of EHN, Stanford, $\check{Le}_i(\zeta_i)$, and ZUSSP vs. ZASSP.

%% Talk about temperature, $Y_{PAH}$, $f_V$, conditional radial statistics at various distances downstream, etc.
