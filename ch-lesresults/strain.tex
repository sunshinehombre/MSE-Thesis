\section{Trends With Global Strain Rate}
\label{sec:lesresults:strain}

LES with the $Z$-Activated Soot Subfilter PDF and the Strain-Sensitive Transport Approach have also been performed for flames $1/\tau|_H$ and $1/\tau|_L$ from \cref{tab:subfilter:leszussp:ehn}. In \cref{fig:lesresults:strain:allflameslecomparison}, these are compared to the LES that model all gas-phase species with equal effective diffusivities. It can be observed that the choice of transport model does not significantly impact the predictions of the flame temperature, even as the exit strain rate is varied. The gas-phase species that participate in the main heat-releasing reactions, such as \ce{CO} or \ce{CO2}, have fast chemical production rates compared to the rates of local mixing. In fact, their production rates easily adjust to the global strain rate, resulting in similar experimental peak temperatures among the three flames. These species are not identified as strain-sensitive according to \cref{eq:transport:ssta:framework:ssp}, so they are modeled with unity effective Lewis numbers in the Strain-Sensitive Transport Approach. As a result, the corresponding profiles closely follow those of the LES with unity effective Lewis numbers. Both transport models closely match the flame temperatures from the experimental data, suggesting that the diffusive transport of the aforementioned species is not important to the dynamics of the main heat-releasing reactions. Similar conclusions were reached in previous studies~\cite{attili2015,attili2016}.

\begin{figure}[htb]
  \centering
  \includegraphics[width=\linewidth]{ch-lesresults/figures/combined-Tfv-S-ZASSP-Le_comparison}
  \caption[Centerline \texorpdfstring{$\langle T \rangle$}{<T>} \& \texorpdfstring{$\langle f_V \rangle$}{<fV>} from LES of Flames \texorpdfstring{$1/\tau|_H$}{1/t|H}, \texorpdfstring{$1/\tau|_M$}{1/t|M}, and \texorpdfstring{$1/\tau|_L$}{1/t|L} with Various Transport Approaches]{\textit{Left to right} - Time-averaged flame temperatures and soot volume fractions from LES of jet flames $1/\tau|_H$, $1/\tau|_M$, and $1/\tau|_L$, respectively. Note that the soot volume fraction profiles from LES have been premultiplied by a factor of 10 for visibility. Circles indicate experimental data, solid lines are LES with unity effective Lewis numbers for gas-phase species, and dashed lines are LES with the Strain-Sensitive Transport Approach. Colors of profiles are the same as in \cref{fig:lesresults:ssta:f2lecomparison}.}
  \label{fig:lesresults:strain:allflameslecomparison}
\end{figure}

On the other hand, the soot volume fraction is significantly influenced by the global strain rate. The experimental data reveals an inverse relationship between the global strain rate and the centerline soot volume fraction, where jet flame $1/\tau|_H$ has the smallest maximum value of 197 ppb at $x/D = 103$ and jet flame $1/\tau|_L$ has the largest maximum value of 444 ppb at $x/D = 83$. Jet flame $1/\tau|_M$ has an exit strain rate that is 1.7 times larger than that of flame $1/\tau|_L$, yet its peak volume fraction is only a slightly diminished 437 ppb at $x/D = 105$. Mahmoud \etal~\cite{mahmoud2017} have attributed this nonlinearity to the secondary influence of buoyancy, which plays a larger role for the latter two flames. Nonetheless, previous studies have observed this inverse trend with strain rate in both laminar~\cite{decroix2000,huijnen2010,wang2016433} and turbulent~\cite{kent1984,mahmoud2017} nonpremixed flames.

In \cref{fig:lesresults:strain:fvvsstrain}, the LES with strain-sensitive transport and equal effective diffusivities transport generally capture this behavior with respect to the global strain rate, although both still underpredict the normalized maximum centerline volume fraction for jet flame $1/\tau|_M$. Additionally, it is clear in \cref{fig:lesresults:strain:fvvsstrain,fig:lesresults:strain:allflameslecomparison} that the difference between the two approaches is relatively small for jet flame $1/\tau|_H$. At large values of global strain rate, PAH are much more spatially intermittent due to the enhanced rates of turbulent mixing. The overall amount of PAH is reduced, so PAH-based soot growth modes have diminished contributions to the overall volume fraction. Therefore, the proposed model's improvement in the prediction of the soot volume fraction over the unity effective Lewis number approach is expected to be limited.

% The former LES tends to capture the trend better than the latter, although both still underpredict the normalized maximum centerline volume fraction for jet flame $1/\tau|_M$.

\begin{figure}[htb]
  \centering
  \includegraphics[width=0.45\linewidth]{ch-lesresults/figures/fv-vs-tau-S-ZASSP-Le_combined}
  \caption[Normalized Maximum \texorpdfstring{$\langle f_V \rangle$}{<fV>} Versus Exit Strain Rate]{Time-averaged maximum soot volume fractions normalized by the maximum value from from jet flame $1/\tau|_L$ as a function of exit strain rate. Circles indicate experimental data, solid lines with triangles are from LES with unity effective Lewis numbers, and solid lines with squares are from LES with the Strain-Sensitive Transport Approach.}
  \label{fig:lesresults:strain:fvvsstrain}
\end{figure}

On the contrary, PAH are more prevalent at smaller global strain rates, so accurately modeling their transport is critical to predicting the evolution of soot. In \cref{fig:lesresults:strain:allflamesdfvdt}, significant differences exist between the two transport models regarding the source terms for nucleation, condensation, and surface growth. As the global strain rate is decreased, it can be observed that the source term due to combined nucleation and condensation grows in importance further downstream, especially for the LES with the Strain-Sensitive Transport Approach. At the same time, the source term due to surface growth becomes less dominant at upstream locations. In the LES of jet flame $1/\tau|_L$ with the proposed model, the maximum value of the source term due to nucleation and condensation is 1.67 times the peak value of the source term due to surface growth, highlighting the importance of properly modeling the physics of PAH. Conversely, this factor is only 1.25 in the LES with unity effective Lewis numbers. At the elevated global strain rate of jet flame $1/\tau|_M$, the maximum value of the PAH-based source term is 1.14 times larger than that of the source term due to surface growth in the LES with the Strain-Sensitive Transport Approach. Meanwhile, the former is actually smaller than the latter in the LES with unity effective Lewis numbers, reiterating the point that this model is not appropriate for strain-sensitive species such as PAH.

% Conversely, the maximum value of the source term due to surface growth always remains larger than that of the source term due to condensation in the LES with unity effective Lewis numbers. However, this model is not appropriate for strain-sensitive species such as PAH.

\begin{figure}[htb]
  \centering
  \includegraphics[width=\linewidth]{ch-lesresults/figures/combined-dfvdt-S-ZASSP-Le_comparison-nc-combined}
  \caption[Centerline \texorpdfstring{$\langle df_V/dt \rangle$}{<dfV/dt>} from LES of Flames \texorpdfstring{$1/\tau|_H$}{1/t|H}, \texorpdfstring{$1/\tau|_M$}{1/t|M}, and \texorpdfstring{$1/\tau|_L$}{1/t|L} with Various Transport Approaches]{\textit{Left to right} - Time-averaged soot volume fraction source terms from LES of jet flames $1/\tau|_H$, $1/\tau|_M$, and $1/\tau|_L$, respectively. Solid lines are LES with unity effective Lewis numbers for gas-phase species and dashed lines are LES with the Strain-Sensitive Transport Approach. Colors of profiles are the same as in \cref{fig:lesresults:ssta:radialdfvdtf2lecomparison}.}
  \label{fig:lesresults:strain:allflamesdfvdt}
\end{figure}

It is a valid model for species with relatively fast chemistry. It is evident in \cref{fig:lesresults:strain:allflamesdfvdt} that the volume fraction source term for oxidation is fairly insensitive to the global strain rate for both transport models. Soot oxidation by \ce{OH} and \ce{O2} occurs relatively quickly, so this process can adapt to elevated levels of turbulent mixing. As confirmed in \cref{fig:transport:ssta:dependencies:chist}, these species should be modeled with unity effective Lewis numbers. % It is also notable that the volume fraction source term due to nucleation is insensitive to the strain rate. For both transport models, the maximum value of the latter hardly varies, although the axial location of the peak slightly changes as the strain rate is decreased. This behavior agrees with the findings of previous studies, where the production of the smallest, nascent soot particles has been observed to be minimally affected by the global strain rate~\cite{huijnen2010}.
