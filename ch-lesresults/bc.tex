\section{Flow Field Boundary Conditions}
\label{sec:lesresults:bc}

As discussed in \cref{sec:lesresults:comput}, additional LES of three-dimensional, fully developed turbulent periodic pipe flows are required to generate the flow field conditions within each burner. These flow field conditions specify the fuel-side inflow boundary condition of each turbulent nonpremixed jet, so the quality of their enforcement must be assessed through comparison against the experimental data. 

Radial profiles of the streamwise velocity component and the root-mean-square counterpart are shown in \cref{fig:lesresults:bc:uurms} for each case in \cref{tab:lesresults:comput:details}. Each profile is normalized by the mean jet exit velocity of the $1/\tau|_H$ flame. In general, the agreement between LES and the experiments is good. The normalized streamwise velocity component of jet flames $1/\tau|_M$ and $1/\tau|_L$ from LES closely replicates the experimental data within the burner away from the wall and in the coflow. However, the velocity is underpredicted around the near-wall region at $r/D = 0.5$, suggesting that there is less mixing between the fuel stream and air coflow in the simulations. On the other hand, the agreement between LES and the experiments is good in the near-wall region and in the coflow of the flame with the largest exit strain rate ($1/\tau|_H$), but is worse in the region near the centerline. The profile from LES underpredicts the experimental values for $r/D < 0.34$, and is ``flatter'' than the measured profile. The latter observation suggests that the turbulent flow inside the burner may not be fully developed in the experiment.
% This reduced amount of mixing may be a result of the grid resolution at the wall, so additional studies are needed to evaluate the impact of refining the grid near the wall. 

The bottom plots of \cref{fig:lesresults:bc:uurms} contain the normalized Root-Mean-Square (RMS) streamwise velocity profiles as a function of the normalized radius. It can be observed that LES follows the general trends of the experiments, where the normalized RMS velocities are zero at large $r/D$ and reach their maximum values just before $r/D = 0.5$, at which point they steadily decline as $r/D$ approaches zero. However, LES generally underpredicts the normalized RMS velocities near $r/D = 25$ and overpredicts the maximum values, the extent of which increases with the exit strain rate. These deviations are expected to diminish as the grid resolution is refined, especially in the near-wall region. Ultimately, the discrepancies in the fuel-side flow field boundary condition do not affect the downstream results significantly.

\begin{figure}[htb]
  \centering
  \includegraphics[width=\linewidth]{ch-lesresults/figures/U-URMS}
  \caption[Fuel-Side Flow Field Boundary Conditions]{\textit{Top} - Radial profiles of the normalized streamwise velocity component at 2 mm above the burner. \textit{Bottom} - Radial profiles of the normalized streamwise root-mean-square velocity component at 2 mm above the burner. \textit{Left to right} - Jet flames $1/\tau|_H$, $1/\tau|_M$, and $1/\tau|_L$, respectively. Circles indicate the experimental data of Mahmoud \etal~\cite{mahmoud2017}, while solid lines are time-averaged values from LES.
  \label{fig:lesresults:bc:uurms}
\end{figure}
