\section{Experimental Framework}
\label{sec:lesresults:exp}

A series of turbulent nonpremixed ethylene-hydrogen-nitrogen [40/41/19 by volume] jet flames~\cite{mahmoud2017} is used to validate the $Z$-Activated Soot Subfilter PDF of \cref{sec:subfilter:zassp} and the Strain-Sensitive Transport Approach of \cref{sec:transport:ssta}. In this series, three attached flames at atmospheric pressure are maintained at a jet exit Reynolds number of 15,000 while the exit strain rate is varied by altering the ratio of the fuel exit velocity to the fuel jet diameter. Surrounding each burner is a contraction with a square cross-section of dimensions 150 mm by 150 mm, from which a uniform coflow of air at a velocity of 1.1 m/s is emitted. Mahmoud \etal~\cite{mahmoud2017} selected ethylene (\ce{C2H4}) for the fuel due to its relatively well-studied chemical kinetics and high soot yield and included hydrogen (\ce{H2}) in the mixture to ensure the attachment of each flame to the burner. Nitrogen (\ce{N2}) was added to the fuel mixture in order to reduce the amount of soot produced to enable accurate measurements. \cref{tab:subfilter:leszussp:ehn} summarizes other key aspects of the experimental setup, and complete details can be obtained from Mahmoud \etal~\cite{mahmoud2017}.

\begin{table}[htbp]
\centering
\caption[Flow Conditions of Turbulent Nonpremixed \ce{C2H4}/\ce{H2}/\ce{N2} Jet Flames]{Flow conditions of turbulent nonpremixed \ce{C2H4}/\ce{H2}/\ce{N2} jet flames~\cite{mahmoud2017}}
\label{tab:subfilter:leszussp:ehn}
\begin{tabular}{p{0.35\textwidth} p{0.1\textwidth} p{0.12\textwidth} p{0.12\textwidth} p{0.12\textwidth}}
\toprule
\textbf{Flame} & & \bm{$1/\tau|_{H}$} & \bm{$1/\tau|_{M}$} & \bm{$1/\tau|_{L}$} \\
\midrule

Central jet diameter, $D$
& [mm] & 4.4 & 5.8 & 8.0 \\[0.2em]

Mean jet exit velocity, $U$
& [m/s] & 56.8 & 42.4 & 31.5 \\[0.2em]

Exit strain rate, $U/D$
& [s$^{-1}$] & 12,900 & 7300 & 4300 \\[0.2em]

Exit Reynolds number, $Re_D$
& [--] & 15,000 & 15,000 & 15,000 \\[0.2em]

Mean flame length, $L_f$
& [mm] & 710 & 930 & 1060 \\

\bottomrule
\end{tabular}
\end{table}

In \cref{tab:subfilter:leszussp:ehn}, the flame length, $L_f$, was estimated from five instantaneous photographs of the flames. It is a measure of the distance from the jet exit to the most visible downstream portion of the flame and is repeatable to within 5\% uncertainty~\cite{mahmoud2017}. Measurements of flame temperature and soot volume fraction were taken from thermocouple readings at the centerline and from laser-induced incandescence (LII) signals, respectively. LII was performed with a pulsed Nd:YAG laser at 1064 nm, which emitted a laser sheet with a height of 24 mm that was aligned vertically with the burner axis and transmitted horizontally through the flame. To minimize the effects of fluence variation from attenuation and beam steering, the laser fluence was maintained above 0.5 J/cm$^2$, and statistical data were sampled only from the laser-in half of the flame~\cite{mahmoud2017,sun2015}.

The incandescence was filtered at 430$\pm$10 nm and collected with an intensified CCD camera. The CCD camera's gate-width was set to 50 ns in order to decrease the sensitivity of the signal to the soot particle size~\cite{vanderwal1996,mahmoud2017}. Conversion of the LII signal to the soot volume fraction was accomplished by calibrating against a flat premixed \ce{C2H4}/air flame on a McKenna burner. The volume fraction detection threshold for the collected data is 1 part per billion (ppb), and the uncertainty of the measurements is 25\%.

% Discuss Adelaide flame series details.
