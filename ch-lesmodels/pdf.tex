\section{Small-Scale Quantities}
\label{sec:lesmodels:presumedpdf}

LES is a technique where large-scale, unsteady turbulent motions are computed explicitly while small-scale motions are modeled. The separation of scales is achieved through filtering the velocity and scalar fields (generically represented by $Q(x_j,t)$) to decompose them into the sum of a resolved component $\mean{Q}(x_j,t)$ and a subfilter-scale component $Q(x_j,t)-\mean{Q}(x_j,t)$. A general filtering operation can be expressed as
\begin{equation}\label{eq:lesmodels:presumedpdf:filter}
  \mean{Q}(x_j,t) = \int F(r_j,x_j)Q(x_j - r_j, t)dr_j,
\end{equation}
where integration is over the entire domain. The filter kernel $F(r_j,x_j)$ satisfies the normalization condition
\begin{equation}\label{eq:lesmodels:presumedpdf:kernel}
  \int F(r_j,x_j)dr_j = 1,
\end{equation}
and uses cutoff length and time scales to separate unresolved small-scale quantities from $\mean{Q}(x_j,t)$. In combustion, variations in density are non-zero. Therefore, it is convenient to use density-weighted filtering (Favre filtering) in LES of turbulent combustion, where the resolved components are expressed as $\tf{Q}(x_j,t) = \mean{\rho(x_j,t)Q(x_j,t)}/\mean{\rho(x_j,t)}$.

When these filtering operations are applied to the conservation equations for mass, momentum, and energy or to scalar transport equations, subfilter contributions to the chemical source term, turbulent transport, and molecular transport terms arise. For instance, the density-weighted filtered transport equation for a scalar $Q$ is given by
\begin{equation}\label{eq:lesmodels:presumedpdf:scalar}
  \ftrder{\tf{Q}} = -\dsff[\tf{Q}]{\tf{u_j Q}} + \pder[]{x_j}\left( \mean{\rho}\tf{D_Q\pder[Q]{x_j}} \right) + \fst[m]{Q},
\end{equation}
where the first term on the right-hand side is known as the subfilter scalar flux and contains the effects of unresolved turbulent transport, the second term contains subfilter contributions to molecular diffusion fluxes, and the third term is the filtered source term. These unclosed terms represent small-scale interactions between soot, turbulence, and combustion chemistry. In particular, the filtered source term is usually closed with a density-weighted joint PDF $\tf{P}$ of the thermochemical and soot variables. Density-weighted filtered functions of these quantities are obtained from a convolution of the equation of state with $\tf{P}$:
\begin{equation}\label{eq:lesmodels:presumedpdf:filteredfuncs}
  \tf{\phi}(\xi,\mc{M}_j) = \iint \mc{J}(\xi,\mc{M}_j)\tf{P}(\xi,\mc{M}_j) d\xi d\mc{M}_j,
\end{equation}
where $\xi$ is the vector of thermochemical variables and $\mc{M}_j$ is the state vector encompassing the moments $M_{x,y}$ and the weight of the delta function $N_0$ allocated for nucleating particles.

The joint subfilter PDF $\tf{P}$ is unknown and must be modeled using transported PDF or presumed PDF methods. In the transported PDF approach, a transport equation is solved for the subfilter PDF~\cite{pope1985,pope1991}. This equation is high-dimensional, so conventional discretization techniques are computationally intractable. Instead, Lagrangian particle methods are employed, where a set of notional particles that track fluid particles is solved for. The advantage of the class of transported PDF methods is that all local phenomena are naturally described by the subfilter PDF, which provides information about subfilter fluctuations and correlations at a single point. Therefore, chemical source terms do not require additional closure because chemical reactions are one-point phenomena. However, processes such as molecular mixing are dependent on gradients and are inherently two-point, non-local phenomena. Mixing models such as the Interaction with Exchange of the Mean (IEM)~\cite{dopazo1974} and Modified Curl~\cite{janicka1979} have been proposed, where quantities are relaxed to local filtered values to mimic homogenization through diffusion. However, such approaches fail to use information about localness in composition space. More complex mixing models using the concept of the Euclidean minimum spanning tree~\cite{subramaniam1998} or the shadow position~\cite{pope2013} introduce localness in composition space but still suffer from the large computational cost associated with evolving the Lagrangian particles.

% In contrast, presumed PDF methods assume a form of the joint subfilter PDF. Reasoning based on physical insights about subfilter interactions between the scalars and turbulence is utilized to deduce the form of the PDF. Previous works have indicated that using the beta distribution for the mixture fraction and conditional delta distributions for the progress variable and heat loss parameter is a good model for the thermochemical portion of the subfilter PDF when considering passive scalars in a two-feed system~\cite{cook1994,jimenez1997,wall2000,ihme2008}. Ihme \etal~\cite{ihme2005} arrived at the same conclusion with the FPV approach for reactive scalars. Since presumed PDF methods do not rely on mixing models, they generally match or even exceed the performance of transported PDF methods, provided that the form of the PDF is appropriate. Therefore, the presumed PDF approach will be adopted in this work.

In contrast, presumed PDF methods assume a form of the joint subfilter PDF. Reasoning based on physical insights about subfilter interactions between the scalars and turbulence is utilized to deduce the form of the PDF. Since presumed PDF methods do not rely on mixing models, they generally match or even exceed the performance of transported PDF methods, provided that the form of the PDF is appropriate. Therefore, the presumed PDF approach will be adopted in this work. An approximate form of the overall joint subfilter PDF is determined by first decomposing $\tf{P}$ into a thermochemical PDF and a PDF of the soot scalars conditioned on the thermochemical variables. By Bayes' theorem,
\begin{equation}\label{eq:lesmodels:presumedpdf:bayes}
  \tf{P}(\xi,\mc{M}_j) = \tf{P}(\xi)P(\mc{M}_j|\xi),
\end{equation}
where $\tf{P}(\xi)$ includes a beta distribution for the mixture fraction and $P(\mc{M}_j|\xi)$ is unknown. Further discussion on the complete forms of $\tf{P}(\xi)$ and $P(\mc{M}_j|\xi)$ will be deferred until \cref{ch:subfilter}.

In practical implementation, each flamelet solution in the database is convoluted with a beta distribution for the mixture fraction and tabulated as a function of the filtered mixture fraction, subfilter mixture fraction variance, filtered progress variable, and filtered heat loss parameter. The transport equation for the filtered mixture fraction is derived by spatially filtering \cref{eq:lesmodels:combust:map:z} and is given by
\begin{equation}\label{eq:lesmodels:presumedpdf:filteredz}
  \ftrder{\tf{Z}} = -\dsff[\tf{Z}]{\tf{u_j Z}} + \fdt[Z]{\tf{Z}} + \fst[m]{Z},
\end{equation}
where the first and last terms on the right-hand side are unclosed. The former is the subfilter flux, which can be modeled with a dynamic procedure~\cite{germano1991,lilly1992,moin1991}. The filtered source term is closed through the presumed subfilter PDF given in \cref{eq:lesmodels:presumedpdf:trimarg}. Similarly, the transport equations for the filtered progress variable and filtered heat loss parameter are given by
\begin{equation}\label{eq:lesmodels:presumedpdf:filteredc}
  \ftrder{\tf{C}} = -\dsff[\tf{C}]{\tf{u_j C}} + \fdt[C]{\tf{C}} + \mean{\left( \frac{\dot{m}_{C_{\text{conv}}}}{C^*} \right)}
\end{equation}
and
\begin{equation}\label{eq:lesmodels:presumedpdf:filteredh}
  \ftrder{\tf{H}} = -\dsff[\tf{H}]{\tf{u_j H}} + \fdt[H]{\tf{H}} + \mean{\dot{\rho}H} + \fst[q]{\text{RAD}},
\end{equation}
respectively. The remaining database parameter is the subfilter mixture fraction variance, which will be procured in the manner of Mueller and Pitsch~\cite{mueller2012}. Rather than solving for the subfilter variance directly, a transport equation for the filtered square of the mixture fraction is evaluated:
\begin{equation}\label{eq:lesmodels:presumedpdf:filteredzsq}
  \begin{split}
    \ftrder{\tf{Z^2}} &= -\dsff[\tf{Z^2}]{\tf{u_j Z^2}} + \fdt[Z]{\tf{Z^2}} \\
    &- 2\mean{\rho}\tf{D}_Z \pder[\tf{Z}]{x_j}\pder[\tf{Z}]{x_j} - \mean{\rho}\tf{\chi}_{\text{sgs}} - \mean{\dot{\rho}Z^2} + 2\mean{\dot{m}_Z Z},
  \end{split}
\end{equation}
where $\tf{\chi}_{\text{sgs}}$ is the subfilter scalar dissipation rate. This quantity is estimated with a linear relaxation model~\cite{ihme200890}:
\begin{equation}\label{eq:lesmodels:presumedpdf:chisgs}
  \tf{\chi}_{\text{sgs}} = \text{C}_{\chi}\frac{\nu_t}{\Delta^2}\tf{Z_{\text{V}}},
\end{equation}
where $\text{C}_{\chi} \approx 20$, the mechanical timescale is determined from the eddy viscosity $\nu_t$ and filter width $\Delta$, and $\tf{Z_{\text{V}}} = \tf{Z^2} - \tf{Z}^2$ is the subfilter mixture fraction variance.

