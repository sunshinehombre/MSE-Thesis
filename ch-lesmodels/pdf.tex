\section{Presumed PDF Approach}
\label{sec:lesmodels:presumedpdf}

LES only resolves the larger features of a turbulent flow, so small-scale interactions between soot, turbulence, and combustion chemistry need to be modeled. These subfilter interactions are modeled with a density-weighted joint PDF $\tf{P}$ of the thermochemical and soot variables. Density-weighted filtered (Favre filtered) functions of these quantities are obtained from a convolution of the equation of state with $\tf{P}$:
\begin{equation}\label{eq:lesmodels:presumedpdf:filteredfuncs}
  \tf{\phi}(\xi,\mc{M}_j) = \iint \mc{J}(\xi,\mc{M}_j)\tf{P}(\xi,\mc{M}_j) d\xi d\mc{M}_j,
\end{equation}
where $\xi$ is the vector of thermochemical variables and $\mc{M}_j$ is the state vector encompassing the moments $M_{x,y}$ and the weight of the delta function $N_0$ allocated for nucleating particles.

The joint subfilter PDF $\tf{P}$ is unknown and must be modeled using transported PDF or presumed PDF methods. In the transported PDF approach, a transport equation is solved for the subfilter PDF~\cite{pope1985,pope1991}. Often this equation is high-dimensional, and therefore conventional Eulerian discretization techniques are computationally intractable. Instead, Lagrangian particle methods are employed, where a set of notional particles that track fluid particles is solved for. The advantage of the class of transported PDF methods is that all local phenomena are naturally described by the subfilter PDF, which provides information about subfilter fluctuations and correlations at a single point. Thus, chemical source terms do not require additional closure because chemical reactions are one-point phenomena. However, processes such as molecular mixing are dependent on gradients and therefore are inherently two-point, non-local phenomena. Mixing models have been proposed such as the Interaction with Exchange of the Mean (IEM)~\cite{dopazo1974} and Modified Curl~\cite{janicka1979}, where quantities are relaxed to local filtered values to mimic homogenization through diffusion. However, such crude approaches fail to use information about localness in composition space. More complex mixing models using the concept of the Euclidean minimum spanning tree~\cite{subramaniam1998} or the shadow position~\cite{pope2013} introduce localness in composition space, but still suffer from the large computational cost associated with evolving the Lagrangian particles.

In contrast, presumed PDF methods assume a form of the joint subfilter PDF. Reasoning based on physical insights about subfilter interactions between the scalars and turbulence is utilized to deduce the form of the PDF. Previous works have indicated that the beta distribution is a good model for the thermochemical portion of the subfilter PDF when considering passive scalars in a two-feed system~\cite{cook1994,jimenez1997,wall2000}. Ihme et al.~\cite{ihme2005} arrived at the same conclusion with the FPV approach for reactive scalars. Since presumed PDF methods do not rely on ad hoc mixing models, they generally match or exceed the performance of transported PDF methods, provided that the form of the PDF is appropriate. Thus, the presumed PDF approach will be adopted in this work.

An approximate form of the overall joint subfilter PDF is determined by first decomposing $\tf{P}$ into a thermochemical PDF and a PDF of the soot scalars conditioned on the thermochemical variables. By Bayes' theorem,
\begin{equation}\label{eq:lesmodels:presumedpdf:bayes}
  \tf{P}(\xi,\mc{M}_j) = \tf{P}(\xi)P(\mc{M}_j|\xi),
\end{equation}
where $\tf{P}(\xi)$ includes a beta distribution and $P(\mc{M}_j|\xi)$ is unknown. The form of the latter has been approximated previously through consideration of the dynamics of soot. In DNS studies of a nitrogen-diluted, \textit{n}-heptane/air turbulent nonpremixed jet, PAH were found to have much slower formation chemistry compared to the main heat-releasing chemistry, which is represented by the thermochemical variables $\xi$~\cite{attili2014,bisetti2012}. PAH are soot precursors and therefore soot itself is characterized by even longer timescales. Consequently, Mueller and Pitsch proposed that the timescales of the soot scalars and thermochemical variables are disparate enough such that the former should not depend on the latter~\cite{subfilterpdf2011}. This argument shall be used to simplify the subfilter PDF of the soot scalars conditioned on $\xi$ to a marginal PDF of only the soot scalars. The expression for the spatially Favre filtered functions of \cref{eq:lesmodels:presumedpdf:filteredfuncs} then becomes
\begin{equation}\label{eq:lesmodels:presumedpdf:separated}
  \tf{\phi}(\xi,\mc{M}_j) = \iint \mc{J}(\xi,\mc{M}_j)\tf{P}(\xi)P(\mc{M}_j) d\xi d\mc{M}_j.
\end{equation}
By replacing the functional relation $\mc{J}$ with \cref{eq:lesmodels:combust:producteos}, \cref{eq:lesmodels:presumedpdf:separated} can be further simplified to
\begin{equation}\label{eq:lesmodels:presumedpdf:indep}
  \tf{\phi}(Z,C,H,\mc{M}_j) = \iiint \mc{G}(Z,C,H)\tf{P}(Z,C,H) dHdCdZ \times \int \mc{K}(\mc{M}_j)P(\mc{M}_j) d\mc{M}_j,
\end{equation}
where the thermochemical and soot components are now completely independent. Mueller and Pitsch noted that the time scale separation argument could be violated during the oxidation of soot, when there are enhanced interactions between soot and the major gas-phase chemistry, and during surface growth near the flame. These situations will be examined more closely in \cref{ch:subfilter}.

The challenging task of modeling the joint subfilter PDF has been reduced to developing models for the thermochemical subfilter PDF and the soot subfilter PDF. Discussion of the latter will be deferred to \cref{ch:subfilter}. The thermochemical subfilter PDF for the RFPV approach is obtained from Ihme and Pitsch~\cite{ihme2008}. First, two quantities are introduced to uniquely identify each flamelet solution in the database of thermochemical states: $\Lambda = C(Z_{st})$ and $\Phi = H(Z_{st})$. The thermochemical equation of state \cref{eq:lesmodels:combust:rfpv} then becomes
\begin{equation}\label{eq:lesmodels:presumedpdf:eos}
  \xi = \mc{G}(Z, C, H) = \mg(Z, \Lambda, \Phi).
\end{equation}
The spatially Favre filtered thermochemical functions are given by
\begin{equation}\label{eq:lesmodels:presumedpdf:filteredeos}
  \begin{split}
    \tf{\xi} &= \iiint \mc{G}(Z,C,H)\tf{P}(Z,C,H) dHdCdZ \\
    &= \iiint \mg(Z,\Lambda,\Phi)\tf{P}(Z,\Lambda,\Phi) d\Phi d\Lambda dZ.
  \end{split}
\end{equation}
Since $Z$, $\Lambda$, and $\Phi$ are defined to be independent, the thermochemical subfilter PDF can be expressed as the product of three marginal distributions:
\begin{equation}\label{eq:lesmodels:presumedpdf:trimarg}
  \tf{P}(Z,\Lambda,\Phi) = \beta(Z;\tf{Z},\tf{Z_{\text{V}}})\delta(\Lambda - \tf{\Lambda})\delta(\Phi - \tf{\Phi}),
\end{equation}
where the mixture fraction is modeled with a beta distribution~\cite{cook1994,jimenez1997,wall2000}, the subfilter mixture fraction variance is defined as $\tf{Z_{\text{V}}} = \tf{Z^2} - \tf{Z}^2$, and $\Lambda$ and $\Phi$ are modeled with delta distributions~\cite{ihme2008}. Assuming \cref{eq:lesmodels:combust:rfpv} is unique, a bijective inversion may be used to conveniently reexpress a dependence on $\tf{\Lambda}$ and $\tf{\Phi}$ as a dependence on $\tf{C}$ and $\tf{H}$. In practical implementation, each flamelet solution in the database is convoluted with the beta distribution for the mixture fraction and tabulated as a function of the filtered mixture fraction, subfilter mixture fraction variance, filtered progress variable, and filtered heat loss parameter.

The transport equation for the filtered mixture fraction is derived by spatially filtering \cref{eq:lesmodels:combust:map:z} and is given by
\begin{equation}\label{eq:lesmodels:presumedpdf:filteredz}
  \ftrder{\tf{Z}} = \dsff[\tf{Z}]{\tf{u_j Z}} + \fdt[Z]{\tf{Z}} + \fst[m]{Z},
\end{equation}
where the first and last terms on the right-hand side are unclosed. The former is the subfilter flux, which can be modeled with a dynamic procedure~\cite{germano1991,lilly1992,moin1991}. The filtered source term is closed through the presumed subfilter PDF given in \cref{eq:lesmodels:presumedpdf:trimarg}. Similarly, the transport equations for the filtered progress variable and filtered heat loss parameter are given by
\begin{equation}\label{eq:lesmodels:presumedpdf:filteredc}
  \ftrder{\tf{C}} = \dsff[\tf{C}]{\tf{u_j C}} + \fdt[C]{\tf{C}} + \mean{\left( \frac{\dot{m}_{C_{\text{conv}}}}{C^*} \right)}
\end{equation}
and
\begin{equation}\label{eq:lesmodels:presumedpdf:filteredh}
  \ftrder{\tf{H}} = \dsff[\tf{H}]{\tf{u_j H}} + \fdt[H]{\tf{H}} + \mean{\dot{\rho}H} + \fst[q]{\text{RAD}},
\end{equation}
respectively. The remaining database parameter is the subfilter mixture fraction variance, which will be procured in the manner of Mueller and Pitsch~\cite{mueller2012}. Rather than solving for the subfilter variance directly, a transport equation for the filtered square of the mixture fraction is evaluated:
\begin{equation}\label{eq:lesmodels:presumedpdf:filteredzsq}
  \begin{split}
    \ftrder{\tf{Z^2}} &= \dsff[\tf{Z^2}]{\tf{u_j Z^2}} + \fdt[Z]{\tf{Z^2}} \\
    &- 2\mean{\rho}\tf{D}_Z \pder[\tf{Z}]{x_j}\pder[\tf{Z}]{x_j} - \mean{\rho}\tf{\chi}_{\text{sgs}} - \mean{\dot{\rho}Z^2} + 2\mean{\dot{m}_Z Z},
  \end{split}
\end{equation}
where $\tf{\chi}_{\text{sgs}}$ is the subfilter scalar dissipation rate. This quantity is estimated with a linear relaxation model~\cite{ihme200890}:
\begin{equation}\label{eq:lesmodels:presumedpdf:chisgs}
  \tf{\chi}_{\text{sgs}} = \text{C}_{\chi}\frac{\nu_t}{\Delta^2}\tf{Z_{\text{V}}},
\end{equation}
where $\text{C}_{\chi} \approx 20$, the mechanical timescale is determined from the eddy viscosity $\nu_t$ and filter width $\Delta$, and $\tf{Z_{\text{V}}}$ is the subfilter mixture fraction variance as defined previously.
