\section{Turbulent Combustion}
\label{sec:lesmodels:combust}

In this section, effective strategies for incorporating combustion chemistry into LES are outlined. The turbulent combustion model must be able to accurately predict the thermal and chemical structure of the nonpremixed flame as fuel and oxidizer chemically react to form products such as carbon dioxide and water vapor. For sooting flames, the model also needs to account for the formation and evolution of soot precursors in addition to the soot itself. A brute-force method would involve selecting a chemical mechanism that models the above phenomena and solving coupled transport equations for all species within the mechanism. However, such an approach is impractical for realistic chemical mechanisms, which may contain thousands of species and tens of thousands of reactions~\cite{law2007}.

A more computationally tractable method is the Flamelet/Progress Variable (FPV) approach for nonpremixed combustion~\cite{pierce2004}. In the FPV approach, the local thermochemical state is described by the solutions to the steady flamelet equations~\cite{peters1984} and is parameterized by the mixture fraction $Z$ and reaction progress variable $C$:
\begin{equation}\label{eq:lesmodels:combust:fpv}
  \xi = \mc{F}(Z,C),
\end{equation}
where $\xi$ encompasses the thermochemical variables (density, temperature, species mass fractions, species source terms, etc.) and $\mc{F}$ is the functional relationship obtained from the solutions to the steady flamelet equations. To reduce computational overhead, these solutions are often computed \textit{a priori} and are stored in a database that is accessed during LES. Overall, the mapping given by \cref{eq:lesmodels:combust:fpv} has reduced the set of transported scalar variables to $Z$ and $C$, drastically decreasing the number of required transport equations.

However, this technique is insufficient for sooting flames due to its inability to predict the effects of thermal radiation. Soot evolution is characterized by long timescales and a sensitivity to the thermochemical state. Radiative thermal losses occur over similar temporal scales and can significantly alter the local thermochemical state. Therefore, radiative thermal losses cannot be neglected. The Radiative Flamelet/Progress Variable (RFPV) approach has extended the FPV approach to include these losses~\cite{ihme2008} and therefore will be the foundation for the turbulent combustion model in this work. The database of solutions to the steady flamelet equations is expanded to include solutions with radiative losses, and a heat loss parameter $H$ is now added to the local thermochemical equation of state
\begin{equation}\label{eq:lesmodels:combust:rfpv}
  \xi = \mc{G}(Z, C, H),
\end{equation}
where $\mc{G}$ is the functional relationship obtained from the augmented flamelet database.

For sooting flames, certain quantities such as the source terms in \cref{eq:lesmodels:soot:ndf:momtransport} are dependent on the soot scalars in addition to the thermochemical quantities. Therefore, a more general equation of state is formulated as~\cite{mueller2012}
\begin{equation}\label{eq:lesmodels:combust:jointeos}
  \phi = \mc{J}(\xi, \mc{M}_j),
\end{equation}
where $\phi$ is now any quantity that could depend jointly on the thermochemical variables $\xi$ and the soot scalars $\mc{M}_j$. The state vector $\mc{M}_j$ encompasses the moments $M_{x,y}$ and the weight of the delta function allocated for nucleating particles $N_0$. Mueller and Pitsch~\cite{subfilterpdf2011} proposed that \cref{eq:lesmodels:combust:jointeos} could be simplified by assuming that all of the quantities needed in the soot model can be written as the product of a function that depends only on the thermochemical state and a function that depends only on the soot scalars. Therefore, the functional relation $\mc{J}$ can be expressed as~\cite{mueller2012}
\begin{equation}\label{eq:lesmodels:combust:producteos}
  \phi = \mc{G}(Z, C, H)\mc{K}(\mc{M}_j),
\end{equation}
where $\mc{K}$ is unity if $\phi$ solely depends on the thermochemical quantities. This form will be utilized in \cref{ch:subfilter}.

%The reasoning behind this simplification will be revisited in \cref{ch:subfilter}.

In the following subsections, further details about the RFPV model will be discussed in the context of sooting flames. Definitions of the mixture fraction $Z$, progress variable $C$, and heat loss parameter $H$ are outlined to ensure that the thermochemical equation of state (\cref{eq:lesmodels:combust:rfpv}) is a unique function in the presence of soot formation. Afterwards, the nonpremixed flamelet equations are presented with additional terms to account for soot formation.


\subsection{Definition of Flame Structure Parameters}
\label{sec:lesmodels:combust:map}

For two-feed systems with a single fuel stream and single oxidizer stream, the mixture fraction can be defined to be a conserved scalar satisfying the transport equation~\cite{pitsch1998}
\begin{equation}\label{eq:lesmodels:combust:origz}
\trder{Z} = \pder[]{x_j}\left( \rho D_Z \pder[Z]{x_j} \right),
\end{equation}
where $Z$ is one in the fuel stream and zero in the oxidizer stream and $D_Z$ is chosen such that $Le_Z$ is unity. Such a definition is advantageous, for no assumptions about the species Lewis numbers have been made. However, the mixture fraction cannot be defined as a conserved scalar in sooting flames. During the nucleation of soot, PAH are extracted from the gas-phase, causing the mixture to be locally leaned. The approach of Mueller and Pitsch~\cite{mueller2012} is utilized to account for this phenomenon, where a source term is introduced into \cref{eq:lesmodels:combust:origz}:
\begin{equation}\label{eq:lesmodels:combust:map:z}
  \trder{Z} = \pder[]{x_j}\left( \rho D_Z \pder[Z]{x_j} \right) + \dot{m}_Z.
\end{equation}
The source term $\dot{m}_Z$ is defined similarly to Bilger's mixture fraction based on element mass fractions~\cite{bilger1989,mueller2012}:
\begin{equation}\label{eq:lesmodels:combust:map:zsource}
  \dot{m}_Z = \frac{\frac{\dot{m}_{\ce{C}} - \dot{\rho}Z_{\ce{C},\ce{O}}}{\nu_{\ce{C}}W_{\ce{C}}} + \frac{\dot{m}_{\ce{H}} - \dot{\rho}Z_{\ce{H},\ce{O}}}{\nu_{\ce{H}}W_{\ce{H}}} + 2\frac{\dot{\rho}Z_{\ce{O},\ce{O}} - \dot{m}_{\ce{O}}}{\nu_{\ce{O2}}W_{\ce{O2}}}}{\frac{Z_{\ce{C},\ce{F}} - Z_{\ce{C},\ce{O}}}{\nu_{\ce{C}}W_{\ce{C}}} + \frac{Z_{\ce{H},\ce{F}} - Z_{\ce{H},\ce{O}}}{\nu_{\ce{H}}W_{\ce{H}}} + 2\frac{Z_{\ce{O},\ce{O}} - Z_{\ce{O},\ce{F}}}{\nu_{\ce{O2}}W_{\ce{O2}}}},
\end{equation}
where $\dot{m}_k$ are element mass source terms, $\dot{\rho}$ is the source term in the continuity equation for the removal of gas-phase PAH, $Z_{k,l}$ are the mass fractions of element $k$ in the stream indicated by $l$, $\nu_k$ are the stoichiometric coefficients of the global reaction between fuel and oxidizer, and $W_{k}$ are the element molar masses. In the limit of unity Lewis numbers for all species, \cref{eq:lesmodels:combust:map:z,eq:lesmodels:combust:map:zsource} provide a description of the mixture fraction that is equivalent to Bilger's mixture fraction based on element mass fractions~\cite{bilger1989}. This definition of mixture fraction ultimately leads to additional terms in the flamelet equations of \cref{sec:lesmodels:combust:flamelet} that acknowledge the presence of soot.

The reaction progress variable has been defined in previous works~\cite{pierce2004,ihme2008} as
\begin{equation}\label{eq:lesmodels:combust:map:cprev}
  C_{\text{conv}} = Y_{\ce{CO2}} + Y_{\ce{CO}} + Y_{\ce{H2O}} + Y_{\ce{H2}}.
\end{equation}
However, this conventional definition is not appropriate in the context of sooting flames. PAH have large amounts of carbon relative to hydrogen, so their removal from the gas-phase tends to lower the local $\ce{C}$/$\ce{H}$ ratio. As a result, the local effective fuel composition is changed. To address this phenomena, Mueller and Pitsch~\cite{mueller2012} redefined the reaction progress variable through the transport equation given by
\begin{equation}\label{eq:lesmodels:combust:map:c}
  \trder{C} = \pder[]{x_j}\left( \rho D_C \pder[C]{x_j} \right) + \frac{\dot{m}_{C_{\text{conv}}}}{C^*},
\end{equation}
where the diffusion coefficient $D_C$ is chosen such that $Le_C$ is unity, $\dot{m}_{C_{\text{conv}}}$ is the source term for the conventional progress variable as defined in \cref{eq:lesmodels:combust:map:cprev}, and $C^*$ is a normalizing factor that accounts for the change in stoichiometry due to the removal of gas-phase PAH. $C^*$ ensures that the equation of state (\cref{eq:lesmodels:combust:jointeos}) remains unique even as the local $\ce{C}$/$\ce{H}$ ratio changes due to the nucleation of soot. In non-sooting flames, $C^*$ is constant because the local $\ce{C}$/$\ce{H}$ ratio is constant (in the absence of differential diffusion effects). Further details about $C^*$ can be found in Mueller and Pitsch~\cite{mueller2012}.
%With \cref{eq:lesmodels:combust:map:c}, the progress variable adopts the value of zero in both the fuel and oxidizer streams.

The remaining mapping quantity to fully specify the thermochemical state given by \cref{eq:lesmodels:combust:rfpv} is the heat loss parameter $H$. This quantity provides a measure for the extent to which radiative thermal losses affect the enthalpy. A simple definition for the heat loss parameter is the enthalpy itself or the enthalpy deficit. The latter option is more convenient, for the adiabatic boundary has a well-defined value of zero in the limit of unity Lewis numbers for all species. However, the presence of soot can affect the adiabatic boundary for the enthalpy deficit, and the inclusion of molecular transport introduces complications associated with tracking the enthalpy fluxes of different species. For either situation, the transport equation definition for the heat loss parameter $H$ from Mueller and Pitsch~\cite{mueller2012} is more convenient:
\begin{equation}\label{eq:lesmodels:combust:map:heatloss}
  \trder{H} = \pder[]{x_j}\left( \rho D_H \pder[H]{x_j} \right) + \dot{\rho}H + \dot{q}_{\text{RAD}},
\end{equation}
where the diffusion coefficient $D_H$ is chosen such that $Le_H$ is unity and $\dot{q}_{\text{RAD}}$ is the radiation source term. The second term on the right-hand side is present to ensure that $H$ is constant everywhere if $\dot{q}_{\text{RAD}}$ is zero. Initially, the heat loss parameter is zero throughout the nonpremixed system. As the cumulative effect of radiation increases, the heat loss parameter acquires increasingly negative values.


\subsection{Flamelet Equations}
\label{sec:lesmodels:combust:flamelet}

The thermochemical equation of state obtained from the RFPV approach (\cref{eq:lesmodels:combust:rfpv}) is specified by the solutions to the steady flamelet equations. These solutions, which are computed \textit{a priori} and stored in a database that is accessed during LES, provide a description of the thermal and chemical structure of a nonpremixed flame. The species flamelet equation is derived from the conservation equation for the species mass fractions:
\begin{equation}\label{eq:lesmodels:combust:flamelet:consy}
  \trder{Y_k} = -\pder[]{x_j}\left( \rho Y_k V_{k,j} \right) + \dot{m}_k,
\end{equation}
where $Y_k$ is the mass fraction of species $k$, $V_{k,j}$ is the diffusion velocity of species $k$ in the direction indicated by $j$, and $\dot{m}_k$ is the chemical source term for species $k$. The diffusion velocity has contributions from mass diffusion in the presence of concentration gradients, pressure gradients, body forces, and temperature gradients (a second order effect known as Soret diffusion)~\cite{law2006}. By only considering diffusion through concentration gradients, the diffusion velocity is described by the Stefan-Maxwell equation. However, evaluating the linear system associated with the Stefan-Maxwell equation can be computationally expensive, so further simplifications are often made with the Curtiss-Hirschfelder approximation~\cite{curtiss1949} or Fick's law of diffusion~\cite{fick1855}. The Curtiss-Hirschfelder approximation is given by
\begin{equation}\label{eq:lesmodels:combust:flamelet:curtisshirschfelder}
  V_{k,j} = -\frac{1}{X_k}D_k\pder[X_k]{x_j} + \sum\limits_{k} \frac{Y_k}{X_k}D_k\pder[X_k]{x_j},
\end{equation}
where $X_k$ is the mole fraction of species $k$, $D_k = (1 - Y_k)/\sum\limits_{k \neq l} (X_k/D_{k,l})$ is the mixture-averaged diffusivity for species $k$, and the second term on the right-hand side is a correction velocity that enforces the conservation of mass. By assuming constant unity Lewis numbers, \cref{eq:lesmodels:combust:flamelet:curtisshirschfelder} is simplified to Fick's law:
\begin{equation}\label{eq:lesmodels:combust:flamelet:fick}
  V_{k,j} = -\frac{1}{Y_k}D\pder[Y_k]{x_j},
\end{equation}
where $D$ is the constant diffusion coefficient. In \cref{ch:transport}, this assumption will be revisited.

% Fick's law assumes all species have the same constant Lewis number and is given by
%% \begin{equation}\label{eq:lesmodels:combust:flamelet:fick}
%%   V_{k,j} = -\frac{1}{Y_k}D\pder[Y_k]{x_j},
%% \end{equation}
%% where $D$ is the constant diffusion coefficient. For turbulent flames at sufficiently high Reynolds numbers, it is often assumed that the turbulence mixes all species indiscriminately~\cite{pitsch19981057}. Therefore, $D$ will be selected such that $Le_k$ is unity. Note that, in \cref{ch:transport}, this assumption will be revisited.

The temperature flamelet equation is obtained from the conservation equation for energy in terms of temperature under the assumption of constant thermodynamic pressure:
\begin{equation}\label{eq:lesmodels:combust:flamelet:const}
  \trder{T} = \frac{1}{c_p}\left[ \pder[]{x_j}\left( \lambda\pder[T]{x_j} \right) + \sum\limits_{k} \rho c_{p,k} D\pder[Y_k]{x_j}\pder[T]{x_j} - \sum\limits_{k} h_k\dot{m}_k + \dot{q}_{\text{RAD}} \right],
\end{equation}
where $\lambda$ is the thermal conductivity, $c_p$ is the specific heat at constant pressure, $c_{p,k}$ is the specific heat at constant pressure for species $k$, and $h_k$ is the specific enthalpy of species $k$. Fick's law is implemented in the second term on the right-hand side.

The flamelet equations are derived by transforming the physical space coordinate system to a mixture fraction-based coordinate system~\cite{peters1984}. It is assumed that the gradients along the flame surface are negligible compared to the gradients normal to the flame surface, which allows the flame structure to be described with only the mixture fraction. For sooting flames, Mueller~\cite{muellerphd} determined that the flamelet equations possess additional terms due to the source terms in \cref{eq:lesmodels:combust:map:z} and the continuity equation. For unity Lewis numbers, \cref{eq:lesmodels:combust:flamelet:consy} is transformed to
\begin{equation}\label{eq:lesmodels:combust:flamelet:flamelety}
  \rho\pder[Y_k]{\tau} = \frac{\rho\chi}{2}\sder[Y_k]{Z} + \dot{m}_k - \dot{\rho}Y_k + (\dot{\rho}Z - \dot{m}_Z)\pder[Y_k]{Z},
\end{equation}
where $\chi = 2D_Z(\partial Z/\partial x_j)^2$ is the scalar dissipation rate, the third term on the right-hand side is due to the source term of the continuity equation, and the last term on the right-hand side is a convective term in mixture fraction space due to the density and mixture fraction source terms. The latter two new terms ensure that mass is conserved and that the database of flamelet solutions remains unique, although they do not play a substantial role in the dynamics of \cref{eq:lesmodels:combust:flamelet:flamelety}.

Similarly, \cref{eq:lesmodels:combust:flamelet:const} is transformed to obtain the flamelet equation for temperature:
\begin{equation}\label{eq:lesmodels:combust:flamelet:flamelett}
  \begin{split}
    \rho c_p \pder[T]{\tau} &= \frac{\rho c_p \chi}{2}\sder[T]{Z} + \frac{\rho\chi}{2}\pder[c_p]{Z}\pder[T]{Z} - \sum\limits_{k} h_k\dot{m}_k + \sum\limits_{k} \frac{\rho c_{p,k}\chi}{2}\pder[Y_k]{Z}\pder[T]{Z} + \dot{q}_{\text{RAD}} \\
    &+ \dot{H} + c_p(\dot{\rho}Z - \dot{m}_Z)\pder[T]{Z},
  \end{split}
\end{equation}
where $\dot{H}$ is the enthalpy source term due to the extraction of PAH from the gas-phase and the last term on the second line is the convection term in mixture fraction space that is analogous to the one discussed in \cref{eq:lesmodels:combust:flamelet:flamelety}. In addition to these quantities, flamelet equations for the progress variable $C$ and heat loss parameter $H$ need to be derived. Performing the coordinate transform on \cref{eq:lesmodels:combust:map:c} provides the flamelet equation for the progress variable:
\begin{equation}\label{eq:lesmodels:combust:flamelet:flameletc}
  \rho\pder[C]{\tau} = \frac{\rho\chi}{2}\sder[C]{Z} + \frac{\dot{m}_{C_{\text{conv}}}}{C^*} - \dot{\rho}C + (\dot{\rho}Z - \dot{m}_Z)\pder[C]{Z},
\end{equation}
and performing a similar operation on \cref{eq:lesmodels:combust:map:heatloss} leads to the flamelet equation for the heat loss parameter:
\begin{equation}\label{eq:lesmodels:combust:flamelet:flameleth}
  \rho\pder[H]{\tau} = \frac{\rho\chi}{2}\sder[H]{Z} + (\dot{\rho}Z - \dot{m}_Z)\pder[H]{Z} + \dot{q}_{\text{RAD}}.
\end{equation}
Since the latter two quantities are not explicit functions of the species mass fractions and temperature, evaluation of \cref{eq:lesmodels:combust:flamelet:flameletc,eq:lesmodels:combust:flamelet:flameleth} is required to produce the complete parameterization of the database of flamelet solutions.

The database is typically filled using the procedure of Ihme and Pitsch~\cite{ihme2008}, where the unsteady forms of \crefrange{eq:lesmodels:combust:flamelet:flamelety}{eq:lesmodels:combust:flamelet:flameleth} are evaluated in order to fill the heat loss parameter space. However, such a procedure is computationally expensive due to the need to resolve the temporal coordinate. In this work, the approach of Carbonell \etal~\cite{carbonell2009} is adopted instead, where the steady forms of \crefrange{eq:lesmodels:combust:flamelet:flamelety}{eq:lesmodels:combust:flamelet:flameleth} are solved and the radiation source term of \cref{eq:lesmodels:combust:flamelet:flamelett} is weighted by a radiation loss coefficient $\Omega_{\text{RAD}} \in [0,1]$ to span the heat loss parameter space. The adiabatic form of \cref{eq:lesmodels:combust:flamelet:flamelett} is specified by $\Omega_{\text{RAD}} = 0$. Radiative thermal losses are introduced by increasing the value of $\Omega_{\text{RAD}}$ while still solving the steady forms of \crefrange{eq:lesmodels:combust:flamelet:flamelety}{eq:lesmodels:combust:flamelet:flameleth}. Note that in the limit of very small values of scalar dissipation rate, no steady solutions with radiation exist.

The adiabatic upper, middle, and lower branches of the S-curve as well as nonadiabatic solutions are shown in \cref{fig:lesmodels:combust:flamelet:tvschi}. These flamelet solutions are uniquely parameterized by $Z$, $C$, and $H$, with the latter two quantities obtained from evaluating \cref{eq:lesmodels:combust:flamelet:flameletc,eq:lesmodels:combust:flamelet:flameleth}. Validation of uniqueness with the specified definitions of mixture fraction and progress variable is available in Mueller~\cite{muellerphd}.

%% The aforementioned adiabatic and nonadiabatic, steady, stable-burning solutions form the upper branches of the S-curve as shown in \cref{fig:lesmodels:combust:flamelet:tvschi}. The S-curve contains the upper, middle, and lower branches that are formed by the flamelet solutions in the flame temperature/scalar dissipation rate plane at an arbitrary mixture fraction. The adiabatic, steady, unstable-burning, middle branch and adiabatic, steady, non-burning, lower branch are obtained through \crefrange{eq:lesmodels:combust:flamelet:flamelety}{eq:lesmodels:combust:flamelet:flameleth} as well, although the initial positions within the $T|Z_{st}$--$\chi|Z_{st}$ plane are different than those of the upper branch solutions. Thus, the set of solutions to the laminar nonpremixed flamelet equations is uniquely parameterized by $Z$, $C$, and $H$, with the latter two quantities obtained from evaluating \cref{eq:lesmodels:combust:flamelet:flameletc,eq:lesmodels:combust:flamelet:flameleth}. Validation of uniqueness with the specified definitions of mixture fraction and progress variable is available in Mueller~\cite{muellerphd}.

\begin{figure}[htb]
  \centering
  \includegraphics[width=0.6\linewidth]{ch-lesmodels/figures/TvsChi-connected}
  \caption[Flamelet Solutions of RFPV Model, $T|Z_{st}$ vs. $\chi|Z_{st}$]{Solutions to the flamelet equations of the RFPV model for a $\ce{C2H4}$/$\ce{H2}$/$\ce{N2}$ [40/41/19 by volume] nonpremixed flame at atmospheric pressure~\cite{mahmoud2017}. The adiabatic solutions are the red circles, magenta triangles, and the black square, which represent the stable-burning, unstable-burning, and non-burning solutions, respectively. The nonadiabatic solutions are the diamonds, where each set of solutions linked with a dashed line possesses a distinct, nonzero $\Omega_{\text{RAD}}$.}
  \label{fig:lesmodels:combust:flamelet:tvschi}
\end{figure}
