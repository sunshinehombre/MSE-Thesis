\chapter{Modeling Framework for Large Eddy Simulation\label{ch:lesmodels}}

This chapter lays the foundation for modeling the evolution of soot in large eddy simulation (LES) of turbulent nonpremixed combustion. In LES of turbulent flows, the large scale, geometry-dependent motions are resolved while small-scale, universal turbulent structures are modeled. The predictive capabilities of LES are dependent on the quality of the models used for the small-scale, subfilter phenomena. Accurately capturing the evolution of soot requires the integration of models for soot, turbulent combustion, and polycyclic aromatic hydrocarbons (PAH) into the LES framework.

The distribution of soot particle sizes, known as the Number Density Function (NDF), has been found to be bimodal due to the presence of small, nascent particles and larger, more mature aggregates. Modeling the NDF is achieved by first parameterizing it with the volume and surface area of particles. However, since the NDF is high-dimensional, the exact form cannot be solved for directly due to the associated computational costs. Thus, the Method of Moments is used to obtain the average quantities, known as moments, of the NDF. In this chapter, the transport equation that governs the evolution of these moments is introduced. This equation has a source term that requires modeling, so discussion of closure through the Hybrid Method of Moments is presented as well.

Turbulent combustion is modeled through the flamelet approach~\cite{peters1984}, where a three-dimensional turbulent flame is represented as an ensemble of locally one-dimensional laminar flamelets embedded in a turbulent flow field. This method relies on the separation of scales between the turbulence and the flame's reaction zone, allowing the thermochemical properties of the flame to be decoupled from the flow field. As a result, the flame structure can be evaluated \textit{a priori}, parameterized with a reduced set of variables, and stored in a database that is accessed during LES. In this chapter, modifications to the classical flamelet approach accounting for the formation of soot and radiative thermal losses are explained. Closure of subfilter quantities with the presumed PDF approach is also outlined.

Lastly, special consideration is given to gas-phase PAH. Since their chemistry is slow relative to the exothermic combustion chemistry, a transport equation is required to model the resulting unsteady effects. Discussion on modeling the source term for a lumped PAH species is included.

%% Using LES is advantageous because it captures unsteady flow features like swirl and recirculation that are neglected by other methods such as Reynolds-Averaged Navier-Stokes (RANS), without being as computationally intensive as full-fidelity Direct Numerical Simulation (DNS). Its predictive capabilities can lead to highly optimized designs through rapid iteration. The potential of LES to innovate the next generation of environmentally-friendly automobile or aircraft engines is the main reason it has been selected as the underlying modeling architecture.

\section{Soot}
\label{sec:lesmodels:soot}

\subsection{Modeling the Particle Size Distribution}
\label{sec:lesmodels:soot:ndf}

The distribution of soot particle sizes in a system can be described by the Number Density Function (NDF), $N$. Previous studies have suggested that the soot NDF is bimodal, accounting for newly formed 'primary' particles from persistent nucleation and larger aggregrates from various growth modes~\cite{zhao2003,zhao2005,netzell2007}. Incipient particles are roughly spherical with diameters on the order of tens of nanometers, while aggregates comprised of monodisperse primaries have fractal geometries with length scales on the order of hundreds of nanometers~\cite{vanderwal1999}. In order to account for the geometrical complexities, the NDF requires a multivariate parameterization such as mass or volume and surface area~\cite{patterson2007,mueller2009,hmom2009}, the number of agglomerates and primary particles~\cite{park2004}, or even volume, surface area, and the number of surface hydrogenated carbon sites~\cite{blanquart2009}. In this work, a bivariate parameterization (volume $V$ and surface area $S$) of the NDF is used to give the functional form $N = N(V, S; t, x_j)$. Note that the NDF cannot be solved for directly due to its high dimensionality (six degrees of freedom), and so statistical models are required.

For application to LES, the Method of Moments is one of the only tractable techniques due to its lower computational expense. In this approach, mean quantities of the NDF are solved for and complete information about the distribution is lost. These quantities are moments given by
\begin{equation}\label{eq:lesmodels:soot:ndf:moments}
  M_{x,y} = \sum\limits_{i=0}^\infty V_i^x S_i^y N_i,
\end{equation}
where discrete summation over $i$ implies summation over all particles sizes and $N_i$ is the number density of particles with volume $V_i$ and surface area $S_i$. The subscripts $x$ and $y$ indicate the order of the moment in volume and surface area, respectively.

The NDF evolves according to the Population Balance Equation (PBE)~\cite{friedlander2000}
\begin{equation}\label{eq:lesmodels:soot:ndf:pbe}
  \tder{N_i} - \pder[]{x_j}\left( 0.55\frac{\nu}{T}\pder[T]{x_j}N_i \right) = \dot{N_i},
\end{equation}
where the third term is the thermophoresis of particles~\cite{waldmann1966} and the source term on the right-hand side incorporates the various physical and chemical processes that govern soot evolution. These processes include nucleation from polycyclic aromatic hydrocarbon (PAH) dimers~\cite{blanquart2009,schuetz2002,frenklach1991,wang2011}, particle coagulation~\cite{kazakov1998,hmom2009}, soot growth from PAH condensation~\cite{blanquart2009,hmom2009}, soot growth through the $\ce{H}$-abstraction, $\ce{C2H2}$-addition (HACA) surface reaction mechanism~\cite{frenklach1985,frenklach1991}, oxidation~\cite{stanmore2001,neoh1981,kazakov1995}, an oxidation-induced fragmentation~\cite{neoh1984,mueller2011}. Note that molecular diffusion has been neglected, as a soot particle with a diameter of 10 nm has a Schmidt number of 287 at standard atmospheric conditions~\cite{friedlander2000}. This corresponds to a molecular diffusion coefficient of $D = 5.24\times 10^{-4}$ cm$^2$/sec, so the diffusion velocity of soot is minimal as a result. For LES, tracking the evolution of the moments is more relevant. A transport equation is derived by taking moments of \cref{eq:lesmodels:soot:ndf:moments},
\begin{equation}\label{eq:lesmodels:soot:ndf:momtransport}
  \tder{M_{x,y}} - \pder[]{x_j}\left( 0.55\frac{\nu}{T}\pder[T]{x_j}M_{x,y} \right) = \dot{M}_{x,y}.
\end{equation}
For convenience, a total velocity is defined that combines the convective and thermophoretic terms
\begin{equation}\label{eq:lesmodels:soot:ndf:totvel}
  u_j^* = u_j - 0.55\frac{\nu}{T}\pder[T]{x_j}.
\end{equation}

Although the Method of Moments has a relatively smaller computational cost, it faces the problem of closure. Evaluation of the source term $\dot{M}_{x,y}$ in \cref{eq:lesmodels:soot:ndf:momtransport} depends on moments that are not directly solved for, requiring further modeling. Two moment methods that provide closure are the Method of Moments with Interpolative Closure (MOMIC) and the Direct Quadrature Method of Moments (DQMOM).

In MOMIC~\cite{frenklach2002,frenklach1987,frenklach1994}, transport equations are solved for a set of moments. Source term closure is then achieved by polynomial interpolation of the logarithm of these known moments. For a bivariate parameterization of the NDF with volume and surface area, the interpolation is given by
\begin{equation}\label{eq:lesmodels:soot:ndf:momic}
  M_{x,y}^{\text{MOMIC}} = \exp\left( \sum\limits_{r=0}^{R} \sum\limits_{k=0}^{r} a_{r,k}x^k y^{r-k} \right),
\end{equation}
where $R$ is the order of the polynomial interpolation. The constant coefficients $a_{r,k}$ are determined by taking the logarithm of \cref{eq:lesmodels:soot:ndf:momic} and solving the resulting linear system with the known moments. This system is well-conditioned, although as the order $R$ is increased, the number of additional moment transport equations increases.

In DQMOM~\cite{marchisio2005}, the NDF is thought of as a summation of multi-dimensional Dirac delta functions, where the moments are approximated by Gauss quadrature. Transport equations are solved for the weights and locations of these delta functions, rather than for the moments of the NDF. The moments of the NDF are given by
\begin{equation}\label{eq:lesmodels:soot:ndf:dqmom}
  M_{x,y}^{\text{DQMOM}} = \sum\limits_{i=1}^{P} N_i V_i^x S_i^y,
\end{equation}
where $P$ is the total number of delta functions used in the quadrature approximation, $N_i$ are the weights of the delta functions, and $V_i$ and $S_i$ are the abscissas (locations) of the delta functions. Closure of the source terms of the transport equations for the weights and abscissas is achieved by using the source terms of a specified independent set of moments. However, depending on the shape of the NDF at any point and the selected set of moments, the procedure of closure can involve inverting an ill-conditioned linear system.

As mentioned at the beginning of this section, the NDF has been found to be bimodal in sooting flames with persistent nucleation. The ability to predict the contributions of primary particles and larger aggregates is a major characteristic that distinguishes between the aforementioned methods. MOMIC has been shown to be unsuccessful at capturing the influence of incipient particles, while DQMOM represents the bimodality of the NDF well~\cite{mueller2009}. DQMOM has this capability because it is able to allocate a single delta function for the incipient particles, which remains nearly fixed at the size of these nucleated particles~\cite{blanquart2009}. All other delta functions can then be allocated to resolve the distribution of larger particles and aggregates.

A third moment method, the Hybrid Method of Moments (HMOM), adopts the advantages of DQMOM and MOMIC to provide a bimodal description of the NDF that is numerically well-conditioned. In this work, the Hybrid Method of Moments will be used.


\subsection{Hybrid Method of Moments}
\label{sec:lesmodels:soot:hmom}

In HMOM, an arbitrary moment is given by
\begin{equation}\label{eq:lesmodels:soot:hmom:m}
  M_{x,y}^{\text{HMOM}} = N_0 V_0^x S_0^y + \exp\left( \sum\limits_{r=0}^{R} \sum\limits_{k=0}^{r} a_{r,k}x^k y^{r-k} \right),
\end{equation}
where the first term on the right-hand side is a delta function that is allocated to account for nucleating soot particles. $N_0$ is the weight of this delta function that is located at the fixed coordinates given by $V_0$ and $S_0$. Note that in DQMOM, this delta function is not completely immobile. However, it does not move much and therefore no significant error is expected by fixing its location. The second term on the right-hand side of \cref{eq:lesmodels:soot:hmom:m} is the contribution from MOMIC and accounts for the presence of larger soot aggregates. Unlike MOMIC, determination of the weight $N_0$ and the coefficients $a_{r,k}$ is nontrivial. However, if $N_0$ is known, then the system given by \cref{eq:lesmodels:soot:hmom:m} can be easily inverted as in MOMIC to evaluate $a_{r,k}$. Thus, as in DQMOM, the weight of the delta function $N_0$ will be determined through a transport equation similar to the one defined in \cref{eq:lesmodels:soot:ndf:momtransport}. For first order polynomial interpolation ($R = 1$), the source term of the transport equation for $N_0$ is given by
\begin{equation}\label{eq:lesmodels:soot:hmom:ndot}
  \dot{N}_0 = \lim_{\alpha,\beta\to\infty} \frac{\dot{M}_{-\alpha,-\beta}}{V_0^{-\alpha} S_0^{-\beta}},
\end{equation}
where expressions for the source term in the numerator are obtained from the various processes that govern the evolution of soot particles. Note that in order to determine $a_{r,k}$ given $R = 1$, a set of three additional moments need to be evaluated. These quantities are the total number density $M_{0,0}$, the total particle volume $M_{1,0}$, and the total particle surface area $M_{0,1}$. Further information about the modeling of the physical processes in \cref{eq:lesmodels:soot:hmom:ndot} as well as other details of HMOM can be found in Mueller et al.~\cite{hmom2009,mueller2009,mueller2011}.

\section{Turbulent Combustion}
\label{sec:lesmodels:combust}

In this section, effective strategies for incorporating combustion chemistry into LES are covered. The turbulent combustion model must be able to accurately predict the thermal and chemical structure of the nonpremixed flame as fuel is oxidized into products. For sooting flames, the model also needs to account for the formation and evolution of soot precursors in addition to the soot itself. A brute-force method would involve selecting a chemical mechanism that models the above phenomena and solving coupled transport equations for all the species within the mechanism. However, such an approach is impractical for realistic chemical mechanisms, where there may be thousands of species and tens of thousands of reactions~\cite{law2007}.

A more computationally tractable method is the Flamelet/Progress Variable (FPV) approach for nonpremixed combustion~\cite{pierce2004}. In the FPV approach, the local thermochemical equation of state is described by the solutions of the steady flamelet equations~\cite{peters1984} and is parameterized by the mixture fraction $Z$ and reaction progress variable $C$
\begin{equation}\label{eq:lesmodels:combust:fpv}
  \xi_i = \mathcal{F}(Z,C),
\end{equation}
where $\xi_i$ encompasses the thermochemical variables (density, temperature, species mass fractions, species source terms, etc.) and $\mathcal{F}$ is the functional relationship obtained from the solutions to the steady flamelet equations. To reduce computational overhead, these solutions are often computed \textit{a priori} and are stored in a database that is accessed during LES. Overall, the mapping given by Equation~\ref{eq:lesmodels:combust:fpv} has reduced the set of transported scalar variables to $Z$ and $C$, thereby drastically decreasing the number of required transport equations.

However, this technique is insufficient for sooting flames due to its inability to predict the effects of thermal radiation. Soot evolution is characterized by long time scales and a sensitivity to the thermochemical state. Radiative thermal losses occur on similar temporal scales and can significantly alter the local thermochemical state. Therefore, radiative losses must be included. The Radiation Flamelet/Progress Variable (RFPV) approach has extended the FPV approach to include these losses~\cite{ihme2008} and therefore will be the foundation for the turbulent combustion model in this work. The database of solutions to the steady flamelet equations is expanded to include solutions with radiative losses, and a heat loss parameter $H$ is now added to the local thermochemical equation of state:
\begin{equation}\label{eq:lesmodels:combust:rfpv}
  \xi_i = \mathcal{G}(Z, C, H),
\end{equation}
where $\mathcal{G}$ is the functional relationship obtained from the augmented flamelet database.

For sooting flames, certain quantities such as the source terms in Equation~\ref{eq:lesmodels:soot:ndf:momtransport} are dependent on the soot scalars in addition to the thermochemical quantities. Therefore, a more general equation of state is formulated as
\begin{equation}\label{eq:lesmodels:combust:jointeos}
  \phi = \mathcal{J}(\xi_i, \mathcal{M_j}),
\end{equation}
where $\phi$ is now any quantity that could depend jointly on the thermochemical variables $\xi_i$ and the soot scalars $\mathcal{M_j}$. The state-space vector $\mathcal{M_j}$ encompasses the moments $M_{x,y}$ and the weight of the delta function allocated for nucleating particles $N_0$. Mueller et al.~\cite{subfilterpdf2011} proposed that Equation~ref{eq:lesmodels:combust:jointeos} could be simplified by assuming that all of the quantities needed in the soot model can be written as the product of a function that depends only on the thermochemical state and a function that depends only on the soot scalars. Ergo, the functional relation $\mathcal{J}$ can be expressed as
\begin{equation}\label{eq:lesmodels:combust:producteos}
  \phi = \mathcal{G}(Z, C, H)\mathcal{K}(\mathcal{M_j})),
\end{equation}
where $\mathcal{K}$ is unity if $\phi$ solely depends on the thermochemical quantities. The reasoning behind this simplification will be revisited in Chapter~ref{ch:subfiltermodeling}.

In the following sections, further details about the RFPV model (Equation~\ref{eq:lesmodels:combust:rfpv}) will be discussed in the context of sooting flames. Definitions of the mixture fraction $Z$, progress variable $C$, and heat loss parameter $H$ are outlined to ensure that the equation of state (Equation~\ref{eq:lesmodels:combust:rfpv}) is a unique function in the presence of soot formation. Afterwards, the nonpremixed flamelet equations are presented with additional terms to account for soot formation.


\subsection{Definition of Flame Structure Parameters}
\label{sec:lesmodels:combust:map}




\subsection{Flamelet Equations}
\label{sec:lesmodels:combust:flamelet}

Information about flamelet equations.

\section{Small-Scale Quantities}
\label{sec:lesmodels:presumedpdf}

LES is a technique where large-scale, unsteady turbulent motions are computed explicitly while small-scale motions are modeled. The separation of scales is achieved through filtering the velocity and scalar fields (generically represented by $Q(x_j,t)$) to decompose them into the sum of a resolved component $\mean{Q}(x_j,t)$ and a subfilter-scale component $Q(x_j,t)-\mean{Q}(x_j,t)$. A general filtering operation can be expressed as
\begin{equation}\label{eq:lesmodels:presumedpdf:filter}
  \mean{Q}(x_j,t) = \int F(r_j,x_j)Q(x_j - r_j, t)dr_j,
\end{equation}
where integration is over the entire domain. The filter kernel $F(r_j,x_j)$ satisfies the normalization condition
\begin{equation}\label{eq:lesmodels:presumedpdf:kernel}
  \int F(r_j,x_j)dr_j = 1,
\end{equation}
and uses cutoff length and time scales to separate unresolved small-scale quantities from $\mean{Q}(x_j,t)$. In combustion, variations in density are non-zero. Therefore, it is convenient to use density-weighted filtering (Favre filtering) in LES of turbulent combustion, where the resolved components are expressed as $\tf{Q}(x_j,t) = \mean{\rho(x_j,t)Q(x_j,t)}/\mean{\rho(x_j,t)}$.

When these filtering operations are applied to the conservation equations for mass, momentum, and energy or to scalar transport equations, subfilter contributions to the chemical source term, turbulent transport, and molecular transport terms arise. For instance, the density-weighted filtered transport equation for a scalar $Q$ is given by
\begin{equation}\label{eq:lesmodels:presumedpdf:scalar}
  \ftrder{\tf{Q}} = \pder[]{x_j}\left( \mean{\rho}\tf{D_Q\pder[Q]{x_j}} \right) - \dsff[\tf{Q}]{\tf{u_j Q}} + \fst[m]{Q},
\end{equation}
where the first term on the right-hand side contains subfilter contributions to molecular diffusion fluxes, the second term is known as the subfilter scalar flux and contains the effects of unresolved turbulent transport, and the third term is the filtered source term. These unclosed terms represent small-scale interactions between soot, turbulence, and combustion chemistry. In particular, the filtered source term is usually closed with a density-weighted joint PDF $\tf{P}$ of the thermochemical and soot variables. Density-weighted filtered functions of these quantities are obtained from a convolution of the equation of state with $\tf{P}$:
\begin{equation}\label{eq:lesmodels:presumedpdf:filteredfuncs}
  \tf{\phi}(\xi,\mc{M}_j) = \iint \mc{J}(\xi,\mc{M}_j)\tf{P}(\xi,\mc{M}_j) d\xi d\mc{M}_j,
\end{equation}
where $\xi$ is the vector of thermochemical variables and $\mc{M}_j$ is the state vector encompassing the moments $M_{x,y}$ and the weight of the delta function $N_0$ allocated for nucleating particles.

The joint subfilter PDF $\tf{P}$ is unknown and must be modeled using transported PDF or presumed PDF methods. In the transported PDF approach, a transport equation is solved for the subfilter PDF~\cite{pope1985,pope1991}. This equation is high-dimensional, so conventional discretization techniques are computationally intractable. Instead, Lagrangian particle methods are employed, where a set of notional particles that track fluid particles is solved for. The advantage of the class of transported PDF methods is that all local phenomena are naturally described by the subfilter PDF, which provides information about subfilter fluctuations and correlations at a single point. Therefore, chemical source terms do not require additional closure because chemical reactions are one-point phenomena. However, processes such as molecular mixing are dependent on gradients and are inherently two-point, non-local phenomena. Mixing models such as the Interaction with Exchange of the Mean (IEM)~\cite{dopazo1974} and Modified Curl~\cite{janicka1979} have been proposed, where quantities are relaxed to local filtered values to mimic homogenization through diffusion. However, such approaches fail to use information about localness in composition space. More complex mixing models using the concept of the Euclidean minimum spanning tree~\cite{subramaniam1998} or the shadow position~\cite{pope2013} introduce localness in composition space but still suffer from the large computational cost associated with evolving the Lagrangian particles.

% In contrast, presumed PDF methods assume a form of the joint subfilter PDF. Reasoning based on physical insights about subfilter interactions between the scalars and turbulence is utilized to deduce the form of the PDF. Previous works have indicated that using the beta distribution for the mixture fraction and conditional delta distributions for the progress variable and heat loss parameter is a good model for the thermochemical portion of the subfilter PDF when considering passive scalars in a two-feed system~\cite{cook1994,jimenez1997,wall2000,ihme2008}. Ihme \etal~\cite{ihme2005} arrived at the same conclusion with the FPV approach for reactive scalars. Since presumed PDF methods do not rely on mixing models, they generally match or even exceed the performance of transported PDF methods, provided that the form of the PDF is appropriate. Therefore, the presumed PDF approach will be adopted in this work.

In contrast, presumed PDF methods assume a form of the joint subfilter PDF. Reasoning based on physical insights about subfilter interactions between the scalars and turbulence is utilized to deduce the form of the PDF. Since presumed PDF methods do not rely on mixing models, they generally match or even exceed the performance of transported PDF methods, provided that the form of the PDF is appropriate. Therefore, the presumed PDF approach will be adopted in this work. An approximate form of the overall joint subfilter PDF is determined by first decomposing $\tf{P}$ into a thermochemical PDF and a PDF of the soot scalars conditioned on the thermochemical variables. By Bayes' theorem,
\begin{equation}\label{eq:lesmodels:presumedpdf:bayes}
  \tf{P}(\xi,\mc{M}_j) = \tf{P}(\xi)P(\mc{M}_j|\xi),
\end{equation}
where $\tf{P}(\xi)$ includes a beta distribution for the mixture fraction and $P(\mc{M}_j|\xi)$ is unknown. Further discussion on the complete forms of $\tf{P}(\xi)$ and $P(\mc{M}_j|\xi)$ will be deferred until \cref{ch:subfilter}.

In practical implementation, each flamelet solution in the database is convoluted with a beta distribution for the mixture fraction and tabulated as a function of the filtered mixture fraction, subfilter mixture fraction variance, filtered progress variable, and filtered heat loss parameter. The transport equation for the filtered mixture fraction is derived by spatially filtering \cref{eq:lesmodels:combust:map:z} and is given by
\begin{equation}\label{eq:lesmodels:presumedpdf:filteredz}
  \ftrder{\tf{Z}} = -\dsff[\tf{Z}]{\tf{u_j Z}} + \fdt[Z]{\tf{Z}} + \fst[m]{Z},
\end{equation}
where the first and last terms on the right-hand side are unclosed. The former is the subfilter flux, which can be modeled with a dynamic procedure~\cite{germano1991,lilly1992,moin1991}. The filtered source term is closed through the presumed subfilter PDF given in \cref{eq:lesmodels:presumedpdf:trimarg}. Similarly, the transport equations for the filtered progress variable and filtered heat loss parameter are given by
\begin{equation}\label{eq:lesmodels:presumedpdf:filteredc}
  \ftrder{\tf{C}} = -\dsff[\tf{C}]{\tf{u_j C}} + \fdt[C]{\tf{C}} + \mean{\left( \frac{\dot{m}_{C_{\text{conv}}}}{C^*} \right)}
\end{equation}
and
\begin{equation}\label{eq:lesmodels:presumedpdf:filteredh}
  \ftrder{\tf{H}} = -\dsff[\tf{H}]{\tf{u_j H}} + \fdt[H]{\tf{H}} + \mean{\dot{\rho}H} + \fst[q]{\text{RAD}},
\end{equation}
respectively. The remaining database parameter is the subfilter mixture fraction variance, which will be procured in the manner of Mueller and Pitsch~\cite{mueller2012}. Rather than solving for the subfilter variance directly, a transport equation for the filtered square of the mixture fraction is evaluated:
\begin{equation}\label{eq:lesmodels:presumedpdf:filteredzsq}
  \begin{split}
    \ftrder{\tf{Z^2}} &= -\dsff[\tf{Z^2}]{\tf{u_j Z^2}} + \fdt[Z]{\tf{Z^2}} \\
    &- 2\mean{\rho}\tf{D}_Z \pder[\tf{Z}]{x_j}\pder[\tf{Z}]{x_j} - \mean{\rho}\tf{\chi}_{\text{sgs}} - \mean{\dot{\rho}Z^2} + 2\mean{\dot{m}_Z Z},
  \end{split}
\end{equation}
where $\tf{\chi}_{\text{sgs}}$ is the subfilter scalar dissipation rate. This quantity is estimated with a linear relaxation model~\cite{ihme200890}:
\begin{equation}\label{eq:lesmodels:presumedpdf:chisgs}
  \tf{\chi}_{\text{sgs}} = \text{C}_{\chi}\frac{\nu_t}{\Delta^2}\tf{Z_{\text{V}}},
\end{equation}
where $\text{C}_{\chi} \approx 20$, the mechanical timescale is determined from the eddy viscosity $\nu_t$ and filter width $\Delta$, and $\tf{Z_{\text{V}}} = \tf{Z^2} - \tf{Z}^2$ is the subfilter mixture fraction variance.


\section{Polycyclic Aromatic Hydrocarbons}
\label{sec:lesmodels:pah}

Understanding the formation and evolution of soot is central to this work, so accurately modeling the generation and consumption of gas-phase PAH that participate in nucleation and condensation is vital. A simple LES model for tracking the evolution of PAH would involve obtaining the PAH mass fraction from the RFPV combustion model. However, Attili et al.~\cite{attili2014} and Bisetti et al.~\cite{bisetti2012} noted in DNS studies of a nitrogen-diluted, \textit{n}-heptane/air turbulent nonpremixed flame that the PAH mass fraction deviates significantly from the values calculated by the steady flamelet approximation. The chemistry of PAH such as naphthalene is much slower than the chemistry of species that participate in the significantly exothermic reactions. As a result, PAH are unable to immediately adjust to the rapidly changing scalar dissipation rate field. They require a longer time to respond to the fluctuating field, so the unsteady term in the flamelet equations cannot be neglected.

To account for these unsteady phenomena, Mueller and Pitsch~\cite{mueller2012} developed a transport equation model for PAH that was inspired by a similar approach for slowly evolving $\ce{NO}$~\cite{ihme2008}. The spatially filtered transport equation is given by
\begin{equation}\label{eq:lesmodels:pah:pahtrans}
  \ftrder{\tf{Y}_{\text{PAH}}} = \dsff[\tf{Y}_{\text{PAH}}]{\tf{u_j Y}_{\hspace{-3pt}\text{PAH}}} + \fdt[\text{PAH}]{\tf{Y}_{\text{PAH}}} + \fst[m]{\text{PAH}},
\end{equation}
where $\tf{Y}_{\text{PAH}}$ is the filtered mass fraction of a lumped PAH species and $\fst[m]{\text{PAH}}$ is the sum of the chemical source terms of all the PAH species. Under the assumption of unity Lewis number, all PAH species will have a diffusivity governed by $D_{\text{PAH}}$ and \cref{eq:lesmodels:pah:pahtrans} is the exact transport equation for a lumped PAH. However, if all PAH do not have the same molecular diffusivity, then the diffusivity of naphthalene is preferred for $D_{\text{PAH}}$ as it is generally the PAH species with the largest concentration. In this circumstance, \cref{eq:lesmodels:pah:pahtrans} is an approximate model for the evolution of the combined PAH mass fractions.

The unclosed terms in \cref{eq:lesmodels:pah:pahtrans} include the first and third terms on the right-hand side. The former is simply the subfilter flux and can be modeled using a conventional dynamic procedure as mentioned previously. Following Mueller and Pitsch~\cite{mueller2012}, the filtered chemical source term can be decomposed into a production term $\dot{m}_{+}$, a consumption term $\dot{m}_{-}$, and an additional consumption term due to the extraction of PAH from the gas-phase to form soot (dimerization) $\dot{m}_{D}$. For each individual PAH species $k$, the chemical production term is independent of $Y_k$, the chemical consumption term depends linearly on $Y_k$, and the dimerization term depends quadratically on $Y_k$. The source term of the lumped species will be decomposed similarly:
\begin{equation}\label{eq:lesmodels:pah:pahsrc}
  \fst[m]{\text{PAH}} = \mean{(\dot{m}_{+})} + \mean{\left( \frac{\dot{m}_{-}}{Y_{\text{PAH}}} \right) Y_{\text{PAH}}} + \mean{\left( \frac{\dot{m}_{D}}{{Y_{\text{PAH}}}^2} \right) {Y_{\text{PAH}}}^2},
\end{equation}
where the terms in parentheses are obtained directly from the RFPV combustion model because they are nearly independent of the PAH mass fraction. The last two terms on the right-hand side of \cref{eq:lesmodels:pah:pahsrc} are unclosed filtered correlations. Mueller and Pitsch~\cite{mueller2012} have proposed closure through a scale-similarity assumption between the transported PAH mass fraction and the PAH mass fraction computed from the steady flamelet equations. This model is given by
\begin{equation}\label{eq:lesmodels:pah:similarity}
  \frac{\tf{Y}_{\text{PAH}}}{\tf{Y}_{\text{PAH}}^{\text{RFPV}}} = \frac{Y_{\text{PAH}}^{''}}{Y_{\text{PAH}}^{\text{RFPV}''}},
\end{equation}
where the subscript $^{\text{RFPV}}$ designates quantities that are from the RFPV combustion model and the subscript $^{''}$ indicates the ``fluctuating'' LES quantities given by
\begin{equation}\label{eq:lesmodels:pah:residual}
  Y_{\text{PAH}}^{''} = Y_{\text{PAH}} - \tf{Y}_{\text{PAH}}.
\end{equation}
With this closure model, the filtered source term of the lumped PAH species is
\begin{equation}\label{eq:lesmodels:pah:pahsrcmodeled}
  \fst[m]{\text{PAH}} = \mean{\dot{m}}_{+}^{\text{RFPV}} + \mean{\dot{m}}_{-}^{\text{RFPV}}\left( \frac{\tf{Y}_{\text{PAH}}}{\tf{Y}_{\text{PAH}}^{\text{RFPV}}} \right) + \mean{\dot{m}}_{D}^{\text{RFPV}}\left( \frac{\tf{Y}_{\text{PAH}}}{\tf{Y}_{\text{PAH}}^{\text{RFPV}}} \right)^2.
\end{equation}

