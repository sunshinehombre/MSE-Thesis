\chapter{Modeling Framework for Large Eddy Simulation\label{ch:lesmodels}}

This chapter lays the foundation for modeling the evolution of soot in large eddy simulation (LES) of turbulent nonpremixed combustion. In LES of turbulent, nonreacting flows, the large scale, geometry-dependent motions are resolved while small-scale, universal turbulent structures are modeled. Using LES is advantageous because it captures unsteady flow features like swirl and recirculation that are neglected by other methods such as Reynolds-Averaged Navier-Stokes (RANS) without being as computationally intensive as full-fidelity Direct Numerical Simulation (DNS). Its predictive capabilities can lead to highly optimized designs through rapid iteration. The potential of LES to innovate the next generation of environmentally-friendly automobile or aircraft engines is the main reason it has been selected as the underlying modeling architecture.

Models for combustion are incorporated into this framework. In the flamelet approach~\cite{peters1984}, a three-dimensional turbulent flame is represented as an ensemble of locally one-dimensional laminar flamelets embedded in a turbulent flowfield. This method relies on the scale separation between 

Models for soot, turbulent combustion, the presumed PDF approach, and PAH are discussed.

\section{Soot}
\label{sec:lesmodels:soot}

%\subsection{Modeling the Particle Size Distribution}
%\label{sec:lesmodels:soot:ndf}

The distribution of soot particle sizes in a system can be described by the Number Density Function (NDF), $N$. Previous studies have suggested that the soot NDF is bimodal, accounting for newly formed primary particles from persistent nucleation and larger aggregrates from various growth modes~\cite{zhao2003,zhao2005,netzell2007}. Incipient particles are roughly spherical with diameters on the order of nanometers, while aggregates comprised of monodisperse primaries have fractal geometries with length scales on the order of hundreds of nanometers~\cite{vanderwal1999}. In order to account for the geometrical complexities, the NDF requires a multivariate parameterization such as mass/volume and surface area~\cite{patterson2007,mueller2009,hmom2009}, the number of agglomerates and primary particles~\cite{park2004}, or even volume, surface area, and the number of surface hydrogenated carbon sites~\cite{blanquart2009}. In this work, a bivariate parameterization (volume $V$ and surface area $S$) of the NDF is used to give the functional form $N = N(V, S; t, x_j)$. Note that the NDF cannot be solved for directly due to its high dimensionality (six degrees of freedom), so statistical models are required.

For application to LES, the Method of Moments is one of the only tractable techniques due to its lower computational expense. In this approach, mean quantities of the NDF are solved for and complete information about the distribution is lost. These quantities are moments given by
\begin{equation}\label{eq:lesmodels:soot:ndf:moments}
  M_{x,y} = \sum\limits_{i=0}^\infty V_i^x S_i^y N_i,
\end{equation}
where discrete summation over $i$ implies summation over all particles sizes and $N_i$ is the number density of particles with volume $V_i$ and surface area $S_i$. The subscripts $x$ and $y$ indicate the order of the moment in volume and surface area, respectively.

The NDF evolves according to the Population Balance Equation (PBE)~\cite{friedlander2000}
\begin{equation}\label{eq:lesmodels:soot:ndf:pbe}
  \tder{N_i} - \pder[]{x_j}\left( 0.55\frac{\nu}{T}\pder[T]{x_j}N_i \right) = \dot{N_i},
\end{equation}
where the third term is the thermophoresis of particles~\cite{waldmann1966} and the source term on the right-hand side incorporates the various physical and chemical processes that govern soot evolution. These processes include nucleation from polycyclic aromatic hydrocarbon (PAH) dimers~\cite{blanquart2009,schuetz2002,frenklach1991,wang2011}, particle coagulation~\cite{kazakov1998,hmom2009}, soot growth from PAH condensation~\cite{blanquart2009,hmom2009}, soot growth through the $\ce{H}$-abstraction, $\ce{C2H2}$-addition (HACA) surface reaction mechanism~\cite{frenklach1985,frenklach1991}, oxidation~\cite{stanmore2001,neoh1981,kazakov1995}, and oxidation-induced fragmentation~\cite{neoh1985,mueller2011}. Note that molecular diffusion has been neglected, for a soot particle with a diameter of 10 nm has a Schmidt number of 287 at standard atmospheric conditions~\cite{friedlander2000}. This corresponds to a molecular diffusion coefficient of $D = 5.24\times 10^{-4}$ cm$^2$/sec, so the diffusion velocity of soot can be considered to be minimal. For LES, tracking the evolution of the moments is more relevant. A transport equation is derived by taking moments of \cref{eq:lesmodels:soot:ndf:moments},
\begin{equation}\label{eq:lesmodels:soot:ndf:momtransport}
  \tder{M_{x,y}} - \pder[]{x_j}\left( 0.55\frac{\nu}{T}\pder[T]{x_j}M_{x,y} \right) = \dot{M}_{x,y}.
\end{equation}
For convenience, a total velocity is defined that combines the convective and thermophoretic terms:
\begin{equation}\label{eq:lesmodels:soot:ndf:totvel}
  u_j^* = u_j - 0.55\frac{\nu}{T}\pder[T]{x_j}.
\end{equation}

Although the Method of Moments has a relatively smaller computational cost, it faces the problem of closure. Evaluation of the source term $\dot{M}_{x,y}$ in \cref{eq:lesmodels:soot:ndf:momtransport} depends on moments that are not directly solved for, requiring further modeling. Two moment methods that provide closure are the Method of Moments with Interpolative Closure (MOMIC)~\cite{frenklach2002,frenklach1987,frenklach1994} and the Direct Quadrature Method of Moments (DQMOM)~\cite{marchisio2005}.

In MOMIC, explicit transport equations are solved for a set of moments. Source term closure is achieved by polynomial interpolation of the logarithm of these known moments. For a bivariate parameterization of the NDF with volume and surface area, the interpolation is given by
\begin{equation}\label{eq:lesmodels:soot:ndf:momic}
  M_{x,y}^{\text{MOMIC}} = \exp\left( \sum\limits_{r=0}^{R} \sum\limits_{k=0}^{r} a_{r,k}x^k y^{r-k} \right),
\end{equation}
where $R$ is the order of the polynomial interpolation. The constant coefficients $a_{r,k}$ are determined by taking the logarithm of \cref{eq:lesmodels:soot:ndf:momic} and solving the resulting linear system with the known moments. This system is well-conditioned, although as the order $R$ is increased, the number of additional moment transport equations increases.

In DQMOM, the NDF is thought of as a summation of multi-dimensional Dirac delta functions, with the moments approximated by Gauss quadrature. Transport equations are solved for the weights and locations of these delta functions, rather than for the moments of the NDF. The moments of the NDF are given by
\begin{equation}\label{eq:lesmodels:soot:ndf:dqmom}
  M_{x,y}^{\text{DQMOM}} = \sum\limits_{i=1}^{P} N_i V_i^x S_i^y,
\end{equation}
where $P$ is the total number of delta functions used in the quadrature approximation, $N_i$ are the weights of the delta functions, and $V_i$ and $S_i$ are the abscissas (locations) of the delta functions. Closure of the source terms of the transport equations for the weights and abscissas is achieved by using the source terms of a specified independent set of moments. However, depending on the shape of the NDF at any point and the selected set of moments, the procedure of closure can involve inverting an ill-conditioned linear system.

As mentioned at the beginning of this section, the NDF has been found to be bimodal in sooting flames with persistent nucleation. The ability to predict the contributions of primary particles and larger aggregates is a significant characteristic that distinguishes between the aforementioned methods. MOMIC has been shown to be unsuccessful at capturing the influence of incipient particles, while DQMOM represents the bimodality of the NDF well~\cite{mueller2009}. Extensions to MOMIC that do account for bimodal distributions have been proposed~\cite{frenklach2002}, but these have been derived for univariate distributions and cannot be directly extended to bivariate or multi-variate distributions. DQMOM has this capability because it is able to allocate a single delta function for the incipient particles, which remains nearly fixed at the size of these nucleated particles~\cite{blanquart2009}. All other delta functions can then be allocated to resolve the distribution of larger particles and aggregates.

%A third moment method, the Hybrid Method of Moments (HMOM), adopts the advantages of DQMOM and MOMIC to provide a bimodal description of the NDF that is numerically well-conditioned. In this work, HMOM will be used.


%\subsection{Hybrid Method of Moments}
%\label{sec:lesmodels:soot:hmom}

A third moment method, the Hybrid Method of Moments (HMOM)~\cite{hmom2009}, adopts the advantages of DQMOM and MOMIC to provide a bimodal description of the NDF that is numerically well-conditioned. In HMOM, an arbitrary moment is given by
\begin{equation}\label{eq:lesmodels:soot:hmom:m}
  M_{x,y}^{\text{HMOM}} = N_0 V_0^x S_0^y + \exp\left( \sum\limits_{r=0}^{R} \sum\limits_{k=0}^{r} a_{r,k}x^k y^{r-k} \right),
\end{equation}
where the first term on the right-hand side is a delta function that is allocated to account for incipient soot particles. $N_0$ is the weight of this delta function that is located at the fixed coordinates given by $V_0$ and $S_0$. Note that in DQMOM, this delta function is not completely immobile. However, it does not move much, so no significant error is expected by fixing its location~\cite{hmom2009}. The second term on the right-hand side of \cref{eq:lesmodels:soot:hmom:m} is the contribution from MOMIC and accounts for the presence of larger soot aggregates. Unlike MOMIC, determination of the weight $N_0$ and the coefficients $a_{r,k}$ is nontrivial from a given set of moments, requiring the inversion of a nonlinear system. However, if $N_0$ is known, then the system given by \cref{eq:lesmodels:soot:hmom:m} can be inverted to evaluate $a_{r,k}$ with the ease of MOMIC. Therefore, as in DQMOM, the weight of the delta function $N_0$ is determined through a transport equation similar to the one defined in \cref{eq:lesmodels:soot:ndf:momtransport}. For first order polynomial interpolation ($R = 1$), the source term of the transport equation for $N_0$ is given by
\begin{equation}\label{eq:lesmodels:soot:hmom:ndot}
  \dot{N}_0 = \lim_{\alpha,\beta\to\infty} \frac{\dot{M}_{-\alpha,-\beta}}{V_0^{-\alpha} S_0^{-\beta}},
\end{equation}
where expressions for the source term in the numerator are obtained from the various processes that govern the evolution of soot particles. Note that in order to determine $a_{r,k}$ given $R = 1$, a set of three additional moments need to be evaluated. These quantities are the total number density $M_{0,0}$, the total particle volume $M_{1,0}$, and the total particle surface area $M_{0,1}$. Further information about the modeling of the physical processes in \cref{eq:lesmodels:soot:hmom:ndot} as well as other details of HMOM can be found in Mueller \etal~\cite{hmom2009,mueller2009,mueller2011}. In this work, HMOM will be used to provide closure.

\section{Turbulent Combustion}
\label{sec:lesmodels:combust}

In this section, effective strategies for incorporating combustion chemistry into LES are outlined. The turbulent combustion model must be able to accurately predict the thermal and chemical structure of the nonpremixed flame as fuel and oxidizer chemically react to form products such as carbon dioxide and water vapor. For sooting flames, the model also needs to account for the formation and evolution of soot precursors in addition to the soot itself. A brute-force method would involve selecting a chemical mechanism that models the above phenomena and solving coupled transport equations for all species within the mechanism. However, such an approach is impractical for realistic chemical mechanisms, which may contain thousands of species and tens of thousands of reactions~\cite{law2007}.

A more computationally tractable method is the Flamelet/Progress Variable (FPV) approach for nonpremixed combustion~\cite{pierce2004}. In the FPV approach, the local thermochemical state is described by the solutions to the steady flamelet equations~\cite{peters1984} and is parameterized by the mixture fraction $Z$ and reaction progress variable $C$:
\begin{equation}\label{eq:lesmodels:combust:fpv}
  \xi = \mc{F}(Z,C),
\end{equation}
where $\xi$ encompasses the thermochemical variables (density, temperature, species mass fractions, species source terms, etc.) and $\mc{F}$ is the functional relationship obtained from the solutions to the steady flamelet equations. To reduce computational overhead, these solutions are often computed \textit{a priori} and are stored in a database that is accessed during LES. Overall, the mapping given by \cref{eq:lesmodels:combust:fpv} has reduced the set of transported scalar variables to $Z$ and $C$, drastically decreasing the number of required transport equations.

However, this technique is insufficient for sooting flames due to its inability to predict the effects of thermal radiation. Soot evolution is characterized by long timescales and a sensitivity to the thermochemical state. Radiative thermal losses occur over similar temporal scales and can significantly alter the local thermochemical state. Therefore, radiative thermal losses cannot be neglected. The Radiation Flamelet/Progress Variable (RFPV) approach has extended the FPV approach to include these losses~\cite{ihme2008} and therefore will be the foundation for the turbulent combustion model in this work. The database of solutions to the steady flamelet equations is expanded to include solutions with radiative losses, and a heat loss parameter $H$ is now added to the local thermochemical equation of state
\begin{equation}\label{eq:lesmodels:combust:rfpv}
  \xi = \mc{G}(Z, C, H),
\end{equation}
where $\mc{G}$ is the functional relationship obtained from the augmented flamelet database.

For sooting flames, certain quantities such as the source terms in \cref{eq:lesmodels:soot:ndf:momtransport} are dependent on the soot scalars in addition to the thermochemical quantities. Therefore, a more general equation of state is formulated as~\cite{mueller2012}
\begin{equation}\label{eq:lesmodels:combust:jointeos}
  \phi = \mc{J}(\xi, \mc{M}_j),
\end{equation}
where $\phi$ is now any quantity that could depend jointly on the thermochemical variables $\xi$ and the soot scalars $\mc{M}_j$. The state vector $\mc{M}_j$ encompasses the moments $M_{x,y}$ and the weight of the delta function allocated for nucleating particles $N_0$. Mueller and Pitsch~\cite{subfilterpdf2011} proposed that \cref{eq:lesmodels:combust:jointeos} could be simplified by assuming that all of the quantities needed in the soot model can be written as the product of a function that depends only on the thermochemical state and a function that depends only on the soot scalars. Therefore, the functional relation $\mc{J}$ can be expressed as~\cite{mueller2012}
\begin{equation}\label{eq:lesmodels:combust:producteos}
  \phi = \mc{G}(Z, C, H)\mc{K}(\mc{M}_j),
\end{equation}
where $\mc{K}$ is unity if $\phi$ solely depends on the thermochemical quantities. This form will be utilized in \cref{ch:subfilter}.

%The reasoning behind this simplification will be revisited in \cref{ch:subfilter}.

In the following subsections, further details about the RFPV model will be discussed in the context of sooting flames. Definitions of the mixture fraction $Z$, progress variable $C$, and heat loss parameter $H$ are outlined to ensure that the thermochemical equation of state (\cref{eq:lesmodels:combust:rfpv}) is a unique function in the presence of soot formation. Afterwards, the nonpremixed flamelet equations are presented with additional terms to account for soot formation.


\subsection{Definition of Flame Structure Parameters}
\label{sec:lesmodels:combust:map}

For two-feed systems with a single fuel stream and single oxidizer stream, the mixture fraction can be defined to be a conserved scalar satisfying the transport equation~\cite{pitsch1998}
\begin{equation}\label{eq:lesmodels:combust:origz}
\trder{Z} = \pder[]{x_j}\left( \rho D_Z \pder[Z]{x_j} \right),
\end{equation}
where $Z$ is one in the fuel stream and zero in the oxidizer stream and $D_Z$ is chosen such that $Le_Z$ is unity. Such a definition is advantageous, for no assumptions about the species Lewis numbers have been made. However, the mixture fraction cannot be defined as a conserved scalar in sooting flames. During the nucleation of soot, PAH are extracted from the gas-phase, causing the mixture to be locally leaned. The approach of Mueller and Pitsch~\cite{mueller2012} is utilized to account for this phenomenon, where a source term is introduced into \cref{eq:lesmodels:combust:origz}:
\begin{equation}\label{eq:lesmodels:combust:map:z}
  \trder{Z} = \pder[]{x_j}\left( \rho D_Z \pder[Z]{x_j} \right) + \dot{m}_Z.
\end{equation}
The source term $\dot{m}_Z$ is defined similarly to Bilger's mixture fraction based on element mass fractions~\cite{bilger1989,mueller2012}:
\begin{equation}\label{eq:lesmodels:combust:map:zsource}
  \dot{m}_Z = \frac{\frac{\dot{m}_{\ce{C}} - \dot{\rho}Z_{\ce{C},\ce{O}}}{\nu_{\ce{C}}W_{\ce{C}}} + \frac{\dot{m}_{\ce{H}} - \dot{\rho}Z_{\ce{H},\ce{O}}}{\nu_{\ce{H}}W_{\ce{H}}} + 2\frac{\dot{\rho}Z_{\ce{O},\ce{O}} - \dot{m}_{\ce{O}}}{\nu_{\ce{O2}}W_{\ce{O2}}}}{\frac{Z_{\ce{C},\ce{F}} - Z_{\ce{C},\ce{O}}}{\nu_{\ce{C}}W_{\ce{C}}} + \frac{Z_{\ce{H},\ce{F}} - Z_{\ce{H},\ce{O}}}{\nu_{\ce{H}}W_{\ce{H}}} + 2\frac{Z_{\ce{O},\ce{O}} - Z_{\ce{O},\ce{F}}}{\nu_{\ce{O2}}W_{\ce{O2}}}},
\end{equation}
where $\dot{m}_k$ are element mass source terms, $\dot{\rho}$ is the source term in the continuity equation for the removal of gas-phase PAH, $Z_{k,l}$ are the mass fractions of element $k$ in the stream indicated by $l$, $\nu_k$ are the stoichiometric coefficients of the global reaction between fuel and oxidizer, and $W_{k}$ are the element molar masses. In the limit of unity Lewis numbers for all species, \cref{eq:lesmodels:combust:map:z,eq:lesmodels:combust:map:zsource} provide a description of the mixture fraction that is equivalent to Bilger's mixture fraction based on element mass fractions~\cite{bilger1989}. This definition of mixture fraction ultimately leads to additional terms in the flamelet equations of \cref{sec:lesmodels:combust:flamelet} that acknowledge the presence of soot.

The reaction progress variable has been defined in previous works~\cite{pierce2004,ihme2008} as
\begin{equation}\label{eq:lesmodels:combust:map:cprev}
  C_{\text{conv}} = Y_{\ce{CO2}} + Y_{\ce{CO}} + Y_{\ce{H2O}} + Y_{\ce{H2}}.
\end{equation}
However, this conventional definition is not appropriate in the context of sooting flames. PAH have large amounts of carbon relative to hydrogen, so their removal from the gas-phase tends to lower the local $\ce{C}$/$\ce{H}$ ratio. As a result, the local effective fuel composition is changed. To address this phenomena, Mueller and Pitsch~\cite{mueller2012} redefined the reaction progress variable through the transport equation given by
\begin{equation}\label{eq:lesmodels:combust:map:c}
  \trder{C} = \pder[]{x_j}\left( \rho D_C \pder[C]{x_j} \right) + \frac{\dot{m}_{C_{\text{conv}}}}{C^*},
\end{equation}
where the diffusion coefficient $D_C$ is chosen such that $Le_C$ is unity, $\dot{m}_{C_{\text{conv}}}$ is the source term for the conventional progress variable as defined in \cref{eq:lesmodels:combust:map:cprev}, and $C^*$ is a normalizing factor that accounts for the change in stoichiometry due to the removal of gas-phase PAH. $C^*$ ensures that the equation of state (\cref{eq:lesmodels:combust:jointeos}) remains unique even as the local $\ce{C}$/$\ce{H}$ ratio changes due to the nucleation of soot. In non-sooting flames, $C^*$ is constant because the local $\ce{C}$/$\ce{H}$ ratio is constant (in the absence of differential diffusion effects). Further details about $C^*$ can be found in Mueller and Pitsch~\cite{mueller2012}.
%With \cref{eq:lesmodels:combust:map:c}, the progress variable adopts the value of zero in both the fuel and oxidizer streams.

The remaining mapping quantity to fully specify the thermochemical state given by \cref{eq:lesmodels:combust:rfpv} is the heat loss parameter $H$. This quantity provides a measure for the extent to which radiative thermal losses affect the enthalpy. A simple definition for the heat loss parameter is the enthalpy itself or the enthalpy deficit. The latter option is more convenient, for the adiabatic boundary has a well-defined value of zero in the limit of unity Lewis numbers for all species. However, the presence of soot can affect the adiabatic boundary for the enthalpy deficit, and the inclusion of molecular transport introduces complications associated with tracking the enthalpy fluxes of different species. For either situation, the transport equation definition for the heat loss parameter $H$ from Mueller and Pitsch~\cite{mueller2012} is more convenient:
\begin{equation}\label{eq:lesmodels:combust:map:heatloss}
  \trder{H} = \pder[]{x_j}\left( \rho D_H \pder[H]{x_j} \right) + \dot{\rho}H + \dot{q}_{\text{RAD}},
\end{equation}
where the diffusion coefficient $D_H$ is chosen such that $Le_H$ is unity and $\dot{q}_{\text{RAD}}$ is the radiation source term. The second term on the right-hand side is present to ensure that $H$ is constant everywhere if $\dot{q}_{\text{RAD}}$ is zero. Initially, the heat loss parameter is zero throughout the nonpremixed system. As the cumulative effect of radiation increases, the heat loss parameter acquires increasingly negative values.


\subsection{Flamelet Equations}
\label{sec:lesmodels:combust:flamelet}

The thermochemical equation of state obtained from the RFPV approach (\cref{eq:lesmodels:combust:rfpv}) is specified by the solutions to the steady flamelet equations. These solutions, which are computed \textit{a priori} and stored in a database that is accessed during LES, provide a description of the thermal and chemical structure of a nonpremixed flame. The species flamelet equation is derived from the conservation equation for the species mass fractions:
\begin{equation}\label{eq:lesmodels:combust:flamelet:consy}
  \trder{Y_k} = -\pder[]{x_j}\left( \rho Y_k V_{k,j} \right) + \dot{m}_k,
\end{equation}
where $Y_k$ is the mass fraction of species $k$, $V_{k,j}$ is the diffusion velocity of species $k$ in the direction indicated by $j$, and $\dot{m}_k$ is the chemical source term for species $k$. The diffusion velocity has contributions from mass diffusion in the presence of concentration gradients, pressure gradients, body forces, and temperature gradients (a second order effect known as Soret diffusion)~\cite{law2006}. By only considering diffusion through concentration gradients, the diffusion velocity is described by the Stefan-Maxwell equation. However, evaluating the linear system associated with the Stefan-Maxwell equation can be computationally expensive, so further simplifications are often made with the Curtiss-Hirschfelder approximation~\cite{curtiss1949} or Fick's law of diffusion~\cite{fick1855}. The Curtiss-Hirschfelder approximation is given by
\begin{equation}\label{eq:lesmodels:combust:flamelet:curtisshirschfelder}
  V_{k,j} = -\frac{1}{X_k}D_k\pder[X_k]{x_j} + \sum\limits_{k} \frac{Y_k}{X_k}D_k\pder[X_k]{x_j},
\end{equation}
where $X_k$ is the mole fraction of species $k$, $D_k = (1 - Y_k)/\sum\limits_{k \neq l} (X_k/D_{k,l})$ is the mixture-averaged diffusivity for species $k$, and the second term on the right-hand side is a correction velocity to ensure the conservation of mass. By assuming constant unity Lewis numbers, \cref{eq:lesmodels:combust:flamelet:curtisshirschfelder} is simplified to Fick's law:
\begin{equation}\label{eq:lesmodels:combust:flamelet:fick}
  V_{k,j} = -\frac{1}{Y_k}D\pder[Y_k]{x_j},
\end{equation}
where $D$ is the constant diffusion coefficient. In \cref{ch:transport}, this assumption will be revisited.

% Fick's law assumes all species have the same constant Lewis number and is given by
%% \begin{equation}\label{eq:lesmodels:combust:flamelet:fick}
%%   V_{k,j} = -\frac{1}{Y_k}D\pder[Y_k]{x_j},
%% \end{equation}
%% where $D$ is the constant diffusion coefficient. For turbulent flames at sufficiently high Reynolds numbers, it is often assumed that the turbulence mixes all species indiscriminately~\cite{pitsch19981057}. Therefore, $D$ will be selected such that $Le_k$ is unity. Note that, in \cref{ch:transport}, this assumption will be revisited.

The temperature flamelet equation is obtained from the conservation equation for energy in terms of temperature under the assumption of constant thermodynamic pressure:
\begin{equation}\label{eq:lesmodels:combust:flamelet:const}
  \trder{T} = \frac{1}{c_p}\left[ \pder[]{x_j}\left( \lambda\pder[T]{x_j} \right) + \sum\limits_{k} \rho c_{p,k} D\pder[Y_k]{x_j}\pder[T]{x_j} - \sum\limits_{k} h_k\dot{m}_k + \dot{q}_{\text{RAD}} \right],
\end{equation}
where $\lambda$ is the thermal conductivity, $c_p$ is the specific heat at constant pressure, $c_{p,k}$ is the specific heat at constant pressure for species $k$, and $h_k$ is the specific enthalpy of species $k$. Fick's law is implemented in the second term on the right-hand side.

The flamelet equations are derived by transforming the physical space coordinate system to a mixture fraction-based coordinate system~\cite{peters1984}. It is assumed that the gradients along the flame surface are negligible compared to the gradients normal to the flame surface, which allows the flame structure to be described with only the mixture fraction. For sooting flames, Mueller~\cite{muellerphd} determined that the flamelet equations possess additional terms due to the source terms in \cref{eq:lesmodels:combust:map:z} and the continuity equation. For unity Lewis numbers, \cref{eq:lesmodels:combust:flamelet:consy} is transformed to
\begin{equation}\label{eq:lesmodels:combust:flamelet:flamelety}
  \rho\pder[Y_k]{\tau} = \frac{\rho\chi}{2}\sder[Y_k]{Z} + \dot{m}_k - \dot{\rho}Y_k + (\dot{\rho}Z - \dot{m}_Z)\pder[Y_k]{Z},
\end{equation}
where $\chi = 2D_Z(\partial Z/\partial x_j)^2$ is the scalar dissipation rate, the third term on the right-hand side is due to the source term of the continuity equation, and the last term on the right-hand side is a convective term in mixture fraction space due to the density and mixture fraction source terms. The latter two new terms ensure that mass is conserved and that the database of flamelet solutions remains unique, although they do not play a substantial role in the dynamics of \cref{eq:lesmodels:combust:flamelet:flamelety}.

Similarly, \cref{eq:lesmodels:combust:flamelet:const} is transformed to obtain the flamelet equation for temperature:
\begin{equation}\label{eq:lesmodels:combust:flamelet:flamelett}
  \begin{split}
    \rho c_p \pder[T]{\tau} &= \frac{\rho c_p \chi}{2}\sder[T]{Z} + \frac{\rho\chi}{2}\pder[c_p]{Z}\pder[T]{Z} - \sum\limits_{k} h_k\dot{m}_k + \sum\limits_{k} \frac{\rho c_{p,k}\chi}{2}\pder[Y_k]{Z}\pder[T]{Z} + \dot{q}_{\text{RAD}} \\
    &+ \dot{H} + c_p(\dot{\rho}Z - \dot{m}_Z)\pder[T]{Z},
  \end{split}
\end{equation}
where $\dot{H}$ is the enthalpy source term due to the extraction of PAH from the gas-phase and the last term on the second line is the convection term in mixture fraction space that is analogous to the one discussed in \cref{eq:lesmodels:combust:flamelet:flamelety}. In addition to these quantities, flamelet equations for the progress variable $C$ and heat loss parameter $H$ need to be derived. Performing the coordinate transform on \cref{eq:lesmodels:combust:map:c} provides the flamelet equation for the progress variable:
\begin{equation}\label{eq:lesmodels:combust:flamelet:flameletc}
  \rho\pder[C]{\tau} = \frac{\rho\chi}{2}\sder[C]{Z} + \frac{\dot{m}_{C_{\text{conv}}}}{C^*} - \dot{\rho}C + (\dot{\rho}Z - \dot{m}_Z)\pder[C]{Z},
\end{equation}
and performing a similar operation on \cref{eq:lesmodels:combust:map:heatloss} leads to the flamelet equation for the heat loss parameter:
\begin{equation}\label{eq:lesmodels:combust:flamelet:flameleth}
  \rho\pder[H]{\tau} = \frac{\rho\chi}{2}\sder[H]{Z} + (\dot{\rho}Z - \dot{m}_Z)\pder[H]{Z} + \dot{q}_{\text{RAD}}.
\end{equation}
Since the latter two quantities are not explicit functions of the species mass fractions and temperature, evaluation of \cref{eq:lesmodels:combust:flamelet:flameletc,eq:lesmodels:combust:flamelet:flameleth} is required to produce the complete parameterization of the database of flamelet solutions.

The database is typically filled using the procedure of Ihme and Pitsch~\cite{ihme2008}, where the unsteady forms of \crefrange{eq:lesmodels:combust:flamelet:flamelety}{eq:lesmodels:combust:flamelet:flameleth} are evaluated in order to fill the heat loss parameter space. However, such a procedure is computationally expensive due to the need to resolve the temporal coordinate. In this work, the approach of Carbonell \etal~\cite{carbonell2009} is adopted instead, where the steady forms of \crefrange{eq:lesmodels:combust:flamelet:flamelety}{eq:lesmodels:combust:flamelet:flameleth} are solved and the radiation source term of \cref{eq:lesmodels:combust:flamelet:flamelett} is weighted by a radiation loss coefficient $\Omega_{\text{RAD}} \in [0,1]$ to span the heat loss parameter space. The adiabatic form of \cref{eq:lesmodels:combust:flamelet:flamelett} is specified by $\Omega_{\text{RAD}} = 0$. Radiative thermal losses are introduced by increasing the value of $\Omega_{\text{RAD}}$ while still solving the steady forms of \crefrange{eq:lesmodels:combust:flamelet:flamelety}{eq:lesmodels:combust:flamelet:flameleth}. Note that in the limit of very small values of scalar dissipation rate, no steady solutions with radiation exist.

The adiabatic upper, middle, and lower branches of the S-curve as well as nonadiabatic solutions are shown in \cref{fig:lesmodels:combust:flamelet:tvschi}. These flamelet solutions are uniquely parameterized by $Z$, $C$, and $H$, with the latter two quantities obtained from evaluating \cref{eq:lesmodels:combust:flamelet:flameletc,eq:lesmodels:combust:flamelet:flameleth}. Validation of uniqueness with the specified definitions of mixture fraction and progress variable is available in Mueller~\cite{muellerphd}.

%% The aforementioned adiabatic and nonadiabatic, steady, stable-burning solutions form the upper branches of the S-curve as shown in \cref{fig:lesmodels:combust:flamelet:tvschi}. The S-curve contains the upper, middle, and lower branches that are formed by the flamelet solutions in the flame temperature/scalar dissipation rate plane at an arbitrary mixture fraction. The adiabatic, steady, unstable-burning, middle branch and adiabatic, steady, non-burning, lower branch are obtained through \crefrange{eq:lesmodels:combust:flamelet:flamelety}{eq:lesmodels:combust:flamelet:flameleth} as well, although the initial positions within the $T|Z_{st}$--$\chi|Z_{st}$ plane are different than those of the upper branch solutions. Thus, the set of solutions to the laminar nonpremixed flamelet equations is uniquely parameterized by $Z$, $C$, and $H$, with the latter two quantities obtained from evaluating \cref{eq:lesmodels:combust:flamelet:flameletc,eq:lesmodels:combust:flamelet:flameleth}. Validation of uniqueness with the specified definitions of mixture fraction and progress variable is available in Mueller~\cite{muellerphd}.

\begin{figure}[htb]
  \centering
  \includegraphics[width=0.6\linewidth]{ch-lesmodels/figures/TvsChi-connected}
  \caption[Flamelet Solutions of RFPV Model, $T|Z_{st}$ vs. $\chi|Z_{st}$]{Solutions to the flamelet equations of the RFPV model for a $\ce{C2H4}$/$\ce{H2}$/$\ce{N2}$ [40/41/19 by volume] nonpremixed flame at atmospheric pressure~\cite{mahmoud2017}. The adiabatic solutions are the red circles, magenta triangles, and the black square, which represent the stable-burning, unstable-burning, and non-burning solutions, respectively. The nonadiabatic solutions are the diamonds, where each set of solutions linked with a dashed line possesses a distinct, nonzero $\Omega_{\text{RAD}}$.}
  \label{fig:lesmodels:combust:flamelet:tvschi}
\end{figure}

\section{Presumed PDF Approach}
\label{sec:lesmodels:presumedpdf}

LES only resolves the larger features of a turbulent flow, so small-scale interactions between soot, turbulence, and combustion chemistry need to be modeled. These subfilter interactions are modeled with a density-weighted joint PDF $\tf{P}$ of the thermochemical and soot variables. Density-weighted filtered (Favre filtered) functions of these quantities are obtained from a convolution of the equation of state with $\tf{P}$:
\begin{equation}\label{eq:lesmodels:presumedpdf:filteredfuncs}
  \tf{\phi}(\xi,\mc{M}_j) = \iint \mc{J}(\xi,\mc{M}_j)\tf{P}(\xi,\mc{M}_j) d\xi d\mc{M}_j,
\end{equation}
where $\xi$ is the vector of thermochemical variables and $\mc{M}_j$ is the state-space vector encompassing the moments $M_{x,y}$ and the weight of the delta function $N_0$ allocated for nucleating particles.

The joint subfilter PDF $\tf{P}$ is unknown and must be modeled using transported PDF or presumed PDF methods. In the transported PDF approach, a transport equation is solved for the subfilter PDF~\cite{pope1985,pope1991}. Often this equation is high-dimensional, and therefore conventional Eulerian discretization techniques are computationally intractable. Instead, Lagrangian particle methods are employed, where a set of notional particles that track fluid particles is solved for. The advantage of the class of transported PDF methods is that all local phenomena are naturally described by the subfilter PDF, which provides information about subfilter fluctuations and correlations at a single point. Thus, chemical source terms do not require additional closure because chemical reactions are one-point phenomena. However, processes such as molecular mixing are dependent on gradients and therefore are inherently two-point, non-local phenomena. Mixing models have been proposed such as the Interaction with Exchange of the Mean (IEM)~\cite{dopazo1974} and Modified Curl~\cite{janicka1979}, where quantities are relaxed to local filtered values to mimic homogenization through diffusion. However, such crude approaches fail to use information about localness in composition space. More complex mixing models using the concept of the Euclidean minimum spanning tree~\cite{subramaniam1998} or the shadow position~\cite{pope2013} introduce localness in composition space, but still suffer from the large computational cost associated with evolving the Lagrangian particles.

In contrast, presumed PDF methods assume a form of the joint subfilter PDF. Reasoning based on physical insights about subfilter interactions between the scalars and turbulence is utilized to deduce the form of the PDF. Previous works have indicated that the beta distribution is a good model for the thermochemical portion of the subfilter PDF when considering passive scalars in a two-feed system~\cite{cook1994,jimenez1997,wall2000}. Ihme et al.~\cite{ihme2005} arrived at the same conclusion with the FPV approach for reactive scalars. Since presumed PDF methods do not rely on ad hoc mixing models, they generally match or exceed the performance of transported PDF methods, provided that the form of the PDF is appropriate. Thus, the presumed PDF approach will be adopted in this work.

An approximate form of the overall joint subfilter PDF is determined by first decomposing $\tf{P}$ into a thermochemical PDF and a PDF of the soot scalars conditioned on the thermochemical variables. By Bayes' theorem,
\begin{equation}\label{eq:lesmodels:presumedpdf:bayes}
  \tf{P}(\xi,\mc{M}_j) = \tf{P}(\xi)P(\mc{M}_j|\xi),
\end{equation}
where $\tf{P}(\xi)$ includes a beta distribution and $P(\mc{M}_j|\xi)$ is unknown. The form of the latter has been approximated previously through consideration of the dynamics of soot. In DNS studies of an \textit{n}-heptane/air turbulent nonpremixed jet, PAH were found to have much slower formation chemistry compared to the main heat-releasing chemistry, which is represented by the thermochemical variables $\xi$~\cite{attili2014,bisetti2012}. PAH are soot precursors and therefore soot itself is characterized by even longer timescales. Consequently, Mueller and Pitsch proposed that the time scales of the soot scalars and thermochemical variables are disparate enough such that the former should not depend on the latter~\cite{subfilterpdf2011}. This argument shall be used to simplify the subfilter PDF of the soot scalars conditioned on $\xi$ to a marginal PDF of only the soot scalars. The expression for the spatially Favre filtered functions of \cref{eq:lesmodels:presumedpdf:filteredfuncs} then becomes
\begin{equation}\label{eq:lesmodels:presumedpdf:separated}
  \tf{\phi}(\xi,\mc{M}_j) = \iint \mc{J}(\xi,\mc{M}_j)\tf{P}(\xi)P(\mc{M}_j) d\xi d\mc{M}_j.
\end{equation}
By replacing the functional relation $\mc{J}$ with \cref{eq:lesmodels:combust:producteos}, \cref{eq:lesmodels:presumedpdf:separated} can be further simplified to
\begin{equation}\label{eq:lesmodels:presumedpdf:indep}
  \tf{\phi}(Z,C,H,\mc{M}_j) = \iiint \mc{G}(Z,C,H)\tf{P}(Z,C,H) dHdCdZ \times \int \mc{K}(\mc{M}_j)P(\mc{M}_j) d\mc{M}_j,
\end{equation}
where the thermochemical and soot components are now completely independent. Mueller and Pitsch noted that the time scale separation argument could be violated during the oxidation of soot, when there are enhanced interactions between soot and the major gas-phase chemistry, and during surface growth near the flame. These situations will be examined more closely in \cref{ch:subfilter}.

The challenging task of modeling the joint subfilter PDF has been reduced to developing models for the thermochemical subfilter PDF and the soot subfilter PDF. Discussion of the latter will be deferred to \cref{ch:subfilter}. The thermochemical subfilter PDF for the RFPV approach is obtained from Ihme and Pitsch~\cite{ihme2008}. First, two quantities are introduced to uniquely identify each flamelet solution in the database of thermochemical states: $\Lambda = C(Z_{st})$ and $\Phi = H(Z_{st})$. The thermochemical equation of state \cref{eq:lesmodels:combust:rfpv} then becomes
\begin{equation}\label{eq:lesmodels:presumedpdf:eos}
  \xi = \mc{G}(Z, C, H) = \mg(Z, \Lambda, \Phi).
\end{equation}
The spatially Favre filtered thermochemical functions are given by
\begin{equation}\label{eq:lesmodels:presumedpdf:filteredeos}
  \begin{split}
    \tf{\xi} &= \iiint \mc{G}(Z,C,H)\tf{P}(Z,C,H) dHdCdZ \\
    &= \iiint \mg(Z,\Lambda,\Phi)\tf{P}(Z,\Lambda,\Phi) d\Phi d\Lambda dZ.
  \end{split}
\end{equation}
Since $Z$, $\Lambda$, and $\Phi$ are defined to be independent, the thermochemical subfilter PDF can be expressed as the product of three marginal distributions:
\begin{equation}\label{eq:lesmodels:presumedpdf:trimarg}
  \tf{P}(Z,\Lambda,\Phi) = \beta(Z;\tf{Z},\tf{Z_{\text{V}}})\delta(\Lambda - \tf{\Lambda})\delta(\Phi - \tf{\Phi}),
\end{equation}
where the mixture fraction is modeled with a beta distribution~\cite{cook1994,jimenez1997,wall2000}, the subfilter mixture fraction variance is defined as $\tf{Z_{\text{V}}} = \tf{Z^2} - \tf{Z}^2$, and $\Lambda$ and $\Phi$ are modeled with delta distributions~\cite{ihme2008}. Assuming \cref{eq:lesmodels:combust:rfpv} is unique, a bijective inversion may be used to conveniently reexpress a dependence on $\tf{\Lambda}$ and $\tf{\Phi}$ as a dependence on $\tf{C}$ and $\tf{H}$. In practical implementation, each flamelet solution in the database is convoluted with the beta distribution for the mixture fraction and tabulated as a function of the filtered mixture fraction, subfilter mixture fraction variance, filtered progress variable, and filtered heat loss parameter.

The transport equation for the filtered mixture fraction is derived by spatially filtering \cref{eq:lesmodels:combust:map:z} and is given by
\begin{equation}\label{eq:lesmodels:presumedpdf:filteredz}
  \ftrder{\tf{Z}} = \dsff[\tf{Z}]{\tf{u_j Z}} + \fdt[Z]{\tf{Z}} + \fst[m]{Z},
\end{equation}
where the first and last terms on the right-hand side are unclosed. The former is the subfilter flux, which can be modeled with a dynamic procedure~\cite{germano1991,lilly1992,moin1991}. The filtered source term is closed through the presumed subfilter PDF given in \cref{eq:lesmodels:presumedpdf:trimarg}. Similarly, the transport equations for the filtered progress variable and filtered heat loss parameter are given by
\begin{equation}\label{eq:lesmodels:presumedpdf:filteredc}
  \ftrder{\tf{C}} = \dsff[\tf{C}]{\tf{u_j C}} + \fdt[C]{\tf{C}} + \mean{\left( \frac{\dot{m}_{C_{\text{conv}}}}{C^*} \right)}
\end{equation}
and
\begin{equation}\label{eq:lesmodels:presumedpdf:filteredh}
  \ftrder{\tf{H}} = \dsff[\tf{H}]{\tf{u_j H}} + \fdt[H]{\tf{H}} + \mean{\dot{\rho}H} + \fst[q]{\text{RAD}},
\end{equation}
respectively. The remaining database parameter is the subfilter mixture fraction variance, which will be procured in the manner of Mueller and Pitsch~\cite{mueller2012}. Rather than solving for the subfilter variance directly, a transport equation for the filtered square of the mixture fraction is evaluated:
\begin{equation}\label{eq:lesmodels:presumedpdf:filteredzsq}
  \begin{split}
    \ftrder{\tf{Z^2}} &= \dsff[\tf{Z^2}]{\tf{u_j Z^2}} + \fdt[Z]{\tf{Z^2}} \\
    &- 2\mean{\rho}\tf{D}_Z \pder[\tf{Z}]{x_j}\pder[\tf{Z}]{x_j} - \mean{\rho}\tf{\chi}_{\text{sgs}} - \mean{\dot{\rho}Z^2} + 2\mean{\dot{m}_Z Z},
  \end{split}
\end{equation}
where $\tf{\chi}_{\text{sgs}}$ is the subfilter scalar dissipation rate. This quantity is estimated with a linear relaxation model~\cite{ihme200890}:
\begin{equation}\label{eq:lesmodels:presumedpdf:chisgs}
  \tf{\chi}_{\text{sgs}} = \text{C}_{\chi}\frac{\nu_t}{\Delta^2}\tf{Z_{\text{V}}},
\end{equation}
where $\text{C}_{\chi} \approx 20$, the mechanical timescale is determined from the eddy viscosity $\nu_t$ and filter width $\Delta$, and $\tf{Z_{\text{V}}}$ is the subfilter mixture fraction variance as defined previously.

\section{Polycyclic Aromatic Hydrocarbons}
\label{sec:lesmodels:pah}

Understanding the formation and evolution of soot is central to this work, so accurately modeling the generation and consumption of gas-phase PAH that participate in nucleation and condensation is vital. A simple LES model for tracking the evolution of PAH would involve obtaining the PAH mass fraction from the RFPV combustion model. However, Attili \etal~\cite{attili2014} and Bisetti \etal~\cite{bisetti2012} noted in DNS studies of a nitrogen-diluted, \textit{n}-heptane/air turbulent nonpremixed flame that the PAH mass fraction deviates significantly from the values calculated by the steady flamelet approximation. The chemistry of PAH such as naphthalene is much slower than the chemistry of species that participate in the significantly exothermic reactions. As a result, PAH are unable to immediately adjust to the rapidly changing scalar dissipation rate field. They require a longer time to respond to the fluctuating field, so the unsteady term in the flamelet equations cannot be neglected.

To account for these unsteady phenomena, Mueller and Pitsch~\cite{mueller2012} developed a transport equation model for PAH that was inspired by a similar approach for slowly evolving $\ce{NO}$~\cite{ihme2008}. The spatially filtered transport equation is given by
\begin{equation}\label{eq:lesmodels:pah:pahtrans}
  \ftrder{\tf{Y}_{\text{PAH}}} = -\dsff[\tf{Y}_{\text{PAH}}]{\tf{u_j Y}_{\hspace{-3pt}\text{PAH}}} + \fdt[\text{PAH}]{\tf{Y}_{\text{PAH}}} + \fst[m]{\text{PAH}},
\end{equation}
where $\tf{Y}_{\text{PAH}}$ is the filtered mass fraction of a lumped PAH species and $\fst[m]{\text{PAH}}$ is the sum of the chemical source terms of all the PAH species. Under the assumption of unity Lewis number, all PAH species will have a diffusivity governed by $D_{\text{PAH}}$ and \cref{eq:lesmodels:pah:pahtrans} is the exact transport equation for a lumped PAH. However, if all PAH do not have the same molecular diffusivity, then the diffusivity of naphthalene is preferred for $D_{\text{PAH}}$ as it is generally the PAH species with the largest concentration. In this circumstance, \cref{eq:lesmodels:pah:pahtrans} is an approximate model for the evolution of the combined PAH mass fraction.

The unclosed terms in \cref{eq:lesmodels:pah:pahtrans} include the first and third terms on the right-hand side. The former is simply the subfilter flux and can be modeled using a conventional dynamic procedure as mentioned previously. Following Mueller and Pitsch~\cite{mueller2012}, the filtered chemical source term can be decomposed into a production term $\dot{m}_{+}$, a consumption term $\dot{m}_{-}$, and an additional consumption term due to the extraction of PAH from the gas-phase to form soot (dimerization) $\dot{m}_{D}$. For each individual PAH species $k$, the chemical production term is independent of $Y_k$, the chemical consumption term depends linearly on $Y_k$, and the dimerization term depends quadratically on $Y_k$. The source term of the lumped species will be decomposed similarly:
\begin{equation}\label{eq:lesmodels:pah:pahsrc}
  \fst[m]{\text{PAH}} = \mean{(\dot{m}_{+})} + \mean{\left( \frac{\dot{m}_{-}}{Y_{\text{PAH}}} \right) Y_{\text{PAH}}} + \mean{\left( \frac{\dot{m}_{D}}{{Y_{\text{PAH}}}^2} \right) {Y_{\text{PAH}}}^2},
\end{equation}
where the terms in parentheses are obtained directly from the RFPV combustion model because they are nearly independent of the PAH mass fraction. The last two terms on the right-hand side of \cref{eq:lesmodels:pah:pahsrc} are unclosed filtered correlations. Mueller and Pitsch~\cite{mueller2012} have proposed closure through a scale-similarity assumption between the transported PAH mass fraction and the PAH mass fraction computed from the steady flamelet equations. This model is given by
\begin{equation}\label{eq:lesmodels:pah:similarity}
  \frac{\tf{Y}_{\text{PAH}}}{\tf{Y}_{\text{PAH}}^{\text{RFPV}}} = \frac{Y_{\text{PAH}}^{''}}{Y_{\text{PAH}}^{\text{RFPV}''}},
\end{equation}
where the subscript $^{\text{RFPV}}$ designates quantities that are from the RFPV combustion model and the subscript $^{''}$ indicates the ``fluctuating'' LES quantities given by
\begin{equation}\label{eq:lesmodels:pah:residual}
  Y_{\text{PAH}}^{''} = Y_{\text{PAH}} - \tf{Y}_{\text{PAH}}.
\end{equation}
With this closure model, the filtered source term of the lumped PAH species is
\begin{equation}\label{eq:lesmodels:pah:pahsrcmodeled}
  \fst[m]{\text{PAH}} = \mean{\dot{m}}_{+}^{\text{RFPV}} + \mean{\dot{m}}_{-}^{\text{RFPV}}\left( \frac{\tf{Y}_{\text{PAH}}}{\tf{Y}_{\text{PAH}}^{\text{RFPV}}} \right) + \mean{\dot{m}}_{D}^{\text{RFPV}}\left( \frac{\tf{Y}_{\text{PAH}}}{\tf{Y}_{\text{PAH}}^{\text{RFPV}}} \right)^2.
\end{equation}

