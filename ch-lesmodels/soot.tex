\section{Soot}
\label{sec:lesmodels:soot}

%\subsection{Modeling the Particle Size Distribution}
%\label{sec:lesmodels:soot:ndf}

The distribution of soot particle sizes in a system can be described by the Number Density Function (NDF), $N$. Previous studies have suggested that the soot NDF is bimodal, accounting for newly formed primary particles from persistent nucleation and larger aggregrates from various growth modes~\cite{zhao2003,zhao2005,netzell2007}. Incipient particles are roughly spherical with diameters on the order of nanometers, while aggregates comprised of monodisperse primaries have fractal geometries with length scales on the order of hundreds of nanometers~\cite{vanderwal1999}. In order to account for the geometrical complexities, the NDF requires a multivariate parameterization such as mass/volume and surface area~\cite{patterson2007,mueller2009,hmom2009}, the number of agglomerates and primary particles~\cite{park2004}, or even volume, surface area, and the number of surface hydrogenated carbon sites~\cite{blanquart2009}. In this work, a bivariate parameterization (volume $V$ and surface area $S$) of the NDF is used to give the functional form $N = N(V, S; t, x_j)$. Note that the NDF cannot be solved for directly due to its high dimensionality (six degrees of freedom), so statistical models are required.

For application to LES, the Method of Moments is one of the only tractable techniques due to its lower computational expense. In this approach, mean quantities of the NDF are solved for and complete information about the distribution is lost. These quantities are moments given by
\begin{equation}\label{eq:lesmodels:soot:ndf:moments}
  M_{x,y} = \sum\limits_{i=0}^\infty V_i^x S_i^y N_i,
\end{equation}
where discrete summation over $i$ implies summation over all particles sizes and $N_i$ is the number density of particles with volume $V_i$ and surface area $S_i$. The subscripts $x$ and $y$ indicate the order of the moment in volume and surface area, respectively.

The NDF evolves according to the Population Balance Equation (PBE)~\cite{friedlander2000}
\begin{equation}\label{eq:lesmodels:soot:ndf:pbe}
  \tder{N_i} - \pder[]{x_j}\left( 0.55\frac{\nu}{T}\pder[T]{x_j}N_i \right) = \dot{N_i},
\end{equation}
where the third term is the thermophoresis of particles~\cite{waldmann1966} and the source term on the right-hand side incorporates the various physical and chemical processes that govern soot evolution. These processes include nucleation from polycyclic aromatic hydrocarbon (PAH) dimers~\cite{blanquart2009,schuetz2002,frenklach1991,wang2011}, particle coagulation~\cite{kazakov1998,hmom2009}, soot growth from PAH condensation~\cite{blanquart2009,hmom2009}, soot growth through the $\ce{H}$-abstraction, $\ce{C2H2}$-addition (HACA) surface reaction mechanism~\cite{frenklach1985,frenklach1991}, oxidation~\cite{stanmore2001,neoh1981,kazakov1995}, and oxidation-induced fragmentation~\cite{neoh1985,mueller2011}. Note that molecular diffusion has been neglected, for a soot particle with a diameter of 10 nm has a Schmidt number of 287 at standard atmospheric conditions~\cite{friedlander2000}. This corresponds to a molecular diffusion coefficient of $D = 5.24\times 10^{-4}$ cm$^2$/sec, so the diffusion velocity of soot can be considered to be minimal. For LES, tracking the evolution of the moments is more relevant. A transport equation is derived by taking moments of \cref{eq:lesmodels:soot:ndf:moments},
\begin{equation}\label{eq:lesmodels:soot:ndf:momtransport}
  \tder{M_{x,y}} - \pder[]{x_j}\left( 0.55\frac{\nu}{T}\pder[T]{x_j}M_{x,y} \right) = \dot{M}_{x,y}.
\end{equation}
For convenience, a total velocity is defined that combines the convective and thermophoretic terms:
\begin{equation}\label{eq:lesmodels:soot:ndf:totvel}
  u_j^* = u_j - 0.55\frac{\nu}{T}\pder[T]{x_j}.
\end{equation}

Although the Method of Moments has a relatively smaller computational cost, it faces the problem of closure. Evaluation of the source term $\dot{M}_{x,y}$ in \cref{eq:lesmodels:soot:ndf:momtransport} depends on moments that are not directly solved for, requiring further modeling. Two moment methods that provide closure are the Method of Moments with Interpolative Closure (MOMIC)~\cite{frenklach2002,frenklach1987,frenklach1994} and the Direct Quadrature Method of Moments (DQMOM)~\cite{marchisio2005}.

In MOMIC, explicit transport equations are solved for a set of moments. Source term closure is achieved by polynomial interpolation of the logarithm of these known moments. For a bivariate parameterization of the NDF with volume and surface area, the interpolation is given by
\begin{equation}\label{eq:lesmodels:soot:ndf:momic}
  M_{x,y}^{\text{MOMIC}} = \exp\left( \sum\limits_{r=0}^{R} \sum\limits_{k=0}^{r} a_{r,k}x^k y^{r-k} \right),
\end{equation}
where $R$ is the order of the polynomial interpolation. The constant coefficients $a_{r,k}$ are determined by taking the logarithm of \cref{eq:lesmodels:soot:ndf:momic} and solving the resulting linear system with the known moments. This system is well-conditioned, although as the order $R$ is increased, the number of additional moment transport equations increases.

In DQMOM, the NDF is thought of as a summation of multi-dimensional Dirac delta functions, with the moments approximated by Gauss quadrature. Transport equations are solved for the weights and locations of these delta functions, rather than for the moments of the NDF. The moments of the NDF are given by
\begin{equation}\label{eq:lesmodels:soot:ndf:dqmom}
  M_{x,y}^{\text{DQMOM}} = \sum\limits_{i=1}^{P} N_i V_i^x S_i^y,
\end{equation}
where $P$ is the total number of delta functions used in the quadrature approximation, $N_i$ are the weights of the delta functions, and $V_i$ and $S_i$ are the abscissas (locations) of the delta functions. Closure of the source terms of the transport equations for the weights and abscissas is achieved by using the source terms of a specified independent set of moments. However, depending on the shape of the NDF at any point and the selected set of moments, the procedure of closure can involve inverting an ill-conditioned linear system.

As mentioned at the beginning of this section, the NDF has been found to be bimodal in sooting flames with persistent nucleation. The ability to predict the contributions of primary particles and larger aggregates is a significant characteristic that distinguishes between the aforementioned methods. MOMIC has been shown to be unsuccessful at capturing the influence of incipient particles, while DQMOM represents the bimodality of the NDF well~\cite{mueller2009}. Extensions to MOMIC that do account for bimodal distributions have been proposed~\cite{frenklach2002}, but these have been derived for univariate distributions and cannot be directly extended to bivariate or multi-variate distributions. DQMOM has this capability because it is able to allocate a single delta function for the incipient particles, which remains nearly fixed at the size of these nucleated particles~\cite{blanquart2009}. All other delta functions can then be allocated to resolve the distribution of larger particles and aggregates.

%A third moment method, the Hybrid Method of Moments (HMOM), adopts the advantages of DQMOM and MOMIC to provide a bimodal description of the NDF that is numerically well-conditioned. In this work, HMOM will be used.


%\subsection{Hybrid Method of Moments}
%\label{sec:lesmodels:soot:hmom}

A third moment method, the Hybrid Method of Moments (HMOM)~\cite{hmom2009}, adopts the advantages of DQMOM and MOMIC to provide a bimodal description of the NDF that is numerically well-conditioned. In HMOM, an arbitrary moment is given by
\begin{equation}\label{eq:lesmodels:soot:hmom:m}
  M_{x,y}^{\text{HMOM}} = N_0 V_0^x S_0^y + \exp\left( \sum\limits_{r=0}^{R} \sum\limits_{k=0}^{r} a_{r,k}x^k y^{r-k} \right),
\end{equation}
where the first term on the right-hand side is a delta function that is allocated to account for incipient soot particles. $N_0$ is the weight of this delta function that is located at the fixed coordinates given by $V_0$ and $S_0$. Note that in DQMOM, this delta function is not completely immobile. However, it does not move much, so no significant error is expected by fixing its location~\cite{hmom2009}. The second term on the right-hand side of \cref{eq:lesmodels:soot:hmom:m} is the contribution from MOMIC and accounts for the presence of larger soot aggregates. Unlike MOMIC, determination of the weight $N_0$ and the coefficients $a_{r,k}$ is nontrivial from a given set of moments, requiring the inversion of a nonlinear system. However, if $N_0$ is known, then the system given by \cref{eq:lesmodels:soot:hmom:m} can be inverted to evaluate $a_{r,k}$ with the ease of MOMIC. Therefore, as in DQMOM, the weight of the delta function $N_0$ is determined through a transport equation similar to the one defined in \cref{eq:lesmodels:soot:ndf:momtransport}. For first order polynomial interpolation ($R = 1$), the source term of the transport equation for $N_0$ is given by
\begin{equation}\label{eq:lesmodels:soot:hmom:ndot}
  \dot{N}_0 = \lim_{\alpha,\beta\to\infty} \frac{\dot{M}_{-\alpha,-\beta}}{V_0^{-\alpha} S_0^{-\beta}},
\end{equation}
where expressions for the source term in the numerator are obtained from the various processes that govern the evolution of soot particles. Note that in order to determine $a_{r,k}$ given $R = 1$, a set of three additional moments need to be evaluated. These quantities are the total number density $M_{0,0}$, the total particle volume $M_{1,0}$, and the total particle surface area $M_{0,1}$. Further information about the modeling of the physical processes in \cref{eq:lesmodels:soot:hmom:ndot} as well as other details of HMOM can be found in Mueller \etal~\cite{hmom2009,mueller2009,mueller2011}. In this work, HMOM will be used to provide closure.
