% ********************************************************************** %
% Dissertation template and document class for Princeton University      %
% Author  : Jeffrey Scott Dwoskin <jdwoskin@princeton.edu>               %
% Adapted from: http://www.math.princeton.edu/graduate/tex/puthesis.html %
% ********************************************************************** %

%%% For print copies
%% set 'singlespace' option to set entire thesis to single space, and define "\printmode" to remove all hyperlinks for printed copies of the thesis.
%% Delete all output files before changing this mode -- it will turn hyperref package on and off
%\documentclass[12pt,lot, lof, singlespace]{puthesis}
%\newcommand{\printmode}{}

%%% For the electronic copy, use doublespacing, define "\proquestmode" to use outlined links, instead of colored links. 
\documentclass[12pt,lot,lof]{puthesis} % 06/06/2017 jklew: MAKE SURE TO UNCOMMENT FOR FINAL VERSION
\newcommand{\proquestmode}{}
% I prefer proquestmode to be off for electronic copies for normal use, since the colored links are less distracting. However when printed in black and white, the colored links are difficult to read. 

%%% For early drafts without some of the frontmatter
% Also see the "ifodd" command below to disable more frontmatter
%\documentclass[12pt]{puthesis}

%%%%%%%%%%%%%%%%%%%%%%%%%%%%%%%%%%%%%%%%%%%%%%%%%%%%%%%%%%%%%\
%%%% Author & title page info

\title{Soot Modeling in Large Eddy Simulation of Turbulent Nonpremixed Combustion}

\submitted{September 2017}  % degree conferral date (January, April, June, September, or November)
\copyrightyear{2017}  % year in which the copyright is secured by publication of the dissertation.
\author{Jeffry Kimura Lew}
\adviser{Professor Michael E. Mueller}  %replace with the full name of your adviser
%\departmentprefix{Program in}  % defaults to "Department of", but programs need to change this.
\department{Mechanical and Aerospace Engineering}

%%%%%%%%%%%%%%%%%%%%%%%%%%%%%%%%%%%%%%%%%%%%%%%%%%%%%%%%%%%%%\
%%%% Tweak float placements
% From: http://mintaka.sdsu.edu/GF/bibliog/latex/floats.html "Controlling LaTeX Floats"
% and based on: http://www.tex.ac.uk/cgi-bin/texfaq2html?label=floats
% LaTeX defaults listed at: http://people.cs.uu.nl/piet/floats/node1.html

% Alter some LaTeX defaults for better treatment of figures:
    % See p.105 of "TeX Unbound" for suggested values.
    % See pp. 199-200 of Lamport's "LaTeX" book for details.
    %   General parameters, for ALL pages:
    \renewcommand{\topfraction}{0.85}	% max fraction of floats at top
    \renewcommand{\bottomfraction}{0.6}	% max fraction of floats at bottom
    %   Parameters for TEXT pages (not float pages):
    \setcounter{topnumber}{2}
    \setcounter{bottomnumber}{2}
    \setcounter{totalnumber}{4}     % 2 may work better
    \setcounter{dbltopnumber}{2}    % for 2-column pages
    \renewcommand{\dbltopfraction}{0.66}	% fit big float above 2-col. text
    \renewcommand{\textfraction}{0.15}	% allow minimal text w. figs
    %   Parameters for FLOAT pages (not text pages):
    \renewcommand{\floatpagefraction}{0.66}	% require fuller float pages
	% N.B.: floatpagefraction MUST be less than topfraction !!
    \renewcommand{\dblfloatpagefraction}{0.66}	% require fuller float pages

% The documentclass already sets parameters to make a high penalty for widows and orphans. 

%%%%%%%%%%%%%%%%%%%%%%%%%%%%%%%%%%%%%%%%%%%%%%%%%%%%%%%%%%%%%\
%%%% Use packages

%\usepackage{amsfonts}

% Proper font encoding, especially for dashes in bibliography
\usepackage[english]{babel}
\usepackage[utf8]{inputenc}
\usepackage[T1]{fontenc}

% To allow for text in math mode
\usepackage{amsmath}    

%%% For figures
\usepackage{graphicx}
%\usepackage{subfig,rotate}

%%% for comments
\usepackage{verbatim}

%%% For tables
\usepackage{multirow}
% Longtable lets you have tables that span multiple pages.
\usepackage{longtable}

% Booktabs produces far nicer tables than the standard LaTeX tables.
%   see: http://en.wikibooks.org/wiki/LaTeX/Tables
\usepackage{booktabs}

%set parameters for longtable:
% default caption width is 4in for longtable, but wider for normal tables
\setlength{\LTcapwidth}{\textwidth}

% So citations are in numerical order and grouped together if possible (i.e., 3-5 instead of 3,4,5)
\usepackage{cite}

% For the degrees symbol
\usepackage{textcomp,gensymb}

% To write chemical reactions easily
\usepackage[version=4]{mhchem}

% For extra wide tildes
\usepackage{yhmath}

%%%%%%%%%%%%%%%%%%%%%%%%%%%%%%%%%%%%%%%%%%%%%%%%%%%%%%%%%%
%%% Printed vs. online formatting
\ifdefined\printmode

% Printed copy
% url package understands urls (with proper line-breaks) without hyperlinking them
\usepackage{url}


\else

\ifdefined\proquestmode
%ProQuest copy -- http://www.princeton.edu/~mudd/thesis/Submissionguide.pdf

% ProQuest requires a double spaced version (set previously). They will take an electronic copy, so we want links in the pdf, but also copies may be printed or made into microfilm in black and white, so we want outlined links instead of colored links.
\usepackage{hyperref}
\hypersetup{bookmarksnumbered}

% copy the already-set title and author to use in the pdf properties
\makeatletter
\hypersetup{pdftitle=\@title,pdfauthor=\@author}
\makeatother

\else
% Online copy

% adds internal linked references, pdf bookmarks, etc

% turn all references and citations into hyperlinks:
%  -- not for printed copies
% -- automatically includes url package
% options:
%   colorlinks makes links by coloring the text instead of putting a rectangle around the text.
\usepackage{hyperref}
\hypersetup{colorlinks,bookmarksnumbered}

% copy the already-set title and author to use in the pdf properties
\makeatletter
\hypersetup{pdftitle=\@title,pdfauthor=\@author}
\makeatother

% make the page number rather than the text be the link for ToC entries
%\hypersetup{linktocpage}
\fi % proquest or online formatting
\fi % printed or online formatting

% For grouped references to equations
% Needs to be placed last, after all ``\usepackage'' commands
\usepackage[capitalise]{cleveref}


%%%%%%%%%%%%%%%%%%%%%%%%%%%%%%%%%%%%%%%%%%%%%%%%%%%%%%%%%%%%%\
%%%% Define commands

% Define any custom commands that you want to use.
% For example, highlight notes for future edits to the thesis
%\newcommand{\todo}[1]{\textbf{\emph{TODO:}#1}}

  %% Commands for writing equations
  \newcommand*\mc[1]{\mathcal{#1}}                % To change argument to script font
  \newcommand*\mean[1]{\overline{#1}}             % Average qtys. or LES low-pass filter
  \newcommand*\tf[1]{\widetilde{#1}}              % Density-weighted filter (LES low-pass filtered qtys.)
  \newcommand{\tff}[1]{\widetilde{\overline{#1}}} % Combine above two 
  \newcommand{\pder}[2][]{\frac{\partial#1}{\partial#2}}            % Partial derivative
  \newcommand{\sder}[2][]{\frac{\partial^{2}#1}{\partial#2^{2}}}      % Second partial derivative
  \newcommand{\tder}[1]{\pder[#1]{t} + \pder[u_j#1]{x_j}}            % Total derivative
  \newcommand{\trder}[1]{\pder[\rho #1]{t} + \pder[\rho u_j#1]{x_j}} % Total derivative (conservative form)

  % Density-weighted LES filtered total derivative (conservative form)
  \newcommand{\ftrder}[1]{\pder[\mean{\rho}#1]{t} + \pder[\mean{\rho}\tf{u}_j#1]{x_j}}

  % Divergence of subfilter flux
  \newcommand{\dsff}[2][]{\pder[]{x_j}\left( \mean{\rho}\tf{u}_j#1 - \mean{\rho}#2 \right)}

  % Density-weighted LES filtered diffusion term
  \newcommand{\fdt}[2][]{\pder[]{x_j}\left( \mean{\rho}\tf{D}_{#1}\pder[#2]{x_j} \right)}

  % LES filtered source term
  \newcommand{\fst}[2][]{\mean{\dot{#1}}_{#2}}

  % Modified script G in Presumed PDF section of Ch. 2
  \newcommand{\mg}[0]{\mc{G}^{\heartsuit}}
  

% create an environment that will indent text
% see: http://latex.computersci.org/Reference/ListEnvironments
% 	\raggedright makes them left aligned instead of justified
\newenvironment{indenttext}{
    \begin{list}{}{ \itemsep 0in \itemindent 0in
    \labelsep 0in \labelwidth 0in
    \listparindent 0in
    \topsep 0in \partopsep 0in \parskip 0in \parsep 0in
    \leftmargin 1em \rightmargin 0in
    \raggedright
    }
    \item
  }
  {\end{list}}

% another environment that's an indented list, with no spaces between items -- if we want multiple items/lines. Useful in tables. Use \item inside the environment.
% 	\raggedright makes them left aligned instead of justified
\newenvironment{indentlist}{
    \begin{list}{}{ \itemsep 0in \itemindent 0in
    \labelsep 0in \labelwidth 0in
    \listparindent 0in
    \topsep 0in \partopsep 0in \parskip 0in \parsep 0in
    \leftmargin 1em \rightmargin 0in
    \raggedright
    }

  }
  {\end{list}}

%%%%%%%%%%%%%%%%%%%%%%%%%%%%%%%%%%%%%%%%%%%%%%%%%%%%%%%%%%%%%\
%%%% Front-matter

% For early drafts, you may want to disable some of the frontmatter. Simply change this to "\ifodd 1" to do so.
\ifodd 1 % 06/06/2017 jklew: DEFAULT IS 0
% front-matter disabled while writing chapters
%%%\renewcommand{\maketitlepage}{}
\renewcommand*{\makecopyrightpage}{}
\renewcommand*{\makeabstract}{}

% you can just skip the \acknowledgements and \dedication commands to leave out these sections.

\else

% 06/21/2017 DEFAULT FOR BELOW IS UNCOMMENTED
\abstract{
% Abstract can be any length, but should be max 350 words for a Dissertation for ProQuest's print indicies (150 words for a Master's Thesis) or it will be truncated for those uses.
Two major components developed are a model for the small-scale interactions between soot, turbulence, and chemistry as well as a model for the transport of strain-sensitive species in turbulent nonpremixed combustion. The former is based on the presumed subfilter PDF method, where the distribution of blah is assumed. It contains a dependence on theh mixture fraction to address the lack of soot in fuel-lean regions and has the form of a double-delta distribution to account for the high spatial intermittency of soot. The strain-sensitive transport approach is a model that transports species governed by relatively slow formation chemistry with molecular diffusion and transports species governed by fast kinetics with equal effective diffusivities. A strain-sensitivity parameter is formulated to categorize each species. The final objective of this thesis is to validate these models in Large Eddy Simulations (LES). A series of three laboratory-scale ethylene-hydrogen-nitrogen simple jet flames are used as the testbench.

In chapter 3, the Z-activated soot subfilter PDF is a model for the small-scale interactions between soot, chemistry, and turbulence that accounts for the spatial intermittency of soot as well as the distribution of soot in mixture fraction space. It has the form of a double delta distribution to account for sooting and non-sooting modes within an LES filter width. It excludes the presence of soot at lean mixtures in order to prevent non-physical oxidation from occurring. The filtered moment source terms for oxidation and surface growth are evaluated with this model. The Z-activated soot subfilter PDF was found to reduce the oxidation rate at a fixed filter width compared to the subfilter PDF that did not account for soot's distribution in mixture fraction space. Meanwhile, the source term for surface growth was minimally affected by this model, for surface growth is confined to fuel rich regions. A increase in the LES filter width results in a intensified decrease in the oxidation rate, while the surface growth source term is hardly affected.

In chapter 4, the strain-sensitive transport approach was developed to appropriately model the transport of species that are governed by relatively slow formation chemistry like PAH. PAH are confined to spatially intermittent regions of low scalar dissipation rate that are on the order of the Kolmogorov scale or smaller, where transport is solely dictated by molecular diffusion. This approach is based on the nonpremixed flamelet equations accounting for differential diffusion effects. Identification of species that should be modeled with molecular diffusion is achieved through the strain-sensitivity parameter, which compares the rate of local mixing to the formation chemistry rate of a particular species. This parameter is incorporated in the the flamelet framework through a new definition of the effective species Lewis number. An \textit{a priori} analysis using flamelet solutions reveal that the proposed model matches the flame temperature and acetylene mass fraction profiles from solutions using a unity Lewis number assumption, while the naphthalene mass fraction profile with the proposed model approaches the profile using full detailed transport. Sensitivity studies with the strain-sensitivity parameter reveal that the same species are identified as strain-sensitive as the fuel mixture, chemical mechanism, and stoichiometric scalar dissipation rate are varied.

In chapter 5, the proposed models are validated against experimental data from a series of turbulent nonpremixed ethylene-hydrogen-nitrogen simple jet flames with varying amounts of global strain at a constant Reynolds number of 15,000. LES with the Z-activated soot subfilter PDF and unity effective Lewis numbers demonstrated good agreement with experimental profiles for the centerline flame temperature but underpredicted the centerline soot volume fraction by roughly two orders of magnitude. Minimal differences improvements of LES with the Z-uniform soot subfilter PDF and unity effective Lewis numbers were observed due to only a slight reduction in the maximum rate of oxidation. LES with the Z-activated soot subfilter PDF and the strain-sensitive transport approach showed good agreement with experimental profiles for the flame temperature, and exhibited an improved prediction of the soot volume fraction over LES with the Z-activated soot subfilter PDF and unity effective Lewis numbers. LES results were found to be sensitive to the chemical mechanism, although the maximum volume fraction was underpredicted by a factor of 10 at best, suggesting that the model is still missing some aspect of the physics of soot evolution. Nevertheless, the LES with the proposed models were able to capture trends with the global strain rate.

%% This is a \LaTeX{} template and document class for Ph.D. dissertations at Princeton University. It was created in 2010 by Jeffrey Dwoskin, and adapted from a template provided by the math department. Their original version is available at: \url{http://www.math.princeton.edu/graduate/tex/puthesis.html}

%% This is \textbf{NOT} an official document. Please verify the current Mudd Library dissertation requirements~\cite{mudd2009} and any department-specific requirements before using this template or document class.


%% Your abstract can be any length, but should be a maximum of 350 words for a Dissertation for ProQuest's print indicies (or 150 words for a Master's Thesis); otherwise it will be truncated for those uses~\cite{proquest2006}.


%% Dwoskin Ph.D. Dissertation Template --- version 1.0, 5/19/2010

}

\acknowledgements{
%I would like to thank...
This thesis would not exist without the support I have received from people in the Princeton community and beyond. I am indebted to my advisor, Professor Michael E. Mueller, for his generosity with guidance and encouragement. I thank him for the myriad of ways he has contributed to my intellectual and professional growth.

I am beholden to my graduate student peers and Professors Elie Bou-Zeid, Luigi Martinelli, and Howard A. Stone for assisting me with general exam preparations. I thank Professors Marcus Hultmark, Yiguang Ju, Chung K. Law, and Alexander Smits for being my examiners. I would also like to express my gratitude to my advisor and Professor Yiguang Ju for serving as my reading committee.

Many aspects of this thesis would not have come to fruition without the cooperation of various outstanding researchers outside Princeton University. I am fortunate to have collaborated with Professor Antonio Attili and Mr. Lukas K. W. Berger at RWTH Aachen University and the following researchers at the University of Adelaide: Mr. Saleh Mahmoud, Professor Zeyad T. Alwahabi, Professor Bassam B. Dally, and Professor Graham J. Nathan.

I am grateful for the companionship of my colleagues in CTRFL: Dr. Temistocle Grenga, Dr. Suo Yang, Dr. Carla Bahri, Dr. Sili Deng, Jonathan F. MacArt, A. Cody Nunno, Sandra Sowah, Bruce A. Perry, and Alex G. Novoselov. The discussions that I had with them helped me overcome obstacles in my research, and their presence in lab reminded me that it is always sunny in D101.

I extend my appreciation to Seongwoo Oh, Justin L. Ripley, Jonathan Shi, and James E. Park for being compassionate cohabitants, and I am thankful for the Big Red comradeship of Seongwoo Oh, Zhaoqi Leng, Anthony Savas, and Daniel Floryan. Finally, I deeply thank my parents for their steadfast encouragement and support.

This work would not be possible without the valuable support in the form of computational time on the TIGRESS high performance computer center at Princeton University, which is jointly supported by the Princeton Institute for Computational Science and Engineering (PICSciE) and the Princeton University Office of Information Technology's Research Computing Department. This thesis carries T-3342 in the records of the Department of Mechanical and Aerospace Engineering.

%% Advisor
%% Thesis committee/generals committee
%% Professors who helped me during the general exam process
%% labmates + other labs + classmates + Daniel floryan
%% Jill Ray and Theresa
%% Susanne Killian
%% Roommates Seong, Justin
%% Ian, Bryan, going to the gym
%% Collaborators such as Bassam Dally
%% Susanne Killian and Shalin Shah
%% Advisors from Cornell (Olivier Desjardins, Elizabeth Fisher, Dr. Brandon Hencey)
%% Parents

%% I would like to thank the Math department for providing the original documentclass file that this class is based upon. I would like to thank my parents, without whom my life would not be possible. I would also like to thank my advisor, my dissertation committee, and my research collaborators because every graduate student needs to do so. And finally, I thank the members of my research group, to whom I leave this template to save you some of the trouble I had to go through getting my dissertation to compile in \LaTeX{}.  

%% Don't forget to ask your advisor if your work was sponsored by a grant that needs to be acknowledged in this section.

%% Lastly, I want to thank my parents for their 

}

\dedication{To my parents.}

\fi  % disable frontmatter


%%%%%%%%%%%%%%%%%%%%%%%%%%%%%%%%%%%%%%%%%%%%%%%%%%%%%%%%%%%%%\
%%%% Hide some chapters

%%% If you want to produce a pdf that includes only certain chapters, specify them with includeonly, in addition to including all chapters below.
%\includeonly{ch-intro/chapter-intro}
%%% You can also specify multiple chapters.
%\includeonly{ch-intro/chapter-intro,ch-usage/chapter-usage}
%\includeonly{chap1,chap2,chap3}

%\includeonly{ch-lesmodels/chapter-lesmodels}


%%%%%%%%%%%%%%%%%%%%%%%%%%%%%%%%%%%%%%%%%%%%%%%%%%%%%%%%%%%%%
%%%% Notes:

% Footnotes should be placed after punctuation.\footnote{place here.}
% Generally, place citations before the period~\cite{anotherauthor}.
% The proper usage for i.e., and e.g., include commas ``(e.g., option A, option B)''

%%%%%%%%%%%%%%%%%%%%%%%%%%%%%%%%%%%%%%%%%%%%%%%%%%%%%%%%%%%%%
%%%% Import chapters

\begin{document}

\makefrontmatter


% If you've disabled frontmatter, you can insert the toc manually
%\tableofcontents\clearpage % 06/06/2017 jklew: DEFAULT IS COMMENTED OUT

% \include lets us split up the document (and each include starts a new page):
\chapter{Introduction\label{ch:intro}}
 
Soot is a carbonaceous product formed from the incomplete combustion of hydrocarbon fuels. Although soot may appear as a dark cloud or plume to the naked eye, it is actually a collection of nanoscale particles, as shown in \cref{fig:intro:dieselsoot}. Soot particles are required to increase the radiant energy of flames in furnaces and boilers, but they are also an undesired waste product in many engineering systems such as internal combustion engines, gas turbine combustors, and combustion-based power plants. If released into the atmosphere, soot can contribute to haze, global warming, and acid rain. Exposure to these particles results in an increased risk for a variety of ailments including strokes~\cite{popeiii2006}, atherosclerosis~\cite{polichetti2009,kennedy2007,popeiii2006}, lung cancer~\cite{kennedy2007,popeiii2006}, and cardiovascular mortality~\cite{polichetti2009,kennedy2007,popeiii2006}, and these adverse health effects have been observed to be closely related the number, composition, and size of particles rather than their mass~\cite{seaton1995,lighty2000}. While the fate of federal regulations for particulate matter under the current administration of President Donald J. Trump is unclear, these regulations are expected to become more stringent in the coming decades. 

\begin{figure}[htb]
  \centering
  \includegraphics[width=0.4\linewidth]{ch-intro/figures/diesel-soot}
  \caption[Soot From Diesel Car Engine]{Transmission Electron Microscope (TEM) image of soot collected from the exhaust of a diesel car engine. Reproduced from Grob\'ety \etal~\cite{grobety2010}.}
  \label{fig:intro:dieselsoot}
\end{figure}

The production of next-generation, environmentally friendly combustion systems will depend on numerical simulation for rapid design optimization. The task of predicting how soot evolves in the turbulent reacting flows of these systems is nontrivial and will require high-fidelity models for soot, combustion, and turbulence. The large range of spatial and temporal scales present in these systems makes the development of each component model extremely challenging. Furthermore, integration into a single framework necessitates accounting for the interactions between models. Today's state-of-the-art simulations have progressed towards achieving this goal, but significant work still lies ahead. 

The overall objective of this thesis is to advance the understanding of how turbulence affects the dynamics of soot. Specifically, the small-scale interactions between soot, chemistry, and turbulence as well as the transport of species in turbulent nonpremixed combustion are analyzed. A survey of previous investigations on both topics is presented in the upcoming sections, but first, a brief overview of the processes that guide the evolution of soot is provided.


%% ***
%% My rough idea for intro:
%% What is soot and why it needs to be studied/modeled
%% Explanation of dynamics of soot/various modes of evolution

%% Goal is to motivate need for ZASSP (closure model for soot transport equation SOOT MODEL that accounts for spatial intermittency of soot (soot-turbulence interactions) as well as its interaction/connection to combustion chemistry). Also need to motivate need for SSTA (transport approach that uses the nonpremixed flamelet framework TURBULENT COMBUSTION MODEL WITH EMPHASIS ON SOOT PRECURSORS with differential diffusion, transporting species with relatively slow chemistry with molecular diffusion and species with fast chemistry with unity effective Lewis numbers).

%% SOOT-TURBULENCE-CHEMISTRY INTERACTIONS:
%% Effects of turbulence on soot, experimental evidence of influence of strain
%% Previous modeling attempts of soot evolution in turbulent combustion: RANS, LES, DNS (highlight findings and weaknesses)
%% Explain how the model I developed will address these deficiencies

%% EFFECT OF TRANSPORT ON SOOT EVOLUTION:
%% Mention species that influence soot evolution and dictate flame structure have a variety of molecular weights and scales.
%% Somehow incorporate experimental findings into this section
%% Previous modeling attempts for transport of species in turbulent combustion (highlight findings and weaknesses, especially with consideration of soot)
%% Explain how the turbulent combustion transport model will address these deficiencies
%% ***

\section{Dynamics of Soot}
\label{sec:intro:dynamics}
%\section{Modeling Framework for Large Eddy Simulation}
%\label{sec:intro:framework}

%% Summarize previous works on soot and turbulent combustion models as well as other closure approaches other than the presumed PDF approach.

%% There are three classes of statistical models that are generally utilized. The most accurate is the Monte Carlo (MC) approach, where a large population of notional particles is used to represent the NDF. The evolution of these particles is determined by assuming that all aerosol processes are governed by stochastic processes~\cite{balthasar2003} or that only the coagulation of particles occurs stochastically~\cite{lucchesi2017}. MC is able to capture the NDF with high accuracy, as thousands of internal coordinates can be used to provide highly detailed descriptions of aggregate structure and chemical composition~\cite{celnik2008,mosbach2009}. However, the computational cost associated with such accuracy constrains the application of MC to simple configurations such as homogeneous reactors~\cite{celnik2007} and one-dimensional laminar premixed flames~\cite{patterson2007}. Additionally, explicitly coupling MC to the gas-phase chemistry is not straightforward~\cite{celnik2007}.

%% The second class of statistical models comprises sectional methods, where the NDF is discretized into bins and equations are solved for the number of particles in each bin. Like MC simulations, sectional methods provide accurate depictions of the complete NDF. However, as the number of internal coordinates used to describe the NDF increases, the associated computational cost can become intractable due to the required number of bins~\cite{gelbard1980}. Sectional methods do possess an advantage over MC, as they are deterministic and can be explicitly coupled to gas-phase chemistry.

The life of a soot particle begins with the formation of Polycyclic Aromatic Hydrocarbons (PAH). The exact mechanism for this process is not yet fully understood, but in general, large aliphatic fuel molecules are oxidized into smaller hydrocarbons by $\beta$-scission and \ce{H}-abstraction~\cite{law2006}. Further reactions eventually lead to the creation of species such as acetylene (\ce{C2H2}), propargyl (\ce{C3H3}}), cyclopentadienyl (\ce{C5H5}), phenyl (\ce{C6H5}), and benzene (\ce{C6H6})~\cite{wang1997,richter2000,wang2011}. Benzene, the first aromatic ring, plays an important role in the production of multi-ring aromatic species. Larger PAH with molecular weights of 500-1000 amu are considered to be the immediate precursors of soot, and their formation can occur through the attachment of \ce{C2}, \ce{C3}, and other small units to benzene~\cite{wang1997,richter2000}. Growth is further fostered through the addition of PAH radicals and through reactions between PAH species, including PAH-PAH radical recombination and addition reactions~\cite{richter2000,wang2011}. Collisions between these gas-phase PAH lead to dimerization, and the collisions between PAH dimers form solid-phase nascent soot particles known as primary particles~\cite{frenklach1991,richter2000,schuetz2002,blanquart2009,wang2011}. This process can take several milliseconds~\cite{richter200,wang2011} and is summarized in the first two frames of \cref{fig:intro:dynamics:sootdynamics}.

\begin{figure}[htb]
  \centering
  \includegraphics[width=\linewidth]{ch-intro/figures/soot-dynamics}
  \caption[Dynamics of Soot]{Various processes that govern the dynamics of soot.}
  \label{fig:intro:dynamics:sootdynamics}
\end{figure}

Once the first primary particles appear, their evolution is dictated by various physical and chemical processes. In the top right frame of \cref{fig:intro:dynamics:sootdynamics}, coagulation is depicted. During coagulation, the number density of particles decreases as existing particles collide and no mass is transferred from gas-phase species to solid-phase particles~\cite{kazakov1995,hmom2009}. There are two limits as to how coagulation can occur. In the limit of pure coalescence, a primary particle is assumed to undergo maximum deformation as it collides with another particle to form a larger spherical particle. This is facilitated by the liquid-like nature of nascent particles as noted in experimental studies~\cite{dobbins1998,dobbins2002}. In the limit of pure aggregation, the colliding particles do not deform and the total surface area is assumed to be preserved in the resulting particle.

Soot can also evolve through two different growth pathways, as shown in the bottom left and middle frames of \cref{fig:intro:dynamics:sootdynamics}. During condensation, a gas-phase PAH dimer collides with a soot particle and attaches to its surface~\cite{hmom2009,blanquart2009}. Surface growth, on the other hand, involves reactions with gas-phase acetylene. Carbon atoms are added to the surface of the soot particle through the \ce{H}-Abstraction \ce{C2H2}-Addition (HACA) mechanism~\cite{frenklach1985,frenklach1991}. Both growth processes influence soot morphology by rendering particles and aggregates more spherical~\cite{mitchell1998,mitchell2003,park2003}.

These growth modes are balanced by destruction, as illustrated in the bottom right frame of \cref{fig:intro:dynamics:sootdynamics}. Oxidation occurs when hydroxyl radicals (and molecular oxygen to a lesser extent) strip carbon atoms from the surface of soot particles, forming products such as \ce{CO} and \ce{CO2}~\cite{kazakov1995,neoh1981,stanmore2001}. Oxidation by hydroxyl radicals proceeds rapidly, but oxidation by molecular oxygen occurs slowly enough such that the internal structures of large aggregates are weakened. This leaves these aggregates susceptible to breaking apart in a process known as fragmentation~\cite{mueller2011,neoh1985}.

Clearly, the lifetime of soot is governed by interactions with species of various weights, lengthscales, and timescales. Accurately predicting the evolution of soot in a turbulent reacting flow requires accounting for these interactions as well as for the influence of the surrounding flow field. Previous research addressing these topics is explored in the upcoming sections.

\section{Soot-Chemistry-Turbulence Interactions}
\label{sec:intro:scti}

The evolution of soot in turbulent nonpremixed combustion is heavily influenced by small-scale interactions between particles, combustion chemistry, and turbulence. Turbulence has been 

TURBULENCE CAUSES INTERMITTENCY
The effects of turbulence are known to be far-reaching. Turbulence is a highly efficient mechanism for mixing, and has a strong influence on how species are transported throughout the system. As mentioned previously, PAH are governed by relatively slow chemistry and are formed only in regions of low scalar dissipation rate~\cite{}. Due to turbulence, these regions are on the order of the Kolmogorov scale or smaller, and are highly spatially intermittent. Soot, which rely on PAH for nucleation and condensation, possess even slower kinetics and hardly diffuse at all due to their very large Schmidt number. As a result, soot is confined to very thin structures that are elongated by turbulent eddies and exhibit a high degree of spatial intermittency. This property has been observed in experiments~\cite{} and in Direct Numerical Simulation (DNS)~\cite{}.

Turbulence also dictates the mixture fraction field.

SOOT EXISTS ONLY AT RICH MIXTURE FRACTIONS
Soot has also been observed to exist only in rich regions of mixture fraction~\cite{DNS,experiment}. In a DNS study by Lignell \etal~\cite{lignell}, it was noted that if the center of curvature of the flame was in the fuel stream, the flame motion was towards the fuel and soot concentrations peaked. However, recent DNS studies by Bisetti \etal~\cite{bisetti} and Mueller and Pitsch~\cite{mueller2012} noted that soot was equally likely to travel towards lean or rich mixtures. If soot traveled to leaner regions, it experienced a brief period of surface growth followed by complete oxidation. Conversely, soot that traveled to richer regions tended to linger and experience growth by condensation as well as coagulation.

A previous model for these small-scale interactions between soot, combustion, and turbulence had proposed that the variables for soot should be independent of the variables for thermochemical quantities such as the mixture fraction. The reasoning behind this assumption was based on the timescale separation between the slow chemistry of soot and its precursors and the much faster heat-releasing chemistry. However, as was previously noted, the evolution of soot is intimately linked to the local mixture fraction, which is a thermochemical quantity. In \cref{ch:subfilter}, a model is developed that incorporates the relationship between soot and combustion chemistry while still accounting for the intermittent nature of soot. The impact of such a model on the source terms for oxidation and surface growth is assessed \textit{a priori} through an existing DNS database.

%% Main points:
%% Previous research has shown that turbulence constrains soot and its precursors to very thin, spatially intermittent regions. Experimental and computational evidence that support the previous statement. Point of work is to incorporate dependence of soot subfilter PDF on thermochemical variables - i.e. soot does not exist in lean regions, only in rich regions.

%% Previously proposed by Mueller and Pitsch that PAH had much slower formation chemistry compared to main heat-releasing chemistry (thermochemical variables). They proposed the timescales of soot scalars and the latter are disparate enough to be independent. They noted this timescale separation argument could be violated during soot oxidation, when there are enhanced interactions between soot and the major gas-phase chemistry, and during surface growth near the flame. Also, this argument only accounts for the timescales of PAH and soot formation, but soot, turbulence, and chemistry interact even after the nucleation of soot particles from PAH dimers.

%% The previous theory assumes soot is uniformly distributed in mixture fraction space, but this is not correct for non-smoking flames (DNS). Soot is oxidized before reaching lean mixtures. The local mixture fraction also guides the mode of evolution of soot. The previous theory will overpredict the rate of oxidation with a lesser impact on the rate of surface growth. The subfilter PDF must exclude the presence of soot at lean mixtures by incorporating the local ratios of fuel and oxidizer.

%% From Michael:
%% Turbulence affects soot:
%% 1) Confines to very thin structures that are stretched into long filaments
%% 2) PAH formed only in regions of low scalar dissipation rate due to its relatively slow chemistry

%% RANS studies with semi-empirical soot models, PAH-based inception models
%% LES studies with semi-empirical, acetylene-based inception model
%% DNS studies
%% Goal of work

\section{Effect of Transport on Soot Evolution}
\label{sec:intro:transport}

In addition to the effects discussed previously, turbulent eddies contribute to various levels of strain in the velocity field. As previously noted, PAH are highly sensitive to strain. In experiments of counterflow diffusion flames~\cite{}, an increase in the strain rate lead to a decrease in the amount of soot observed. Similarly, in turbulent 

Experiments~\cite{} and simulations~\cite{} have found that PAH are highly sensitive to the amount of strain.

Can discuss various approaches to modeling transport for sooting flames. Can discuss diffusion between soot and gas-phase species (DNS paper by Jackie Chen).

Main points:
Discussion about soot PAH precursors and other strain-sensitive species.
Classical theory is that turbulence mixes indiscriminately at sufficiently high Re.
PAH are very sensitive to the local scalar dissipation rate due to their slow formation chemistry
DNS studies suggest PAH are confined to spatially intermittent regions of low scalar dissipation rate that are on the order of the Kolmogorov scale or smaller
At such scales, transport is governed by molecular diffusion
Pitsch and Peters flamelet equations with full differential diffusion
Wang's bimodal transport

\section{Organization of Thesis}
\label{sec:intro:org}

This thesis is organized as follows. In \cref{ch:lesmodels}, the foundation for modeling soot in LES of turbulent nonpremixed combustion is presented. In \cref{ch:subfilter}, the model for small-scale interactions between soot, combustion chemistry, and turbulence is developed and validated \textit{a priori} against a recent DNS database. In \cref{ch:transport}, the transport model for species with relatively slow formation chemistry is presented and validated \textit{a priori} with solutions to the flamelet equations. In \cref{ch:lesresults}, these models are implemented in LES and validated against experimental measurements from a series of three laboratory-scale flames. Finally, in \cref{ch:conclusion}, the results from this thesis are summarized and suggestions for future work are proposed.

The accomplishments and new contributions of this thesis are outlined below.
\begin{itemize}
\item \textbf{$Z$-Activated Soot Subfilter PDF (ZASSP):} Developed a presumed subfilter PDF model to close the filtered transport equations for the soot model. It contains a dependence on the mixture fraction to address the lack of soot in fuel-lean regions and has the form of a double-delta distribution to account for the high spatial intermittency of soot (Chapter 3).
\item \textbf{Strain-Sensitive Transport Approach (SSTA):} Developed a model that transports species governed by relatively slow formation chemistry with molecular diffusion and transports species governed by fast kinetics with equal effective diffusivities. A strain-sensitivity parameter is used to categorize each species (Chapter 4).
\item \textbf{Validation of the Integrated Large Eddy Simulation Model:} Conducted Large Eddy Simulations with ZASSP and SSTA for a series of three laboratory-scale ethylene-hydrogen-nitrogen simple jet flames (Chapter 5).
\end{itemize}



%% This documentclass, \texttt{puthesis.cls}, is setup for a Ph.D. dissertation for Princeton University. The Mudd Library website~\cite{mudd2009} provides detailed specifications for how to format your disseration~\cite{muddthesis2009}. Please review those documents, as the requirements may have changed since this template was created. Also, review the ProQuest Dissertation Guide~\cite{proquest2006}, which has additional formatting rules that are important for the submission of the electronic copy of your dissertation.

%% This template includes many details about the requirements and common practices for writing, printing, and submitting your dissertation. However, this is \textbf{NOT} an official document. It was written by Jeffrey Dwoskin and is current as of May 2010 based on requirements for the Electrical Engineering department, but the requirements may have changed. Please verify all information, deadlines, costs, requirements, and formatting rules with the Mudd Library website~\cite{mudd2009}, with the Graduate School, and with your department.

%\input{ch-intro/intro_contributions}

\chapter{Modeling Framework for Large Eddy Simulation\label{ch:lesmodels}}

This chapter lays the foundation for modeling the evolution of soot in large eddy simulation (LES) of turbulent nonpremixed combustion. In LES of turbulent flows, the large scale, geometry-dependent motions are resolved while small-scale, universal turbulent structures are modeled. The predictive capabilities of LES are dependent on the quality of the models used for the small-scale, subfilter phenomena. Accurately capturing the evolution of soot requires the integration of models for soot, turbulent combustion, and polycyclic aromatic hydrocarbons (PAH) into the LES framework.

The distribution of soot particle sizes, known as the Number Density Function (NDF), has been found to be bimodal due to the presence of small, nascent particles and larger, more mature aggregates. Modeling the NDF is achieved by first parameterizing it with the volume and surface area of particles. However, since the NDF is high-dimensional, the exact form cannot be solved for directly due to the associated computational costs. Thus, the Method of Moments is used to obtain the average quantities, known as moments, of the NDF. In this chapter, the transport equation that governs the evolution of these moments is introduced. This equation has a source term that requires modeling, so discussion of closure through the Hybrid Method of Moments is presented as well.

Turbulent combustion is modeled through the flamelet approach~\cite{peters1984}, where a three-dimensional turbulent flame is represented as an ensemble of locally one-dimensional laminar flamelets embedded in a turbulent flow field. This method relies on the separation of scales between the turbulence and the flame's reaction zone, allowing the thermochemical properties of the flame to be decoupled from the flow field. As a result, the flame structure can be evaluated \textit{a priori}, parameterized with a reduced set of variables, and stored in a database that is accessed during LES. In this chapter, modifications to the classical flamelet approach accounting for the formation of soot and radiative thermal losses are explained. Closure of subfilter quantities with the presumed PDF approach is also outlined.

Lastly, special consideration is given to gas-phase PAH. Since their chemistry is slow relative to the exothermic combustion chemistry, a transport equation is required to model the resulting unsteady effects. Discussion on modeling the source term for a lumped PAH species is included.

%% Using LES is advantageous because it captures unsteady flow features like swirl and recirculation that are neglected by other methods such as Reynolds-Averaged Navier-Stokes (RANS), without being as computationally intensive as full-fidelity Direct Numerical Simulation (DNS). Its predictive capabilities can lead to highly optimized designs through rapid iteration. The potential of LES to innovate the next generation of environmentally-friendly automobile or aircraft engines is the main reason it has been selected as the underlying modeling architecture.

\section{Soot}
\label{sec:lesmodels:soot}

%\subsection{Modeling the Particle Size Distribution}
%\label{sec:lesmodels:soot:ndf}

The distribution of soot particle sizes in a system can be described by the Number Density Function (NDF), $N$. Previous studies have suggested that the soot NDF is bimodal, accounting for newly formed primary particles from persistent nucleation and larger aggregrates from various growth modes~\cite{zhao2003,zhao2005,netzell2007}. Incipient particles are roughly spherical with diameters on the order of nanometers, while aggregates comprised of monodisperse primaries have fractal geometries with length scales on the order of hundreds of nanometers~\cite{vanderwal1999}. In order to account for the geometrical complexities, the NDF requires a multivariate parameterization such as mass/volume and surface area~\cite{patterson2007,mueller2009,hmom2009}, the number of agglomerates and primary particles~\cite{park2004}, or even volume, surface area, and the number of surface hydrogenated carbon sites~\cite{blanquart2009}. In this work, a bivariate parameterization (volume $V$ and surface area $S$) of the NDF is used to give the functional form $N = N(V, S; t, x_j)$. Note that the NDF cannot be solved for directly due to its high dimensionality (six degrees of freedom), so statistical models are required.

For application to LES, the Method of Moments is one of the only tractable techniques due to its lower computational expense. In this approach, mean quantities of the NDF are solved for and complete information about the distribution is lost. These quantities are moments given by
\begin{equation}\label{eq:lesmodels:soot:ndf:moments}
  M_{x,y} = \sum\limits_{i=0}^\infty V_i^x S_i^y N_i,
\end{equation}
where discrete summation over $i$ implies summation over all particles sizes and $N_i$ is the number density of particles with volume $V_i$ and surface area $S_i$. The subscripts $x$ and $y$ indicate the order of the moment in volume and surface area, respectively.

The NDF evolves according to the Population Balance Equation (PBE)~\cite{friedlander2000}
\begin{equation}\label{eq:lesmodels:soot:ndf:pbe}
  \tder{N_i} - \pder[]{x_j}\left( 0.55\frac{\nu}{T}\pder[T]{x_j}N_i \right) = \dot{N_i},
\end{equation}
where the third term is the thermophoresis of particles~\cite{waldmann1966} and the source term on the right-hand side incorporates the various physical and chemical processes that govern soot evolution. These processes include nucleation from polycyclic aromatic hydrocarbon (PAH) dimers~\cite{blanquart2009,schuetz2002,frenklach1991,wang2011}, particle coagulation~\cite{kazakov1998,hmom2009}, soot growth from PAH condensation~\cite{blanquart2009,hmom2009}, soot growth through the $\ce{H}$-abstraction, $\ce{C2H2}$-addition (HACA) surface reaction mechanism~\cite{frenklach1985,frenklach1991}, oxidation~\cite{stanmore2001,neoh1981,kazakov1995}, and oxidation-induced fragmentation~\cite{neoh1985,mueller2011}. Note that molecular diffusion has been neglected, for a soot particle with a diameter of 10 nm has a Schmidt number of 287 at standard atmospheric conditions~\cite{friedlander2000}. This corresponds to a molecular diffusion coefficient of $D = 5.24\times 10^{-4}$ cm$^2$/sec, so the diffusion velocity of soot can be considered to be minimal. For LES, tracking the evolution of the moments is more relevant. A transport equation is derived by taking moments of \cref{eq:lesmodels:soot:ndf:moments},
\begin{equation}\label{eq:lesmodels:soot:ndf:momtransport}
  \tder{M_{x,y}} - \pder[]{x_j}\left( 0.55\frac{\nu}{T}\pder[T]{x_j}M_{x,y} \right) = \dot{M}_{x,y}.
\end{equation}
For convenience, a total velocity is defined that combines the convective and thermophoretic terms:
\begin{equation}\label{eq:lesmodels:soot:ndf:totvel}
  u_j^* = u_j - 0.55\frac{\nu}{T}\pder[T]{x_j}.
\end{equation}

Although the Method of Moments has a relatively smaller computational cost, it faces the problem of closure. Evaluation of the source term $\dot{M}_{x,y}$ in \cref{eq:lesmodels:soot:ndf:momtransport} depends on moments that are not directly solved for, requiring further modeling. Two moment methods that provide closure are the Method of Moments with Interpolative Closure (MOMIC)~\cite{frenklach2002,frenklach1987,frenklach1994} and the Direct Quadrature Method of Moments (DQMOM)~\cite{marchisio2005}.

In MOMIC, explicit transport equations are solved for a set of moments. Source term closure is achieved by polynomial interpolation of the logarithm of these known moments. For a bivariate parameterization of the NDF with volume and surface area, the interpolation is given by
\begin{equation}\label{eq:lesmodels:soot:ndf:momic}
  M_{x,y}^{\text{MOMIC}} = \exp\left( \sum\limits_{r=0}^{R} \sum\limits_{k=0}^{r} a_{r,k}x^k y^{r-k} \right),
\end{equation}
where $R$ is the order of the polynomial interpolation. The constant coefficients $a_{r,k}$ are determined by taking the logarithm of \cref{eq:lesmodels:soot:ndf:momic} and solving the resulting linear system with the known moments. This system is well-conditioned, although as the order $R$ is increased, the number of additional moment transport equations increases.

In DQMOM, the NDF is thought of as a summation of multi-dimensional Dirac delta functions, with the moments approximated by Gauss quadrature. Transport equations are solved for the weights and locations of these delta functions, rather than for the moments of the NDF. The moments of the NDF are given by
\begin{equation}\label{eq:lesmodels:soot:ndf:dqmom}
  M_{x,y}^{\text{DQMOM}} = \sum\limits_{i=1}^{P} N_i V_i^x S_i^y,
\end{equation}
where $P$ is the total number of delta functions used in the quadrature approximation, $N_i$ are the weights of the delta functions, and $V_i$ and $S_i$ are the abscissas (locations) of the delta functions. Closure of the source terms of the transport equations for the weights and abscissas is achieved by using the source terms of a specified independent set of moments. However, depending on the shape of the NDF at any point and the selected set of moments, the procedure of closure can involve inverting an ill-conditioned linear system.

As mentioned at the beginning of this section, the NDF has been found to be bimodal in sooting flames with persistent nucleation. The ability to predict the contributions of primary particles and larger aggregates is a significant characteristic that distinguishes between the aforementioned methods. MOMIC has been shown to be unsuccessful at capturing the influence of incipient particles, while DQMOM represents the bimodality of the NDF well~\cite{mueller2009}. Extensions to MOMIC that do account for bimodal distributions have been proposed~\cite{frenklach2002}, but these have been derived for univariate distributions and cannot be directly extended to bivariate or multi-variate distributions. DQMOM has this capability because it is able to allocate a single delta function for the incipient particles, which remains nearly fixed at the size of these nucleated particles~\cite{blanquart2009}. All other delta functions can then be allocated to resolve the distribution of larger particles and aggregates.

%A third moment method, the Hybrid Method of Moments (HMOM), adopts the advantages of DQMOM and MOMIC to provide a bimodal description of the NDF that is numerically well-conditioned. In this work, HMOM will be used.


%\subsection{Hybrid Method of Moments}
%\label{sec:lesmodels:soot:hmom}

A third moment method, the Hybrid Method of Moments (HMOM)~\cite{hmom2009}, adopts the advantages of DQMOM and MOMIC to provide a bimodal description of the NDF that is numerically well-conditioned. In HMOM, an arbitrary moment is given by
\begin{equation}\label{eq:lesmodels:soot:hmom:m}
  M_{x,y}^{\text{HMOM}} = N_0 V_0^x S_0^y + \exp\left( \sum\limits_{r=0}^{R} \sum\limits_{k=0}^{r} a_{r,k}x^k y^{r-k} \right),
\end{equation}
where the first term on the right-hand side is a delta function that is allocated to account for incipient soot particles. $N_0$ is the weight of this delta function that is located at the fixed coordinates given by $V_0$ and $S_0$. Note that in DQMOM, this delta function is not completely immobile. However, it does not move much, so no significant error is expected by fixing its location~\cite{hmom2009}. The second term on the right-hand side of \cref{eq:lesmodels:soot:hmom:m} is the contribution from MOMIC and accounts for the presence of larger soot aggregates. Unlike MOMIC, determination of the weight $N_0$ and the coefficients $a_{r,k}$ is nontrivial from a given set of moments, requiring the inversion of a nonlinear system. However, if $N_0$ is known, then the system given by \cref{eq:lesmodels:soot:hmom:m} can be inverted to evaluate $a_{r,k}$ with the ease of MOMIC. Therefore, as in DQMOM, the weight of the delta function $N_0$ is determined through a transport equation similar to the one defined in \cref{eq:lesmodels:soot:ndf:momtransport}. For first order polynomial interpolation ($R = 1$), the source term of the transport equation for $N_0$ is given by
\begin{equation}\label{eq:lesmodels:soot:hmom:ndot}
  \dot{N}_0 = \lim_{\alpha,\beta\to\infty} \frac{\dot{M}_{-\alpha,-\beta}}{V_0^{-\alpha} S_0^{-\beta}},
\end{equation}
where expressions for the source term in the numerator are obtained from the various processes that govern the evolution of soot particles. Note that in order to determine $a_{r,k}$ given $R = 1$, a set of three additional moments need to be evaluated. These quantities are the total number density $M_{0,0}$, the total particle volume $M_{1,0}$, and the total particle surface area $M_{0,1}$. Further information about the modeling of the physical processes in \cref{eq:lesmodels:soot:hmom:ndot} as well as other details of HMOM can be found in Mueller \etal~\cite{hmom2009,mueller2009,mueller2011}. In this work, HMOM will be used to provide closure.

\section{Turbulent Combustion}
\label{sec:lesmodels:combust}

In this section, effective strategies for incorporating combustion chemistry into LES are outlined. The turbulent combustion model must be able to accurately predict the thermal and chemical structure of the nonpremixed flame as fuel and oxidizer chemically react to form products such as carbon dioxide and water vapor. For sooting flames, the model also needs to account for the formation and evolution of soot precursors in addition to the soot itself. A brute-force method would involve selecting a chemical mechanism that models the above phenomena and solving coupled transport equations for all species within the mechanism. However, such an approach is impractical for realistic chemical mechanisms, which may contain thousands of species and tens of thousands of reactions~\cite{law2007}.

A more computationally tractable method is the Flamelet/Progress Variable (FPV) approach for nonpremixed combustion~\cite{pierce2004}. In the FPV approach, the local thermochemical state is described by the solutions to the steady flamelet equations~\cite{peters1984} and is parameterized by the mixture fraction $Z$ and reaction progress variable $C$:
\begin{equation}\label{eq:lesmodels:combust:fpv}
  \xi = \mc{F}(Z,C),
\end{equation}
where $\xi$ encompasses the thermochemical variables (density, temperature, species mass fractions, species source terms, etc.) and $\mc{F}$ is the functional relationship obtained from the solutions to the steady flamelet equations. To reduce computational overhead, these solutions are often computed \textit{a priori} and are stored in a database that is accessed during LES. Overall, the mapping given by \cref{eq:lesmodels:combust:fpv} has reduced the set of transported scalar variables to $Z$ and $C$, drastically decreasing the number of required transport equations.

However, this technique is insufficient for sooting flames due to its inability to predict the effects of thermal radiation. Soot evolution is characterized by long timescales and a sensitivity to the thermochemical state. Radiative thermal losses occur over similar temporal scales and can significantly alter the local thermochemical state. Therefore, radiative thermal losses cannot be neglected. The Radiation Flamelet/Progress Variable (RFPV) approach has extended the FPV approach to include these losses~\cite{ihme2008} and therefore will be the foundation for the turbulent combustion model in this work. The database of solutions to the steady flamelet equations is expanded to include solutions with radiative losses, and a heat loss parameter $H$ is now added to the local thermochemical equation of state
\begin{equation}\label{eq:lesmodels:combust:rfpv}
  \xi = \mc{G}(Z, C, H),
\end{equation}
where $\mc{G}$ is the functional relationship obtained from the augmented flamelet database.

For sooting flames, certain quantities such as the source terms in \cref{eq:lesmodels:soot:ndf:momtransport} are dependent on the soot scalars in addition to the thermochemical quantities. Therefore, a more general equation of state is formulated as~\cite{mueller2012}
\begin{equation}\label{eq:lesmodels:combust:jointeos}
  \phi = \mc{J}(\xi, \mc{M}_j),
\end{equation}
where $\phi$ is now any quantity that could depend jointly on the thermochemical variables $\xi$ and the soot scalars $\mc{M}_j$. The state vector $\mc{M}_j$ encompasses the moments $M_{x,y}$ and the weight of the delta function allocated for nucleating particles $N_0$. Mueller and Pitsch~\cite{subfilterpdf2011} proposed that \cref{eq:lesmodels:combust:jointeos} could be simplified by assuming that all of the quantities needed in the soot model can be written as the product of a function that depends only on the thermochemical state and a function that depends only on the soot scalars. Therefore, the functional relation $\mc{J}$ can be expressed as~\cite{mueller2012}
\begin{equation}\label{eq:lesmodels:combust:producteos}
  \phi = \mc{G}(Z, C, H)\mc{K}(\mc{M}_j),
\end{equation}
where $\mc{K}$ is unity if $\phi$ solely depends on the thermochemical quantities. This form will be utilized in \cref{ch:subfilter}.

%The reasoning behind this simplification will be revisited in \cref{ch:subfilter}.

In the following subsections, further details about the RFPV model will be discussed in the context of sooting flames. Definitions of the mixture fraction $Z$, progress variable $C$, and heat loss parameter $H$ are outlined to ensure that the thermochemical equation of state (\cref{eq:lesmodels:combust:rfpv}) is a unique function in the presence of soot formation. Afterwards, the nonpremixed flamelet equations are presented with additional terms to account for soot formation.


\subsection{Definition of Flame Structure Parameters}
\label{sec:lesmodels:combust:map}

For two-feed systems with a single fuel stream and single oxidizer stream, the mixture fraction can be defined to be a conserved scalar satisfying the transport equation~\cite{pitsch1998}
\begin{equation}\label{eq:lesmodels:combust:origz}
\trder{Z} = \pder[]{x_j}\left( \rho D_Z \pder[Z]{x_j} \right),
\end{equation}
where $Z$ is one in the fuel stream and zero in the oxidizer stream and $D_Z$ is chosen such that $Le_Z$ is unity. Such a definition is advantageous, for no assumptions about the species Lewis numbers have been made. However, the mixture fraction cannot be defined as a conserved scalar in sooting flames. During the nucleation of soot, PAH are extracted from the gas-phase, causing the mixture to be locally leaned. The approach of Mueller and Pitsch~\cite{mueller2012} is utilized to account for this phenomenon, where a source term is introduced into \cref{eq:lesmodels:combust:origz}:
\begin{equation}\label{eq:lesmodels:combust:map:z}
  \trder{Z} = \pder[]{x_j}\left( \rho D_Z \pder[Z]{x_j} \right) + \dot{m}_Z.
\end{equation}
The source term $\dot{m}_Z$ is defined similarly to Bilger's mixture fraction based on element mass fractions~\cite{bilger1989,mueller2012}:
\begin{equation}\label{eq:lesmodels:combust:map:zsource}
  \dot{m}_Z = \frac{\frac{\dot{m}_{\ce{C}} - \dot{\rho}Z_{\ce{C},\ce{O}}}{\nu_{\ce{C}}W_{\ce{C}}} + \frac{\dot{m}_{\ce{H}} - \dot{\rho}Z_{\ce{H},\ce{O}}}{\nu_{\ce{H}}W_{\ce{H}}} + 2\frac{\dot{\rho}Z_{\ce{O},\ce{O}} - \dot{m}_{\ce{O}}}{\nu_{\ce{O2}}W_{\ce{O2}}}}{\frac{Z_{\ce{C},\ce{F}} - Z_{\ce{C},\ce{O}}}{\nu_{\ce{C}}W_{\ce{C}}} + \frac{Z_{\ce{H},\ce{F}} - Z_{\ce{H},\ce{O}}}{\nu_{\ce{H}}W_{\ce{H}}} + 2\frac{Z_{\ce{O},\ce{O}} - Z_{\ce{O},\ce{F}}}{\nu_{\ce{O2}}W_{\ce{O2}}}},
\end{equation}
where $\dot{m}_k$ are element mass source terms, $\dot{\rho}$ is the source term in the continuity equation for the removal of gas-phase PAH, $Z_{k,l}$ are the mass fractions of element $k$ in the stream indicated by $l$, $\nu_k$ are the stoichiometric coefficients of the global reaction between fuel and oxidizer, and $W_{k}$ are the element molar masses. In the limit of unity Lewis numbers for all species, \cref{eq:lesmodels:combust:map:z,eq:lesmodels:combust:map:zsource} provide a description of the mixture fraction that is equivalent to Bilger's mixture fraction based on element mass fractions~\cite{bilger1989}. This definition of mixture fraction ultimately leads to additional terms in the flamelet equations of \cref{sec:lesmodels:combust:flamelet} that acknowledge the presence of soot.

The reaction progress variable has been defined in previous works~\cite{pierce2004,ihme2008} as
\begin{equation}\label{eq:lesmodels:combust:map:cprev}
  C_{\text{conv}} = Y_{\ce{CO2}} + Y_{\ce{CO}} + Y_{\ce{H2O}} + Y_{\ce{H2}}.
\end{equation}
However, this conventional definition is not appropriate in the context of sooting flames. PAH have large amounts of carbon relative to hydrogen, so their removal from the gas-phase tends to lower the local $\ce{C}$/$\ce{H}$ ratio. As a result, the local effective fuel composition is changed. To address this phenomena, Mueller and Pitsch~\cite{mueller2012} redefined the reaction progress variable through the transport equation given by
\begin{equation}\label{eq:lesmodels:combust:map:c}
  \trder{C} = \pder[]{x_j}\left( \rho D_C \pder[C]{x_j} \right) + \frac{\dot{m}_{C_{\text{conv}}}}{C^*},
\end{equation}
where the diffusion coefficient $D_C$ is chosen such that $Le_C$ is unity, $\dot{m}_{C_{\text{conv}}}$ is the source term for the conventional progress variable as defined in \cref{eq:lesmodels:combust:map:cprev}, and $C^*$ is a normalizing factor that accounts for the change in stoichiometry due to the removal of gas-phase PAH. $C^*$ ensures that the equation of state (\cref{eq:lesmodels:combust:jointeos}) remains unique even as the local $\ce{C}$/$\ce{H}$ ratio changes due to the nucleation of soot. In non-sooting flames, $C^*$ is constant because the local $\ce{C}$/$\ce{H}$ ratio is constant (in the absence of differential diffusion effects). Further details about $C^*$ can be found in Mueller and Pitsch~\cite{mueller2012}.
%With \cref{eq:lesmodels:combust:map:c}, the progress variable adopts the value of zero in both the fuel and oxidizer streams.

The remaining mapping quantity to fully specify the thermochemical state given by \cref{eq:lesmodels:combust:rfpv} is the heat loss parameter $H$. This quantity provides a measure for the extent to which radiative thermal losses affect the enthalpy. A simple definition for the heat loss parameter is the enthalpy itself or the enthalpy deficit. The latter option is more convenient, for the adiabatic boundary has a well-defined value of zero in the limit of unity Lewis numbers for all species. However, the presence of soot can affect the adiabatic boundary for the enthalpy deficit, and the inclusion of molecular transport introduces complications associated with tracking the enthalpy fluxes of different species. For either situation, the transport equation definition for the heat loss parameter $H$ from Mueller and Pitsch~\cite{mueller2012} is more convenient:
\begin{equation}\label{eq:lesmodels:combust:map:heatloss}
  \trder{H} = \pder[]{x_j}\left( \rho D_H \pder[H]{x_j} \right) + \dot{\rho}H + \dot{q}_{\text{RAD}},
\end{equation}
where the diffusion coefficient $D_H$ is chosen such that $Le_H$ is unity and $\dot{q}_{\text{RAD}}$ is the radiation source term. The second term on the right-hand side is present to ensure that $H$ is constant everywhere if $\dot{q}_{\text{RAD}}$ is zero. Initially, the heat loss parameter is zero throughout the nonpremixed system. As the cumulative effect of radiation increases, the heat loss parameter acquires increasingly negative values.


\subsection{Flamelet Equations}
\label{sec:lesmodels:combust:flamelet}

The thermochemical equation of state obtained from the RFPV approach (\cref{eq:lesmodels:combust:rfpv}) is specified by the solutions to the steady flamelet equations. These solutions, which are computed \textit{a priori} and stored in a database that is accessed during LES, provide a description of the thermal and chemical structure of a nonpremixed flame. The species flamelet equation is derived from the conservation equation for the species mass fractions:
\begin{equation}\label{eq:lesmodels:combust:flamelet:consy}
  \trder{Y_k} = -\pder[]{x_j}\left( \rho Y_k V_{k,j} \right) + \dot{m}_k,
\end{equation}
where $Y_k$ is the mass fraction of species $k$, $V_{k,j}$ is the diffusion velocity of species $k$ in the direction indicated by $j$, and $\dot{m}_k$ is the chemical source term for species $k$. The diffusion velocity has contributions from mass diffusion in the presence of concentration gradients, pressure gradients, body forces, and temperature gradients (a second order effect known as Soret diffusion)~\cite{law2006}. By only considering diffusion through concentration gradients, the diffusion velocity is described by the Stefan-Maxwell equation. However, evaluating the linear system associated with the Stefan-Maxwell equation can be computationally expensive, so further simplifications are often made with the Curtiss-Hirschfelder approximation~\cite{curtiss1949} or Fick's law of diffusion~\cite{fick1855}. The Curtiss-Hirschfelder approximation is given by
\begin{equation}\label{eq:lesmodels:combust:flamelet:curtisshirschfelder}
  V_{k,j} = -\frac{1}{X_k}D_k\pder[X_k]{x_j} + \sum\limits_{k} \frac{Y_k}{X_k}D_k\pder[X_k]{x_j},
\end{equation}
where $X_k$ is the mole fraction of species $k$, $D_k = (1 - Y_k)/\sum\limits_{k \neq l} (X_k/D_{k,l})$ is the mixture-averaged diffusivity for species $k$, and the second term on the right-hand side is a correction velocity to ensure the conservation of mass. By assuming constant unity Lewis numbers, \cref{eq:lesmodels:combust:flamelet:curtisshirschfelder} is simplified to Fick's law:
\begin{equation}\label{eq:lesmodels:combust:flamelet:fick}
  V_{k,j} = -\frac{1}{Y_k}D\pder[Y_k]{x_j},
\end{equation}
where $D$ is the constant diffusion coefficient. In \cref{ch:transport}, this assumption will be revisited.

% Fick's law assumes all species have the same constant Lewis number and is given by
%% \begin{equation}\label{eq:lesmodels:combust:flamelet:fick}
%%   V_{k,j} = -\frac{1}{Y_k}D\pder[Y_k]{x_j},
%% \end{equation}
%% where $D$ is the constant diffusion coefficient. For turbulent flames at sufficiently high Reynolds numbers, it is often assumed that the turbulence mixes all species indiscriminately~\cite{pitsch19981057}. Therefore, $D$ will be selected such that $Le_k$ is unity. Note that, in \cref{ch:transport}, this assumption will be revisited.

The temperature flamelet equation is obtained from the conservation equation for energy in terms of temperature under the assumption of constant thermodynamic pressure:
\begin{equation}\label{eq:lesmodels:combust:flamelet:const}
  \trder{T} = \frac{1}{c_p}\left[ \pder[]{x_j}\left( \lambda\pder[T]{x_j} \right) + \sum\limits_{k} \rho c_{p,k} D\pder[Y_k]{x_j}\pder[T]{x_j} - \sum\limits_{k} h_k\dot{m}_k + \dot{q}_{\text{RAD}} \right],
\end{equation}
where $\lambda$ is the thermal conductivity, $c_p$ is the specific heat at constant pressure, $c_{p,k}$ is the specific heat at constant pressure for species $k$, and $h_k$ is the specific enthalpy of species $k$. Fick's law is implemented in the second term on the right-hand side.

The flamelet equations are derived by transforming the physical space coordinate system to a mixture fraction-based coordinate system~\cite{peters1984}. It is assumed that the gradients along the flame surface are negligible compared to the gradients normal to the flame surface, which allows the flame structure to be described with only the mixture fraction. For sooting flames, Mueller~\cite{muellerphd} determined that the flamelet equations possess additional terms due to the source terms in \cref{eq:lesmodels:combust:map:z} and the continuity equation. For unity Lewis numbers, \cref{eq:lesmodels:combust:flamelet:consy} is transformed to
\begin{equation}\label{eq:lesmodels:combust:flamelet:flamelety}
  \rho\pder[Y_k]{\tau} = \frac{\rho\chi}{2}\sder[Y_k]{Z} + \dot{m}_k - \dot{\rho}Y_k + (\dot{\rho}Z - \dot{m}_Z)\pder[Y_k]{Z},
\end{equation}
where $\chi = 2D_Z(\partial Z/\partial x_j)^2$ is the scalar dissipation rate, the third term on the right-hand side is due to the source term of the continuity equation, and the last term on the right-hand side is a convective term in mixture fraction space due to the density and mixture fraction source terms. The latter two new terms ensure that mass is conserved and that the database of flamelet solutions remains unique, although they do not play a substantial role in the dynamics of \cref{eq:lesmodels:combust:flamelet:flamelety}.

Similarly, \cref{eq:lesmodels:combust:flamelet:const} is transformed to obtain the flamelet equation for temperature:
\begin{equation}\label{eq:lesmodels:combust:flamelet:flamelett}
  \begin{split}
    \rho c_p \pder[T]{\tau} &= \frac{\rho c_p \chi}{2}\sder[T]{Z} + \frac{\rho\chi}{2}\pder[c_p]{Z}\pder[T]{Z} - \sum\limits_{k} h_k\dot{m}_k + \sum\limits_{k} \frac{\rho c_{p,k}\chi}{2}\pder[Y_k]{Z}\pder[T]{Z} + \dot{q}_{\text{RAD}} \\
    &+ \dot{H} + c_p(\dot{\rho}Z - \dot{m}_Z)\pder[T]{Z},
  \end{split}
\end{equation}
where $\dot{H}$ is the enthalpy source term due to the extraction of PAH from the gas-phase and the last term on the second line is the convection term in mixture fraction space that is analogous to the one discussed in \cref{eq:lesmodels:combust:flamelet:flamelety}. In addition to these quantities, flamelet equations for the progress variable $C$ and heat loss parameter $H$ need to be derived. Performing the coordinate transform on \cref{eq:lesmodels:combust:map:c} provides the flamelet equation for the progress variable:
\begin{equation}\label{eq:lesmodels:combust:flamelet:flameletc}
  \rho\pder[C]{\tau} = \frac{\rho\chi}{2}\sder[C]{Z} + \frac{\dot{m}_{C_{\text{conv}}}}{C^*} - \dot{\rho}C + (\dot{\rho}Z - \dot{m}_Z)\pder[C]{Z},
\end{equation}
and performing a similar operation on \cref{eq:lesmodels:combust:map:heatloss} leads to the flamelet equation for the heat loss parameter:
\begin{equation}\label{eq:lesmodels:combust:flamelet:flameleth}
  \rho\pder[H]{\tau} = \frac{\rho\chi}{2}\sder[H]{Z} + (\dot{\rho}Z - \dot{m}_Z)\pder[H]{Z} + \dot{q}_{\text{RAD}}.
\end{equation}
Since the latter two quantities are not explicit functions of the species mass fractions and temperature, evaluation of \cref{eq:lesmodels:combust:flamelet:flameletc,eq:lesmodels:combust:flamelet:flameleth} is required to produce the complete parameterization of the database of flamelet solutions.

The database is typically filled using the procedure of Ihme and Pitsch~\cite{ihme2008}, where the unsteady forms of \crefrange{eq:lesmodels:combust:flamelet:flamelety}{eq:lesmodels:combust:flamelet:flameleth} are evaluated in order to fill the heat loss parameter space. However, such a procedure is computationally expensive due to the need to resolve the temporal coordinate. In this work, the approach of Carbonell \etal~\cite{carbonell2009} is adopted instead, where the steady forms of \crefrange{eq:lesmodels:combust:flamelet:flamelety}{eq:lesmodels:combust:flamelet:flameleth} are solved and the radiation source term of \cref{eq:lesmodels:combust:flamelet:flamelett} is weighted by a radiation loss coefficient $\Omega_{\text{RAD}} \in [0,1]$ to span the heat loss parameter space. The adiabatic form of \cref{eq:lesmodels:combust:flamelet:flamelett} is specified by $\Omega_{\text{RAD}} = 0$. Radiative thermal losses are introduced by increasing the value of $\Omega_{\text{RAD}}$ while still solving the steady forms of \crefrange{eq:lesmodels:combust:flamelet:flamelety}{eq:lesmodels:combust:flamelet:flameleth}. Note that in the limit of very small values of scalar dissipation rate, no steady solutions with radiation exist.

The adiabatic upper, middle, and lower branches of the S-curve as well as nonadiabatic solutions are shown in \cref{fig:lesmodels:combust:flamelet:tvschi}. These flamelet solutions are uniquely parameterized by $Z$, $C$, and $H$, with the latter two quantities obtained from evaluating \cref{eq:lesmodels:combust:flamelet:flameletc,eq:lesmodels:combust:flamelet:flameleth}. Validation of uniqueness with the specified definitions of mixture fraction and progress variable is available in Mueller~\cite{muellerphd}.

%% The aforementioned adiabatic and nonadiabatic, steady, stable-burning solutions form the upper branches of the S-curve as shown in \cref{fig:lesmodels:combust:flamelet:tvschi}. The S-curve contains the upper, middle, and lower branches that are formed by the flamelet solutions in the flame temperature/scalar dissipation rate plane at an arbitrary mixture fraction. The adiabatic, steady, unstable-burning, middle branch and adiabatic, steady, non-burning, lower branch are obtained through \crefrange{eq:lesmodels:combust:flamelet:flamelety}{eq:lesmodels:combust:flamelet:flameleth} as well, although the initial positions within the $T|Z_{st}$--$\chi|Z_{st}$ plane are different than those of the upper branch solutions. Thus, the set of solutions to the laminar nonpremixed flamelet equations is uniquely parameterized by $Z$, $C$, and $H$, with the latter two quantities obtained from evaluating \cref{eq:lesmodels:combust:flamelet:flameletc,eq:lesmodels:combust:flamelet:flameleth}. Validation of uniqueness with the specified definitions of mixture fraction and progress variable is available in Mueller~\cite{muellerphd}.

\begin{figure}[htb]
  \centering
  \includegraphics[width=0.6\linewidth]{ch-lesmodels/figures/TvsChi-connected}
  \caption[Flamelet Solutions of RFPV Model, $T|Z_{st}$ vs. $\chi|Z_{st}$]{Solutions to the flamelet equations of the RFPV model for a $\ce{C2H4}$/$\ce{H2}$/$\ce{N2}$ [40/41/19 by volume] nonpremixed flame at atmospheric pressure~\cite{mahmoud2017}. The adiabatic solutions are the red circles, magenta triangles, and the black square, which represent the stable-burning, unstable-burning, and non-burning solutions, respectively. The nonadiabatic solutions are the diamonds, where each set of solutions linked with a dashed line possesses a distinct, nonzero $\Omega_{\text{RAD}}$.}
  \label{fig:lesmodels:combust:flamelet:tvschi}
\end{figure}

\section{Presumed PDF Approach}
\label{sec:lesmodels:presumedpdf}

LES only resolves the larger features of a turbulent flow, so small-scale interactions between soot, turbulence, and combustion chemistry need to be modeled. These subfilter interactions are modeled with a density-weighted joint PDF $\tf{P}$ of the thermochemical and soot variables. Density-weighted filtered (Favre filtered) functions of these quantities are obtained from a convolution of the equation of state with $\tf{P}$:
\begin{equation}\label{eq:lesmodels:presumedpdf:filteredfuncs}
  \tf{\phi}(\xi,\mc{M}_j) = \iint \mc{J}(\xi,\mc{M}_j)\tf{P}(\xi,\mc{M}_j) d\xi d\mc{M}_j,
\end{equation}
where $\xi$ is the vector of thermochemical variables and $\mc{M}_j$ is the state-space vector encompassing the moments $M_{x,y}$ and the weight of the delta function $N_0$ allocated for nucleating particles.

The joint subfilter PDF $\tf{P}$ is unknown and must be modeled using transported PDF or presumed PDF methods. In the transported PDF approach, a transport equation is solved for the subfilter PDF~\cite{pope1985,pope1991}. Often this equation is high-dimensional, and therefore conventional Eulerian discretization techniques are computationally intractable. Instead, Lagrangian particle methods are employed, where a set of notional particles that track fluid particles is solved for. The advantage of the class of transported PDF methods is that all local phenomena are naturally described by the subfilter PDF, which provides information about subfilter fluctuations and correlations at a single point. Thus, chemical source terms do not require additional closure because chemical reactions are one-point phenomena. However, processes such as molecular mixing are dependent on gradients and therefore are inherently two-point, non-local phenomena. Mixing models have been proposed such as the Interaction with Exchange of the Mean (IEM)~\cite{dopazo1974} and Modified Curl~\cite{janicka1979}, where quantities are relaxed to local filtered values to mimic homogenization through diffusion. However, such crude approaches fail to use information about localness in composition space. More complex mixing models using the concept of the Euclidean minimum spanning tree~\cite{subramaniam1998} or the shadow position~\cite{pope2013} introduce localness in composition space, but still suffer from the large computational cost associated with evolving the Lagrangian particles.

In contrast, presumed PDF methods assume a form of the joint subfilter PDF. Reasoning based on physical insights about subfilter interactions between the scalars and turbulence is utilized to deduce the form of the PDF. Previous works have indicated that the beta distribution is a good model for the thermochemical portion of the subfilter PDF when considering passive scalars in a two-feed system~\cite{cook1994,jimenez1997,wall2000}. Ihme et al.~\cite{ihme2005} arrived at the same conclusion with the FPV approach for reactive scalars. Since presumed PDF methods do not rely on ad hoc mixing models, they generally match or exceed the performance of transported PDF methods, provided that the form of the PDF is appropriate. Thus, the presumed PDF approach will be adopted in this work.

An approximate form of the overall joint subfilter PDF is determined by first decomposing $\tf{P}$ into a thermochemical PDF and a PDF of the soot scalars conditioned on the thermochemical variables. By Bayes' theorem,
\begin{equation}\label{eq:lesmodels:presumedpdf:bayes}
  \tf{P}(\xi,\mc{M}_j) = \tf{P}(\xi)P(\mc{M}_j|\xi),
\end{equation}
where $\tf{P}(\xi)$ includes a beta distribution and $P(\mc{M}_j|\xi)$ is unknown. The form of the latter has been approximated previously through consideration of the dynamics of soot. In DNS studies of an \textit{n}-heptane/air turbulent nonpremixed jet, PAH were found to have much slower formation chemistry compared to the main heat-releasing chemistry, which is represented by the thermochemical variables $\xi$~\cite{attili2014,bisetti2012}. PAH are soot precursors and therefore soot itself is characterized by even longer timescales. Consequently, Mueller and Pitsch proposed that the time scales of the soot scalars and thermochemical variables are disparate enough such that the former should not depend on the latter~\cite{subfilterpdf2011}. This argument shall be used to simplify the subfilter PDF of the soot scalars conditioned on $\xi$ to a marginal PDF of only the soot scalars. The expression for the spatially Favre filtered functions of \cref{eq:lesmodels:presumedpdf:filteredfuncs} then becomes
\begin{equation}\label{eq:lesmodels:presumedpdf:separated}
  \tf{\phi}(\xi,\mc{M}_j) = \iint \mc{J}(\xi,\mc{M}_j)\tf{P}(\xi)P(\mc{M}_j) d\xi d\mc{M}_j.
\end{equation}
By replacing the functional relation $\mc{J}$ with \cref{eq:lesmodels:combust:producteos}, \cref{eq:lesmodels:presumedpdf:separated} can be further simplified to
\begin{equation}\label{eq:lesmodels:presumedpdf:indep}
  \tf{\phi}(Z,C,H,\mc{M}_j) = \iiint \mc{G}(Z,C,H)\tf{P}(Z,C,H) dHdCdZ \times \int \mc{K}(\mc{M}_j)P(\mc{M}_j) d\mc{M}_j,
\end{equation}
where the thermochemical and soot components are now completely independent. Mueller and Pitsch noted that the time scale separation argument could be violated during the oxidation of soot, when there are enhanced interactions between soot and the major gas-phase chemistry, and during surface growth near the flame. These situations will be examined more closely in \cref{ch:subfilter}.

The challenging task of modeling the joint subfilter PDF has been reduced to developing models for the thermochemical subfilter PDF and the soot subfilter PDF. Discussion of the latter will be deferred to \cref{ch:subfilter}. The thermochemical subfilter PDF for the RFPV approach is obtained from Ihme and Pitsch~\cite{ihme2008}. First, two quantities are introduced to uniquely identify each flamelet solution in the database of thermochemical states: $\Lambda = C(Z_{st})$ and $\Phi = H(Z_{st})$. The thermochemical equation of state \cref{eq:lesmodels:combust:rfpv} then becomes
\begin{equation}\label{eq:lesmodels:presumedpdf:eos}
  \xi = \mc{G}(Z, C, H) = \mg(Z, \Lambda, \Phi).
\end{equation}
The spatially Favre filtered thermochemical functions are given by
\begin{equation}\label{eq:lesmodels:presumedpdf:filteredeos}
  \begin{split}
    \tf{\xi} &= \iiint \mc{G}(Z,C,H)\tf{P}(Z,C,H) dHdCdZ \\
    &= \iiint \mg(Z,\Lambda,\Phi)\tf{P}(Z,\Lambda,\Phi) d\Phi d\Lambda dZ.
  \end{split}
\end{equation}
Since $Z$, $\Lambda$, and $\Phi$ are defined to be independent, the thermochemical subfilter PDF can be expressed as the product of three marginal distributions:
\begin{equation}\label{eq:lesmodels:presumedpdf:trimarg}
  \tf{P}(Z,\Lambda,\Phi) = \beta(Z;\tf{Z},\tf{Z_{\text{V}}})\delta(\Lambda - \tf{\Lambda})\delta(\Phi - \tf{\Phi}),
\end{equation}
where the mixture fraction is modeled with a beta distribution~\cite{cook1994,jimenez1997,wall2000}, the subfilter mixture fraction variance is defined as $\tf{Z_{\text{V}}} = \tf{Z^2} - \tf{Z}^2$, and $\Lambda$ and $\Phi$ are modeled with delta distributions~\cite{ihme2008}. Assuming \cref{eq:lesmodels:combust:rfpv} is unique, a bijective inversion may be used to conveniently reexpress a dependence on $\tf{\Lambda}$ and $\tf{\Phi}$ as a dependence on $\tf{C}$ and $\tf{H}$. In practical implementation, each flamelet solution in the database is convoluted with the beta distribution for the mixture fraction and tabulated as a function of the filtered mixture fraction, subfilter mixture fraction variance, filtered progress variable, and filtered heat loss parameter.

The transport equation for the filtered mixture fraction is derived by spatially filtering \cref{eq:lesmodels:combust:map:z} and is given by
\begin{equation}\label{eq:lesmodels:presumedpdf:filteredz}
  \ftrder{\tf{Z}} = \dsff[\tf{Z}]{\tf{u_j Z}} + \fdt[Z]{\tf{Z}} + \fst[m]{Z},
\end{equation}
where the first and last terms on the right-hand side are unclosed. The former is the subfilter flux, which can be modeled with a dynamic procedure~\cite{germano1991,lilly1992,moin1991}. The filtered source term is closed through the presumed subfilter PDF given in \cref{eq:lesmodels:presumedpdf:trimarg}. Similarly, the transport equations for the filtered progress variable and filtered heat loss parameter are given by
\begin{equation}\label{eq:lesmodels:presumedpdf:filteredc}
  \ftrder{\tf{C}} = \dsff[\tf{C}]{\tf{u_j C}} + \fdt[C]{\tf{C}} + \mean{\left( \frac{\dot{m}_{C_{\text{conv}}}}{C^*} \right)}
\end{equation}
and
\begin{equation}\label{eq:lesmodels:presumedpdf:filteredh}
  \ftrder{\tf{H}} = \dsff[\tf{H}]{\tf{u_j H}} + \fdt[H]{\tf{H}} + \mean{\dot{\rho}H} + \fst[q]{\text{RAD}},
\end{equation}
respectively. The remaining database parameter is the subfilter mixture fraction variance, which will be procured in the manner of Mueller and Pitsch~\cite{mueller2012}. Rather than solving for the subfilter variance directly, a transport equation for the filtered square of the mixture fraction is evaluated:
\begin{equation}\label{eq:lesmodels:presumedpdf:filteredzsq}
  \begin{split}
    \ftrder{\tf{Z^2}} &= \dsff[\tf{Z^2}]{\tf{u_j Z^2}} + \fdt[Z]{\tf{Z^2}} \\
    &- 2\mean{\rho}\tf{D}_Z \pder[\tf{Z}]{x_j}\pder[\tf{Z}]{x_j} - \mean{\rho}\tf{\chi}_{\text{sgs}} - \mean{\dot{\rho}Z^2} + 2\mean{\dot{m}_Z Z},
  \end{split}
\end{equation}
where $\tf{\chi}_{\text{sgs}}$ is the subfilter scalar dissipation rate. This quantity is estimated with a linear relaxation model~\cite{ihme200890}:
\begin{equation}\label{eq:lesmodels:presumedpdf:chisgs}
  \tf{\chi}_{\text{sgs}} = \text{C}_{\chi}\frac{\nu_t}{\Delta^2}\tf{Z_{\text{V}}},
\end{equation}
where $\text{C}_{\chi} \approx 20$, the mechanical timescale is determined from the eddy viscosity $\nu_t$ and filter width $\Delta$, and $\tf{Z_{\text{V}}}$ is the subfilter mixture fraction variance as defined previously.

\section{Polycyclic Aromatic Hydrocarbons}
\label{sec:lesmodels:pah}

Understanding the formation and evolution of soot is central to this work, so accurately modeling the generation and consumption of gas-phase PAH that participate in nucleation and condensation is vital. A simple LES model for tracking the evolution of PAH would involve obtaining the PAH mass fraction from the RFPV combustion model. However, Attili \etal~\cite{attili2014} and Bisetti \etal~\cite{bisetti2012} noted in DNS studies of a nitrogen-diluted, \textit{n}-heptane/air turbulent nonpremixed flame that the PAH mass fraction deviates significantly from the values calculated by the steady flamelet approximation. The chemistry of PAH such as naphthalene is much slower than the chemistry of species that participate in the significantly exothermic reactions. As a result, PAH are unable to immediately adjust to the rapidly changing scalar dissipation rate field. They require a longer time to respond to the fluctuating field, so the unsteady term in the flamelet equations cannot be neglected.

To account for these unsteady phenomena, Mueller and Pitsch~\cite{mueller2012} developed a transport equation model for PAH that was inspired by a similar approach for slowly evolving $\ce{NO}$~\cite{ihme2008}. The spatially filtered transport equation is given by
\begin{equation}\label{eq:lesmodels:pah:pahtrans}
  \ftrder{\tf{Y}_{\text{PAH}}} = -\dsff[\tf{Y}_{\text{PAH}}]{\tf{u_j Y}_{\hspace{-3pt}\text{PAH}}} + \fdt[\text{PAH}]{\tf{Y}_{\text{PAH}}} + \fst[m]{\text{PAH}},
\end{equation}
where $\tf{Y}_{\text{PAH}}$ is the filtered mass fraction of a lumped PAH species and $\fst[m]{\text{PAH}}$ is the sum of the chemical source terms of all the PAH species. Under the assumption of unity Lewis number, all PAH species will have a diffusivity governed by $D_{\text{PAH}}$ and \cref{eq:lesmodels:pah:pahtrans} is the exact transport equation for a lumped PAH. However, if all PAH do not have the same molecular diffusivity, then the diffusivity of naphthalene is preferred for $D_{\text{PAH}}$ as it is generally the PAH species with the largest concentration. In this circumstance, \cref{eq:lesmodels:pah:pahtrans} is an approximate model for the evolution of the combined PAH mass fraction.

The unclosed terms in \cref{eq:lesmodels:pah:pahtrans} include the first and third terms on the right-hand side. The former is simply the subfilter flux and can be modeled using a conventional dynamic procedure as mentioned previously. Following Mueller and Pitsch~\cite{mueller2012}, the filtered chemical source term can be decomposed into a production term $\dot{m}_{+}$, a consumption term $\dot{m}_{-}$, and an additional consumption term due to the extraction of PAH from the gas-phase to form soot (dimerization) $\dot{m}_{D}$. For each individual PAH species $k$, the chemical production term is independent of $Y_k$, the chemical consumption term depends linearly on $Y_k$, and the dimerization term depends quadratically on $Y_k$. The source term of the lumped species will be decomposed similarly:
\begin{equation}\label{eq:lesmodels:pah:pahsrc}
  \fst[m]{\text{PAH}} = \mean{(\dot{m}_{+})} + \mean{\left( \frac{\dot{m}_{-}}{Y_{\text{PAH}}} \right) Y_{\text{PAH}}} + \mean{\left( \frac{\dot{m}_{D}}{{Y_{\text{PAH}}}^2} \right) {Y_{\text{PAH}}}^2},
\end{equation}
where the terms in parentheses are obtained directly from the RFPV combustion model because they are nearly independent of the PAH mass fraction. The last two terms on the right-hand side of \cref{eq:lesmodels:pah:pahsrc} are unclosed filtered correlations. Mueller and Pitsch~\cite{mueller2012} have proposed closure through a scale-similarity assumption between the transported PAH mass fraction and the PAH mass fraction computed from the steady flamelet equations. This model is given by
\begin{equation}\label{eq:lesmodels:pah:similarity}
  \frac{\tf{Y}_{\text{PAH}}}{\tf{Y}_{\text{PAH}}^{\text{RFPV}}} = \frac{Y_{\text{PAH}}^{''}}{Y_{\text{PAH}}^{\text{RFPV}''}},
\end{equation}
where the subscript $^{\text{RFPV}}$ designates quantities that are from the RFPV combustion model and the subscript $^{''}$ indicates the ``fluctuating'' LES quantities given by
\begin{equation}\label{eq:lesmodels:pah:residual}
  Y_{\text{PAH}}^{''} = Y_{\text{PAH}} - \tf{Y}_{\text{PAH}}.
\end{equation}
With this closure model, the filtered source term of the lumped PAH species is
\begin{equation}\label{eq:lesmodels:pah:pahsrcmodeled}
  \fst[m]{\text{PAH}} = \mean{\dot{m}}_{+}^{\text{RFPV}} + \mean{\dot{m}}_{-}^{\text{RFPV}}\left( \frac{\tf{Y}_{\text{PAH}}}{\tf{Y}_{\text{PAH}}^{\text{RFPV}}} \right) + \mean{\dot{m}}_{D}^{\text{RFPV}}\left( \frac{\tf{Y}_{\text{PAH}}}{\tf{Y}_{\text{PAH}}^{\text{RFPV}}} \right)^2.
\end{equation}


\chapter{Soot-Chemistry-Turbulence Interactions\label{ch:subfilter}}

In this chapter, two models for interactions between soot, turbulence, and combustion chemistry at subfilter scales are presented. These models are validated in an \textit{a priori} analysis of filtered moment source terms for oxidation and surface growth using the DNS database of Attili \etal~\cite{attili2014}.

%% Summary: Whole point of the chapter is to show that including a mixture fraction dependence in the soot subfilter model may lead to better predictions of $f_v$.

% include other files for sections of this chapter. These use the 'input' command since each section within a chapter should not start a new page.
% If you want to swap the order of sections, it is as simple as reversing the order you include them.
\section{Joint Subfilter PDF}
\label{sec:subfilter:joint}

As a consequence of \cref{eq:lesmodels:presumedpdf:bayes}, the challenging task of modeling the joint subfilter PDF is reduced to developing models for the thermochemical subfilter PDF and the soot subfilter PDF. The thermochemical subfilter PDF for the RFPV approach is obtained from Ihme and Pitsch~\cite{ihme2008}. First, two quantities are introduced to uniquely identify each flamelet solution in the database of thermochemical states: $\Lambda = C(Z_{st})$ and $\Phi = H(Z_{st})$. The thermochemical equation of state (\cref{eq:lesmodels:combust:rfpv}) then becomes
\begin{equation}\label{eq:lesmodels:presumedpdf:eos}
  \xi = \mc{G}(Z, C, H) = \mg(Z, \Lambda, \Phi).
\end{equation}
The spatially Favre filtered thermochemical functions are given by
\begin{equation}\label{eq:lesmodels:presumedpdf:filteredeos}
  \begin{split}
    \tf{\xi} &= \iiint \mc{G}(Z,C,H)\tf{P}(Z,C,H) dHdCdZ \\
    &= \iiint \mg(Z,\Lambda,\Phi)\tf{P}(Z,\Lambda,\Phi) d\Phi d\Lambda dZ.
  \end{split}
\end{equation}
Since $Z$, $\Lambda$, and $\Phi$ are defined to be independent, the thermochemical subfilter PDF can be expressed as the product of three marginal distributions:
\begin{equation}\label{eq:lesmodels:presumedpdf:trimarg}
  \tf{P}(Z,\Lambda,\Phi) = \beta(Z;\tf{Z},\tf{Z_{\text{V}}})\delta(\Lambda - \tf{\Lambda})\delta(\Phi - \tf{\Phi}),
\end{equation}
where the mixture fraction has been modeled with a beta distribution~\cite{cook1994,jimenez1997,wall2000}, the subfilter mixture fraction variance is defined as $\tf{Z_{\text{V}}} = \tf{Z^2} - \tf{Z}^2$, and $\Lambda$ and $\Phi$ are modeled with delta distributions~\cite{ihme2008}. Assuming \cref{eq:lesmodels:combust:rfpv} is unique, a bijective inversion may be used to conveniently reexpress a dependence on $\tf{\Lambda}$ and $\tf{\Phi}$ as a dependence on $\tf{C}$ and $\tf{H}$.

The subfilter PDF of the soot scalars conditioned on the thermochemical variables in \cref{eq:lesmodels:presumedpdf:bayes} has been approximated previously through physical arguments. In previous works, PAH were found to have much slower formation chemistry compared to the main heat-releasing chemistry, which is represented by the thermochemical variables $\xi$~\cite{attili2014,bisetti2012}. PAH are soot precursors, so soot itself is characterized by even longer timescales. Consequently, Mueller and Pitsch~\cite{subfilterpdf2011} proposed that the timescales of the soot scalars and thermochemical variables are disparate enough such that the former should not depend on the latter. This argument was used to simplify the subfilter PDF of the soot scalars conditioned on $\xi$ to a marginal PDF of only the soot scalars. The expression for the spatially Favre filtered functions of \cref{eq:lesmodels:presumedpdf:filteredfuncs} then becomes
\begin{equation}\label{eq:lesmodels:presumedpdf:separated}
  \tf{\phi}(\xi,\mc{M}_j) = \iint \mc{J}(\xi,\mc{M}_j)\tf{P}(\xi)P(\mc{M}_j) d\xi d\mc{M}_j.
\end{equation}
By replacing the functional relation $\mc{J}$ with \cref{eq:lesmodels:combust:producteos}, \cref{eq:lesmodels:presumedpdf:separated} can be further simplified to
\begin{equation}\label{eq:lesmodels:presumedpdf:indep}
  \tf{\phi}(Z,C,H,\mc{M}_j) = \iiint \mc{G}(Z,C,H)\tf{P}(Z,C,H) dHdCdZ \times \int \mc{K}(\mc{M}_j)P(\mc{M}_j) d\mc{M}_j,
\end{equation}
where the thermochemical and soot components are now completely independent. A model for the subfilter PDF of the soot scalars will be presented in \cref{sec:subfilter:zussp}. Mueller and Pitsch noted that the timescale separation argument could be violated during the oxidation of soot, when there are enhanced interactions between soot and the major gas-phase chemistry, and during surface growth near the flame. These situations will be examined more closely in \cref{sec:subfilter:zassp}.

%% The challenging task of modeling the joint subfilter PDF was then reduced to developing models for the thermochemical subfilter PDF and the soot subfilter PDF. Discussion of the latter will be deferred to \cref{ch:subfilter}. The thermochemical subfilter PDF for the RFPV approach is obtained from Ihme and Pitsch~\cite{ihme2008}. First, two quantities are introduced to uniquely identify each flamelet solution in the database of thermochemical states: $\Lambda = C(Z_{st})$ and $\Phi = H(Z_{st})$. The thermochemical equation of state (\cref{eq:lesmodels:combust:rfpv}) then becomes
%% \begin{equation}\label{eq:lesmodels:presumedpdf:eos}
%%   \xi = \mc{G}(Z, C, H) = \mg(Z, \Lambda, \Phi).
%% \end{equation}
%% The spatially Favre filtered thermochemical functions are given by
%% \begin{equation}\label{eq:lesmodels:presumedpdf:filteredeos}
%%   \begin{split}
%%     \tf{\xi} &= \iiint \mc{G}(Z,C,H)\tf{P}(Z,C,H) dHdCdZ \\
%%     &= \iiint \mg(Z,\Lambda,\Phi)\tf{P}(Z,\Lambda,\Phi) d\Phi d\Lambda dZ.
%%   \end{split}
%% \end{equation}
%% Since $Z$, $\Lambda$, and $\Phi$ are defined to be independent, the thermochemical subfilter PDF can be expressed as the product of three marginal distributions:
%% \begin{equation}\label{eq:lesmodels:presumedpdf:trimarg}
%%   \tf{P}(Z,\Lambda,\Phi) = \beta(Z;\tf{Z},\tf{Z_{\text{V}}})\delta(\Lambda - \tf{\Lambda})\delta(\Phi - \tf{\Phi}),
%% \end{equation}
%% where the mixture fraction has been modeled with a beta distribution~\cite{cook1994,jimenez1997,wall2000}, the subfilter mixture fraction variance is defined as $\tf{Z_{\text{V}}} = \tf{Z^2} - \tf{Z}^2$, and $\Lambda$ and $\Phi$ are modeled with delta distributions~\cite{ihme2008}. Assuming \cref{eq:lesmodels:combust:rfpv} is unique, a bijective inversion may be used to conveniently reexpress a dependence on $\tf{\Lambda}$ and $\tf{\Phi}$ as a dependence on $\tf{C}$ and $\tf{H}$.

\section{Z-Uniform Soot Subfilter PDF}
\label{sec:subfilter:zussp}

Description of ZUSSP.

% \section{Evaluation in LES}
\label{sec:subfilter:leszussp}

LES of a series of turbulent, nonpremixed, ethylene-hydrogen-nitrogen [40/41/19 by volume] jet flames~\cite{mahmoud2017} is used to validate the $Z$-uniform soot subfilter PDF (\cref{eq:subfilter:zussp:dd}) in the framework presented in \cref{ch:lesmodels}. This is achieved by comparing thermocouple readings of flame temperature and laser-induced incandescence (LII) measurements of soot volume fraction with the profiles from LES. In this series, three flames are maintained at a jet exit Reynolds number of 15,000 while the exit strain rate is varied by altering the jet diameter and fuel exit velocity. A uniform coflow of air at a velocity of 1.1 m/s surrounds these flames. \cref{tab:subfilter:leszussp:ehn} summarizes other key aspects of the experimental setup, and complete details can be obtained from Mahmoud et al.~\cite{mahmoud2017}.

\begin{table}[htbp]
\centering
\caption[Flow Conditions of Turbulent Nonpremixed \ce{C2H4}/\ce{H2}/\ce{N2} Jet Flames]{Flow conditions of turbulent nonpremixed \ce{C2H4}/\ce{H2}/\ce{N2} jet flames.}
\label{tab:subfilter:leszussp:ehn}
\begin{tabular}{p{0.35\textwidth} p{0.1\textwidth} p{0.12\textwidth} p{0.12\textwidth} p{0.12\textwidth}}
\toprule
\textbf{Flame} & & \bm{$1/\tau|_{H}$} & \bm{$1/\tau|_{M}$} & \bm{$1/\tau|_{L}$} \\
\midrule

Central jet diameter, $D$
& [mm] & 4.4 & 5.8 & 8.0 \\[0.2em]

Mean jet exit velocity, $U$
& [m/s] & 56.8 & 42.4 & 31.5 \\[0.2em]

Exit strain rate, $U/D$
& [s$^{-1}$] & 12,900 & 7300 & 4300 \\[0.2em]

Exit Reynolds number, $Re_D$
& [--] & 15,000 & 15,000 & 15,000 \\[0.2em]

Mean flame length, $L_f$
& [mm] & 710 & 930 & 1060 \\

\bottomrule
\end{tabular}
\end{table}

A brief description of the numerical approach is provided here and will be addressed fully in \cref{ch:lesresults}. Grid filtered LES is achieved through NGA, a low-Mach number flow solver~\cite{desjardins2008}. The computational domain of each flame consists of a structured grid with $192 \times 96 \times 32$ points in the streamwise, radial, and circumferential directions, respectively. This corresponds to a domain with lengths $200D$ in the streamwise direction and $60D$ in the radial direction, where the grid is stretched for both. The flamelet solutions in the database are computed with FlameMaster, a solver for 0D and 1D flame calculations~\cite{flamemaster}. These solutions utilize the chemical mechanism of Blanquart et al.~\cite{blanquart2009588}, which encompasses the high temperature combustion of fuels from methane to iso-octane and places emphasis on the formation of soot precursors up to cyclopenta[cd]pyrene (\ce{C18H10}). The mechanism of Narayanaswamy et al.~\cite{narayanaswamy2010} is incorporated into the base mechanism to model the high-temperature oxidation of aromatic species.

In \cref{fig:subfilter:leszussp:zusspleseval}, it is clear that time-averaged, centerline flame temperature is generally well-predicted by LES in all flames. In the recorded experiments, the flame temperature rises to a peak value around $x/D = 100$, at which point it declines due to the presence of soot. The flame with the highest exit strain rate ($1/\tau|_H$, left plot) possesses a small region of uniform temperature around $x/D = 0$ due to the lifting of the flame in LES. This feature is not present in the experiment, and therefore explains the slight deviation from the empirical values until about $x/D = 50$. In all three flames, the peak mean temperature from LES closely matches the magnitude of the corresponding experimental value (1800 to 1900 K), but is shifted slightly downstream.

\begin{figure}[htb]
  \begin{center}
    \includegraphics[width=\linewidth]{ch-subfiltermodeling/figures/combined-Tfv-S-ZUSSP-Le_1}
    \caption[LES Validation of \texorpdfstring{$Z$}{Z}-Uniform Soot Subfilter PDF]{\textit{Left to right} - LES of jet flames $1/\tau|_H$, $1/\tau|_M$, and $1/\tau|_L$, respectively. Circles indicate experimental data while solid lines are time-averaged values from LES. Profiles in black correspond to centerline flame temperatures while profiles in blue represent centerline soot volume fractions. Note that the soot volume fraction profiles from LES have been premultiplied by a factor of 10 for visibility.}
    \label{fig:subfilter:leszussp:zusspleseval}
  \end{center}
\end{figure}

Most obvious in \cref{fig:subfilter:leszussp:zusspleseval} is the underprediction of the time-averaged, centerline soot volume fraction by LES. In all three flames, the volume fraction is underestimated by more than two orders of magnitude relative to the empirical values. Furthermore, the mean soot volume fractions from LES peak slightly earlier and approach zero sooner at approximately $x/D = 130$. These phenomena signal the presence of intense oxidation further upstream compared to the experiments. Such a disparity between the predicted and measured profiles raises the question of what could be modeled inaccurately. To determine the source of this inconsistency, the assumptions and deficiencies of the models need to be revisited.

The soot volume fraction is a measure of the total volume of soot relative to the total volume of gas. The amount of soot that is present depends on the rates of nucleation, condensation, surface growth, and oxidation. An underprediction of the volume fraction suggests that soot growth through nucleation, condensation, and surface growth is insufficient, or that soot oxidation is overwhelmingly excessive. Expressions for the filtered moment source terms of these modes are similar in form to \cref{eq:lesmodels:presumedpdf:separated}. For example, the filtered moment source term for oxidation is given by
\begin{equation}\label{eq:subfilter:leszussp:ox}
  \begin{split}
    \mean{\dot{M}}_{x,y}^{ox} &= \iint k_{ox}(Z)\mc{M}_j P(\mc{M}_j)\pz dZ d\mc{M}_j \\
    &= \hat{k}_{ox}\mean{M}_j,
  \end{split}
\end{equation}
where $k_{ox}(Z)$ is the oxidation coefficient, $\pz$ is modeled with a beta distribution as in \cref{eq:lesmodels:presumedpdf:trimarg}, and the soot subfilter PDF $P(\mc{M}_j)$ is provided in \cref{eq:subfilter:zussp:dd}. Contributions from the thermochemical and soot variables are made distinct in the second line, where $\hat{k}_{ox} = \int k_{ox}(Z)\pz dZ$ and $\mean{M}_j = \int \mc{M}_j P(\mc{M}_j)d\mc{M}_j$. \cref{eq:subfilter:leszussp:ox} incorporates the joint PDF simplification of Mueller and Pitsch~\cite{subfilterpdf2011}, which postulates that the soot scalars should be independent of the thermochemical variables due to the long timescales of PAH and soot formation relative to the timescales of the highly exothermic combustion chemistry. However, they also noted that such an assumption could be violated when surface growth near the flame is the dominant growth mechanism or during the oxidation of soot, when interactions with gas-phase chemistry are enhanced. The likelihood of the former is smaller, as the DNS studies of Bisetti et al.~\cite{bisetti2012} demonstrated that soot growth through PAH-based nucleation and condensation dominates acetylene-based surface growth in turbulent nonpremixed combustion. Thus, this work hypothesizes that the excessive oxidation of soot, due to the inappropriate simplification of the joint subfilter PDF \cref{eq:lesmodels:presumedpdf:bayes}, is the reason for the dramatic underprediction of the soot volume fraction as shown in \cref{fig:subfilter:leszussp:zusspleseval}.

As a result of the decorrelation of the soot scalars from the thermochemical variables, the current soot subfilter PDF implictly assumes that soot is uniformly distributed in mixture fraction space. However, for non-smoking flames, this is qualitatively incorrect as there should be zero soot in fuel-lean regions of the flame. This is evident in \cref{fig:subfilter:leszussp:ysvsz}, reproduced from the 3D DNS of a nitrogen-diluted, \textit{n}-heptane/air turbulent nonpremixed planar jet flame~\cite{attili2014}. The lack of soot at mixture fractions below stoichiometric can be explained by the preferential transport of soot generated in the region $0.3 < Z < 0.5$ to richer regions by turbulent fluctuations and by the oxidation of all soot during transport towards the stoichiometric surface. 

\begin{figure}[htb]
  \begin{center}
    \includegraphics[width=0.7\linewidth]{ch-subfiltermodeling/figures/dns_Ysoot_vs_Z}
    \caption[DNS of Turbulent Nonpremixed \ce{C7H16}/\ce{N2} Jet Flame, \texorpdfstring{$\langle Y_{\text{s}}|Z \rangle$}{<Ys|Z>} vs. \texorpdfstring{$Z$}{Z}]{Mean soot mass fraction conditioned on mixture fraction at various times in a 3D DNS of an \textit{n}-heptane/nitrogen [15/85 by volume] and air turbulent nonpremixed planar jet flame, reproduced from Attili et al.~\cite{attili2014}. The stoichiometric mixture fraction ($Z_{st} = 0.147$) is demarcated with the vertical dashed line. \textit{Left} - 1 ms (filled squares), 2 ms (crosses), 3 ms (open squares), 4 ms (circles), and 5 ms (triangles). \textit{Right} - 6 ms (stars), 8 ms (circles), 10 ms (open squares), and 20 ms (filled squares). The small gray dots represent the soot mass fraction field at 20 ms.}
    \label{fig:subfilter:leszussp:ysvsz}
  \end{center}
\end{figure}

The non-uniform nature of soot in mixture fraction space is captured by the flamelet solutions that are accessed during LES as well. Filtered moment source term coefficients for oxidation, surface growth, and the combination of nucleation and condensation are plotted in \cref{fig:subfilter:leszussp:kvsz}. It is clear that different soot evolution modes are dominant over distinct regions of mixture fraction even as the fuel mixture and stoichiometric scalar dissipation rate are varied. Soot growth through PAH-based nucleation and condensation prevails at very rich values of mixture fraction, whereas acetylene-based surface growth is more dominant at moderately rich mixture fractions. It is noteable that for a fixed fuel mixture (middle and right plots), a reduction in the value of stoichiometric scalar dissipation rate induces a large increase in the coefficient for nucleation and condensation while the coefficient for surface growth is lessened. This trend is due to the increasing effectiveness of PAH chemistry at smaller values of scalar dissipation rate. Thus, PAH-based soot growth is supported at leaner conditions, which ultimately supplants acetylene-based surface growth.

\begin{figure}[htb]
  \begin{center}
    \includegraphics[width=\linewidth]{ch-subfiltermodeling/figures/flamelet_sootcoeffs_le1}
    \caption[Soot Growth and Oxidation Coefficients, 1/\texorpdfstring{$\tau$}{t} vs. \texorpdfstring{$Z$}{Z}]{Flamelet calculations of soot growth and oxidation coefficients for filtered moment source terms as a function of mixture fraction. The blue lines are the oxidation coefficients, the red lines represent the surface growth coefficients, and the green lines are the coefficients for nucleation and condensation. \textit{Left} - Fuel mixture \textit{n}-heptane/nitrogen [15/85 by volume] used in the DNS of \cref{fig:subfilter:zussp:chisensitivity,fig:subfilter:leszussp:ysvsz} at $\chi_{st} = 10\ s^{-1}$. \textit{Middle} \& \textit{Right} - Fuel mixture \ce{C2H4}/\ce{H2}/\ce{N2} [40/41/19 by volume] associated with the LES of \cref{fig:subfilter:leszussp:zusspleseval} at $\chi_{st} = 10\ s^{-1}$ and $\chi_{st} = 0.1\ s^{-1}$, respectively.}
    \label{fig:subfilter:leszussp:kvsz}
  \end{center}
\end{figure}

Soot oxidation is the dominant mode at mixture fractions below and slightly above the stoichiometric value. Since the rates associated with the high-temperature oxidation chemistry are comparable with those of the main heat-releasing chemistry~\cite{guo2016}, it is anticipated that the magnitude of the peak oxidation coefficient will not vary much as the fuel mixture or stoichiometric scalar dissipation rate is modified. Indeed, this phenomenon is evident in \cref{fig:subfilter:leszussp:kvsz}. Additionally, the oxidation coefficient at stoichiometric mixture fraction is at least an order of magnitude larger than the maximum value of the surface growth coefficient. Thus, the complete oxidation of soot by \ce{OH} (and \ce{O2} to a lesser degree) is expected during transport towards stoichiometric regions. However, the soot subfilter PDF given by \cref{eq:subfilter:zussp:dd} allows for the existence of soot in fuel-lean areas, as it assumes that soot is uniformly distributed in mixture fraction space. The presence of large, non-zero oxidation coefficients at lean values of mixture fraction is therefore concerning, as it potentially leads to substantial filtered moment source terms (\cref{eq:subfilter:leszussp:ox}). This contribution to the oxidation rate is artificial and could be an explanation for the underpredicted soot volume fraction in \cref{fig:subfilter:leszussp:zusspleseval}. To prevent this non-physical oxidation from occurring, the subfilter PDF must depend on the thermochemical variables to exclude the presence of soot at lean mixtures $Z < Z_{st}$. A soot subfilter PDF that addresses this point is introduced in the following section.

\section{Z-Activated Soot Subfilter PDF}
\label{sec:subfilter:zassp}

Develop formulation for ZASSP.

\section{\textit{A Priori} Analysis}
\label{sec:subfilter:dns}

\subsection{DNS Database}
\label{sec:subfilter:dns:database}

The effectiveness of the proposed $Z$-activated soot subfilter PDF at reducing the oxidation rate is validated through an \textit{a priori} analysis using the database from a three-dimensional DNS of a temporally evolving, turbulent nonpremixed planar jet flame at atmospheric pressure~\cite{attili2014}. In this simulation, a central fuel slab consisting of an \textit{n}-heptane/nitrogen [15/85 by volume] mixture at 400 K is surrounded on either side by an air coflow at 800 K. The initial velocity field of the fuel jet is obtained from an instantaneous realization of a turbulent channel flow at $Re_{\tau} = 390$ with a centerline value of $U_c = 8.74$ m/s. The surrounding air flows in the opposite direction of the fuel jet with a streamwise velocity of the same magnitude to give a jet Reynolds number $Re = 2U_c H/\nu \approx 15,000$.

Oxidation of \textit{n}-heptane is modeled with a reduced mechanism comprising 47 species and 290 reactions that accounts for the formation of PAH up to naphthalene. The soot model, as described in \cref{ch:lesmodels,ch:subfilter}, is used in both the DNS and the \textit{a priori} analysis. In the DNS, the soot population is described with seven statistical moments whereas only three moments and the weight of the delta function are used in the \textit{a priori} investigation. The choice to use a reduced set of moments allows for validation of the proposed soot model in a configuration that will most likely be implemented in LES.

The DNS domain is discretized with $N_x \times N_y \times N_z = 1024 \times 1024 \times 512$ grid points, where the homogeneous region of the domain ($|y/H| \le 2.8$) has a grid spacing of $h = 91\ \mu$m. For a filter width $\Delta$, the \textit{a priori} study utilizes a subset of the domain of size $N_x \times N_y \times (\Delta/h + 1)$, where low-pass filtering is done only within the homogeneous mesh region. Additionally, analysis is performed on a snapshot of the DNS at $t = 5$ ms and at a stoichiometric scalar dissipation rate of $\chi_{st} = 20$ s$^{-1}$. Other key properties of the DNS are summarized in \cref{tab:subfilter:dns:params}, and complete details can be found in Attili et al.~\cite{attili2014}.

\begin{table}[htbp]
\centering
\caption[Parameters for DNS of Turbulent Nonpremixed \ce{C7H16}/\ce{N2} Jet Flame]{Parameters for DNS of a turbulent nonpremixed \ce{C7H16}/\ce{N2} jet flame}
\label{tab:subfilter:dns:params}
\begin{tabular}{p{0.46\textwidth} p{0.15\textwidth} p{0.2\textwidth}}
\toprule
Initial jet width, $H$
& [mm] & 15 \\[0.2em]

Domain size, $L_x \times L_y \times L_z$
& [mm] & $94 \times 105 \times 47$ \\[0.2em]

Time step, $\Delta t$
& [$\mu$s] & 4 \\[0.2em]

Minimum Kolmogorov scale, $\eta$
& [$\mu$m] & 110 \\[0.2em]

Kinematic viscosity of fuel mixture, $\nu$
& [m$^2$s$^{-1}$] & $1.7 \times 10^{-5}$ \\[0.2em]

Stoichiometric mixture fraction, $Z_{st}$
& [--] & 0.147 \\

\bottomrule
\end{tabular}
\end{table}

In order to validate the proposed LES model against DNS, a low-pass filter must be applied to the chosen snapshot of the DNS database. A three-dimensional, clipped and renormalized Gaussian filter kernel is employed and is given by
\begin{equation}\label{eq:subfilter:dns:kernel}
  F(x,y,z) = \kappa^3\exp\left[ \frac{-6(x^2 + y^2 + z^2)}{\Delta^2} \right],
\end{equation}
where $\Delta$ is the filter width and $\kappa$ is a renormalization constant that ensures the following relation holds for a grid spacing $h$:
\begin{equation}\label{eq:subfilter:dns:unity}
  \sum\limits_{x = -\Delta/2h}^{\Delta/2h} \sum\limits_{y = -\Delta/2h}^{\Delta/2h} \sum\limits_{z = -\Delta/2h}^{\Delta/2h} F(x,y,z) = 1.
\end{equation}
Note that the above relations require a homogeneous mesh such that $\kappa = \kappa_j$, $\Delta = \Delta_j$, and $h = h_j$, where the index $j \in \{ x,y,z \}$. As evident in \cref{eq:subfilter:dns:unity}, the filter kernel is active over a cube of $(\Delta/h + 1)^3$ grid points.


\subsection{Oxidation Source Term}
\label{sec:subfilter:dns:ox}

The filtered moment source term for oxidation is evaluated with the soot subfilter PDFs of \cref{eq:subfilter:leszussp:ox,eq:subfilter:zassp:ox}. The goal of the proposed model is to increase the overall amount of soot, so this analysis will focus on the oxidation source term contribution to the total volume fraction, $\fst[M]{1,0}^{ox}$. Oxidation of only the outermost surface is considered to reduce modeling complexity and computational cost, so the expression for the source term depends on the filtered moment for the total soot surface area $\mean{M}_{0,1}$.

Several approximations need to be made with regards to \cref{eq:subfilter:zassp:ox} before the \textit{a priori} analysis can be performed. First, the oxidation coefficient $k_{ox}(Z)$ is not present in the DNS database, as it was originally obtained from a flamelet model. However, spatial fields of density and mixture fraction are available, and the oxidation coefficient can be calculated as a function of space. Thus, $k_{ox}(Z)$ will be estimated with a density-weighted conditional average
\begin{equation}\label{eq:subfilter:dns:condkox}
  \begin{split}
    k_{ox}(Z) &\approx \{ k_{ox}(x_j)|Z(x_j) \} \\
    &= \frac{<\rho(x_j)k_{ox}(x_j)|Z(x_j)>}{<\rho(x_j)|Z(x_j)>},
  \end{split}
\end{equation}
where the angle brackets $< \cdot >$ denote the conditional averaging operator and the curly brackets $\{ \cdot \}$ denote the density-weighted conditional averaging operator. This quantity is plotted in \cref{fig:subfilter:dns:kox}, where it is evident that the oxidation coefficient at mixture fractions $Z < 0.1$ and $Z > 0.4$ is well-approximated by \cref{eq:subfilter:dns:condkox}. However, the large spread of the DNS field in $0.1 < Z < 0.4$ indicates that additional conditional averaging against the scalar dissipation rate or another variable could be performed for a more accurate approximation.

\begin{figure}[htb]
  \centering
  \includegraphics[width=0.4\linewidth]{ch-subfiltermodeling/figures/koxvsz}
  \caption[Approximation for Oxidation Coefficient, \texorpdfstring{$k_{ox}(Z)$}{kox(Z)}]{Density-weighted, conditionally averaged approximation for the oxidation coefficient $k_{ox}(Z)$. The gray dots represent $k_{ox}(x_j)|Z(x_j)$ from DNS and the red circles are the approximation $\{ k_{ox}(x_j)|Z(x_j) \}$, evaluated with 200 bins. The vertical black dashed line marks the location of stoichiometric mixture fraction $Z_{st} = 0.147$.}
  \label{fig:subfilter:dns:kox}
\end{figure}

The second quantity that needs to be estimated is the thermochemical subfilter PDF $\pz$, since the DNS database does not have the infinite number of points required to construct the true subfilter distribution. As before, this work will presume the form of a beta PDF for $\pz$. The error associated with this assumption can be discerned by juxtaposing two versions of \cref{eq:subfilter:leszussp:ox} that use the same $Z$-uniform soot subfilter PDF but approximate $k_{ox}(Z)$ with and without $\pz$. The form of the filtered moment source term for oxidation that excludes $\pz$ shall be referred to as the ``ZUSSP without $\pz$'' case and is given by
\begin{equation}\label{eq:subfilter:dns:mzusspwithoutpz}
  \fst[M]{1,0}^{ox} = \{ \tf{k_{ox}(x_j)|Z(x_j)} \} \cdot \mean{M_{0,1}(x_j)},
\end{equation}
where the $\tf{(\cdot)}$ operator denotes the density-weighted version of the low-pass filtering operator $\mean{(\cdot)}$. Use of the $Z$-uniform soot subfilter PDF implies that the soot scalars are assumed to be independent of the thermochemical variables. This property is evident in \cref{eq:subfilter:dns:mzusspwithoutpz}, where the two components are distinctly separated.

The source term using $\pz$ will be referred to as the ``ZUSSP with $\pz$'' case, and is given by
\begin{equation}\label{eq:subfilter:dns:mzusspwithpz}
  \fst[M]{1,0}^{ox} = \int\limits_0^1 \{ k_{ox}|Z \}\pz dZ \cdot \mean{M_{0,1}(x_j)},
\end{equation}
where $\{ k_{ox}|Z \}$ is the density-weighted, conditionally averaged oxidation coefficient that is not a function of space. \cref{eq:subfilter:dns:mzusspwithoutpz,eq:subfilter:dns:mzusspwithpz} are first individually compared to the filtered moment source term from DNS, which will be referred to as the ``DNS'' case and is given by
\begin{equation}\label{eq:subfilter:dns:mdns}
  \fst[M]{1,0}^{ox} = \mean{\{ k_{ox}(x_j)|Z(x_j)\} \cdot M_{0,1}(x_j)}.
\end{equation}
These comparisons are visible for a filter width of $\Delta/h = 32$ in \cref{fig:subfilter:dns:erroronbetaox}. The top two plots show that \cref{eq:subfilter:dns:mzusspwithoutpz,eq:subfilter:dns:mzusspwithpz} generally overpredict the oxidation rate compared to the value from DNS. However, when they are compared with each other in the bottom left plot, the sample standard deviation is roughly an order of magnitude smaller. This suggests that the error associated with presuming the form of a beta distribution for $\pz$ is less than the error associated with using the $Z$-uniform soot subfilter PDF to evaluate the filtered moment source term for oxidation.

The validity of presuming the beta distribution is more easily elucidated in the bottom right plot of \cref{fig:subfilter:dns:erroronbetaox}, where the cumulative distribution function (CDF) is plotted for $\fst[M]{1,0}^{ox}$. It can be observed that the distance between the lines representing \cref{eq:subfilter:dns:mzusspwithoutpz,eq:subfilter:dns:mzusspwithpz} is generally much less than the deviation between the latter and the source term from DNS. However, this trend is not valid when the magnitude of the source term is on the order of $10^2$ to $10^3$, where the error of presuming a beta distribution is commensurate with the error from the form of the soot subfilter PDF. Nevertheless, the most important values of the source term are the largest, where the error associated with presuming the form of a beta distribution for $\pz$ is relatively small.  This trend holds true for larger filter widths as well.

\begin{figure}[ht]
  \centering
  \begin{subfigure}[b]{0.375\linewidth}
    \centering
    \includegraphics[width=\linewidth]{ch-subfiltermodeling/figures/lin-Mox3vsMox6-r3D-32}
    %\vspace{1ex}
  \end{subfigure}%%
  \begin{subfigure}[b]{0.375\linewidth}
    \centering
    \includegraphics[width=\linewidth]{ch-subfiltermodeling/figures/lin-Mox4vsMox6-r3D-32}
    %\vspace{1ex}
  \end{subfigure}
  \begin{subfigure}[b]{0.375\linewidth}
    \centering
    \includegraphics[width=\linewidth]{ch-subfiltermodeling/figures/lin-Mox4vsMox3-r3D-32}
  \end{subfigure}%%
  \begin{subfigure}[b]{0.375\linewidth}
    \centering
    \includegraphics[width=\linewidth]{ch-subfiltermodeling/figures/cdf-ox-ZUSSP-r3D-32}
  \end{subfigure}
  \caption[Error Associated with \texorpdfstring{$\pz = \beta(Z;\tf{Z},\tf{Z_V})$}{P(Z) = B(Z;Z,ZV)} for \texorpdfstring{$\fst[M]{1,0}^{ox}$}{M1,0ox}]{Filtered moment source term for oxidation at $t = 5$ ms evaluated with \cref{eq:subfilter:dns:mzusspwithoutpz,eq:subfilter:dns:mzusspwithpz,eq:subfilter:dns:mdns} for a filter width of $\Delta/h = 32$. In the first three plots, the red lines represent a one-to-one correspondence and the sample standard deviation is indicated at the top left corner. In the fourth, bottom right plot, the solid red line is the ``DNS'' case, the blue dashed line is the ``ZUSSP without $\pz$'' case, and the blue dash-dotted line is the ``ZUSSP with $\pz$'' case.}
  \label{fig:subfilter:dns:erroronbetaox}
\end{figure}

Now that the form of the thermochemical subfilter PDF $\pz$ has been confirmed, the filtered moment source term for oxidation using the $Z$-activated soot subfilter PDF can be evaluated. The latter shall be referred to as the ``ZASSP with $\pz$'' case and its expression is given by
\begin{equation}\label{eq:subfilter:dns:mzasspwithpz}
  \fst[M]{1,0}^{ox} = \frac{\int\limits_0^1 \{ k_{ox}|Z \}H(Z - Z_{st})\pz dZ}{\int\limits_0^1 H(Z - Z_{st})\pz dZ} \cdot \mean{M_{0,1}(x_j)}.
\end{equation}

This case is plotted against the source term from DNS in \cref{fig:subfilter:dns:zasspcomparison}. In the left-hand plot, it is clear that the filtered moment source term for oxidation evaluated with the proposed model still overpredicts the oxidation rate when compared to the values from DNS. However, when contrasted with the top right plot of \cref{fig:subfilter:dns:erroronbetaox}, the magnitudes of the largest source terms are reduced by nearly half and the standard deviation is decreased. A direct comparison between the source terms using the $Z$-uniform and $Z$-activated soot subfilter PDFs, available in the middle plot of \cref{fig:subfilter:dns:zasspcomparison}, demonstrates that the latter tends to produce smaller oxidation rates than the former. This is to be expected, as the $Z$-activated soot subfilter PDF was formulated to eliminate the unphysical contributions to the oxidation rate that the $Z$-uniform soot subfilter PDF possessed.

The CDF in the right-hand plot of \cref{fig:subfilter:dns:zasspcomparison} more clearly reveals the extent of the oxidation source term reduction. It is evident that the proposed model has decreased the largest values of the source term relative to the ``ZUSSP with $\pz$'' case, albeit the effect is not enough to reach the values of the source terms from DNS. Nevertheless, the effectiveness of the proposed model is expected to increase with the filter width due to the expanded presence of lean subfilter regions, which is a consequence of the enlargened variance in mixture fraction. This point will be explored in \cref{sec:subfilter:dns:fw}.

\begin{figure}[ht]
  \centering
  \begin{subfigure}[b]{0.33\linewidth}
    \centering
    \includegraphics[width=\linewidth]{ch-subfiltermodeling/figures/lin-Mox5vsMox6-r3D-32}
  \end{subfigure}%%
  \begin{subfigure}[b]{0.33\linewidth}
    \centering
    \includegraphics[width=\linewidth]{ch-subfiltermodeling/figures/lin-Mox5vsMox4-r3D-32}
  \end{subfigure}%%
  \begin{subfigure}[b]{0.33\linewidth}
    \centering
    \includegraphics[width=\linewidth]{ch-subfiltermodeling/figures/cdf-ox-ZASSP-r3D-32}
  \end{subfigure}
  \caption[Comparison of ZASSP with $\pz$ to DNS \& ZUSSP with $\pz$ for \texorpdfstring{$\fst[M]{1,0}^{ox}$}{M1,0ox}]{Comparison of the ``ZASSP with $\pz$'' case to the ``DNS'' and ``ZUSSP with $\pz$'' cases for the same conditions as in \cref{fig:subfilter:dns:erroronbetaox}. In the right-hand plot, the solid red line is the ``DNS'' case, the blue dash-dotted line is the ``ZUSSP with $\pz$'' case, and the magenta dash-dotted line is the ``ZASSP with $\pz$'' case.}
  \label{fig:subfilter:dns:zasspcomparison}
\end{figure}


\subsection{Surface Growth Source Term}
\label{sec:subfilter:dns:sg}

Analysis of surface growth source term.


\subsection{Effects of Filter Width}
\label{sec:subfilter:dns:fw}

Investigation of filter width variation effects.




\chapter{Effect of Transport on PAH Evolution\label{ch:transport}}

The lifetime of soot is governed by the balance between growth and oxidation. Nucleation from collisions between PAH dimers~\cite{blanquart2009,schuetz2002,frenklach1991,wang2011}, condensation of PAH dimers on existing soot particles~\cite{blanquart2009,hmom2009}, and surface growth through the HACA surface reaction mechanism~\cite{frenklach1985,frenklach1991} are processes that increase soot mass by extracting species from the gas-phase. Conversely, oxidation by \ce{OH} and \ce{O2} removes carbon from particles~\cite{stanmore2001,neoh1981,kazakov1995} and can lead to their fragmentation and destruction~\cite{neoh1985,mueller2011}. Clearly, the dynamics of soot are governed by species with a broad range of chemical timescales and molecular weights. Therefore, accurately capturing interactions with gas-phase species is crucial when developing predictive models for soot evolution.

In \cref{ch:lesmodels}, the modeling foundation for LES was established. In particular, the classical nonpremixed flamelet equations for species and temperature~\cite{peters1984} were presented in \cref{sec:lesmodels:combust:flamelet} with modifications accounting for the formation of soot and radiative thermal losses. These equations were derived under the assumption that the species' effective Lewis numbers are unity. This assumption is based on the reasoning that turbulence mixes indiscriminately at sufficiently large Reynolds number~\cite{pitsch19981057}.

%% These equations were derived by transforming the physical space coordinate system of \cref{eq:lesmodels:combust:flamelet:consy,eq:lesmodels:combust:flamelet:const} to a mixture-fraction-based coordinate system. The latter conservation equations used Fick's law to evaluate the diffusion velocity, so all species were assumed to have the same constant diffusion coefficient $D$. A further simplication was made by selecting $D$ such that the species' effective Lewis numbers are unity, for it is often assumed that turbulence mixes all species indiscriminately at sufficiently high Reynolds numbers~\cite{pitsch19981057}.

However, DNS studies~\cite{bisetti2012,attili2014} have suggested that this assumption may be inappropriate for PAH, which are very sensitive to the local scalar dissipation rate due to their slow formation chemistry. These species are confined to spatially intermittent regions of low scalar dissipation rate that are on the order of the Kolmogorov scale or smaller~\cite{vaishnavi2008}. At such scales, transport is solely dictated by molecular diffusion. Therefore, a method to identify such strain-sensitive species and properly account for their interactions with turbulence is desired. The development of a model that addresses these points is the focus of this chapter.

%% Summary: Main point of chapter is to show that existing transport models do not properly account for the presence of soot. As a result, a strain sensitive transport model is developed.

%% Include a brief introduction explaining why accurately capturing the transport of gas-phase species is important for developing predictive models for soot evolution in LES (growth and oxixidation processes).

% include other files for sections of this chapter. These use the 'input' command since each section within a chapter should not start a new page.
% If you want to swap the order of sections, it is as simple as reversing the order you include them. 
\section{Overview of Gas-Phase Species Transport}
\label{sec:transport:overview}

Provide an overview of different transport modeling approaches and discuss their deficiencies.


\subsection{Equal Effective Diffusivities}
\label{sec:transport:overview:le1}

Summary of equal effective diffusivities approach.


\subsection{Considerations for PAH}
\label{sec:transport:overview:pah}

Explain why PAH requires another mode of transport (sensitivity to $\chi$ due to slow chemistry suggests molecular transport is more appropriate).


\subsection{Molecular Transport}
\label{sec:transport:overview:lei}

Summary of detailed transport approach.


\subsection{Bimodal Transport}
\label{sec:transport:overview:bimodal}

Summary of approaches that account for equal effective diffusivities as well as detailed transport (Wang).

\section{Strain-Sensitive Transport Approach}
\label{sec:transport:ssta}

Details on proposed strain-sensitive transport approach.


\subsection{Model Development}
\label{sec:transport:ssta:framework}

Introduce definitions of strain-sensitivity parameter as well as effective species Lewis number.


\subsection{\textit{A Priori} Analysis}
\label{sec:transport:ssta:dns}

Compare flamelet calculations of strain-sensitivity parameter to DNS database.


\subsection{Strain-Sensitivity Parameter Dependencies}
\label{sec:transport:ssta:dependencies}

Discuss dependencies on stoichiometric scalar dissipation rate, fuel mixture, chemical mechanism, and note any discrepancies between flamelet calculations and our intuition (ethylene, propane, etc.)


\chapter{Large Eddy Simulation of Turbulent Nonpremixed Jet Flames\label{ch:lesresults}}

Summary: This chapter covers all details on the experimental and computational framework used and the LES results that have incorporated the previous models discussed.

\section{Experimental Framework}
\label{sec:lesresults:exp}

LES of a series of turbulent, nonpremixed, ethylene-hydrogen-nitrogen [40/41/19 by volume] jet flames~\cite{mahmoud2017} is used to validate the $Z$-uniform soot subfilter PDF (\cref{eq:subfilter:zussp:dd}) in the framework presented in \cref{ch:lesmodels}. This is achieved by comparing thermocouple readings of flame temperature and laser-induced incandescence (LII) measurements of soot volume fraction with the profiles from LES. In this series, three flames are maintained at a jet exit Reynolds number of 15,000 while the exit strain rate is varied by altering the jet diameter and fuel exit velocity. A uniform coflow of air at a velocity of 1.1 m/s surrounds these flames. \cref{tab:subfilter:leszussp:ehn} summarizes other key aspects of the experimental setup, and complete details can be obtained from Mahmoud \etal~\cite{mahmoud2017}.

\begin{table}[htbp]
\centering
\caption[Flow Conditions of Turbulent Nonpremixed \ce{C2H4}/\ce{H2}/\ce{N2} Jet Flames]{Flow conditions of turbulent nonpremixed \ce{C2H4}/\ce{H2}/\ce{N2} jet flames.}
\label{tab:subfilter:leszussp:ehn}
\begin{tabular}{p{0.35\textwidth} p{0.1\textwidth} p{0.12\textwidth} p{0.12\textwidth} p{0.12\textwidth}}
\toprule
\textbf{Flame} & & \bm{$1/\tau|_{H}$} & \bm{$1/\tau|_{M}$} & \bm{$1/\tau|_{L}$} \\
\midrule

Central jet diameter, $D$
& [mm] & 4.4 & 5.8 & 8.0 \\[0.2em]

Mean jet exit velocity, $U$
& [m/s] & 56.8 & 42.4 & 31.5 \\[0.2em]

Exit strain rate, $U/D$
& [s$^{-1}$] & 12,900 & 7300 & 4300 \\[0.2em]

Exit Reynolds number, $Re_D$
& [--] & 15,000 & 15,000 & 15,000 \\[0.2em]

Mean flame length, $L_f$
& [mm] & 710 & 930 & 1060 \\

\bottomrule
\end{tabular}
\end{table}

% Discuss Adelaide flame series details.

\section{Computational Framework}
\label{sec:lesresults:comput}

Grid-filtered LES is achieved through NGA, a finite difference code for low-Mach number turbulent reacting flows~\cite{desjardins2008}. The conservation equations for mass and momentum are spatially discretized with second-order, central difference operators, and the transport equations for the scalars are spatially discretized with the third-order, Weighted Essentially Non-Oscillatory scheme~\cite{jiang1996}. All subfilter stresses and fluxes are closed with the dynamic procedure~\cite{germano1991,lilly1992,moin1991} with Lagrangian averaging over flow path lines~\cite{meneveau1996,reveillon1996}.

The computational domain of each flame consists of a structured grid with $192 \times 96 \times 32$ points in the streamwise, radial, and circumferential directions, respectively. This corresponds to a domain with lengths $200D$ in the streamwise direction and $60D$ in the radial direction, where the grid is stretched for both. In the vicinity of the burner wall, grid points are clustered in the radial direction. The inflow boundary conditions for each domain require an auxiliary simulation of a three-dimensional, fully developed turbulent periodic pipe flow with an average streamwise velocity that matches the corresponding mean jet exit velocity for each case of \cref{tab:subfilter:leszussp:ehn}. The data from these simulations become the inflow conditions within the burner. The air coflow is prescribed as a bulk flow of magnitude 1.1 m/s. The downstream exit of the domain is modeled with a convective outflow that globally conserves mass~\cite{akselvoll1996}, and the outer edge of the domain in the radial direction is modeled with a slip wall.

Each simulation utilizes 128 processors. The specific computational cost of each LES is between 2.05 $\mu$s/s and 8.11 $\mu$s/s and varies with the combination of models used. For each LES, statistics were collected over a period in the range of 308 ms to 1216 ms, with an average of 652 ms. This corresponds to an average of 21.8 flow through times. The total computational cost of each LES ranged from approximately 114,000 cpu-hours to 235,000 cpu-hours for a cumulative total of 1.9 million cpu-hours over all simulations with a number of model variations. %The latter and other simulation details are presented in \cref{tab:lesresults:comput:details}. NEED TO FINISH THIS TABLE.

%% \begin{table}[htbp]
%% \centering
%% \caption[LES Computational Details]{LES Computational Details}
%% \label{tab:lesresults:comput:details}
%% \begin{tabular}{p{0.35\textwidth} p{0.1\textwidth} p{0.12\textwidth} p{0.12\textwidth} p{0.12\textwidth}}
%% \toprule
%% \textbf{Flame} & & \bm{$1/\tau|_{H}$} & \bm{$1/\tau|_{M}$} & \bm{$1/\tau|_{L}$} \\
%% \midrule

%% Central jet diameter, $D$
%% & [mm] & 4.4 & 5.8 & 8.0 \\[0.2em]

%% Mean jet exit velocity, $U$
%% & [m/s] & 56.8 & 42.4 & 31.5 \\[0.2em]

%% Exit strain rate, $U/D$
%% & [s$^{-1}$] & 12,900 & 7300 & 4300 \\[0.2em]

%% Exit Reynolds number, $Re_D$
%% & [--] & 15,000 & 15,000 & 15,000 \\[0.2em]

%% Mean flame length, $L_f$
%% & [mm] & 710 & 930 & 1060 \\

%% \bottomrule
%% \end{tabular}
%% \end{table}


\subsection{Radiation Model}
\label{sec:lesresults:comput:rad}

In \cref{sec:lesmodels:combust}, the RFPV approach~\cite{ihme2008} was selected for the turbulent combustion model to account for radiative losses. Radiation is modeled with an optically thin gray gas approach, where the absorption coefficients for \ce{CO2}, \ce{H2O}, \ce{CH4}, and \ce{CO} are obtained from Barlow \etal~\cite{barlow2001}. Since soot radiation is included in addition to gas-phase radiation, the corresponding absorption coefficients are taken from Hubbard and Tien~\cite{hubbard1978}.


\subsection{Chemical Mechanisms}
\label{sec:lesresults:comput:chem}

The databases of solutions to the steady flamelet equations with radiative losses are computed with FlameMaster, a solver for 0D combustion and 1D laminar flame calculations~\cite{flamemaster}. Seven databases were generated to account for different combinations of chemical mechanisms, soot subfilter PDFs, and subfilter transport models. Five of the databases utilize the chemical mechanism of Blanquart \etal~\cite{blanquart2009588}, which encompasses the high temperature combustion of fuels from methane to iso-octane and places emphasis on the formation of soot precursors up to cyclopenta[cd]pyrene (\ce{C18H10}). The mechanism of Narayanaswamy \etal~\cite{narayanaswamy2010} is incorporated into this base mechanism to model the high-temperature oxidation of aromatic species. The combined mechanism contains 158 species and 1804 reactions, and will be referred to as BN1 henceforth. The remaining two databases use the chemical mechanism of Wang \etal~\cite{wang2013}, which contains the high temperature kinetics of \ce{C1-C4} hydrocarbons fuels and accounts for the formation and growth of PAH up to coronene (\ce{C24H12}). This mechanism, referred to as KM2, includes 202 species and 1351 reactions.

These databases are parameterized by $\tf{Z}$, $\tf{Z_V}$, $\tf{C}$, and $\tf{H}$, as previously discussed in \cref{sec:lesmodels:presumedpdf}, and have a resolution of 100 divisions for each parameter. Each database accounts for soot variables in addition to the quantities required for the RFPV approach~\cite{ihme2008}, so the LES code has been parallelized using a hybrid method of one MPI process per cluster node and sixteen OpenMP threads per node to accommodate the large memory requirements.

\section{Flow Field Boundary Conditions}
\label{sec:lesresults:bc}


\section{Z-Activated Soot Subfilter PDF}
\label{sec:lesresults:zassp}

Main points I want to make about ZASSP: The centerline temperature is still well predeicted compmared to the ZUSSP modmel, but the centerline volume fraction has barely changed when compare to the ZUSSP model. Looking at the radial statistics, it is clear tht he volume fraction is severely underpediceted as well. The flame structue is also too thin. Can look at the dfvdt source terms for mucl+cond, surface growth, and oxidation in the centerline and radial directions vs. Z (think if this makes any sense to show both). Effect of ZASSP supposed to be strongest for oxidation vs surface growth, so show instantaneous or time averaged 2D color plots of fv, number density (?), primary particle diameter (?), and intermittency (?). 

In \cref{fig:subfilter:leszussp:zusspleseval}, it is clear that time-averaged, centerline flame temperature is generally well-predicted by LES in all flames. In the recorded experiments, the flame temperature rises to a peak value around $x/D = 100$, at which point it declines due to the presence of soot. The flame with the highest exit strain rate ($1/\tau|_H$, left plot) possesses a small region of uniform temperature around $x/D = 0$ due to the lifting of the flame in LES. This feature is not present in the experiment, and therefore explains the slight deviation from the empirical values until about $x/D = 50$. In all three flames, the peak mean temperature from LES closely matches the magnitude of the corresponding experimental value (1800 to 1900 K), but is shifted slightly downstream.

\begin{figure}[htb]
  \centering
  \includegraphics[width=\linewidth]{ch-subfiltermodeling/figures/combined-Tfv-S-ZUSSP-Le_1}
  \caption[LES Validation of \texorpdfstring{$Z$}{Z}-Uniform Soot Subfilter PDF]{\textit{Left to right} - LES of jet flames $1/\tau|_H$, $1/\tau|_M$, and $1/\tau|_L$, respectively. Circles indicate experimental data while solid lines are time-averaged values from LES. Profiles in black correspond to centerline flame temperatures while profiles in blue represent centerline soot volume fractions. Note that the soot volume fraction profiles from LES have been premultiplied by a factor of 10 for visibility.}
  \label{fig:subfilter:leszussp:zusspleseval}
\end{figure}

Most obvious in \cref{fig:subfilter:leszussp:zusspleseval} is the underprediction of the time-averaged, centerline soot volume fraction by LES. In all three flames, the volume fraction is underestimated by more than two orders of magnitude relative to the empirical values. Furthermore, the mean soot volume fractions from LES peak slightly earlier and approach zero sooner at approximately $x/D = 130$. These phenomena signal the presence of intense oxidation further upstream compared to the experiments. Such a disparity between the predicted and measured profiles raises the question of what is modeled inaccurately. To determine the source of this inconsistency, the assumptions and deficiencies of the models need to be revisited.

The soot volume fraction is a measure of the total volume of soot relative to the total volume of gas. The amount of soot that is present depends on the rates of nucleation, condensation, surface growth, and oxidation. An underprediction of the volume fraction suggests that soot growth through nucleation, condensation, and surface growth is insufficient, or that soot oxidation is overwhelmingly excessive. Expressions for the filtered moment source terms of these modes are similar in form to \cref{eq:lesmodels:presumedpdf:separated}. For example, the filtered moment source term for oxidation is given by
\begin{equation}\label{eq:subfilter:leszussp:ox}
  \begin{split}
    \mean{\dot{M}}_{x,y}^{ox} &= \iint k_{ox}(Z)\mc{M}_j P(\mc{M}_j)\pz dZ d\mc{M}_j \\
    &= \hat{k}_{ox}\mean{M}_j,
  \end{split}
\end{equation}
where $k_{ox}(Z)$ is the oxidation coefficient, $\pz$ is modeled with a beta distribution as in \cref{eq:lesmodels:presumedpdf:trimarg}, and the soot subfilter PDF $P(\mc{M}_j)$ is provided in \cref{eq:subfilter:zussp:dd}. Contributions from the thermochemical and soot variables are made distinct in the second line, where $\hat{k}_{ox} = \int k_{ox}(Z)\pz dZ$ and $\mean{M}_j = \int \mc{M}_j P(\mc{M}_j)d\mc{M}_j$. \cref{eq:subfilter:leszussp:ox} incorporates the joint PDF simplification of Mueller and Pitsch~\cite{subfilterpdf2011}, which postulates that the soot scalars should be independent of the thermochemical variables due to the long timescales of PAH and soot formation relative to the timescales of the highly exothermic combustion chemistry. However, they also noted that such an assumption could be violated when surface growth near the flame is the dominant growth mechanism or during the oxidation of soot, when interactions with gas-phase chemistry are enhanced. The likelihood of the former is smaller, as the DNS studies of Bisetti \etal~\cite{bisetti2012} demonstrated that soot growth through PAH-based nucleation and condensation dominates acetylene-based surface growth in turbulent nonpremixed combustion. Therefore, this work hypothesizes that the excessive oxidation of soot, due to the simplification of the joint subfilter PDF (\cref{eq:lesmodels:presumedpdf:bayes}), is the reason for the dramatic underprediction of the soot volume fraction as shown in \cref{fig:subfilter:leszussp:zusspleseval}.

As a result of the decorrelation of the soot scalars from the thermochemical variables, the current soot subfilter PDF implictly assumes that soot is uniformly distributed in mixture fraction space. However, for non-smoking flames, this is qualitatively incorrect as there should be zero soot in fuel-lean regions of the flame. This is evident in \cref{fig:subfilter:leszussp:ysvsz}, reproduced from the 3D DNS of a nitrogen-diluted, \textit{n}-heptane/air turbulent nonpremixed planar jet flame~\cite{attili2014}. The lack of soot at mixture fractions below stoichiometric can be explained by the preferential transport of soot generated in the region $0.3 < Z < 0.5$ to richer mixture fractions by turbulent fluctuations and by the oxidation of all soot during transport towards the stoichiometric surface. 

%% \begin{figure}[htb]
%%   \centering
%%   \includegraphics[width=0.7\linewidth]{ch-subfiltermodeling/figures/dns_Ysoot_vs_Z}
%%   \caption[DNS of Turbulent Nonpremixed \ce{C7H16}/\ce{N2} Jet Flame, \texorpdfstring{$\langle Y_{\text{s}}|Z \rangle$}{<Ys|Z>} vs. \texorpdfstring{$Z$}{Z}]{Mean soot mass fraction conditioned on mixture fraction at various times in a 3D DNS of an \textit{n}-heptane/nitrogen [15/85 by volume] and air turbulent nonpremixed planar jet flame, reproduced from Attili \etal~\cite{attili2014}. The stoichiometric mixture fraction ($Z_{st} = 0.147$) is demarcated with the vertical dashed line. \textit{Left} - 1 ms (filled squares), 2 ms (crosses), 3 ms (open squares), 4 ms (circles), and 5 ms (triangles). \textit{Right} - 6 ms (stars), 8 ms (circles), 10 ms (open squares), and 20 ms (filled squares). The small gray dots represent the soot mass fraction field at 20 ms.}
%%   \label{fig:subfilter:leszussp:ysvsz}
%% \end{figure}

The non-uniform nature of soot in mixture fraction space is captured by the flamelet solutions that are accessed during LES as well. Filtered moment source term coefficients for oxidation, surface growth, and the combination of nucleation and condensation are plotted in \cref{fig:subfilter:leszussp:kvsz}. It is clear that different soot evolution modes are dominant over distinct regions of mixture fraction even as the fuel mixture and stoichiometric scalar dissipation rate are varied. Soot growth through PAH-based nucleation and condensation prevails at very rich values of mixture fraction, whereas acetylene-based surface growth is more dominant at moderately rich mixture fractions. It is notable that for a fixed fuel mixture (middle and right plots), a reduction in the value of stoichiometric scalar dissipation rate induces a large increase in the coefficient for nucleation and condensation while the coefficient for surface growth is lessened. This trend is due to the increasing effectiveness of PAH chemistry at smaller values of scalar dissipation rate. Therefore, PAH-based soot growth is supported at leaner conditions, which ultimately supplants acetylene-based surface growth.

%% \begin{figure}[htb]
%%   \centering
%%   \includegraphics[width=\linewidth]{ch-subfiltermodeling/figures/flamelet_sootcoeffs_le1}
%%   \caption[Soot Growth and Oxidation Coefficients, 1/\texorpdfstring{$\tau$}{t} vs. \texorpdfstring{$Z$}{Z}]{Flamelet calculations of soot growth and oxidation coefficients for filtered moment source terms as a function of mixture fraction. The blue lines are the oxidation coefficients, the red lines represent the surface growth coefficients, and the green lines are the coefficients for nucleation and condensation. \textit{Left} - Fuel mixture \textit{n}-heptane/nitrogen [15/85 by volume] used in the DNS of \cref{fig:subfilter:zussp:chisensitivity,fig:subfilter:leszussp:ysvsz} at $\chi_{st} = 10\ s^{-1}$. \textit{Middle} \& \textit{Right} - Fuel mixture \ce{C2H4}/\ce{H2}/\ce{N2} [40/41/19 by volume] used in the LES of \cref{fig:subfilter:leszussp:zusspleseval} at $\chi_{st} = 10\ s^{-1}$ and $\chi_{st} = 0.1\ s^{-1}$, respectively.}
%%   \label{fig:subfilter:leszussp:kvsz}
%% \end{figure}

Soot oxidation is the dominant mode at mixture fractions below and slightly above the stoichiometric value. Since the rates associated with the high-temperature oxidation chemistry are comparable with those of the main heat-releasing chemistry~\cite{guo2016}, it is anticipated that the magnitude of the peak oxidation coefficient will not vary much as the fuel mixture or stoichiometric scalar dissipation rate is modified. Indeed, this phenomenon is evident in \cref{fig:subfilter:leszussp:kvsz}. Additionally, the oxidation coefficient at stoichiometric mixture fraction is at least an order of magnitude larger than the maximum value of the surface growth coefficient. Thus, the complete oxidation of soot by \ce{OH} (and \ce{O2} to a lesser degree) is expected during transport towards stoichiometric regions. However, the soot subfilter PDF given by \cref{eq:subfilter:zussp:dd} permits the existence of soot in fuel-lean areas, for it assumes that soot is uniformly distributed in mixture fraction space. Therefore, the presence of large, non-zero oxidation coefficients at lean values of mixture fraction is concerning, for it potentially leads to substantial filtered moment source terms (\cref{eq:subfilter:leszussp:ox}). This contribution to the oxidation rate is artificial and could be an explanation for the underpredicted soot volume fraction in \cref{fig:subfilter:leszussp:zusspleseval}. To prevent this non-physical oxidation from occurring, the subfilter PDF must depend on the thermochemical variables to exclude the presence of soot at lean mixtures $Z < Z_{st}$. A soot subfilter PDF that addresses this point is introduced in the following section.

Perhaps have a comparison of flamelets first to give hints at what to expect in terms of $Y_{PAH}$ and $f_V$. Include comparison of EHN, Stanford, $\check{Le}_i(\zeta_i)$, and ZUSSP vs. ZASSP.

Talk about temperature, $Y_{PAH}$, $f_V$, conditional radial statistics at various distances downstream, etc.

\section{Strain-Sensitive Transport Approach}
\label{sec:lesresults:ssta}

%% what i want to say in this section: start with a comparison of unity to zeta to lei for flame 2. discuss centerline T and fv.
%% Then bring in radial statistics for unity vs zeta vs lek.

%% next, discuss zeta vs unity across all three flames for centerline T and fv structure.
%% bring in radial statistics for zeta vs unity and discuss T and fv.
%% Then look at radial conditional statistics at xD = 49, 71, 94, and 117 for dfv/dt for nucleation + condensation, surface growth, and oxidation to compare zeta and lek?
%% lastly, bring up point of different chemical mechanism - km2.
%% Show centerline T and fv for KM2 with le1 and le zeta, compare with existing models for flame 2.
%% finally, compare radial statistics for KM2 flame 2 against stanford mechanisms.

Results from the LES using the $Z$-Activated Soot Subfilter PDF were presented in \cref{sec:lesresults:zassp} under the assumption of unity effective Lewis numbers for all gas-phase species. This assumption is appropriate when turbulent transport dominates molecular diffusion within the fuel-oxidizer mixing zone and when the species' length scales are comparable to the length scale of the latter. In \cref{ch:transport}, this was found to be inappropriate for PAH, whose formation chemistry is slow relative to turbulent mixing. PAH are confined to regions of low scalar dissipation rate that are on the order of the Kolmogorov scale or smaller~\cite{vaishnavi2008}, so they are governed by molecular transport. The Strain-Sensitive Transport Approach was developed in \cref{sec:transport:ssta} to more accurately model the behavior of PAH and similar species with relatively slow chemistry. In this section, LES results with the aforementioned approach will be compared to experimental data and data from LES with other transport models. The goal is to evaluate the ability of the proposed transport model to more accurately predict the evolution of soot, so the $Z$-Activated Soot Subfilter PDF will be implemented in all cases of interest.

In \cref{fig:lesresults:ssta:f2lecomparison}, the time-averaged, centerline flame temperature is plotted for flame $1/\tau|_M$ of \cref{tab:subfilter:leszussp:ehn}. It is clear that the LES with unity effective Lewis numbers and the Strain-Sensitive Transport Approach predict the flame temperature fairly accurately, while the LES with detailed transport for gas-phase species predicts a much longer flame with a higher maximum temperature. The peak mean temperatures of the former two LES closely match the magnitude of the experimental value but are shifted slightly downstream as in \cref{fig:lesresults:zassp:ctrlineleseval}. The LES with full detailed transport accounts for the small Lewis number of the hydrogen radical, which causes a convective velocity towards below-stoichiometric regions in mixture fraction space. Consequently, the flame temperature is observed to peak in the leaner regions further downstream, where the hydrogren radicals participate more frequently in reactions that contribute to an elevated maximum flame temperature. % The LES with full detailed transport accounts for the high diffusivity of the hydrogen radical, which promotes its presence at locations far downstream and allows for more frequent participation in various reactions that contribute to an elevated maximum flame temperature.

\begin{figure}[H]
  \centering
  \includegraphics[width=0.6\linewidth]{ch-lesresults/figures/2f-Tfv-S-ZASSP-Le_comparison}
  \caption[Centerline $\langle T \rangle$ \& $\langle f_V \rangle$ from LES of Flame $1/\tau|_M$ with Various Transport Approaches]{Time-averaged flame temperatures and soot volume fractions from LES of jet flame $1/\tau|_M$. Note that the volume fraction profiles from LES have been premultiplied by a factor of 10 for visibility. Circles indicate experimental data, solid lines are LES with unity effective Lewis numbers for gas-phase species, dashed lines are LES with the Strain-Sensitive Transport Approach, and dash-dotted lines are LES with detailed transport for all gas-phase species. Profiles in black correspond to flame temperatures while profiles in blue represent soot volume fractions.}
  \label{fig:lesresults:ssta:f2lecomparison}
\end{figure}

As before, the time-averaged, centerline soot volume fractions are underpredicted by all simulations. For the LES with unity effective Lewis numbers and the Strain-Sensitive Transport Approach, the volume fractions peak at $x/D \approx 80$, versus the experimental maximum at $x/D \approx 100$, and approach zero earlier at $x/D \approx 130$ to 140. On the other hand, the LES that models all gas-phase species with molecular transport predicts a larger soot volume fraction at locations further downstream and has a volume fraction maximum at nearly the same location as the experiment. Its peak volume fraction is more than four times larger than that of the LES with unity effective Lewis numbers and slightly more than twice the maximum value from the LES with the Strain-Sensitive Transport Approach. The increased volume fraction further downstream can be attributed to the elongated flame structure, which enables soot to accumulate for a longer period of time before oxidation. % These two phenomena result from modeling the hydrogen radical with molecular transport, which allows it to diffuse further downstream and contribute to soot growth through the \ce{H}-abstraction, \ce{C2H2}-addition surface reaction mechanism~\cite{frenklach1985,frenklach1991}.

\begin{figure}[H]
  \centering
  \includegraphics[width=\linewidth]{ch-lesresults/figures/dfvdt-S-ZASSP-Le_comparison-nc-combined-cropped}
  \caption[Radial $\langle df_V/dt \rangle/|\langle df_V/dt|_{\text{max}} \rangle|$ from LES with Various Transport Approaches]{\textit{Left to right} - Time-averaged soot volume fraction source terms from LES of jet flame $1/\tau|_M$ with unity effective Lewis numbers, strain-sensitive transport, and detailed transport, respectively. These are normalized by the absolute value of the maximum source term contribution within each flame. \textit{Bottom to top} - Increasing streamwise locations of $x/D = 49, 71, 94,$ and 117, respectively. Profiles in dark green are source terms from nucleation, profiles in green are source terms from condensation, profiles in red are source terms from surface growth, and profiles in blue are source terms from oxidation.}
  \label{fig:lesresults:ssta:radialdfvdtf2lecomparison}
\end{figure}

Additionally, the hydrogen radical is modeled with molecular transport, which allows it to diffuse further downstream and contribute to soot growth through the \ce{H}-abstraction, \ce{C2H2}-addition surface reaction mechanism~\cite{frenklach1985,frenklach1991}. This behavior can be seen more clearly in \cref{fig:lesresults:ssta:radialdfvdtf2lecomparison}, where the LES with detailed transport possesses a normalized time-averaged soot volume fraction source term for surface growth that has a proportionally larger contribution when compared to the LES with the other transport models. In fact, at $x/D = 94$, it becomes the most dominant component for the LES with detailed transport. Conversely, the LES with unity effective Lewis numbers and the Strain-Sensitive Transport Approach have oxidation by \ce{OH} and \ce{O2} as the most prominent source term at the same axial location.

However, the hydrogen radical is not identified as a species with relatively slow chemistry in \cref{fig:transport:ssta:dependencies:chist}. It is not confined to regions of low scalar dissipation rate that are on the order of the Kolmogorov scale or smaller, so modeling its behavior with molecular diffusion is inappropriate. This point is clearly demonstrated by the inaccurate flame temperature profile of the LES with detailed transport in \cref{fig:lesresults:ssta:f2lecomparison}. The Strain-Sensitive Transport Approach more accurately models its physics as well as the physics of species with relatively slow chemistry. In \cref{fig:lesresults:ssta:f2lecomparison}, the LES with the latter model generally predicts a larger soot volume fraction over the LES with unity effective Lewis numbers. A closer look at the relative contributions of the various volume fraction source terms in \cref{fig:lesresults:ssta:radialdfvdtf2lecomparison} reveals that at $x/D = 49$, the LES with the proposed model has nucleation and condensation occurring at a larger proportion than in the LES with unity effective Lewis numbers. Such a trend is responsible for the increased volume fraction at that location and further downstream, as observed in \cref{fig:lesresults:ssta:f2lecomparison}.

%nucleation, condensation, surface growth, and oxidation occurring at roughly the same proportions as the LES with unity effective Lewis numbers. However, by $x/D = 71$, the LES with the strain-sensitive transport approach predicts a higher amount of condensation relative to nucleation for $r/D \le 2$ when compared to the LES with unity effective Lewis numbers. Such a trend is responsible for the increased volume fraction in the region around $x/D = 71$, as observed in \cref{fig:lesresults:ssta:f2lecomparison}.

The previously discussed results use the chemical mechanism from Blanquart \etal~\cite{blanquart2009} and Narayanaswamy \etal~\cite{narayanaswamy2010} (identified as the BN1 mechanism), which places emphasis on the formation of soot precursors up to cyclopenta[cd]pyrene (\ce{C18H10}). The sensitivity of soot evolution to the kinetics was also assessed by performing LES with the chemical mechanism from Wang \etal~\cite{wang2013} (identified as the KM2 mechanism), while modeling gas-phase species with unity effective Lewis numbers and with the Strain-Sensitive Transport Approach. In \cref{fig:lesresults:ssta:f2lemechcomparison}, it is evident that the time-averaged, centerline flame temperatures from these LES closely match the experimental data. However, the soot volume fractions are obviously increased. The peak time-averaged volume fraction from the LES using the KM2 mechanism with unity effective Lewis numbers is 29 ppb, which is roughly tenfold larger than the maximum value from the corresponding LES with the BN1 mechanism (2.9 ppb). The LES using the KM2 mechanism with the Strain-Sensitive Transport Approach has a peak volume fraction of 46.7 ppb, which is within a factor of ten of the peak experimental value and is about eight times larger than the maximum value from the corresponding LES with the BN1 mechanism (6.0 ppb). Clearly, the results are sensitive to the chemical kinetics. However, the LES with the KM2 mechanism still underpredicts the soot volume fraction by a factor of ten, suggesting that the model is still missing some aspect of the physics of soot evolution, and uncertainties in chemical kinetics alone are not enough to explain the discrepancies. % The KM2 mechanism accounts for PAH species up to coronene (\ce{C24H12}), which has seven aromatic rings compared to the five fused rings of cyclopenta[cd]pyrene. The LES results suggest that the kinetics of these larger PAH may play an important role in the evolution of soot, as was found in previous studies~\cite{wang2013,selvaraj2016}. 

\begin{figure}[htb]
  \centering
  \includegraphics[width=\linewidth]{ch-lesresults/figures/2f-Tfv-KS-ZASSP-Le_comparison-split}
  \caption[Centerline \texorpdfstring{$\langle T \rangle$}{<T>} \& \texorpdfstring{$\langle f_V \rangle$}{<fV>} from LES of Flame \texorpdfstring{$1/\tau|_M$}{1/t|M} with Various Transport Approaches and Chemical Mechanisms]{Time-averaged flame temperatures and soot volume fractions from LES of jet flame $1/\tau|_M$. Note that the volume fraction profiles from LES have been premultiplied by a factor of 10 for visibility. All symbols and lines are the same as in \cref{fig:lesresults:ssta:f2lecomparison}, where the chemical mechanism from Blanquart \etal~\cite{blanquart2009} and Narayanaswamy \etal~\cite{narayanaswamy2010} was used. However, results from LES with the chemical mechanism of Wang \etal~\cite{wang2013} have been added - double dotted lines are LES with unity effective Lewis numbers, and dash-double dotted lines are LES with the Strain-Sensitive Transport Approach.}
  \label{fig:lesresults:ssta:f2lemechcomparison}
\end{figure}

The evolution of soot is clearly sensitive to the kinetics of PAH precursors, as evidenced by the increase in the maximum volume fraction by a factor of 7.7 to 9.8 with the KM2 chemical mechanism compared to that from Blanquart \etal~\cite{blanquart2009} and Narayanaswamy \etal~\cite{narayanaswamy2010}. However, note that accurately modeling the interactions of PAH and other species that influence the development of soot particles with turbulence cannot be neglected. The Strain-Sensitive Transport Approach elevates predictions of the peak volume fraction by a preliminary factor of 1.6 to 2.1 over the equal effective diffusivities approach. Improvements to the proposed models, which will be discussed in \cref{sec:conclusion:future}, could potentially make the sensitivity to the turbulent interactions as influential as the sensitivity to the chemical mechanism.

%% Include comparison of EHN, Stanford, ZASSP, and $Le = 1$ vs. $\check{Le}_k(\zeta_k)$ vs. $Le_k$.

%% Also include the effect of the chemical mechanism with a comparison of EHN, Stanford, ZASSP, and $\check{Le}_k(\zeta_k)$ with EHN, KM2, ZASSP, and $Le = 1$ vs. $\check{Le}_k(\zeta_k)$.

%% Talk about temperature, $Y_{PAH}$, $f_V$, conditional radial statistics at various distances downstream, etc.

\section{Trends With Global Strain Rate}
\label{sec:lesresults:strain}

LES with the $Z$-Activated Soot Subfilter PDF and the Strain-Sensitive Transport Approach have also been performed for flames $1/\tau|_H$ and $1/\tau|_L$ from \cref{tab:subfilter:leszussp:ehn}. In \cref{fig:lesresults:strain:allflameslecomparison}, these are compared to the LES that model all gas-phase species with equal effective diffusivities. It can be observed that the choice of transport model does not significantly impact the predictions of the flame temperature, even as the exit strain rate is varied. The gas-phase species that participate in the main heat-releasing reactions, such as \ce{CO} or \ce{CO2}, have fast chemical production rates compared to the rates of local mixing. In fact, their production rates easily adjust to the global strain rate, resulting in similar experimental peak temperatures among the three flames. These species are not identified as strain-sensitive according to \cref{eq:transport:ssta:framework:ssp}, so they are modeled with unity effective Lewis numbers in the Strain-Sensitive Transport Approach. As a result, the corresponding profiles closely follow those of the LES with unity effective Lewis numbers. Both transport models closely match the flame temperatures from the experimental data, suggesting that the diffusive transport of the aforementioned species is not important to the dynamics of the main heat-releasing reactions. Similar conclusions were reached in previous studies~\cite{attili2015,attili2016}.

\begin{figure}[htb]
  \centering
  \includegraphics[width=\linewidth]{ch-lesresults/figures/combined-Tfv-S-ZASSP-Le_comparison}
  \caption[Centerline $\langle T \rangle$ \& $\langle f_V \rangle$ from LES of Flames $1/\tau|_H$, $1/\tau|_M$, and $1/\tau|_L$ with Various Transport Approaches]{\textit{Left to right} - Time-averaged flame temperatures and soot volume fractions from LES of jet flames $1/\tau|_H$, $1/\tau|_M$, and $1/\tau|_L$, respectively. Note that the soot volume fraction profiles from LES have been premultiplied by a factor of 10 for visibility. Circles indicate experimental data, solid lines are LES with unity effective Lewis numbers for gas-phase species, and dashed lines are LES with the Strain-Sensitive Transport Approach. Colors of profiles are the same as in \cref{fig:lesresults:ssta:f2lecomparison}.}
  \label{fig:lesresults:strain:allflameslecomparison}
\end{figure}

On the other hand, the soot volume fraction is significantly influenced by the global strain rate. The experimental data reveals an inverse relationship between the global strain rate and the centerline soot volume fraction, where jet flame $1/\tau|_H$ has the smallest maximum value of 197 ppb at $x/D = 103$ and jet flame $1/\tau|_L$ has the largest maximum value of 444 ppb at $x/D = 83$. Jet flame $1/\tau|_M$ has an exit strain rate that is 1.7 times larger than that of flame $1/\tau|_L$, yet its peak volume fraction is only a slightly diminished 437 ppb at $x/D = 105$. Mahmoud \etal~\cite{mahmoud2017} have attributed this nonlinearity to the secondary influence of buoyancy, which plays a larger role for the latter two flames. Nonetheless, previous studies have observed this inverse trend with strain rate in both laminar~\cite{decroix2000,huijnen2010,wang2016433} and turbulent~\cite{kent1984,mahmoud2017} nonpremixed flames.

In \cref{fig:lesresults:strain:fvvsstrain}, the LES with strain-sensitive transport and equal effective diffusivities transport generally capture this behavior with respect to the global strain rate, although both still underpredict the normalized maximum centerline volume fraction for jet flame $1/\tau|_M$. Additionally, it is clear in \cref{fig:lesresults:strain:fvvsstrain,fig:lesresults:strain:allflameslecomparison} that the difference between the two approaches is relatively small for jet flame $1/\tau|_H$. At large values of global strain rate, PAH are much more spatially intermittent due to the enhanced rates of turbulent mixing. The overall amount of PAH is reduced, so PAH-based soot growth modes have diminished contributions to the overall volume fraction. Therefore, the proposed model's improvement in the prediction of the soot volume fraction over the unity effective Lewis number approach is expected to be limited.

% The former LES tends to capture the trend better than the latter, although both still underpredict the normalized maximum centerline volume fraction for jet flame $1/\tau|_M$.

\begin{figure}[htb]
  \centering
  \includegraphics[width=0.45\linewidth]{ch-lesresults/figures/fv-vs-tau-S-ZASSP-Le_combined}
  \caption[Normalized Maximum $\langle f_V \rangle$ Versus Exit Strain Rate]{Time-averaged maximum soot volume fractions normalized by the maximum value from from jet flame $1/\tau|_L$ as a function of exit strain rate. Circles indicate experimental data, solid lines with triangles are from LES with unity effective Lewis numbers, and solid lines with squares are from LES with the Strain-Sensitive Transport Approach.}
  \label{fig:lesresults:strain:fvvsstrain}
\end{figure}

On the contrary, PAH are more prevalent at smaller global strain rates, so accurately modeling their transport is critical to predicting the evolution of soot. In \cref{fig:lesresults:strain:allflamesdfvdt}, significant differences exist between the two transport models regarding the source terms for nucleation, condensation, and surface growth. As the global strain rate is decreased, it can be observed that the source term due to combined nucleation and condensation grows in importance further downstream, especially for the LES with the Strain-Sensitive Transport Approach. At the same time, the source term due to surface growth becomes less dominant at upstream locations. In the LES of jet flame $1/\tau|_L$ with the proposed model, the maximum value of the source term due to nucleation and condensation is 1.67 times the peak value of the source term due to surface growth, highlighting the importance of properly modeling the physics of PAH. Conversely, this factor is only 1.25 in the LES with unity effective Lewis numbers. At the elevated global strain rate of jet flame $1/\tau|_M$, the maximum value of the PAH-based source term is 1.14 times larger than that of the source term due to surface growth in the LES with the Strain-Sensitive Transport Approach. Meanwhile, the former is actually smaller than the latter in the LES with unity effective Lewis numbers, reiterating the point that this model is not appropriate for strain-sensitive species such as PAH.

% Conversely, the maximum value of the source term due to surface growth always remains larger than that of the source term due to condensation in the LES with unity effective Lewis numbers. However, this model is not appropriate for strain-sensitive species such as PAH.

\begin{figure}[htb]
  \centering
  \includegraphics[width=\linewidth]{ch-lesresults/figures/combined-dfvdt-S-ZASSP-Le_comparison-nc-combined}
  \caption[Centerline $\langle df_V/dt \rangle$ from LES of Flames $1/\tau|_H$, $1/\tau|_M$, and $1/\tau|_L$ with Various Transport Approaches]{\textit{Left to right} - Time-averaged soot volume fraction source terms from LES of jet flames $1/\tau|_H$, $1/\tau|_M$, and $1/\tau|_L$, respectively. Solid lines are LES with unity effective Lewis numbers for gas-phase species and dashed lines are LES with the Strain-Sensitive Transport Approach. Colors of profiles are the same as in \cref{fig:lesresults:ssta:radialdfvdtf2lecomparison}.}
  \label{fig:lesresults:strain:allflamesdfvdt}
\end{figure}

It is a valid model for species with relatively fast chemistry. It is evident in \cref{fig:lesresults:strain:allflamesdfvdt} that the volume fraction source term for oxidation is fairly insensitive to the global strain rate for both transport models. Soot oxidation by \ce{OH} and \ce{O2} occurs relatively quickly, so this process can adapt to elevated levels of turbulent mixing. As confirmed in \cref{fig:transport:ssta:dependencies:chist}, these species should be modeled with unity effective Lewis numbers. % It is also notable that the volume fraction source term due to nucleation is insensitive to the strain rate. For both transport models, the maximum value of the latter hardly varies, although the axial location of the peak slightly changes as the strain rate is decreased. This behavior agrees with the findings of previous studies, where the production of the smallest, nascent soot particles has been observed to be minimally affected by the global strain rate~\cite{huijnen2010}.


\chapter{Conclusion\label{ch:conclusion}}

This thesis has presented several advancements in modeling the evolution of soot and its precursors in turbulent nonpremixed combustion. These models were developed for Large Eddy Simulation (LES), where the geometry-dependent, large-scale phenomena are resolved and the small-scale features are modeled. LES provides improved predictions of turbulent mixing over simulations with Reynolds-Averaged Navier-Stokes (RANS), especially for large-scale phenomena such as swirl, separation, and recirculation that can impact soot dynamics. The evolution of soot is also heavily influenced by its physics and chemistry occurring in the unresolved scales, where the combustion community's understanding is far from complete. Thus, the principal focus of this thesis was to provide novel insight and develop LES models for the interactions between soot, chemistry, and turbulence as well as for the transport of species with relatively slow chemistry at these small scales. The main results from model development, validation, and application are summarized below.

The framework for modeling the evolution of soot in LES of turbulent nonpremixed combustion was outlined in \cref{ch:lesmodels}. The Method of Moments was chosen to model the Number Density Function (NDF) of soot particles due to its lower computational expense. This statistical approach uses a bivariate volume $V$ and surface area $S$ description of soot to account for their geometrical complexities and involves solving equations that track the evolution of moments of the NDF. Closure of these equations was achieved through the Hybrid Method of Moments (HMOM)~\cite{hmom2009}, which accounts for the bimodal nature of the NDF resulting from the presence of small, incipient particles and larger, more mature aggregates. HMOM provides models for the physical and chemical processes that govern soot evolution including nucleation, coagulation, condensation, surface growth, oxidation, and fragmentation.

This soot model was integrated into a turbulent combustion model that describes the thermal and chemical structure of the nonpremixed flame and accounts for the formation of soot. In the flamelet approach~\cite{peters1984}, a three-dimensional turbulent flame is conceptualized as an ensemble of locally one-dimensional flame structures embedded in a turbulent flow field. Since thermal radiation can occur on similar temporal scales as soot evolution, the Radiation Flamelet/Progress Variable (RFPV) approach~\cite{ihme2008} with adaptions from Carbonell \etal~\cite{carbonell2009} was selected as the foundation for the turbulent combustion model. In this approach, the local thermochemical state is described by solutions to the steady flamelet equations that are augmented with radiative losses. These solutions are computed \textit{a priori} and are stored in a database that is accessed during LES through a reduced set of parameters that includes the filtered mixture fraction and its subfilter variance, the filtered progress variable, and the filtered heat loss parameter. Following Mueller and Pitsch~\cite{mueller2012}, the transport equation definitions for these parameters were modified to account for the extraction of PAH from the gas-phase and provide a unique parameterization of the thermochemical state.

However, previous works~\cite{attili2014,bisetti2012} have found that the use of the steady flamelet approximation leads to inaccurate predictions for the mass fractions of gas-phase PAH due to their relatively slow chemistry. To address this point, the transport equation model for PAH from Mueller and Pitsch~\cite{mueller2012} was incorporated into the LES modeling framework, where the chemical source term of the transport equation accounts for the production, consumption, and dimerization of PAH.

\section{Major Contributions}
\label{sec:conclusion:contributions}

Outline major contributions


\subsection{Subfilter Modeling Advancements}
\label{sec:conclusion:contributions:subfilter}

Summarize subfilter modeling advancements.


\subsection{Transport for Strain-Sensitive Species}
\label{sec:conclusion:contributions:transport}

Summarize advancements for transport modeling.
 % Contributions
\section{Recommendations for Future Work}
\label{sec:conclusion:future}

Discuss future work.


\subsection{Z-Activated Soot Subfilter PDF}
\label{sec:conclusion:future:zassp}

Future work for ZASSP.


\subsection{Strain-Sensitive Transport Approach}
\label{sec:conclusion:future:ssta}

Future work for SSTA.

%% Future work should include options in the template for a masters thesis or an undergraduate senior thesis. It should also support running headings in the headers using the `headings' pagestyle.  The print mode and proquest mode included in the template might also be candidates to include in the class itself.
 % Future work

\appendix % all chapters following will be labeled as appendices
\include{ch-appendicies/implementation}
\include{ch-appendicies/printing}
%\include{ch-pastwork/chapter-pastwork}
%\include{ch-usage/chapter-usage}

% Make the bibliography single spaced
\singlespacing
\bibliographystyle{acm}

% add the Bibliography to the Table of Contents
\cleardoublepage
\ifdefined\phantomsection
  \phantomsection  % makes hyperref recognize this section properly for pdf link
\else
\fi
\addcontentsline{toc}{chapter}{Bibliography}

% include your .bib file
\bibliography{thesis}

\end{document}

